\section{Przykładowa lista ekwipunku uczestnika\\obozu wędrownego w~górach\label{lista-ekwipunku-uczestnika}}
\subsection{Podstawa}
\begin{itemize}
\item Jedziemy w~góry, wszystko trzeba będzie nosić, więc bierzemy najmniej jak się da. Bieliznę i~inne ubrania będziemy prać, więc bierzemy tylko po kilka sztuk.
\item Jeśli czegoś zabraknie, to na miejscu kupimy, nie jedziemy na odludzie, na miejscu są sklepy.
\item Plecak ma być spakowany porządnie, wszystko ma być posortowane i~włożone do worków na mrożonki.
\item Każdy sam pakuje swój plecak w~taki sposób, aby wiedział, gdzie co jest.
\item Wszystko powinno być wewnątrz plecaka, nic na zewnątrz, a~jeśli już, to porządnie przytroczone trokami, żadnych sznurków.
\item W~plecaku musi być jeszcze miejsce na jedzenie obozowe, które będziemy nosić ze sobą.
\item W~przypadku wątpliwości dotyczących kupowania nowego sprzętu polecam testy sprzętu: \href{http://www.ngt.pl}{http://www.ngt.pl} i~opinie użytkowników: \href{http://ngt.pl/forum/}{http://ngt.pl/forum/}.
\end{itemize}

\subsection{Obuwie}
\begin{checklist}
\item Buty trekingowe na porządnej podeszwie, sznurowane. Nie muszą być z~membraną. Mają być wygodne i~rozchodzone. Za duże o~1/2 numeru.
\item Sandały (nie skórzane) --- powinny mieć paski pod całą podeszwą, a nie tylko mocowane po bokach.
\item Adidasy lub inne lekkie buty do chodzenia/biegania ,,nie na szlaku''.
\end{checklist}

\subsection{Bielizna}
\begin{checklist}
\item Majtki --- maksymalnie 6~par --- jeśli ktoś ma z~tkaniny termoaktywnej to wziąć koniecznie.
\item Cienkie skarpetki z coolmaxem --- 3 pary.
\item Grube skarpetki z coolmaxem --- 3 pary.
\item Koszulki z krótkim rękawem --- 3 pary --- jeśli ktoś ma z~tkaniny termoaktywnej to wziąć koniecznie.
\end{checklist}

\subsection{Odzież}
Nie brać ubrań, szczególnie spodni i~krótkich spodenek, w~jasnych kolorach, bo szybko się brudzą.
\begin{checklist}
\item Spodnie długie, najlepiej M-65 lub podobne --- 1 para.
\item Krótkie spodenki z~nogawkami nad kolana --- 2 pary.
\item Pasek do spodni.
\item Koszula flanelowa z~długim rękawem.
\item Polar 200 albo 300.
\item Kurtka od deszczu, z~kapturem, cienka, z~membraną, nie brać kurtek zimowych.
\item Spodnie od deszczu z~cienkiej membrany --- kto ma, niekoniecznie.
\item Kapelusz z rondem / czapka z daszkiem.
\item Chustka na głowę / apaszka.
\item Koszulka drużyny i~umundurowanie letnie, jeśli jest.
\item Kąpielówki.
\item Zestaw na powrót (lekkie cienkie długie spodnie, majtki, koszulka, cienkie skarpetki --- porządnie zapakowane w~wodoszczelny worek).
\end{checklist}

\subsection{Akcesoria kuchenne}
\begin{checklist}
\item Menażka okrągła, żadnych jajowatych, ,,lornetek'' ani innego wojskowego złomu.
\item Łyżka.
\item Widelec.
\item Nóż w~pokrowcu / futerale: finka lub scyzoryk (naostrzony).
\item Kubek, najlepiej termiczny.
\end{checklist}

\subsection{Kosmetyki i~przybory toaletowe}
\begin{checklist}
\item Ręcznik mały.
\item Ręcznik kąpielowy.
\item Mydło.
\item Pasta do zębów.
\item Szampon do włosów.
\item Szczoteczka do zębów.
\item Środek przeciw komarom.
\item Obcinaczka do paznokci / skórek.
\item Grzebień.
\item Krem ochronny z~filtrem.
\item Chusteczki higieniczne.
\item Papier toaletowy.
\end{checklist}
Wszystkie płynne kosmetyki warto przelać do mniejszych buteleczek, około 100~ml., aby nie nosić potem zbędnego ciężaru.

\subsection{Spanie}
\begin{checklist}
\item Piżama / coś do spania.
\item Śpiwór w pokrowcu / worku kompresyjnym.
\end{checklist}

\subsection{Transport}
\begin{checklist}
\item Plecak, dopasowany do wzrostu --- po spakowaniu wszystkiego plecak nie może ważyć więcej niż 1/4 wagi ciała.
\item Pokrowiec przeciwdeszczowy na plecak.
\item Troki zapasowe do plecaka.
\item Woreczki foliowe do mrożonek, bez metalowych klipsów, 3 litrowe --- 1 paczka.
\end{checklist}

\subsection{Pozostałe}
\begin{checklist}
\item Lekarstwa osobiste (na alergię, chorobę lokomocyjną i~inne).
\item Sznurek / linka ok. 2 metry.
\item Długopis.
\item Cienki zeszyt w kratkę --- 16 kartek / notatnik / notes.
\item Pieniądze na drobne wydatki (w~małym portfelu).
\item Ważna legitymacja szkolna ze zdjęciem właściciela i~wszystkimi wymaganymi pieczątkami.
\item Okulary słoneczne.
\item Okulary korekcyjne (jeśli ktoś nosi).
\item Latarka --- czołówka z~zapasową żarówką.
\item Igła i~nici.
\end{checklist}