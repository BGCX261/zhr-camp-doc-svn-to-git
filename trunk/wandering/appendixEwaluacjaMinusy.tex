\section{Ewaluacja obozu --- minusy\label{ewaluacja-minusy}}
Poniżej znajduje się autentyczna lista minusów z~komentarzem opracowanym po pewnym obozie, na który pojechały wspólnie: drużyna harcerek, drużyna wędrowniczek i~drużyna wędrowników. Ewaluacja została przeprowadzona przez kadrę podobozu harcerek. Od myślników wymieniono pojedyncze głosy instruktorek.
\subsection{Dezorientacja wśród kadry}
Nie wszyscy znają plan dnia po dojściu do schroniska. W~przyszłości należy zapoznać całą kadrę z~najistotniejszymi ustaleniami i, oczywiście, dać jej prawo głosu.
\begin{itemize}
\item[-] Plan obozu, czyli wszystkie zajęcia, powinny być omówione z~wszystkimi osobami z~kadry przed obozem, żeby mogła zostać uzgodniona ich forma oraz aby utrwaliła się kolejność.
\item[-] Przed obozem powinno się przedyskutować kompetencje oboźnej i~komendantki, bo później jedna nie chce wchodzić w~kompetencje drugiej. W~efekcie nikt nie wydaje stosownych decyzji i~robi się zamieszanie.
\item[-] Każda osoba z~kadry powinna mieć wydrukowany, na własny użytek, program obozu.
\item[-] Przypominaniem czegokolwiek po dojściu do schroniska powinna się zająć oboźna lub komendantka podobozu. Głosować nie ma nad czym, bo wszystko jest już dawno ustalone. Błahe decyzje, które należy podjąć w~określonym momencie, powinny być wydawane jednoosobowo bez zbędnych głosowań (informując przy tym inne osoby z~kadry, aby nie zostały wydawane sprzeczne polecenia, ale także dlatego, żeby cała kadra wiedziała co się dzieje w~danej chwili, a~nie dowiadywała się równo z~uczestnikami, a~nawet po nich).
\item[-] Wątpliwości dotyczące logistyki powinny być przedyskutowane w~gronie kadry w~sposób konkretny i~szybki. Jałowe, przeciągające się dyskusje powinny być tępione.
\end{itemize}
\subsection{Członkowie kadry myślą głównie o~zajęciach prowadzonych przez siebie, niedostatecznie interesując się zajęciami prowadzonymi przez innych}
Przed obozem nie zapoznaliśmy się dokładnie z~zajęciami, które miały być prowadzone przez inne osoby, przez co pewne pomysły się powielały. Za rzadko z~własnej inicjatywy pomagaliśmy sobie w~prowadzeniu zajęć, lub np. w~rozłożeniu punktów na grę. Gdy ktoś miał czas, powinien był zaproponować choćby niewielką pomoc.
\begin{itemize}
\item[-] Zajęcia powinny być prowadzone wspólnie i~dobrze przygotowane przed obozem. W~momencie, w~którym trzeba było rozłożyć grę, a~wszyscy byli zmęczeni, każdy liczył, że ktoś inny to zrobi. Dlatego do każdych zajęć powinny być wydelegowane 2~osoby prowadzące, odpowiedzialne za przygotowanie i~przebieg zajęć (1~główna osoba prowadząca i~1~do pomocy w~prowadzeniu zajęć, np. przy drugim końcu stołu albo sali, pilnująca, aby uczestnicy się nie rozpraszali). Pomocnik przydaje się gdy trzeba coś wytłumaczyć jednej lub dwóm osobom nie męcząc grupy powtarzaniem czegoś, co dla reszty jest jasne.
\item[-] Osoba z~niedopracowanymi zajęciami lub potrzebująca chociażby rozstawienia punktów gry nie może bać się prosić o~pomoc. Liczenie, że coś się zrobi samo lub później do niczego nie prowadzi, a~rozpoczęcie zajęć może się opóźnić.
\item[-] Wszystkie osoby z~kadry, które nie były zajęte aktualnymi sprawami organizacyjnymi, powinny brać udział w~zajęciach innych osób, choćby dlatego, aby wiedzieć, jaka wiedza zostaje przekazana ich podopiecznym i~czego mogą wymagać w~ciągu roku lub jaką wiedzę należy zweryfikować ponownie za jakiś czas. Dzisiaj żadna osoba z~kadry nie jest w~stanie stwierdzić, czego ludzie nauczyli się na zajęciach innych osób. Poza tym uczestnicy widzą, że kadra też jest na zajęciach, a~nie ,,siedzi i~się obija albo wariuje z~chłopakami''.
\end{itemize}
\subsection{Spotkania przedobozowe i~zajęcia}
Przed wyjazdem powinniśmy poświęcić więcej czasu sprawie zajęć programowych. Powinny być dopracowane i~ciekawiej przygotowane. Dobrze spotkać się więcej razy wyłącznie w~tym celu i~wymienić się pomysłami, przygotować pewne rzeczy razem, a~jeśli podzielimy się pracą, dać wszystkim dostęp do tego, co stworzyliśmy m.in. aby w~razie czego coś poprawić. Problemem było jeszcze to, że niektóre zajęcia kończyły się przed upływem czasu, który był na nie przeznaczony i~powstawały luki, podczas których ludzie nie mieli nic do roboty więc zakłócali spokój w~schronisku, rozbiegali się itd. Znów rozwiązaniem jest przygotowanie dodatkowych gier i~zajęć do wykorzystania w~takiej sytuacji, np. kiedy osoba odpowiedzialna za jakąś grę jest w~szpitalu.
\begin{itemize}
\item[-] Przed przygotowaniem zajęć sam ich pomysł powinien zostać wspólnie przedyskutowany. Powinien być atrakcyjny i~maksymalnie ,,wypasiony'' --- w~razie problemów jest pełno osób, które znają temat --- można je poprosić o~podpowiedzenie pewnych rozwiązań. Zawsze też można poszukać w~internecie. Jednym słowem --- trzeba się przygotować do tego i~to już od początku kwietnia.
\item[-] Wszystko, co się da, powinno być profesjonalnie przygotowane na długo przed obozem, pokazane reszcie kadry. Powielanie pewnych elementów, jak np. bieganie wg azymutów, na różnych zajęciach nie jest niczym złym, a~utrwala wiedzę i~pokazuje jej użyteczność.
\item[-] Opracowanie zajęć rezerwowych jest konieczne, podobnie jak opracowanie zajęć na czas deszczu. Umarło gdzieś pojęcie pomocy na ochotnika --- każdy był zmęczony, a~przecież kadra powinna się wspierać. Zajęcia powinny być przygotowywane i~prowadzone przez co najmniej 2 osoby. W~tym czasie pozostałe osoby powinny biegać po krzakach i~rozkładać gry.
\item[-] Zajęcia trzeba dostosować do potrzeb uczestników, a~nie kierować się jedynie sztampowym układem zajęć z~każdego obozu. Ważniejszy jest cel, który chcemy osiągnąć.
\item[-] Przed obozem powinnyśmy się zapoznać z~planem zajęć innej osoby by nie powielać formy zajęć lub po prostu doradzić komuś swoimi pomysłami. Przygotowanie nadmiarowych zajęć było by dobrym rozwiązaniem by zapobiec lukom miedzy zajęciami.
\end{itemize}
\subsection{Współpraca drużynowej i~przybocznej}
Nie może być tak, że przyboczna nie uczestniczy w~istotnych dla drużyny rozmowach, nie wie o~nich. Drużynowa powinna często konsultować się z~przyboczną i~podejmować pewne decyzje razem z~nią, a~co najmniej zapoznać ze swoimi planami.
\begin{itemize}
\item[-] Przyboczna musi trzymać rękę na pulsie, być blisko drużynowej i~interesować się wszystkim, a~nie czekać na podanie pod nos informacji i~zaproszenia na spotkanie w~jakiejś sprawie.
\item[-] Przyboczna miała za mało na głowie na tym obozie i~była tak jakby w~połowie uczestnikiem a~w~połowie kadrą --- to nie skończyło się dobrze. Sama powinna wyjść z~inicjatywą, a~nie uciekać od problemów związanych z~drużyną, mówić, że np. ,,i~tak ma za małe doświadczenie i~nic nie wie i~ktoś inny ma decydować''.
\end{itemize}
\subsection{Zajęcia dla wędrowniczek i~harcerek powinny być prowadzone na różnych poziomach lub osobno}
Kiedy jedne słyszą o~pewnych rzeczach pierwszy raz, inne nudzą się, a~czasem nie mają ochoty czynnie uczestniczyć w~zajęciach --- nawet, jeśli nie wszystko wiedzą czy rozumieją. Wiadomo, że niemożliwe jest to, aby drużynowa wędrowniczek sama przygotowała wszystkie zajęcia dla swoich dziewczyn, a~pozostałe osoby --- dla grupy harcerek. Trzeba się rozsądnie podzielić pracą. A~może wędrowniczki same przygotują jakąś grę korzystając z~własnej wiedzy i~umiejętności?
\begin{itemize}
\item[-] Zajęcia mają być atrakcyjne. Aby zainteresować wędrowniczki powinny były być o~wiele bardziej atrakcyjne, żeby utrzymać ich uwagę.
\item[-] Najlepszym pomysłem byłby całkowicie osobny obóz. Połączenie dwóch metod pracy (harcerskiej i~wędrowniczej) nie idzie w~parze. Niektóre wędrowniczki brakiem karności psują zajęcia harcerkom i~dają zły przykład, a~przy tym same się nudzą. Należy ograniczyć ich kontakty do odwiedzin i~sporadycznej pomocy w~czasie wyjazdów. Szczególnie, gdy ,,wędrowniczki'' są w~wieku zastępowych harcerek a~ponadto ich ego nie jest dobrym przykładem. W~ostateczności powinien to być osobny podobóz.
\item[-] Wędrowniczki powinny raczej przygotowywać zajęcia dla harcerek, może to by je usatysfakcjonowało i~poczułyby, że to wcale nie takie proste. Powinno to jednak być bardzo kontrolowane, żeby później nie czuły się jak nie wiadomo kto.
\end{itemize}
\subsection{Drużynowa i~czas spędzany z harcerkami}
Na szlaku nie było widać drużynowej ze swoimi harcerkami, podobnie przy innych okazjach. Z~tego powodu może ona nie wiedzieć o~wszystkim, co się dzieje między harcerkami, jakie mają problemy. A~przecież podczas wędrówki można kogoś lepiej poznać, zrozumieć i~dzięki rozmowie np. domyślić się, jakie zadania na stopień dla danej osoby będą najodpowiedniejsze, dlatego następnym razem trzeba naprawić ten błąd.
\begin{itemize}
\item[-] Drużynowe muszą być świadome, że są kadrą i~całym swym postępowaniem dają przykład innym. Nie można dać sobie chwil na pofolgowanie, bo cały czas jest się obserwowanym chociażby przez przyboczną, która podświadomie chce dorównać. Nawet w~tych złych rzeczach.
\item[-] Na obozie nie powinno się zdarzyć, że ktoś z~kadry nie ma nic do roboty --- zawsze można komuś pomóc w~zajęciach, w~kuchni, czy porozmawiać z~kimś z~uczestników i~przy okazji wpływać na postawę tej osoby.
\item[-] To jest chyba problem dużych obozów (jak np. 25 na obozie wędrownym) gdy nie jesteśmy w~stanie zapanować i~ogarnąć całej grupy ludzi. Rozwiązaniem są wyjazdy mniejszym gronem lub duże zaplecze kadry jednej drużyny, mające świadomość, jak kształtować postawę ludzi w~czasie drogi.
\end{itemize}
\subsection{Brak czasu na pranie}
Często prać można było tylko podczas godziny wyznaczonej na mycie się. Jedni zdążyli, inni nie (np. z~powodu pechowego miejsca w~kolejce), niektórzy w~ogóle nie wpadli na taki pomysł, a~potem chodzili w~brudnych ubraniach. Dobrze jest przeznaczyć np. raz na kilka dni godzinę czy dwie tylko na tę czynność.
\begin{itemize}
\item[-] Najlepiej wyznaczyć taki czas w~dzień wędrówki, po dotarciu do schroniska, albo zaraz następnego dnia rano po śniadaniu.
\item[-] Czas na pranie powinien być wyznaczony osobno poza czasem na mycie się czy jakimś innym czasem wolnym.
\end{itemize}
\subsection{Torby}
Obóz był o~tematyce ekologicznej i~jednym z~jego elementów było zachęcenie uczestniczek do korzystania z~materiałowych toreb zamiast reklamówek. Można było zabierać je na zakupy, aby przyzwyczaić dziewczyny do tego pomysłu. Niestety, wyszywanie na nich haftu łemkowskiego zajęło sporą część obozu. Poza tym niektóre dziewczyny nie chciały już nawet na nie patrzeć, a~więc cały ekologiczny podtekst padł. Następnym razem trzeba zastanowić się, czy zawsze połączenie dwóch pomysłów ma sens.
\begin{itemize}
\item[-] Powinno się zabierać te torby na zakupy obozowe. A~tak wyszło kompletnie bez sensu. A~hafty łemkowskie można było wyszyć na osobistych proporczykach czy jeszcze czymś innym, a~potem ewentualnie naszyć to na taką torbę.
\item[-] Robienie różnych rzeczy powinno mieć coś na celu. Wyszywanie haftu było ciekawe i~ćwiczyło umiejętności hafciarskie (chociaż rozwlekło się w~czasie i~nie miałyśmy na jego temat konkretnych informacji), ale nie miało dalszego przełożenia, stworzone torby były bezużyteczne.
\item[-] Tematyka ekologiczna nie była odczuwalna. Miało na to wpływ pozyskanie środków finansowych z~różnych źródeł --- musieliśmy spełnić określone wymagania, które nie miały zbyt wiele wspólnego z~ekologią.
\end{itemize}
\subsection{Karność wędrowniczek}
Wędrowniczki czuły się w~pewien sposób lepsze od harcerek, chociaż między niektórymi przedstawicielkami obu pionów jest mała lub żadna różnica wieku, a~nawet umiejętności, zaradności itd. Nie dawały dobrego przykładu młodszym dziewczynom, które potem czasem dziwiły się, że wędrowniczkom coś wolno, a~im nie.
\begin{itemize}
\item[-] Postawa i~przemyślenia wynikały jedynie z~nudy, a~przejście do drużyny wędrowniczek rozbudowało ego do niesamowitych rozmiarów (miejsce do pracy śródrocznej). Problemem był także brak autorytetu wśród kadry dziewcząt.
\item[-] Wędrowniczki i~wędrownicy mieli na siebie zły wpływ.
\item[-] Zabrakło autorytetu wśród kadry podobozu harcerek.
\end{itemize}
\subsection{Słaba współpraca z~drużyną wędrowników}
Wędrownicy za mało pomagali przed obozem w~przygotowaniach, a~w~trakcie obozu w~prowadzeniu zajęć, w~sprawach organizacyjnych itd. Czasem nawet nieodpowiednio się zachowywali. Wcześniej mieli obóz drużyny, który był dla nich priorytetem, dlatego teraz było im ,,wszystko jedno''. W~przyszłości, gdyby zaistniała podobna sytuacja, wcześniej trzeba dokładnie określić, jakie są obowiązki członków drużyny męskiej jadącej na obóz z~licznym podobozem żeńskim.
\begin{itemize}
\item[-] Chłopaki nie pomagali przed ani na obozie. A~jeszcze na dodatek często się obijali i~jeśli czegoś nie zrobili to nikt tego nie zrobił. Swoją postawą (rzekomo wędrowniczą) dawali zły przykład.
\item[-] Zachowywali się niedojrzale. Trzeba było zwracać im uwagę, czasem w~podstawowych kwestiach, np. związanych z~kulturą osobistą. Często bardziej utrudniali swoją obecnością, zamiast pomagać. Gdyby taka sytuacja miała miejsce po raz kolejny, należy zadbać o~normalną ilość uczestników z~drużyny męskiej, aby ich kadra się nie nudziła. Poza tym trzeba obiektywnie weryfikować drużyny, z~którymi chce się jechać --- jeśli się ma z~kimś dobre relacje prywatnie, to nie gwarantuje wcale dobrej organizacji wyjazdu.
\item[-] Harcerze zawiedli na całej linii. Myślę, że należy im o~tym bardzo dobitnie powiedzieć, żeby też zrozumieli swój błąd i~nie powtarzali tego przy współpracy z~innymi drużynami.
\end{itemize}
\subsection{Mało czasu dla siebie (odpoczynek)}
Czas wolny wynikał głównie z~tego, że zajęcia ,,się skróciły''. Natomiast czas wolny, który przewidziany był w~planie, był stanowczo za krótki. Ludzie spędzali go głównie stojąc w~kolejce do mycia.
\begin{itemize}
\item[-] Dawna praktyka instruktorska, zawsze mówiła, że uczestnicy na obozie nie mają mieć chwili tzw. czasu wolnego. Być może teraz zmieniły się standardy i~należy równać w~dół do poziomu kolonii, gdzie przed obiadem są słabo przygotowane ,,zajęcia'', a po obiedzie ,,czas wolny'' i~ludzie robią co chcą. Jedyny czas wolny, wg tej praktyki, kiedy ludzie mają chwilę oddechu to:
\begin{itemize}
\item[-] toaleta poranna i~wieczorna (każdy czeka na swoją kolejkę --- przed i~po tym, zanim wszyscy skończą jest czas wolny),
\item[-] czas na pranie (każdy czeka na swoją kolejkę --- przed i~po tym, zanim wszyscy skończą jest czas wolny),
\item[-] cisza poobiednia,
\item[-] chwila czasu przed ciszą nocną i~czas po ciszy nocnej, zanim oboźna się wkurzy, że jest za głośno.
\end{itemize}
Jeśli ma być inaczej, to jest kwestia ustalenia tego na forum Hufca czy Okręgu, bo to raczej powinno być jednolite wszędzie.
\item[-] Wyjść trzeba od solidnego przygotowania zajęć, żeby były ciekawe, intensywne i~odpowiednio długie. Żeby ludzie rzeczywiście odpoczywali w~przewidzianym na to czasie, a~nie szwendali się bez sensu bo zajęcia ,,się skróciły''.
\item[-] Czasu wolnego powinno być jak najmniej, ale w~warunkach prawie codziennego ekstremalnego wysiłku podczas wędrówek, ludzie, w~czasie dnia bez wędrówek, nie mieli już siły aktywnie uczestniczyć w~zajęciach, co też wpływało na ich poziom i~atmosferę --- wiadomo, że gdy ktoś jest zmęczony i~każe mu się iść na bieg, nie będzie zachwycony. Nie było złe to, że dawano ludziom czasami trochę luzu, trzeba tylko pilnować by nie przeszkadzali innym mieszkańcom schroniska.
\item[-] Kwestia sporna i~do ustalenia przed obozem.
\end{itemize}
\subsection{Uczestnicy nie byli kondycyjnie przygotowani do wędrówek i~do długotrwałego wysiłku}
\begin{itemize}
\item[-] Mimo tego, że forma obozu była znana od września ludzie nie zostali przygotowani kondycyjnie do wędrówek. Tylko wędrowniczkom udało się zorganizować pojedyncze krótkie wędrówki przed obozem. Długofalowe przygotowania nie zostały zrealizowane, a~dokładniej: zostały potraktowane jako nieważne. Kadra drużyn nie znalazła w~sobie minimum chęci aby zorganizować w~ciągu roku harcerskiego zajęcia poprawiające kondycję, takie jak: bieganie raz lub dwa razy w~tygodniu, zajęcia na basenie, wędrówki w~trudnym terenie, rajdy. Gorzej --- nikomu nie przyszło do głowy wysłać ludzi na imprezy organizowane przez kogoś innego, np. biegi i~marsze na orientację. Jednym słowem, rady płynące z~doświadczenia komendanta zostały kompletnie zignorowane --- przygotowanie ludzi do obozu i~świadomość konieczności przygotowania kondycyjnego tak naprawdę nie istniała wśród ludzi, a~zwłaszcza wśród kadry. Śródroczna praca drużyn nie miała nic wspólnego z~przygotowaniami do obozu, chociaż miejsce i~cel były znane na początku roku harcerskiego.
\item[-] Drużynowe skupiły się głównie na tym, żeby z~ich drużyn pojechało jak najwięcej osób, wiedząc o~tym, że nie wszystkie dadzą radę. Trzeba obiektywnie i~ostro postawić warunki przed obozem, gdyż później ludzie, nie dający sobie rady, stają się zmorą całej kadry i~innych uczestników, spowalniając tempo marszu i~utrudniając jego przebieg. Dodatkowo ma to bardzo demotywujący wpływ na słabszych uczestników i~obniża morale całego obozu, psując przy okazji atmosferę.
\end{itemize}