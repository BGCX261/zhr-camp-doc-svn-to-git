\section{Umowa rezerwacji noclegów\label{wzor-umowy-rezerwacji}}
\subsection{Wzór}
\begin{center}UMOWA REZERWACJI MIEJSC NOCLEGOWYCH\end{center}

\noindent Zawarta w~dniu \uline{data} w~\uline{miejscowość} pomiędzy
\begin{enumerate}
\item \uline{Nazwa schroniska, szkoły, operatora pola biwakowego itp.} reprezentowanym przez \uline{imię i~nazwisko osoby, która ma prawo podpisywać umowy}, legitymującym się dowodem osobistym \uline{seria i~numer dowodu osobistego}, zwanym w~treści umowy ,,Zleceniobiorcą'', a
\item Okręgiem Pomorskim Związku Harcerstwa Rzeczypospolitej z~siedzibą w~Gdańsku, ul.~Zator Przytockiego~4, 80-245 Gdańsk, NIP: 957-090-39-62, reprezentowanym przez \uline{imię i~nazwisko osoby, która ma prawo podpisywać umowy}, legitymującym się dowodem osobistym \uline{seria i~numer dowodu osobistego}, zwanym w~treści umowy ,,Harcerzami'',
\end{enumerate}
w~sprawie wynajmu \uline{miejsc noclegowych i~wyżywienia} dla uczestników obozu harcerskiego \uline{nazwa obozu i~numery drużyn}.

\begin{center}\S 1\end{center}
Zleceniobiorca oświadcza, że jest właścicielem/operatorem \uline{nazwa schroniska, szkoły, pola biwakowego itp.} znajdującego się w~\uline{adres i~ewentualnie dokładny opis lokalizacji obiektu}.

\begin{center}\S 2\end{center}
Zleceniobiorca dokona rezerwacji miejsc noclegowych na rzecz Harcerzy i~odda im do~korzystania wymieniony w~\S~1 obiekt w~następującym zakresie, np.:
\begin{enumerate}[a)]
\item ilość osób:
\begin{itemize}
\item[-] 20~harcerek,
\item[-] 20~harcerzy,
\end{itemize}
\item ilość kadry:
\begin{itemize}
\item[-] 3~instruktorki,
\item[-] 3~instruktorów,
\end{itemize}
\item termin pobytu (dzień i~godzina):\\
\uline{np. od 7~lipca 2012 godz: 17:00 do 9~lipca 2012 godz: 10:00,}
\item udostępniony obszar/pomieszczenia:\\
\uline{Dokładnie wymienić co zostanie udostępnione dla harcerzy podczas pobytu, np.:}
\begin{itemize}
\item[-] Fragment pola biwakowego o~powierzchni 100~m$^{2}$ i~wymiarach 10~m.~x~10~m., znajdujący się w~odległości co najmniej 25~m. od toalet, umywalni, ustępów, ulicy, \uline{wymienić inne}. 2~toalety, 2~umywalnie z~bieżącą zimną wodą dostępne 24~godz./dobę, miejsce do przygotowywania posiłków. Miejsce do palenia ogniska.
\item[-] 3~sale lekcyjne o~powierzchni co najmniej 50~m$^{2}$ każda, 2~toalety zlokalizowane na tej samej kondygnacji co wspomniane sale lekcyjne, sala gimnastyczna \uline{ze sprzętem: piłki, materace itd.}, szatnia na \uline{ilość} osób, stołówka, kuchnia z~dostępem do lodówki, kuchenki gazowej/elektrycznej/pieca itd. Ławki szkolne oraz krzesła zostaną wyniesione z~sal lekcyjnych, a~szafki oraz pomoce naukowe zostaną zabezpieczone/zaplombowane przez Zleceniobiorcę przed przybyciem Harcerzy.
\item[-] 6~pokoi 3-osobowych na tej samej kondygnacji, wyposażonych w~jednoosobowe łóżka/łóżka piętrowe z~materacami, bez pościeli. 3~łazienki wspólne znajdujące się w~piwnicy schroniska z~dostępem do bieżącej zimnej wody 24~godz./dobę oraz bieżącej ciepłej wody codziennie w~godz. 7:00~-~9:00 i~18:00~-~20:00. W~łazienkach będą natryski dostępne dla Harcerzy codziennie w~godz.: 18:00~-~20:00 z~bieżącą zimną i~ciepłą wodą. Miejsce do palenia ogniska.
\item[-] 4~sale do przeprowadzania zajęć teoretycznych,
\item[-] salę gimnastyczną do przeprowadzania zajęć sportowych,
\item[-] itp.
\end{itemize}
\end{enumerate}

\begin{center}\S 3\end{center}
W~trakcie pobytu Harcerzy w~wymienionym w~\S~1 obiekcie nie będą prowadzone żadne prace budowlane ani remontowe, a~pomieszczenia udostępnione Harcerzom będą wykończone i~w stanie nadającym się do bezpiecznego korzystania.

\begin{center}\S 4\end{center}
Obiekt wymieniony w~\S~1 będzie udostępniony Harcerzom na wyłączność.\\
\uline{Ewentualnie:}\\
Sposób korzystania z~obiektu wymienionego w~\S~1 przez inne osoby w~czasie pobytu harcerzy: \uline{opisać}.\\
Harcerze mogą przebywać w~wymienionym w~\S~1 obiekcie 24~godz./dobę. \uline{(Może być inaczej, np.: Harcerze mogą przebywać w~wymienionym w~\S~1 obiekcie w~godz: 17:00~-~10:00 dnia następnego.)}

\begin{center}\S 5\end{center}
Koszt brutto wynajmu wymienionego w~\S~1 obiektu wynosi np.:
\begin{itemize}
\item[-] namiot/pokój 2-osobowy: \uline{kwota}/dobę/pobyt
\item[-] namiot/pokój/sala lekcyjna \uline{ilość}-osobowy: \uline{kwota}/dobę/pobyt
\item[-] dostęp do sali gimnastycznej: \uline{kwota}/dobę/pobyt
\item[-] dostęp do kuchni: \uline{kwota}/dobę/pobyt
\item[-] pobyt za 1 osobę: \uline{kwota}
\item[-] itd.
\end{itemize}
\newpage
\begin{center}\S 6\end{center}
Koszt brutto wyżywienia dla 1 osoby wynosi odpowiednio:
\begin{itemize}
\item[-] śniadanie: \uline{kwota}
\item[-] obiad: \uline{kwota}
\item[-] kolacja: \uline{kwota}
\end{itemize}

\begin{center}\S 7\end{center}
Zaliczka na poczet dokonania rezerwacji wynosi \uline{kwota}, słownie: \uline{kwota słownie}, płatna na konto Zleceniobiorcy do dnia \uline{data}. W~przypadku nienadesłania zaliczki w~podanym terminie rezerwacja nie będzie dokonana.

Dane Zleceniobiorcy do przelewu zaliczki: \\ \uline{dokładna nazwa i~adres, numer rachunku bankowego.}

\begin{center}\S 8\end{center}
Harcerze mają możliwość dokonania zmiany terminu pobytu oraz ilości osób.

\begin{center}\S 9\end{center}
Rozliczenie nastąpi na podstawie faktycznego ilościowego wykorzystania noclegów w~momencie opuszczenia przez Harcerzy obiektu wymienionego w~\S~1.

Zleceniobiorca po zakończeniu pobytu wystawi fakturę VAT na następujące dane:\\
Okręg Pomorski ZHR, ul.~Zator Przytockiego 4, 80-245 Gdańsk, NIP: 957-090-39-62.

\begin{center}\S 10\end{center}
Po dokonaniu rezerwacji bez zgody Harcerzy nie może nastąpić odwołanie rezerwacji miejsc noclegowych. W~przypadku naruszenia tego postanowienia Zleceniobiorca udostępni taką samą ilość miejsc/pokojów o~tym samym standardzie i~w~tej samej kwocie w~innym obiekcie znajdującym się najdalej w~sąsiedniej miejscowości w~stosunku do obiektu wymienionego w~\S~1.

\begin{center}\S 11\end{center}
W~przypadku odwołania rezerwacji przez Harcerzy w~terminie \uline{ilość} dni przed terminem pobytu wpłacona przez Harcerzy zaliczka przepada na rzecz Zleceniobiorcy.
 
\begin{center}\S 12\end{center}
W~sprawach nie unormowanych niniejszą umową zastosowanie mają:
\begin{itemize}
\item[-] regulamin wymienionego w~\S~1 obiektu (w~załączniku do umowy),
\item[-] odpowiednie przepisy kodeksu cywilnego.
\end{itemize}
\newpage
\begin{center}\S 13\end{center}
Niniejszą umowę sporządzono w dwóch jednobrzmiących egzemplarzach po jednym dla każdej ze stron.
\\
\\
\\
\indent \hspace{1cm} Zleceniobiorca: \hspace{5cm} Harcerze:\\
\\
\\
\indent \hspace{1cm} ........................ \hspace{4.5cm} .........................\\
\\
\\
\\
Załączniki:
\begin{enumerate}
\item Regulamin obiektu \uline{(schroniska, szkoły, pola biwakowego itd. --- przykład na stronie \pageref{regulamin-schroniska})}.
\end{enumerate}
\newpage
\subsection[Regulamin Szkolnego Schroniska Młodzieżowego w~Łodzi]{Regulamin Szkolnego Schroniska Młodzieżowego\\w~Łodzi\footnote{Źródło: \href{http://bip.uml.lodz.pl/edu/organizacja\_przedmiot.php?id=352}{http://bip.uml.lodz.pl/edu/organizacja\_przedmiot.php?id=352}}}
\begin{enumerate}\label{regulamin-schroniska}
\item Prawo do korzystania ze schroniska młodzieżowego w~Łodzi przysługuje młodzieży szkolnej i~studenckiej, nauczycielom i~wychowawcom oraz członkom Polskiego Towarzystwa Schronisk Młodzieżowych, jak również obcokrajowcom należącym do Międzynarodowej Federacji Schronisk Młodzieżowych.

Z~noclegu w~schronisku, w~razie wolnych miejsc, mogą korzystać również inne osoby pod warunkiem przestrzegania obowiązującego regulaminu.
\item Zasady rezerwacji miejsc.

Po uzyskaniu informacji drogą telefoniczną, internetową lub listowną o~dostępności miejsc w~danym terminie, osoby prywatne lub przedstawiciele grup dokonują wstępnej rezerwacji w~recepcji schroniska. Następnie pracownik recepcji przygotowuje umowę dla grupy, wysyła ją faxem, pocztą lub elektronicznie. Instytucja zamawiająca odsyła podpisaną umowę w~ciągu 10~dni od jej otrzymania, przyjmując warunki w~niej zawarte i~deklarując wpłatę zaliczki wyliczoną przez schronisko, zaliczka nie powinna przekraczać kwoty 25~\% rezerwowanej usługi.

Osoby prywatne, rezerwując miejsca na miesiąc wcześniej, powinny wpłacić zaliczkę na adres lub konto Schroniska. Grupy mogą dokonywać rezerwacji z~wyprzedzeniem nawet rocznym. Osoby indywidualne jeden miesiąc przed terminem.

Zwrot zaliczki może nastąpić w~przypadku nieskorzystania ze schroniska z~wyjątkowo ważnych i~udokumentowanych przyczyn uznanych przez dyrektora schroniska, po ewentualnym potrąceniu wcześniej poniesionych kosztów przez schronisko.

W~przypadku sporów sprawę należy kierować do Wydziału Edukacji Urzędu Miasta Łodzi ul.~Sienkiewicza~5. W~przypadku nienadesłania zaliczki 7~dni przed przybyciem do schroniska grupy, dyrektor może anulować zamówienie.
\item Turyści indywidualni w~grupach poniżej 5~osób są przyjmowani na noclegi bezpośrednio w~schronisku, jeżeli są wolne miejsca, z~tym, że pierwszeństwo ma młodzież szkolna i~studencka, a~następnie inne osoby.
\item Przybywający do schroniska (powinno to nastąpić w~godz. 16:00 --- 20:00), wypełniają ,,Kartę gościa'', pokazują dyżurującemu recepcjoniście legitymacje PTSM, legitymacje uczniowskie, studenckie lub dowód osobisty.

W~przypadku grupy --- dowód kierownika grupy i~listę uczestników, regulują należne opłaty, według cennika wywieszonego w~recepcji schroniska w~widocznym miejscu.
\item Ze schroniska korzystać można nie dłużej niż przez 3~kolejne noce, chyba że są wolne miejsca i~dyrektor schroniska wyrazi zgodę na dłuższy pobyt.
\item Osoby płci męskiej i~żeńskiej kwaterują się w~oddzielnych pokojach.
\item Osoby, które przebyły chorobę zakaźną, nie mogą w~okresie kwarantanny korzystać ze schroniska.
\item Kierownik zespołu wycieczkowego i~opiekunowie obowiązani są nocować w~schronisku razem z~uczestnikami. W~przeciwnym przypadku zespół nie może być przyjęty do schroniska.
\item Każdy nocujący otrzymuje świeżą bieliznę pościelową w~recepcji, sam dokonuje powleczenia, a~opuszczając schronisko zdaje bieliznę i~wszystkie wypożyczone przedmioty w recepcji.
\item Od godz. 22.00 do 6.00 obowiązuje w~schronisku cisza nocna z~wygaszeniem światła. Turyści, którzy przychodzą późno lub bardzo rano wychodzą ze schroniska, nie powinni zakłócać wypoczynku pozostałym osobom. Odchylenia od ustalonych godzin ciszy są możliwe tylko w~wyjątkowych przypadkach za wiedzą dyrektora schroniska.
\item Korzystający ze schroniska powinni najpóźniej do godz. 10.00 zasłać łóżka i~sprzątnąć pomieszczenia schroniska.
\item Przygotowywanie posiłków może się odbywać jedynie w~kuchniach samoobsługowych schroniska. Po spożyciu posiłku należy pozmywać naczynia kuchenne, a~kuchnię dokładnie sprzątnąć.
\item Wszelkie zniszczenia i~uszkodzenia przedmiotów stanowiących własność schroniska należy zgłaszać w~recepcji schroniska, następnie dyrektor określa wysokość odszkodowania.
\item W~schronisku obowiązuje schludny ubiór, spokojne i~uprzejme zachowanie się; picie alkoholu i~uprawianie gier hazardowych jest zabronione.
\item Palenie tytoniu w~sypialniach jest zakazane; palić wolno tylko w~miejscu wyznaczonym.
\item Zwierząt do budynku schroniska wprowadzać nie wolno.
\item W~razie przekroczenia regulaminu schroniska lub nieodpowiedniego zachowania się, dyrektor schroniska jest uprawniony do zatrzymania winnemu legitymacji (uczniowskiej, studenckiej, członkowskiej) i~usunięcia go ze schroniska, a~ponadto zawiadamia o~tym właściwą szkołę (uczelnię, organizację).
\item Korzystający ze schroniska mogą wszelkie pozytywne i~negatywne uwagi wpisywać do książki życzeń schroniska lub w~ważnych wypadkach kierować je do Wydziału Edukacji lub Kuratorium Oświaty w~Łodzi.
\item We wszystkich sprawach nie ujętych w~regulaminie, a~dotyczących toku życia w schronisku, jak: zapewnienie porządku, ochrona mienia, przestrzeganie zasad kultury itp., korzystający ze schroniska są obowiązani stosować się do wskazań dyrektora lub osób przez niego upoważnionych do pełnienia dyżuru.
\end{enumerate}
\noindent Regulamin opracowany na podstawie Statutu Schronisk Młodzieżowych w~Polsce i~zgodny ze statutem Szkolnego Schroniska Młodzieżowego w~Łodzi z~roku 2007.