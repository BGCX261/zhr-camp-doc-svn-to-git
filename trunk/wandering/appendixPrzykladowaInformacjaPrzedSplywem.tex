\section{Przykładowa informacja o~spływie\newline (Drawa'2006)\label{info-o-splywie}}
W~niedzielę, 16 lipca, wieczorem dojeżdżamy do Czaplinka. Tam, nad jeziorem rozbijamy biwak na noc.

Wcześniej będziemy musieli podzielić się na pary --- każda para spędzi ze sobą cały spływ w~jednym kajaku. Postaramy się to zrobić tak, aby wszystkie pary były mniej więcej jednakowo silne. Jeśli zrobimy to dobrze, to taki podział pozostanie do końca spływu. Może się jednak okazać, że przeceniliśmy czyjeś siły --- w takim przypadku dokonamy pojedynczych zmian. Oprócz tego utworzymy grupy, które będą spały razem w~jednym namiocie --- te grupy nie będą się również zmieniały przez cały spływ. Postaramy się zrobić tak, aby osoby płynące w~tym samym kajaku spały w~tym samym namiocie. Większość namiotów będzie 4-osobowa, co oznacza, że w~jednym namiocie znajdą się obsady dwóch kajaków. Za transport namiotu odpowiadają obie ekipy, dzieląc się równo ładunkiem. Jednym słowem każda ekipa pilnuje i~transportuje sobie ,,połówkę'' namiotu, w~którym śpi.

Każdy dzień spływu będzie wyglądał podobnie: pobudka całego obozu o~godz. 7:00, potem część osób przygotowuje śniadanie, część składa namioty itp. Po śniadaniu pakujemy sprzęt i~bagaże do kajaków i~wypływamy na szlak. Po drodze będą krótkie przerwy i jedna dłuższa. W~miarę możliwości podczas tej dłuższej przerwy przygotujemy i~zjemy obiad. Po obiedzie płyniemy dalej. Zakupy robimy w~miejscowościach, przez które będziemy przepływać. Pływanie kończymy około godz. 18:00. Na biwaku wyciągamy kajaki z~wody, wypakowujemy namioty i~natychmiast je rozstawiamy. Potem wypakowujemy resztę sprzętu i~bagaży i~przebieramy się w~suche rzeczy. Część osób zabiera się za przygotowanie kolacji, pozostała część porządkuje, myje i~układa kajaki tak, żeby były gotowe na następny dzień. Jeśli będzie to możliwe, to wieczorem spędzimy trochę czasu przy ognisku, ale zależy to od czasu i~regulaminu pola, na którym będziemy biwakować. W~nocy warty będą pilnowały kajaków. Wartę będą pełnić wszyscy bez wyjątku w~każdą noc --- dzięki temu warty będą krótkie --- około 20 minut na osobę.

W~zamykanej komorze na rufie będziemy przewozić większość sprzętu i~bagaży. Dla zachowania wyważenia kajaków część rzeczy będzie zapakowana w~dziobie kajaka. Plecaki w~kajakach muszą być ułożone ,,plecami'' (szelkami) do góry, co oznacza, że podczas pakowania na ,,plecach'' plecaka mają się znaleźć wasze dokumenty, a zaraz pod nimi rzeczy, w~które macie się przebrać na biwaku i~które muszą być suche za wszelką cenę. Rzeczy, które będą na górze plecaka, znajdą się na spodzie, gdy plecak będzie leżał w~kajaku, i~będą najbardziej narażone na zamoczenie.

Mimo, że tylna komora jest zamykana, to nie można zagwarantować, że będzie całkowicie wodoszczelna. Lepiej więc dmuchać na zimne niż spać w~mokrym śpiworze, w~mokrym namiocie i~w~mokrych ubraniach. Dlatego wszystkie rzeczy osobiste zapakujecie w~worki foliowe w~następujący sposób:\\
Każda pojedyncza rzecz ma być zapakowana w~osobny worek foliowy: wkłada się rzecz do woreczka, ściskając woreczek wysysa powietrze, zawiązuje woreczek na ,,kokardkę'', potem wkłada się to do drugiego woreczka, wysysa powietrze i~znowu zawiązuje (nie używajcie żadnych plastikowych klipsów ani metalowych drucików do zawiązywania woreczków, bo dziurawią one sąsiednie worki). Czyli: każda koszulka, majtki, para skarpetek jest włożona osobno w~dwa woreczki. Potem można to sobie posortować w~reklamówki, żeby nie było bałaganu. Tak zabezpieczone wszystkie rzeczy wkładacie do dwóch grubych foliowych worków i~dopiero taką paczkę wkładacie do plecaka. Wtedy zawiązujecie worki i~zapinacie plecak. Alumata (karimata) też ma być zapakowana w~dwa worki. Jeśli wydaje się Wam to przesadą, to pewnie zdziwi was fakt, że taki sposób gwarantuje tylko 70\% bezpieczeństwo. Jeśli nie będzie padał deszcz ani kajak nie nabierze wody to zabezpieczenia się nie przydadzą. Jednak codziennie zdarza się, że: pada deszcz, lub jest duża fala na jeziorze, która przelewa się przez burty, albo ktoś za mocno chlapie wiosłami --- wtedy kajak działa jak wanna i~zbiera wodę w~środku --- w~ skrajnych przypadkach wasze rzeczy będą pływać w~100 litrach brudnej wody i~jeśli zaniechacie pakowania w~worki, albo worki będą dziurawe, albo perforowane, albo gdy zamiast worków użyjecie reklamówek, to wieczorem może się okazać, że:
\begin{itemize}
\item macie mokry śpiwór,
\item macie mokry namiot,
\item macie mokre rzeczy, w~które zamierzaliście przebrać się wieczorem,
\item macie mokre rzeczy, w~których mieliście spać,
\item macie mokre dokumenty, legitymację, portfel, apteczkę, latarkę itd.,
\item jeśli włożyliście wszystkie koszulki albo bieliznę do jednego worka, zamiast każdą rzecz osobno, to okaże się, że nie macie się w~co przebrać, albo dobitniej: nie macie nic suchego, więc możecie zostać w~mokrych kąpielówkach, albo przebrać się w~mokrą bieliznę (przeziębienie gotowe).
\end{itemize}
W kajaku jesteście ubrani~w:
\begin{itemize}
\item koszulkę, na którą będziecie mieli nałożoną i~zapiętą kamizelkę ratunkową,
\item strój kąpielowy / kąpielówki,
\item sandały (na gołe stopy),
\item rękawiczki bez palców,
\item kapelusz,
\item okulary przeciwsłoneczne (kto chce, bo mogą nie dotrwać do końca spływu, ale warto je wziąć).
\end{itemize}
Oprócz tego jesteście posmarowani kremem z filtrem przeciw opalaniu, a~pod ręką (czyli nie w~plecaku, tylko w~osobnym podwójnym worku) macie: krem przeciwsłoneczny, krótkie spodenki, kubek albo menażkę, koszulę flanelową, kurtkę od deszczu --- te rzeczy przydadzą się, gdy zrobi się chłodniej, gdy zacznie padać deszcz lub zacznie wiać wiatr albo na postoju, gdy trzeba będzie iść do sklepu na zakupy.

Rzeczy, w~które będziecie ubrani w~kajaku nie powinny być w~ciemnych kolorach (czyli nie czarne, nie granatowe itp.), bo gdy świeci słońce jest w~nich bardzo gorąco, co może skończyć się przegrzaniem. Wybierzcie ubrania w~jasnych i~żywych kolorach.

Wasza legitymacja szkolna, karta pływacka i~prywatne pieniądze mają być zapakowane w~osobnym, małym, podwójnym worku w~kieszeni plecaka albo na wierzchu plecaka, tak, żebyście mogli je wyjąć w~ciągu minuty bez potrzeby rozpakowywania plecaka. Książeczka RUM musi być zapakowana osobno w~2~worki i~porządnie zabezpieczona, też schowana w~kieszeni albo na wierzchu plecaka.

Oprócz tego w~kajaku macie butelki z~piciem dla was na drogę (,,big łyki''). Dostaniecie także kilka butelek 1,5~litrowych z~wodą pitną, która zostanie wykorzystana do przygotowania obiadu, zrobienia napoju po obiedzie albo kolacji itd. --- tej wody nie będzie wolno wam zużyć bez zgody kadry.

Wszystkie rzeczy, które znajdą się luzem w~kajaku muszą być powiązane ze sobą cienką linką. Ta linka przyda się też gdy trzeba będzie holować kajak przez jakiś uciążliwy odcinek rzeki, a także do suszenia mokrych rzeczy na biwakach.

Nie bierzcie dużych zapasów słodyczy, kosmetyków itd. --- lepiej wziąć trochę więcej pieniędzy i~w~razie czego dokupić sobie rzeczy, które się Wam skończą, niż wozić je ze sobą. Nie będziemy płynąć przez odludzie.
 
\subsection{Co każdy ma zabrać}
\subsubsection{Rzeczy używane podczas pływania\label{podczas_plywania}}
\begin{checklist}
\item koszulki (max. 2 szt.) (najlepiej T-shirt’y, ale jeśli ktoś bardzo chce może wziąć zamiast 1~T-shirt’a koszulkę bez rękawków / na ramiączkach),
\item kąpielówki / strój kąpielowy --- musi być wygodny, będziecie spędzać w~nim całe dnie, nie może być za ciasny ani nigdzie uwierać. Jeśli ktoś ma, może zabrać krótkie spodenki z~pianki,
\item sandały turystyczne z~rzemykami z~pasków parcianych (nie skórzane), ostatecznie klapki. Postarajcie się za wszelką cenę mieć zapinane sandały, które trzymają piętę --- w~tych sandałach będziecie wchodzić często do wody, więc muszą się trzymać mocno na nodze a~nie latać jak klapki,
\item kapelusz materiałowy z~rondem na głowę (ostatecznie czapka z~daszkiem, ale kapelusz jest dużo lepszy),
\item wodoodporny krem z~filtrem (minimum 12) do opalania --- 1~duże opakowanie (nie olejek do opalania, tylko krem ochronny przeciw opalaniu. Dla wrażliwych filtr 16 minimum),
\item cienka, oddychająca kurtka przeciwdeszczowa z~kapturem, ostatecznie płaszcz foliowy (kilka sztuk --- są bardzo nietrwałe),
\item okulary przeciwsłoneczne, najlepiej na sznurku,
\item koszula flanelowa, ciepła bluza lub sweter --- przyda się podczas płynięcia, gdy zrobi się chłodniej. Najlepsza jest koszula flanelowa, przydaje się w~różnych sytuacjach, bo można ją związać u~dołu,
\item krótkie spodenki (najlepiej do połowy uda) --- bądźcie przygotowani na to, że czasem trzeba będzie w~nich wskoczyć do wody, więc nie bierzcie żadnych ,,workowatych'',
\item rękawiczki bez palców (wiosła będą prawdopodobnie aluminiowe --- mogą bardzo brudzić ręce),
\item 10 metrów mocnej linki,
\item kubek / menażka (do wylewania wody z kajaka),
\item 3~butelki ,,big łyki'' (koniecznie z~dużym otworem) co najmniej po pół litra każda (Kto chce może wziąć mały metalowy termos zamiast 1 butelki. Szklane termosy się nie nadają).
\end{checklist}

\subsubsection{Rzeczy używane na postojach\label{na_postojach}}
\begin{checklist}
\item przybory toaletowe i kosmetyki,
\item ręcznik średniej wielkości, zapakowany w~dwa worki,
\item bielizna (bez przesady z~ilością, zawsze można zrobić pranie),
\item T-shirt (max. 1~szt.) --- w~ten T-shirt przebieracie się po przypłynięciu na biwak, żeby nie chodzić w~tym, w~którym płynęliście (bo będzie on po prostu mokry i~przepocony),
\item skarpetki (max. 3 pary),
\item menażka / kubek, niezbędnik,
\item ostry nóż, zabezpieczony w~futerale,
\item śpiwór. Śpiwór składacie odpowiednio, zwijacie w rulon, wkładacie w~dwa mocne duże worki i~to wszystko razem wkładacie do worka kompresyjnego. Potem, ściskając śpiwór, zawiązujecie pierwszy worek foliowy, potem drugi a~na końcu worek kompresyjny. Jeśli macie mały worek kajakowy, to weźcie go zamiast worka kompresyjnego od śpiwora. Tak zapakowany śpiwór na czas podróży ma być przytroczony na zewnątrz plecaka tak, żeby można go było łatwo odczepić,
\item komplet do spania, czyli krótkie spodenki bawełniane (mogą być bokserki) i~T-shirt albo jakaś lekka, cienka piżama. Jeśli ktoś marznie w~śpiworze, to bierze w~zamian coś cieplejszego
\item alumata, ostatecznie karimata --- zapakowana w~dwa worki,
\item buty (lekkie i~małe), najlepiej jakieś lekkie buty trekkingowe pod kostkę, adidasy lub coś podobnego. Buty podczas płynięcia podróżują porządnie zapakowane w~dwa worki, tak jak śpiwór --- czyli nie włożone do plecaka
\item porządne, mocne długie spodnie w~ciemnym kolorze (M-65 lub podobne, jeśli macie), odradzam dżinsy --- schną bardzo długo (rano i~wieczorem nad wodą często jest rosa) --- długie spodnie potrzebne są wieczorami i~rano, gdy jest chłodno,
\item polar rozpinany --- będzie przydatny, podobnie jak długie spodnie, tylko wieczorami i~rano. Koszula flanelowa (bluza lub sweter) mogą być mokre po całym dniu płynięcia --- musicie mieć coś suchego i~ciepłego do włożenia na biwakach,
\item cienkie rękawiczki wełniane z~palcami (zimowe) --- wieczorami i~w~nocy nad wodą bywa bardzo zimno,
\item latarka, najlepiej czołówka (pamiętajcie o~nowych bateriach),
\item środek przeciw komarom (,,Off'' lub coś podobnego).
\end{checklist}

\subsubsection{Przechowywanie i~zabezpieczenie ,,bagażu''}
\begin{checklist}
\item plecak typu kostka lub coś podobnego, ale nie większy, bez stelaża (można wyjąć) albo worek kajakowy, jeśli ktoś ma, o~pojemności takiej jak plecak kostka (czyli 20 litrów),
\item mocne i~grube worki (np. na śmieci) --- cienkie i~perforowane worki się nie nadają,
\item woreczki do produktów spożywczych (mocne, najlepsze są takie jak do mrożonek. Woreczki do kanapek są przeważnie cienkie i~perforowane i~w~większości się nie nadają),
\item zapasowe worki, duże i~małe --- przydadzą się, gdy poprzednie się przedziurawią.
\end{checklist}

\subsubsection{Pozostałe}
\begin{checklist}
\item zegarek (kto chce),
\item chusteczki higieniczne,
\item ważna legitymacja szkolna,
\item karta pływacka (kto ma),
\item pieniądze (bez przesady),
\item apteczka osobista --- leki, które musicie przyjmować (dla alergików itp. itd.),
\item troki parciane (nie skórzane) --- kilka sztuk do przytroczenia śpiwora, alumaty, butów, menażki, ,,big łyków'' itd. Im więcej troków weźmiecie, tym lepiej. Najlepsze są o~długości co najmniej 1~metra. Troki podpiszcie, żeby było wiadomo czyje są.
\end{checklist}

\subsubsection{Zestaw na powrót}
\begin{checklist}
\item bielizna --- 1 para,
\item skarpetki --- 1 para,
\item T-shirt --- 1 sztuka,
\item lekkie, cienkie długie spodnie --- 1 para.
\end{checklist}
Zestaw na powrót ma być zapakowany porządnie, tak jak wszystko, czyli każda rzecz w~dwa worki, do tego to wszystko razem włożone do dwóch porządnych worków i~spakowane na dno plecaka. Wyjmiecie to dopiero po zakończeniu spływu, przed wyruszeniem w~drogę powrotną do domu.

\subsection{Zestaw na wyjazd}
Nie bierzecie żadnych innych rzeczy oprócz już wymienionych, co oznacza, że na podróż z~domu na obóz musicie się ubrać w~rzeczy wymienione powyżej, czyli:
\begin{itemize}
\item bielizna (pkt. \ref{na_postojach}),
\item T-shirt (pkt. \ref{podczas_plywania}),
\item skarpetki (pkt. \ref{na_postojach}),
\item buty (adidasy) (pkt. \ref{na_postojach}),
\item długie spodnie (pkt. \ref{na_postojach}), chyba, że będzie upał, to krótkie spodenki (pkt. \ref{podczas_plywania}),
\item jeśli będzie zimno, to koszula flanelowa (pkt. \ref{podczas_plywania}) albo polar (pkt. \ref{na_postojach}),
\item jeśli będzie padał deszcz, to kurtka od deszczu (pkt. \ref{podczas_plywania}),
\item jeśli nie będzie padał deszcz, to kapelusz na głowę (pkt. \ref{podczas_plywania}),
\item jeśli będzie słońce, to okulary przeciwsłoneczne (pkt. \ref{podczas_plywania}),
\item na drogę weźcie sobie coś do picia w butelki ,,big łyki'' (pkt. \ref{podczas_plywania}) i~jakieś kanapki albo coś do jedzenia, tak, żeby starczyło wam do wieczora (pkt. \ref{podczas_plywania}),
\item sandały, śpiwór, alumatę, menażkę i~wszystko, co się nie zmieściło do plecaka, przytroczcie porządnie trokami do plecaka tak, żebyście mieli wolne ręce.
\end{itemize}