\section{Instrukcja postępowania w~przypadku zdarzenia ubezpieczeniowego mogącego skutkować odpowiedzialnością cywilną instruktora lub Związku\label{instrukcja-postepowania-wypadek}}
\begin{center}(Uchwała Naczelnictwa ZHR nr 190/4 z~dnia 22 kwietnia 2007~r.)\end{center}

Wystąpienie zdarzenia ubezpieczeniowego o~charakterze utraty zdrowia, życia lub mienia osób trzecich. (Np. wypadek śmiertelny; wypadek który zakończył się trwałym uszczerbkiem na zdrowiu; inne zdarzenia, w~trakcie których ktoś poniósł stratę materialną --- pożar lasu, domu, kamienicy, stodoły, zalanie mieszkania, biura, zniszczenie wynajętego lub użyczonego samochodu, poczęcie dziecka przez osobę nieletnią, molestowanie seksualne przez innych uczestników lub instruktorów, mobbing, wyciek danych osobowych do osób nieuprawnionych --- np. informacje o~stanie zdrowia, informacje teleadresowe itp.)
\begin{enumerate}
\item W~miarę możliwości ogranicz negatywne konsekwencje zdarzenia (chroń osoby, które nie zostały pokrzywdzone, udziel pierwszej pomocy osobom poszkodowanym, wzywaj fachową pomoc (lekarz, straż, policja, psycholog itp.).
\item Zadzwoń do przełożonych na poziomie Zarządu Okręgu lub Naczelnictwa i~przedstaw sprawę.
\item Jeżeli sprawa dotyczy niepełnoletniego uczestnika, ustal jedną osobę do kontaktów z~rodzicami tego uczestnika. Osoba ta niezwłocznie powinna powiadomić rodziców o~zdarzeniu i~być stale dostępna pod telefonem komórkowym dla rodziców.
\item Spisz jak najdokładniej wszystkie okoliczności zdarzenia dopóki je jeszcze pamiętasz na specjalnym formularzu. Pamiętaj o~datach i~godzinach. Wyślij taki opis faxem do biura Ubezpieczyciela tel +48 22 452 39 89 jak najszybciej, lecz nie później niż 3~dni od daty zdarzenia.
\item W~razie jakichkolwiek roszczeń wysuwanych przez osoby poszkodowane (rodzice, sąsiedzi, właściciele utraconego mienia, itp.) nie podejmuj żadnych rozmów i~ustaleń. Przekaż nr polisy OC Związku i~powiedz, że nie jesteś upoważniony do odpowiadania na jakiekolwiek pytania. Możesz przekazać nr telefonu i/lub~adresy do Okręgu i~Naczelnictwa z~sugestią, aby osoby poszkodowane tam zgłaszały swoje roszczenia --- najlepiej na piśmie.
\item Bądź miły i~ludzki. Staraj się pomóc osobom poszkodowanym, ale bądź powściągliwy, jeżeli chodzi o~udzielanie informacji. Nigdy do końca nie wiesz, jak odbierze je Twój rozmówca i~jak potem zrelacjonuje przebieg sprawy.
\item Osoba z~biura Ubezpieczyciela możliwie najszybciej będzie starała się udzielić Ci odpowiedniej pomocy, a~być może odwiedzi Cię osobiście i~przejmie sprawę, jeżeli sytuacja będzie tego wymagała.
\item W~przypadku zainteresowania ze strony mediów w~porozumieniu z~osobą z~Okręgu lub Naczelnictwa oraz Ubezpieczycielem przygotuj krótką informację nt. zdarzenia podkreślając słowo wypadek. Nie wpuszczaj mediów na teren obozu (żadnych kontaktów z~uczestnikami i~instruktorami!). W razie problemów proś o~pomoc policję.
\end{enumerate}