%%%%%%%%%%%%%%%%%%%%%%%%%%%%%%%%%%%%%%%%%%%%%%%%%%%%%%%%%%
%%%%%%%%%%%%%%%%%%%%%%%%%%%%%%%%%%%%%%%%%%%%%%%%%%%%%%%%%%
%%                                                      %%
%%           P R E A~M B L E   B E G I~N S              %%
%%                                                      %%
%%%%%%%%%%%%%%%%%%%%%%%%%%%%%%%%%%%%%%%%%%%%%%%%%%%%%%%%%%
%%%%%%%%%%%%%%%%%%%%%%%%%%%%%%%%%%%%%%%%%%%%%%%%%%%%%%%%%%
\documentclass[a5paper,10pt,titlepage,twoside]{article}
\usepackage{polski}
\usepackage[utf8]{inputenc}
\usepackage[OT4]{fontenc} % http://www.opcode.eu.org/more_advanced/latex/
\usepackage[polish]{babel}
% \usepackage[T1]{fontenc}
\usepackage{latexsym,fancyhdr}
\usepackage{geometry} % geometria strony - marginesy, ...
\usepackage{ragged2e} % http://www.tex.ac.uk/cgi-bin/texfaq2html?label=ragright
\usepackage{wallpaper} % ftp://sunsite.icm.edu.pl/pub/CTAN/macros/latex/contrib/wallpaper/
\usepackage{textpos}
\usepackage{indentfirst}
\usepackage{rcs}
\usepackage{enumerate}
\usepackage[hyphens]{url}
\usepackage[pdftex,a4paper=true,colorlinks=true,urlcolor=blue,breaklinks,
pdftitle={Poradnik},
pdfsubject={Poradnik organizowania obozów wedrownych dla drużyn harcerskich},
pdfauthor={Maciej Lipczynski},
pdfkeywords={obóz wedrowny rowerowy splyw zhr akcja letnia finanse dokumentacja kwatery porzadek organizowanie zywienie wedrowka bezpieczenstwo dzien harmonogram plan},
pdfpagemode=UseNone,pdfstartview=Fit, pdfhighlight={/N} % pdfstartview sets the size of the page first displayed (normally page 1, but that can be changed with pdfstartpage). Possible values are Fit, to show the whole page; FitH, to fit the width of the page in the window; or FitB, to fit the width of the contents to the window.
]{hyperref} % nagłowki pdf-a
\usepackage{caption}
\usepackage[normalem]{ulem}
\usepackage{comment}
\usepackage{movie15}
\usepackage{amsfonts}

\providecommand*{\eightnote}{%
  \begingroup
    \fontencoding{U}%
    \fontfamily{wasy}%
    \selectfont
    \symbol{11}%
  \endgroup
}

\DeclareMathSymbol{\bigstar}{\mathord}{AMSa}{"46} % ftp://ftp.ams.org/pub/tex/doc/amsfonts/amsfndoc.pdf

\newcommand*{\thecheckbox}{\hss$\Box$} % http://newsgroups.derkeiler.com/Archive/Comp/comp.text.tex/2005-08/msg01658.html
\newenvironment*{checklist}
{\list{}{%
\renewcommand*{\makelabel}[1]{\thecheckbox}}}
{\endlist}

\newcommand{\tstrut}{\rule[-32pt]{0cm}{42pt}} % 42 is 52 minus 10, where 10 is the font size from documentclass and 52 is about 1.75cm, the row should have 1.75cm height at least

\renewcommand{\captionfont}{\small}

\frenchspacing

%%%
% Define fonts to use in the headers and footers of the songbook.
%%%
\newcommand{\LHeadFont}{\footnotesize\sf}
\newcommand{\CHeadFont}{\footnotesize\sf}
\newcommand{\RHeadFont}{\footnotesize\sf}
\newcommand{\LFootFont}{\footnotesize\sf}
\newcommand{\CFootFont}{\footnotesize\sf}
\newcommand{\RFootFont}{\footnotesize\sf}

%%%
% Define counter for rows numbers
%%%
\newcounter{thecategory} \setcounter{thecategory}{0}
\newcounter{theproblem}[thecategory] \setcounter{theproblem}{0}
\newcommand{\category}{\noindent%
\refstepcounter{thecategory}\arabic{thecategory}.
}
\newcommand{\problem}{\noindent%
\refstepcounter{theproblem}\small{\arabic{thecategory}.\arabic{theproblem}.}
}

%%%
% Turn on and define fancy page heading/footing definition.
%%%
% \pagestyle{fancy}

  \renewcommand{\footrulewidth}{0.5pt} %pozioma kreska w~stopce
  \renewcommand{\headrulewidth}{0pt}
  \fancyhead[LE,RO]{}
  \fancyhead[CE,CO]{}
  \fancyhead[RE,LO]{}

\fancyfoot[LE,RO]{\raisebox{4pt}{\thepage}} % lift the page number a~little bit
\fancyfoot[CE,CO]{}
%\fancyfoot[RE,LO]{\RFootFont ZHR}
\fancyfoot[RE,LO]{\includegraphics[scale=0.2, keepaspectratio]{zhr.png}}

\RCS $Revision: 115 $

\makeatletter
\newcommand{\linia}{\rule{\linewidth}{0.4mm}}
\renewcommand{\maketitle}{\begin{titlepage}
    \begin{flushleft}
    \includegraphics[scale=0.15, keepaspectratio]{lilijka_opom.jpg}
    \end{flushleft}
    \vspace*{1cm}
    \begin{center}\small
    Związek Harcerstwa Rzeczypospolitej\\
    Okręg Pomorski
    \end{center}
    \vspace{1cm}
    \noindent\linia
    \begin{center}
      \LARGE \textsc{\@title}
         \end{center}
     \linia
    \begin{center}
    \@author\\
    \vspace{0.5cm}
    %\scriptsize{Wersja \RCSRevision}
    \end{center}
  \end{titlepage}%
}
\makeatother
\title{Poradnik organizowania obozów wędrownych dla drużyn harcerskich}
\author{Maciej Lipczyński}
\date{\empty} % no date

\geometry{verbose,a5paper,tmargin=1cm,bmargin=1.5cm,lmargin=2cm,rmargin=1cm}

%\overfullrule3pt % uncomment to see overfull

%%%%%%%%%%%%%%%%%%%%%%%%%%%%%%%%%%%%%%%%%%%%%%%%%%%%%%%%%%
%%%%%%%%%%%%%%%%%%%%%%%%%%%%%%%%%%%%%%%%%%%%%%%%%%%%%%%%%%
%%                                                      %%
%%           D O~C U M E N T   B E G I~N S              %%
%%                                                      %%
%%%%%%%%%%%%%%%%%%%%%%%%%%%%%%%%%%%%%%%%%%%%%%%%%%%%%%%%%%
%%%%%%%%%%%%%%%%%%%%%%%%%%%%%%%%%%%%%%%%%%%%%%%%%%%%%%%%%%
\begin{document}
\renewcommand{\headwidth}{11.85cm} % zmiana szerokości nagłówka i~stopki na 7.5cm

\CenterWallPaper{1.0}{net-375-2.jpg} % watermark on every page
%%%
% Uncomment "\maketitle" statement to make a~title (page).
%%%
\maketitle
\pagestyle{empty}
\noindent \scriptsize{Rewizja \RCSRevision}\\
\\
Konsultacja merytoryczna:\\
Maciej Starego\\
Robert Różek\\
Marzena Tomaszewska\\
Małgorzata Grzegorek\\
Alina Nowikowska\\
Anna Reda\\
Joanna Tumasz\\
Aleksandra Ferenc\\
Filip Bisztyga\\
\\
Korekta:\\
Aleksandra Lipczyńska\\
Aleksandra Ferenc\\
\noindent\linia\\
\\
\\
\\
\\
\begin{flushright}
\Large{\textbf{Moje góry łaskawe\footnote{Słowa i~muzyka Gosia Wyka, wykonanie: Ryczące Czterdziestki} \includemovie[text=\eightnote, controls=true, attach=true]{}{}{moje-gory-laskawe.mp3}}}\\
%\Large{\textbf{Moje góry łaskawe\footnote{Słowa i~muzyka: Gosia Wyka, wykonanie: Ryczące Czterdziestki}}}\\
\normalsize{
\vspace{0.7cm}
Moje góry łaskawe dajcie mi święty spokój kiedyś\\
Niechaj żyję tu z~wami a~potem umrę\\
Na zielonej połoninie\\
\vspace{0.7cm}
Moje góry łaskawe dajcie mi święty spokój kiedyś\\
Niechaj żyję tu z~wami a~potem umrę\ldots\\
\vspace{0.7cm}
Domem teraz i~potem bądźcie mi zawsze\\
Bo w~każdym drzewie chowam duszę\\
Niech porywa mnie wiatr\\
Porywa wiatr każdego dnia\\
W~tajemnice górskich dróg\\
\vspace{0.7cm}
Moje góry łaskawe dajcie mi święty spokój kiedyś\\
Niechaj żyję tu z~wami a~potem umrę\\
Kiedy siły mnie opuszczą\\
\vspace{0.7cm}
Moje góry łaskawe dajcie mi święty spokój kiedyś\\
Niechaj żyję tu z~wami a~potem umrę\\
Na zielonej połoninie}
\end{flushright}
\clearpage
\pagestyle{empty}
\small
\tableofcontents
\cleardoublepage
\renewcommand{\abstractname}{\Large Wstęp}
\begin{abstract}
\indent Panuje opinia, że organizowanie obozów wędrownych, spływów kajakowych czy innych form jest łatwiejsze od przygotowania obozu stacjonarnego. Rzeczywistość pokazuje, że to zadanie bardzo trudne. Jak faktycznie sprawa się przedstawia, można sprawdzić organizując obozy obu typów. Po kilku przeprowadzonych obozach każdy będzie miał swoje zdanie. O~przygotowaniach obozów stacjonarnych można znaleźć sporo literatury, natomiast na temat obozów wędrownych dla harcerzy napisano niewiele. W~tym opracowaniu zebrałem doświadczenia swoje, pochodzące z~prowadzonych przeze mnie obozów, jak również doświadczenia i~komentarze kadr obozów, które odwiedzałem na przestrzeni lat jako wizytator. Nie zawiera ono jednak instrukcji jak formalnie zatwierdzić obóz ani wzmianek o~podstawowych kwestiach dotyczących kompletowania kadry, jak również nie określa ,,jedynego słusznego'' programu. Zamieszczone tutaj przykłady, dobre rady, praktyki, ostrzeżenia i~informacje zostały zebrane z~intencją wsparcia dla osób, które mają małe doświadczenie w~organizowaniu i~prowadzeniu obozów wędrownych, albo nie mają go wcale. Uwzględnienie tego wszystkiego wymaga sporej ilości czasu, więc kadra obozu powinna zidentyfikować zadania do wykonania i~dzielić się nimi na bieżąco, aby nie obciążać jednej czy dwóch osób. Identyfikację zadań ułatwia niniejszy poradnik. Wszystkie poruszone aspekty są komentarzami do arkusza wizytacji obozów wędrownych, dlatego tytuł każdej sekcji zawiera odpowiadający numer z~arkusza wizytacji. Oczywiście uwzględnienie wszystkich opisanych punktów nie zagwarantuje udanego obozu, jednak pozwoli na lepsze jego przygotowanie pod wieloma względami. Wszystko zależy od zaangażowania, dobrej woli wszystkich uczestników, nie tylko kadry. Powodzenia!
\end{abstract}
\cleardoublepage
%\tableofcontents
%\thispagestyle{fancy} % this sets the pagestyle for the page with the title as well
% \thispagestyle{empty}
\section{Przygotowania}
\pagestyle{fancy}
Jak sam tytuł tej sekcji podpowiada, zawiera ona wybrane aspekty, które kadra przyszłego obozu powinna rozważyć i~uświadomić sobie w~pierwszej kolejności, po powzięciu decyzji o~chęci wyjazdu na obóz wędrowny. Zaniechanie rzetelnego przygotowania niektórych z~nich będzie miało mniej lub bardziej negatywne skutki, które objawią się w~trakcie obozu, kiedy będzie już za późno na ich poprawienie czy zneutralizowanie.
\subsection{Od czego i~kiedy zacząć}
Przygotowanie obozu wędrownego zaczyna się przeważnie od ,,pomysłu'', że to powinien być właśnie obóz wędrowny. Najczęściej wynika to z~planu śródrocznej pracy drużyny. W~pewnym momencie powstaje myśl, że drużyna powinna spróbować wędrówki w~jakiejś formie, np.: wędrówki piesze, obozy rowerowe, spływy kajakowe, obozy konne itd. --- w~terenach górskich jaki i~nizinnych. Wszystkie one mogą wystąpić w~różnych wersjach, np. z~ekwipunkiem transportowanym przez uczestników lub przewożonym przez inne osoby, z~kilkoma kwaterami lub z~codzienną zmianą kwatery, pod namiotami, w~schroniskach, szkołach, agroturystyce itp., itd. Istnieją także inne formy, jak np. obóz objazdowy, który polega na wynajęciu samochodu i~odwiedzeniu wybranych miejsc w~kraju lub za granicą.

Jakakolwiek forma nie zostałaby wybrana, kolejnym krokiem jest ustalenie terminu, trasy i~miejsc kwaterowania. Te trzy sprawy wiążą się ze sobą bardzo mocno i~należy je rozpatrywać jako całość, gdyż np. termin i~trasa mogą zależeć od jakiegoś wydarzenia, w~którym uczestnicy obozu powinni wziąć udział. Wynika z~tego wprost, że w~danym terminie obóz powinien znaleźć się we właściwym miejscu.

Najmniejsze pole do wyboru pozostawia wybór terminu, gdyż jest on ograniczony trwaniem innych obozów, kursów itp, które zwyczajowo planowane są na wakacje. Po wyborze potencjalnych dat początku i~końca obozu oraz rejonu, przez który obóz ma wędrować, sprawdza się historię tamtejszych warunków pogodowych z~kilku ostatnich lat, aby uniknąć najbardziej prawdopodobnych okresów brzydkiej pogody (patrz pkt. \ref{pogoda} na stronie \pageref{pogoda}). Następnie należy zorientować się w~kalendarzu imprez zaplanowanych w~wybranym czasie, w~docelowym rejonie. Dzięki temu wiadomo jest kiedy i~gdzie warto być.

Trasę należy wybrać po wstępnym ustaleniu terminu obozu. Możliwości jest wiele:
\begin{itemize}
\item ustalić trasę samodzielnie na podstawie informacji o~szlakach turystycznych, atrakcjach, ciekawostkach oraz miejscach, które drużyna ,,koniecznie musi odwiedzić''. Informacje takie można znaleźć w~przewodnikach, bedekerach, czasopismach turystycznych, na rewersach map, w~Internecie,
\item skorzystać z~gotowych propozycji szlaków turystycznych opracowanych przez regionalne oddziały PTTK, \href{http://szlaki.pttk.pl/spis.html}{http://szlaki.pttk.pl/spis.html},
\item wykorzystać listę wycieczek wielodniowych opracowanych przez PTSM (zawierają kwatery w schroniskach PTSM), \href{http://www.ptsm.org.pl/}{http://www.ptsm.org.pl/ --- informator},
\item zasięgnąć rady instruktorów, którzy przeprowadzili w~poprzednich latach swoje obozy,
\item poprosić o~wsparcie opiekuna obozów wędrownych z~ramienia Okręgu,
\item powtórzyć trasę obozu innej drużyny z~poprzednich lat.
\end{itemize}
Ustalona trasa będzie ewoluować w~miarę pojawiania się nowych informacji dotyczących szlaków, kwater, atrakcji, wydarzeń itp., więc nie należy się kurczowo trzymać pierwszej wersji. Pracując nad wytyczeniem trasy należy pamiętać o~aspektach omówionych w~punktach \ref{dlugosc-wedrowek}, \ref{kondycja-uczestnikow}, \ref{atrakcyjnosc}, a~także o~celach programowych --- przecież drużyna ogarniruje obóz, aby je zrealizować. Pobyt w~pewnych miejscach będzie wynikał z~tej konieczności.

Jednym z~czynników, które mają największy wpływ na zmiany projektu trasy są miejsca kwaterowania. Zdarza się, że ciekawy szlak kończy się ,,na pustkowiu'' i~trzeba przejść jeszcze kilka kilometrów aby dotrzeć do schroniska lub miejscowości, w~której można znaleźć kwaterę. Czasami nie ma wyboru, bo na końcu szlaku jest tylko jedno schronisko. Można się na nie zdecydować bez względu na warunki jakie oferuje lub wykorzystać publiczny transport i~pojechać do innej miejscowości (na obozie wędrownym przecież nie trzeba zawsze i~wszędzie chodzić piechotą), gdzie wybór jest większy lub warunki lepsze. Spływy kajakowe i~obozy rowerowe zwyczajowo kwaterują na polach biwakowych lub w~stanicach wodnych, gdzie jest dostęp do łazienek, sanitariatów, natrysków, gastronomii itd. Jeśli nie można znaleźć wolnej kwatery na końcu szlaku, który został wyznaczony na dany dzień, należy rozważyć przejazd z~końca szlaku do innej miejscowości lub zmianę ostatniego fragmentu szlaku, aby dojść w~inne miejsce. Poszukiwanie odpowiednich kwater to żmudne zajęcie, przy którym należy zwrócić szczególną uwagę na warunki lokalowe / biwakowe jakie są oferowane. Wszystko, co jest ,,obiecane'' lub zadeklarowane przez właściciela kwatery powinno zostać zapisane w~umowie najmu / rezerwacji kwatery. Ryzyko z~tym związane opisane jest szczegółowo w~punktach \ref{kwatery} i~\ref{miejsca-noclegow}. Przy okazji wyboru kwater należy uwzględnić także miejsca, w~których obóz będzie mógł zaopatrzyć się w~żywność oraz inne potrzebne artykuły. Pominięcie tego aspektu może skutkować koniecznością organizowania specjalnych, kilkugodzinnych wypraw ,,do sklepu'', które zdezorganizują cały harmonogram. Obóz powinien kwaterować albo trasa powinna przechodzić przez miejscowości gdzie są sklepy, aby można było sprawnie i~szybko zrobić zakupy. W~dni, kiedy zaplanowane są wędrówki, warto skorzystać z~dostępnych lokali gastronomicznych i~zdecydować się na zamówienie obiadu, lub obiado-kolacji na wyznaczoną godzinę. W~ten sposób, po zejściu ze szlaku i~zostawieniu ekwipunku na kwaterze, nie tracąc czasu i~energii, można zapewnić uczestnikom wartościowe posiłki, których przygotowanie wykracza poza standardowe obozowe możliwości.

\subsection{Dokumentacja finansowa i~nie tylko [1]}
\subsubsection{Umowy [1.5]}
Miejsca noclegów powinny zostać zarezerwowane najwcześniej jak to możliwe, czyli po ostatecznym ustaleniu trasy i~terminów. Schroniska wszelkiego rodzaju są rezerwowane przez różne instytucje i~firmy z~reguły z~rocznym wyprzedzeniem. Niektóre nawet na kilka sezonów --- np. przez szkoły wschodnich sztuk walki, które przywożą swoich adeptów co roku w~to samo miejsce, gdzie są odpowiadające im warunki treningowe.

Przy rezerwacji telefonicznej, nie ma możliwości udowodnienia, że została zawarta jakaś umowa, jeżeli druga strona tę okoliczność kwestionuje. Możliwość udowodnienia, że rezerwacja została dokonana, najczęściej oznacza, że umowa została faktycznie zawarta, a~wtedy obie strony ponoszą
odpowiedzialność za poniesione szkody z~tytułu niewykonania umowy na podstawie art. 471 K.~C. Szkodą będzie np. różnica w~cenie zarezerwowanego miejsca, a~kosztem rzeczywiście poniesionym. Forma pisemna wymagana jest jedynie dla celów dowodowych (art. 74 K.~C.). Oznacza to, że umowa mimo niezachowania formy pisemnej jest ważna, ale w~razie sporu przed sądem może być trudno udowodnić sam fakt jej zawarcia.

Rezerwacja powinna być więc dokonana w~formie pisemnej, aby miała moc wiążącą obie strony, tj. tego kto to miejsce oferuje i~tego, kto chce z~niego skorzystać. Umowa\footnote{Wzór umowy rezerwacji w~załączniku \ref{wzor-umowy-rezerwacji}.} powinna jasno precyzować, oprócz ceny, terminu, adresu (ewentualnie dokładnego opisu lokalizacji), przewidywanej liczby osób, dokładną ofertę danego miejsca --- dzięki temu będzie można potem tego wymagać. Wyjaśnione to jest dokładnie w~punktach \ref{kwatery} i~\ref{miejsca-noclegow} na stronie \pageref{kwatery}.

Wszelkiego typu umowy przed podpisaniem powinny zostać skonsultowane z~osobą, która się na tym zna, np.  wyznaczoną przez Zarząd Okręgu. Nieraz można wpaść w~pułapkę podpisując umowę podsuniętą przez usługodawcę jako ,,standardową, którą zawsze podpisują z~klientami'' lub znalezioną w~internecie. W~umowach takich zwykle roi się od klauzul zakazanych, które są niezgodne z~aktualnie obowiązującym prawem, zawierają błędy formalne itd. Zawsze warto przed podpisaniem poinformować drugą stronę, że umowa zostanie skonsultowana przed podpisaniem. Można poprosić o~jej wzór, przesłać scan lub fax do odpowiedniej osoby i~uzyskać potwierdzenie. Osobną kwestią jest kto i~czy w~ogóle ma prawo do podpisywania umów i~w czyim imieniu (drużyny?, hufca?, obozu?) --- tę kwestię również należy dokładnie wyjaśnić z~koordynatorem akcji letniej, ewentualnie członkiem Zarządu Okręgu, ponieważ podpisana umowa może okazać się nieważna jako podpisana przez osobę nieuprawnioną.

Umowę można podpisać również zdalnie, wysyłając ją po prostu pocztą --- druga strona powinna odesłać jedną kopię po podpisaniu. Taka sytuacja jest obarczona sporym ryzykiem, ponieważ nie nastąpiło naoczne sprawdzenie przedmiotu umowy i~nijak nie może być on zweryfikowany. Niejednokrotnie można zostać oszukanym lub wprowadzonym w~błąd.
\\
\\
\small{
\emph{Był kiedyś taki przypadek umowy na rezerwację noclegów w~„Schronisku pod górą”\footnote{Nazwa zmieniona.}. Dane teleadresowe tego schroniska zostały znalezione na odwrocie pewnej starej mapy. Po rozmowie telefonicznej i~ustaleniu szczegółów umowa została spisana, podpisana, wysłana pocztą i~odesłana jak należy. Po przybyciu na miejsce w~trakcie obozu okazało się, że w~schronisku nikt nic nie wie o~tej rezerwacji. Po sprawdzeniu danych teleadresowych okazało się, że są one niewłaściwe, tzn. nieaktualne. A~dokładnie: dane teleadresowe znalezione na odwrocie starej mapy były prywatnymi danymi ówczesnego najemcy tego schroniska. Właściciel schroniska rozwiązał umowę z~tym najemcą kilka lat przed wspomnianym obozem i~wynajął je komu innemu. Co gorsza, ów poprzedni najemca założył swoje własne „Schronisko pod górą” --- kilka kilometrów od oryginalnego. Podczas rozmowy telefonicznej nie raczył wspomnieć o~tym fakcie, lecz nieuczciwie potwierdził, że umowa dotyczy oryginalnego schroniska. Zgodnie z~umową obóz wpłacił zaliczkę na podany numer konta. Można sobie wyobrazić co przeżywała kadra i~uczestnicy tego obozu, kiedy po całym dniu wędrowania przybyli wieczorem do schroniska, gdzie nie było dla nich rezerwacji, której byli pewni. Dodatkowo z~różnych względów nie było możliwe przemieszczenie się obozu do drugiego „Schroniska pod górą”.}
\\
\emph{Na innym obozie okazało się, że zarezerwowane pokoje spełniają kryteria określone w~umowie, ale nieuczciwy właściciel nie wspomniał, że są one w~trakcie remontu. Z~braku jakiegokolwiek pola manewru kadra zdecydowała się zakwaterować uczestników w~tych pokojach. Niestety w~nocy niektórzy uczestnicy zaczęli kaszleć, a~jedna osoba została zabrana przez pogotowie do szpitala. Przyczyną był pył pozostały po wełnie mineralnej, którą ocieplano dach w~tych remontowanych pokojach (na poddaszu). Wełnę przykryto co prawda folią paroizolacyjną, ale nastąpiło to raptem kilka godzin przed zakwaterowaniem. W~nocy unoszący się pył opadł na śpiących uczestników i~spowodował kłopoty.}
}
\\
\\
Biorąc pod uwagę wspomniane historie należy dołożyć wszelkich starań, aby zweryfikować uczciwość drugiej strony podpisującej umowę, a~jeśli dotyczy ona miejsca zakwaterowania upewnić się co do jego zgodności z~tym co obejmuje umowa. Oczywiście warto, a~wręcz należy wykorzystać do tego Internet czytając opinie osób, które skorzystały z~usług, których dotyczy (lub ma dotyczyć) umowa. Przy tej okazji warto zwrócić uwagę na czas pochodzenia tych opinii, bo co komu po opinii sprzed lat, kiedy wszystko mogło być inaczej?

\paragraph{$\bigstar$ Podsumowanie}
\begin{checklist}
\item Miejsca noclegów zarezerwować najwcześniej jak to możliwe.
\item Rezerwacja powinna być w~formie pisemnej.
\item Umowa powinna jasno precyzować, oprócz ceny, terminu, adresu, przewidywanej liczby osób, co dokładnie zostanie zaoferowane w~danym miejscu i~to co obiecał właściciel.
\item Umowa przed podpisaniem powinna zostać skonsultowana z~kompetentną osobą.
\item Umowę można podpisać zdalnie, wysyłając ją pocztą.
\item Sprawdzić opinie osób, które niedawno skorzystały z~usług, których dotyczy umowa.
\end{checklist}

\subsubsection{Lista uczestników [1.8]}
Niektóre schroniska stosują obowiązek meldunkowy, który dawniej był nakazany przepisami. Wymagają kopii listy uczestników i~kadry obozu, która musi zostać złożona w~schronisku zaraz po przybyciu do niego. Nie ma co wdawać się w~dysputy z~osobą przyjmującą obóz do schroniska na temat zasadności tego wymagania, zwłaszcza, że może być ono uzasadnione np. względami bezpieczeństwa w~górach. Zamiast tego warto przed obozem przygotować sobie odpowiednią ilość kopii takiej listy i~wręczać ją na żądanie. Lista powinna być ponumerowana i~zawierać nazwisko, imię, adres zamieszkania (ulicę, kod pocztowy, miejscowość), PESEL wszystkich osób zapisanych na obóz włącznie z~komendantem, powinna posiadać nagłówek lub pieczątkę organizatora i~być podpisana przez komendanta. Jeśli na liście widnieje osoba, która na obóz nie pojechała, to po prostu należy ją wykreślić.

\paragraph{$\bigstar$ Podsumowanie}
\begin{checklist}
\item Przygotować przed obozem kilka kopii listy uczestników obozu.
\end{checklist}

\subsubsection{Ubezpieczenia zdrowotne [1.9]}
Czasami podczas obozu konieczna jest wizyta u~lekarza. W~niektórych przypadkach znajomość numeru PESEL nie wystarcza i~personel placówki medycznej żąda dowodu ubezpieczenia zdrowotnego pacjenta. Dlatego warto zadbać o~to przed wyjazdem i~dopilnować aby odpowiednie informacje znalazły się w~dokumentacji obozowej, najlepiej na kartach obozowych.

Od 1~stycznia 2013 roku wprowadzono system elektronicznej weryfikacji ubezpieczonych. Przy rejestracji w~przychodni czy w~szpitalu najważniejszy jest numer PESEL. Potrzebny także będzie dokument potwierdzający tożsamość pacjenta np. prawo jazdy czy legitymacja szkolna lub studencka. Niestety, system obowiązuje tylko w~tych placówkach, które dobrowolnie zdecydowały się na jego wprowadzenie.

\paragraph{$\bigstar$ Podsumowanie}
\begin{checklist}
\item Zebrać przed obozem informacje o~ubezpieczeniu zdrowotnym uczestników i~kadry.
\end{checklist}

\subsection{Kwatery i~porządek [2]}

\subsubsection{Miejsca noclegów [2.2] \label{miejsca-noclegow}}
Osoby organizujące obóz nagminnie skazują siebie i~uczestników na monotonię miejsc noclegowych tego samego typu, tzn. wszystkie noclegi są zawsze np. w~namiotach, schroniskach PTSM, szkołach itd. Dobrze jest wprowadzić urozmaicenie i~zdecydować się na noclegi w~innych miejscach, zwłaszcza jeśli jest to obóz pod namiotami. Warto np. w~połowie i~pod koniec obozu zorganizować nocleg zamiast pod namiotami w~schronisku z~prawdziwego zdarzenia. Uczestnicy poczują się lepiej --- wyrwani z~namiotowej czy szkolnej monotonii. Nocleg w~schronisku na długo zostanie w~pamięci uczestników. Nie warto iść na łatwiznę tylko poświęcić czas i~zmobilizować kilka osób z~kadry obozu aby wprowadzić różnorodność w~tej materii.
%
%\begin{figure}[htp]
%\centering
%\includegraphics[scale=0.09431787188559634380067816600324]{namioty-na-niemcowej.jpg}\\
%\caption{Namioty na Niemcowej.}\label{fig:namioty-na-niemcowej}
%\end{figure}

Bardzo ważnym aspektem jest stan techniczny oraz wyposażenie miejsc, w~których obóz będzie nocował. Dokonując rezerwacji należy dokładnie ustalić co jest oferowane i~za co kwatermistrz zapłaci. W~wielu przypadkach kadra wpada w~pułapkę naiwności i~zakłada coś, czego rezerwacja nie obejmuje.
\\
\\
\small{
\emph{Wielokrotnie zdarza się, że kadra rezerwując miejsca noclegów w~szkołach, podpisuje umowę na „pobyt i~nocleg w~szkole”. Na tej podstawie kadra błędnie zakłada, że obóz będzie miał dostęp do różnych pomieszczeń, mediów czy obiektów na terenie takiej szkoły. Praktyka pokazuje, że obóz po przybyciu nie ma:
\begin{itemize}
\item dostępu do sal lekcyjnych i~ludzie są zmuszeni spać na korytarzach albo w~sali gimnastycznej (zdarzył się nawet przypadek noclegu na trybunach w~sali gimnastycznej), obiektów sportowych na terenie szkoły, kuchni, lodówki, stołówki, sali komputerowej,
\item bieżącej wody albo jest tylko zimna,
\item ciepłej wody, albo jest jej za mało, bo zależy od czyjegoś widzimisię (np. ,,czy palacz przyjdzie do kotłowni i~napali''),
\end{itemize}
a~oprócz tego:
\begin{itemize}
\item nie można korzystać z~natrysków,
\item wydzielono zbyt mało toalet,
\item nie można zorganizować ogniska, świeczowiska (kominka itp.),
\end{itemize}}}

Gdy noclegi rezerwowane są w~schroniskach PTSM kadra częstokroć nie zdaje sobie sprawy, że większość tych schronisk jest czynna od godz 17:00 do godz 10:00 dnia następnego, z~wyjątkiem tych, które są wyraźnie oznaczone jako całodobowe. Jednym słowem w~trakcie dnia nie ma dostępu do schroniska. Obowiązuje również z~reguły godzina, po której drzwi schroniska są zamykane na noc i~ze względów bezpieczeństwa są otwierane dopiero rano.

Jakiekolwiek by to nie było miejsce, kadra zazwyczaj jest zdziwiona, że obóz nie może się tam ,,dowolnie zachowywać'', że są tam również inne grupy, oazy czy kolonie, że obowiązuje cisza nocna od 22:00 do 6:00, że nie wypada co chwilę zawracać głowy personelowi tego miejsca demaskując przy okazji swoje nieprzygotowanie itd.

Miejsca noclegów muszą być czyste, uczestnicy powinni mieć odpowiednią ilość miejsca dla siebie, muszą mieć miejsce na chwilę prywatności. Bezpieczeństwo uczestników jest jednym z~priorytetów i~nie może być w~żadnym wypadku ignorowane. Należy o~tym pamiętać decydując się na noclegi na dziko, na łące pod chmurką, w~napotkanych bacówkach, czy innych przypadkowych miejscach.
\\
\\
\small{
\emph{Ciężkie przypadki nadużyć i~ignorancji prowadzą do opłakanych skutków. Zdarzyło się na jednym z~obozów, że uczestnicy nosili ze sobą namioty, ale ponieważ było upalnie i~noce były wyjątkowo ciepłe kadra zdecydowała się na noclegi pod chmurką --- bez namiotów --- na łące. Brzmi wspaniale, ale niestety --- wysokie stężenie pyłków traw i~kwiatów rosnących na tej łące wywołało bardzo gwałtowne reakcje alergiczne u~większości uczestników tego obozu. Objawy były na tyle poważne, że kilka osób zostało rano odwiezionych z~tego odludzia do szpitala. Podobna historia wydarzyła się na obozie, kiedy kadra postanowiła zaoszczędzić na noclegach i~zdecydowała, że obóz będzie nocował w~miejscu, do którego dotrą o~godzinie 18:00 danego dnia. Przypadek zrządził, że pewnego dnia nocleg wypadł w~opuszczonej dużej bacówce znajdującej się na zboczu góry, 200 metrów od szlaku. Z~bacówki roztaczał się przepiękny widok na porośniętą bujną roślinnością dolinę górskiego strumienia. Wieczorem zachodzące słońce ułożyło do snu cały obóz. Rano okazało się, że część uczestników źle się czuło; objawy nie były u~wszystkich takie same. Kilka osób odwieziono do szpitala. Okazało się później, że:
\begin{itemize}
\item woda ze strumienia nie nadawała się do picia, bo zawierała duże stężenie pewnych środków chemicznych po niedawnych opryskach przeciwko jakimś szkodnikom,
\item bacówka była wykorzystywana przez pasterzy jako schronienie dla owiec --- insekty schowane w~szparach tej bacówki dotkliwie pokąsały w~nocy uczestników obozu.
\end{itemize}}}

Przykłady te pokazują, że bezmyślność może sprowadzić różnego rodzaju niebezpieczeństwo na uczestników obozu. Nielegalne noclegi w~miejscach do tego nie przeznaczonych, bez uzgodnienia z~właścicielem terenu lub załatwione w~nieprawidłowy sposób nie powinny się zdarzać.

Przed rozpoczęciem dokonywania rezerwacji należy wypisać sobie wymagania dotyczące miejsc noclegowych dla obozu. Będąc przygotowanym do rozmowy w~ten sposób, można uzyskać w~miarę jasny obraz sytuacji w~danej kwaterze, a~osoba dokonująca rezerwacji świadomie decyduje się na określone warunki lokalowe. Kadra obozu nie jest narażona na nieprzyjemne niespodzianki po przybyciu na miejsce.

Opis warunków lokalowych warto zawrzeć w~umowie rezerwacji, dzięki czemu w~razie rozbieżności istnieje szansa na uzyskanie zwrotu części opłat.\\*
~\\*
~\\*
~\\*
~\\*
~\\*
~\\*
~\\*
~\\*
~\\*
\begin{figure}[htp]
\centering
\includegraphics[scale=0.09431787188559634380067816600324]{namioty-na-niemcowej.jpg}\\
\caption{Namioty na terenie Chatki pod Niemcową.}\label{fig:namioty-na-niemcowej}
\end{figure}

\newpage
\paragraph{$\bigstar$ Podsumowanie}
\begin{checklist}
\item Wprowadzić urozmaicenie i~zdecydować się na noclegi w~różnego typu miejscach.
\item Dokonując rezerwacji dokładnie ustalić co jest oferowane i~za co obóz zapłaci.
\item Nie zakładać niczego, czego rezerwacja nie obejmuje.
\item Zapoznać się wcześniej z~regulaminem wybranego miejsca ze względu na różnego typu ograniczenia, takie jak cisza nocna, brak dostępu w~dzień itp.
\item Decydując się na noclegi na dziko pamiętać o~bezpieczeństwie uczestników.
\item Przed rozpoczęciem rezerwowania wypisać sobie wymagania dotyczące miejsc noclegowych dla obozu.
\item Opis warunków lokalowych zawrzeć w~umowie rezerwacji.
\end{checklist}

\subsubsection{Kwatery [2.1] \label{kwatery}}
Kadra obozów wędrownych rezerwuje wszelkiego rodzaju kwatery wybierając je na ogół dlatego, że znajdują się w~okolicy, gdzie akurat kończy się wyznaczona na dany dzień trasa. To powoduje, że obozy kwaterują gdzie popadnie, ewentualnie gdzie się da. Skutek jest przeważnie taki, że kwatery są beznadziejne, a~wszyscy uczestnicy obozu marzą o~tym, aby się z~nich jak najszybciej wynieść.

Szukanie kwater warto zacząć od\ldots szukania atrakcji turystycznych wzdłuż projektowanej trasy. ,,Wzdłuż'' należy rozumieć: ,,nie tylko w~miejscach przez które ta trasa przechodzi''. Nie warto od samego początku kurczowo trzymać się zaprojektowanej trasy, zamiast tego należy przeszukać szeroko rozumiane okolice, nawet w~dość znacznej odległości od linii wyznaczającej oś trasy. W~okolicach atrakcji turystycznych prawie zawsze jest spory wybór miejsc, w~których można zakwaterować obóz. Pod uwagę można wziąć regularne pola namiotowe, miejsca campingowe przy agroturystyce, lokale agroturystyczne, pensjonaty, schroniska, szkoły, bacówki itp. Ważne jest to, aby w~najbliższej okolicy kwatery było coś atrakcyjnego, gdzie można zaprowadzić uczestników obozu (niekoniecznie wszystkich naraz), albo położenie lub wyposażenie takiej kwatery było atrakcyjne samo w~sobie. Takie podejście do wyboru kwater skutkuje możliwością uatrakcyjnienia programu obozu w~miejscach kwaterowania, daje możliwość złapania oddechu między wędrówkami, zatrzymania się w~biegu.

Po wypisaniu sobie i~zaznaczeniu na mapie wybranych lokalizacji należy spróbować zmodyfikować projektowaną trasę tak, aby nadal przebiegała przez ,,punkty, które trzeba odwiedzić z~jakichś względów'' i~jednocześnie poszczególne jej etapy kończyły się w~tych wypisanych lokalizacjach. W~ten sposób projektant trasy pozostanie usatysfakcjonowany, a~program obozu zyska na atrakcyjności.

Trochę inaczej sprawa wygląda na spływach kajakowych --- kwatery znajdują się nad rzeką, którą obóz płynie i~praktycznie nie ma sensu szukać ich gdzieś dalej. Spływy i~obozy rowerowe kwaterują najczęściej na polach namiotowych, które są zaznaczone na mapie szlaku kajakowego czy rowerowego. Niestety, nie można założyć, że sytuacja się nie zmieniła i~informacje podane na mapie się nie zdezaktualizowały, nawet jeśli mapa jest wydana w~tym samym roku, kiedy obóz ma się odbyć --- przecież musiał minąć czas potrzebny na jej wyprodukowanie, czyli informacje częściowo muszą pochodzić z~poprzedniego roku. Często zdarza się, że zaznaczone na mapie pole namiotowe nie istnieje (np. posadzono tam młody las), jest nieczynne (np. kwarantanna ze względu na epidemię wścieklizny wśród zwierząt lub opryski) albo zostało wynajęte komuś w~jakimś celu (np. impreza firmowa). Bywa też odwrotnie, tzn. jakieś świetne pole nie zostało jeszcze zaznaczone albo jest źle zaznaczone. Przed wyruszeniem na obóz warto porozmawiać o~kwaterach z~kimś, kto przemierzał ten szlak w~poprzednim roku, ale już nie wcześniej. Można w~ten sposób uzyskać jakieś potwierdzenie informacji podanych na mapie. Znalezienie namiarów na ajentów czy właścicieli pól namiotowych, a~potem wykonanie kilku rozmów telefonicznych może dostarczyć aktualnych informacji. Opcją jest także rekonesans potencjalnych miejsc kilka tygodni przed obozem --- więcej o~tym w~sekcji \ref{zwiad-kwatermistrzowski} na stronie \pageref{zwiad-kwatermistrzowski}. Doświadczona kadra sprawdza nie tylko te pola namiotowe, na których nocleg wynika wprost z~długości dziennych etapów, ale także te, które w~razie problemów są alternatywą na danym odcinku.
\\
\\
\small{
\emph{Na spływie Słupią w~2011 posługiwano się książeczką Pascala, w~której nie było zaznaczone żadne pole namiotowe w~Wodnicy (to jakieś 8 czy 10~km. przed Ustką). Natomiast na dwóch czy trzech nadrzecznych tablicach z~nazwami kilku najbliższych pól namiotowych i~liczbami kilometrów do nich, znajdowały się takie informacje. To wcale nie były jakieś stare tablice. Rozważano więc nocleg w~tym miejscu. Po dopłynięciu i~wyjściu na brzeg okazało się, że nic tam nie ma tylko zarośnięta sterta gruzu. Ponieważ nie planowano tam noclegu nie stało się nic złego --- było jeszcze wcześnie i~spływ popłynął do Ustki, ale jeśli ktoś liczyłby na nocleg w~tym miejscu i~przypłynął wieczorem, to miałby problem.}}
\\
\\
\small{
\emph{We wrześniu 2008 odbyła się kilkudniowa wyprawa rowerowa w~okolice Elbląga i~Olsztyna. Zabrano porządną, nową mapę campingów, gdzie przy każdym były informacje, do którego dnia września jest czynny i~jakie posiada wyposażenie. We wrześniu już w~ogóle mało które pole namiotowe działa, więc liczono na prawdziwość tych danych. W~przypadku większości pól wszystko się zgadzało, oprócz campingu koło Olsztyna. Mimo, że na aktualnej mapie była informacja, że jest czynny do końca czy prawie końca września (prawdopodobnie miał z~tego powodu więcej gwiazdek), personel nie chciał przyjąć rowerzystów podając absurdalne argumenty --- pewnie po prostu nie chciało czy nie opłacało im się przyjmować małej ilości osób.}}

\paragraph{$\bigstar$ Podsumowanie}
\begin{checklist}
\item Szukanie kwater zacząć od szukania atrakcji turystycznych wzdłuż projektowanej trasy.
\item Nie upierać się przy pierwszej wersji trasy, wprowadzać zmiany w~miarę zdobywania kolejnych informacji.
\item W~okolicach atrakcji turystycznych prawie zawsze jest spory wybór kwater.
\item Atrakcyjna lokalizacja lub wyposażenie kwatery daje większe możliwości programowe.
\item Zaznaczone na mapie pole namiotowe może nie istnieć, być nieczynne albo wynajęte komuś innemu.
\end{checklist}

\subsubsection{Łazienki (umywalnie) i~ich standard [2.3, 2.4]}
Kadra obozu powinna zagwarantować dostęp do łazienek (umywalni) dla uczestników obozu. Ilość tych łazienek powinna być odpowiednio duża, aby wszyscy uczestnicy obozu mogli sprawnie i~stosunkowo szybko z~nich skorzystać wtedy, kiedy przewiduje to rozkład dnia. Zaniedbanie tego aspektu skutkuje tym, że do łazienek tworzą się długie kolejki, albo uczestnicy w~ogóle nie mają możliwości korzystania z~nich i~z~konieczności myją się w~rzece, jeziorze itp.

Dokonując rezerwacji kwater warto sprawdzić w~jakim stanie są łazienki (umywalnie) --- z~tego samego powodu co poprzednio, czyli aby być świadomym co obóz zastanie w~danym miejscu i~uniknąć niemiłych niespodzianek. Bardzo duża ilość właścicieli schronisk czy dyrekcji szkół, które przyjmują obozy na noclegi, nie przykłada w~ogóle wagi do standardu łazienek. Większość jest jakby żywcem wyjętych z~PRL-u: zaniedbane, brudne, odrapane, wilgotne, zagrzybione, zimne, ciemne i~ciasne. Właściciele wychodzą z~założenia, że turyści i~tak zapłacą, bo nie będą mieli innego wyjścia (a właściwie wyboru).

Po całodziennej wędrówce czy innym wysiłku, możliwość wykąpania się jest wręcz zbawienna --- to powinno być oczywiste dla każdego. W~trakcie obozów, czyli w~wakacje, szkoły są nieczynne, a~co za tym idzie --- puste. Nie ma więc sensu aby szkolna kotłownia pracowała. Ciepła woda na takich kwaterach to rzadkość, chyba, że kadra obozu zagwarantowała ją sobie w~umowie. W~schroniskach czy pensjonatach w~małych miejscowościach, na szlaku, gdzie ciepła woda podgrzewana jest na miejscu, zwykle okazuje się, że nie wystarcza jej dla wszystkich lub jest ona dostępna tylko w~określonych godzinach (niekoniecznie pokrywających się z~tymi, na które kadra obozu zaplanowała toaletę w~rozkładzie dnia). Najczęściej jest tak, że natryski są udostępniane tylko na kilka godzin wieczorem i~zazwyczaj jest ich mało. Warto mieć tego świadomość decydując się na daną kwaterę.
\\
\\
\small{
\emph{W~pewnym schronisku w~Bieszczadach, które jest urządzone w~budynku gimnazjum, natryski znajdują się w~niedużym pomieszczeniu, do którego można się dostać przez salę, w~której urządzono suszarnię. Ponieważ schronisko jest duże i~przyjmuje duże grupy obozowe, kolejka do natrysków zaczyna się w~suszarni, a~kończy na korytarzu. Natrysków jest 6. Są to wnęki w~ścianie osłonięte od przejścia tekstylnymi zasłonkami. Jest czysto, ale ubrań ani ręczników nie ma na czym powiesić. W~pomieszczeniu znajduje się kilka krzeseł, z~których można korzystać. Jest ciepła woda. Kolejka jak i~natryski są koedukacyjne.}
\\
\\
\emph{Natryski w~szkole w~pewnej wsi na Podlasiu znajdują się w~części przy hali gimnastycznej. Osobne dla dziewcząt, osobne dla chłopców. Wejście do natrysków wiedzie przez szatnię. Pomieszczenie z~natryskami ma około 35 metrów kwadratowych i mieści 16 pryszniców. Nie są one w~żaden sposób osłonięte, pomieszczenie przedzielone jest 3 ściankami ustawionymi na środku, na każdej stronie takiej ścianki znajdują się dwa prysznice i~dodatkowo po 2 na każdej szczytowej ścianie. Same prysznice są stare, a~głowice skorodowane. Woda jest bardzo zimna. Ciepłą wodę należy zamówić odpowiednio wcześniej, a~i~tak jest ona uzależniona od pojawienia się palacza w~kotłowni. Szkoła jest pusta w~wakacje, więc palacz ma wolne --- ściągnięcie go do kotłowni jest zadaniem raczej skazanym na niepowodzenie --- nie udało się to nawet kiedy organizator uzyskał zapewnienie dyrekcji szkoły, że ciepła woda będzie dostępna.}
\\
\\
\emph{I~ostatni przykład, ze szkoły z~pewnej miejscowości w~Kotlinie Kłodzkiej. Duża szkoła bez części sportowej. Natrysków w~ogóle nie ma, za to schronisko funkcjonuje. Łazienki są dostępne, osobno damskie i~męskie --- po jednej na całe schronisko. Problem braku natrysków został ,,rozwiązany'' w~bardzo ciekawy sposób. Mianowicie wzdłuż ściany znajduje się 7~umywalek --- z~jednej baterii umywalkowej została wykręcona wylewka, a~na jej miejsce podłączono wąż ze słuchawką prysznicową. Drzwi do łazienek nie mają żadnego zamknięcia. Za to jest ciepła woda.}
}

\paragraph{$\bigstar$ Podsumowanie}
\begin{checklist}
\item Ilość toalet (umywalni) powinna być odpowiednia dla liczby uczestników obozu.
\item Dokonując rezerwacji kwater sprawdzić w~jakim stanie są łazienki (umywalnie).
\item Ciepła woda pod natryskami to w~wielu miejscach nadal ,,deficytowy towar''.
\end{checklist}

\subsubsection{Część rekreacyjna i~sportowa [2.9]}
Warto sprawdzić czy w~okolicy kwaterowania są obiekty sportowe lub rekreacyjne, z~których obóz mógłby skorzystać. Chodzi tutaj o~takie miejsca jak baseny, sale gimnastyczne, boiska, sauny, termy, strzelnice sportowe, wypożyczalnie sprzętu wodnego, rowerów itp. Posiadając wiedzę o~nich można je wykorzystać do zajęć obozowych nie tylko w~trakcie niepogody. Rekreacja czy zajęcia sportowe nie są obowiązkowe, ale są doskonałym narzędziem relaksu zarówno dla uczestników jak i~kadry obozu. Brak dostępu do takich obiektów, które znajdują się blisko kwatery i~zostaną ,,wykryte'' prowadzi do zadawania kadrze pytań przez uczestników o~możliwość skorzystania z~nich. Kadra przeważnie odpowiada przecząco, ponieważ program obozu nie uwzględnia czasu na dodatkowe atrakcje. W~takich sytuacjach wychodzi na jaw, że organizatorzy nie przyłożyli się jak należy do rozpoznania możliwości istniejących w~miejscu kwaterowania, lecz skupili się na zarezerwowaniu tylko i~wyłącznie noclegów, a~pobyt w~danej miejscowości to wyłącznie czysta konieczność, tj. noclegu właśnie.

\paragraph{$\bigstar$ Podsumowanie}
\begin{checklist}
\item Obiekty sportowe lub rekreacyjne wykorzystywać do zajęć obozowych nie tylko w~trakcie niepogody.
\end{checklist}

\subsection{Bezpieczeństwo [4]}
\subsubsection{Telefony alarmowe [4.1]}
Przed wyjazdem kadra powinna sporządzić listę aktualnych telefonów i~adresów placówek opieki medycznej znajdujących się wzdłuż trasy wraz z~godzinami ich otwarcia. Nigdy nie wiadomo co, kiedy i~gdzie może się zdarzyć. W~razie konieczności wizyty u~lekarza wspomniana lista z~informacjami przyda się aby trafić za pierwszym razem pod adres, gdzie akurat przyjmuje właściwy lekarz. Druga, podobna lista powinna zawierać dane aptek, aby po wizycie u~lekarza można było zrealizować receptę --- nie traci się wtedy czasu na błąkanie się po okolicznych miejscowościach, zapewne w~towarzystwie chorej osoby, w~poszukiwaniu (czynnej) apteki.

Podczas przygotowywania obu list nie wystarczy tylko znalezienie danych teleadresowych w~internecie --- trzeba jeszcze koniecznie zatelefonować pod te numery i~potwierdzić je same, adresy placówek oraz godziny otwarcia. To ważne, gdyż informacje znalezione w~ten sposób są w~sporej części nieaktualne --- wprowadzono je dawno temu i~prawdopodobnie nigdy nie odświeżano.

Jako ostatnie należy zapisać sobie jeszcze numery telefonów służb ratowniczych działających na terenach, przez które ma przebiegać trasa obozu. Chodzi tutaj o~numery do GOPR, TOPR, WOPR --- aby wezwać natychmiast pomoc gdy zdarzy się wypadek. Wiadomo, że będą one w~każdym telefonie komórkowym należącym do kadry obozu.

\paragraph{$\bigstar$ Podsumowanie}
\begin{checklist}
\item Sporządzić listę telefonów i~adresów placówek opieki medycznej znajdujących się wzdłuż trasy wraz z~godzinami ich otwarcia.
\item Druga, podobna lista powinna zawierać namiary aptek.
\item Zapisać numery telefonów służb ratowniczych działających na terenach, przez które ma przebiegać trasa obozu.
\end{checklist}

\subsubsection{Kadra kwalifikowana [4.7]}
Kwestia zatrudniania koniecznego personelu na obozie niejednokrotnie przychodzi organizatorom ,,z bólem serca'' ze względów finansowych. Mało doświadczony komendant unika ,,zbędnych wydatków'' nie zatrudniając np. ratownika na spływ, czy przewodnika górskiego powyżej 1000 m. n.p.m., przekonany, że nic się nie stanie. I~choć w~większości wypadków zatrudniona osoba nie musi interweniować, to jednak kiedy sytuacja się skomplikuje, coś zawiedzie albo pójdzie nie tak jak powinno, zdarzy się nieszczęście, wtedy odpowiedzialna reakcja i~profesjonalne działanie fachowca okażą się nieocenione.

Tym, którzy myślą o~,,możliwości zaoszczędzenia'' powinno być lżej na sercu, gdy uświadomią sobie, że zatrudniona osoba nie musi być obecna na całym obozie, lecz tylko wtedy, kiedy jest to wymagane. Czyli np. przewodnik górski ---  w~dni, w~które obóz wędruje szlakami powyżej 1000 m. n.p.m., ratownik WOPR --- w~te dni kiedy obóz pływa itd. Takie podejście pozwoli zredukować koszty zatrudnienia przy jednoczesnej zgodności z~przepisami.

\paragraph{$\bigstar$ Podsumowanie}
\begin{checklist}
\item Ratownik na spływie czy przewodnik górski nie musi być obecny na całym obozie, lecz tylko wtedy, kiedy jest to wymagane.
\end{checklist}

\subsubsection{Pogoda [4.9]\label{pogoda}}
Wybór terminu obozu zależy od wielu czynników, nad którymi właściwie nie ma co się rozwodzić. Jeśli jest jakiekolwiek pole manewru i~można zdecydować kiedy odbędzie się obóz, warto przyjrzeć się historycznym danym pogodowym dla wybranego rejonu w~terminach, które ewentualnie wchodzą w~rachubę. Może to dać jakiś pogląd na warunki pogodowe zazwyczaj występujące w~wybranym terminie i~pomóc wybrać optymalnie, aby uniknąć ,,niespodzianek'' jak np. tydzień ulewnych deszczy, które mogą zupełnie nadwyrężyć program, atmosferę, a~także zdrowie uczestników. Opieranie decyzji na danych historycznych nie daje pewności, ale dla niektórych regionów pozwala z~dość dużym prawdopodobieństwem założyć, że najczęściej występujące warunki pogodowe powtórzą się ,,jak zazwyczaj''.
\begin{figure}[htp]
\centering
\includegraphics[scale=0.10803075396825396825396825396825]{P7230128.JPG}\\
~~~~~~~~\\
\includegraphics[scale=0.06345427435387673956262425447316]{DSC_0052.JPG}
\caption{Kiedy w~niektórych rejonach zazwyczaj pada, w~innych, o~tej porze roku, zazwyczaj świeci słońce.}\label{fig:deszcz-i-slonce}
\end{figure}
\\
\\
\small{
\emph{Warunki pogodowe należy sprawdzać w~serwisach, które zajmują się tym tematem na serio. Oto kilka przydatnych adresów:
\begin{description}
\item[Weather Underground] to portal pogodowy monitorujący warunki i~prognozy na całym świecie. Serwis zrzesza największą na świecie sieć osobistych stacji meteorologicznych (ponad 70 stacji w~Polsce i~ponad 19 tys. z~całego świata). Lista polskich stacji jest dostępna pod adresem: \href{http://www.wunderground.com/weatherstation/ListStations.asp?selectedCountry=Poland}{http://www.wunderground.com/ weath\-erstation/ListStations.asp?selectedCountry=Poland}. Dzięki nowej funkcji, dostępnej z~poziomu kontrolki kalendarza nad mapą, można sprawdzić pogodę dla konkretnego dnia z~przeszłości. Dane ze stacji meteorologicznych dostępne są od 2001~r., dane radarowe od 1994~r. Np. dane historyczne dla okolic Rzeszowa można znaleźć pod adresem \href{http://www.wunderground.com/history/airport/EPRZ/2002/7/1/WeeklyHistory.html}{http://www.wunderground.com/history/ air\-port/EPRZ/2002/7/1/WeeklyHistory.html}.
\item[OGIMET] \href{http://www.ogimet.com/index.phtml.en}{http://www.ogimet.com/index.phtml.en} to portal, w~którym można znaleźć m.in. historyczne dane pogodowe. Np. wypełniając formularz na stronie \href{http://www.ogimet.com/gsynres.phtml.en}{http://www.ogimet.com/gsynres.phtml.en} można otrzymać opi\-sy warunków pogodowych dla zadanej sekwencji dni pochodzące z~raportów przygotowywanych przez synoptyków.
\item[Weather Online] udostępnia historię pogody w~postaci wykresów. Na stronie \href{http://www.weatheronline.pl/weather/maps/forecastmaps?LANG=pl&CONT=plpl&R=150}{http:// www.weatheronline.pl/weather/maps/forecastmaps?LANG=pl\&CONT=plpl\& \\ R=150} można wybrać odpowiednią miejscowość lub region.
\item[Weather Spark] to graficznie rozbudowany serwis, który w~bardzo prosty sposób, poprzez nawigowanie osią czasu, łączy informowanie o~prognozie pogody z~prezentowaniem danych historycznych --- wystarczy tylko dla wybranego miejsca cofnąć oś czasu do wybranej daty np.: \href{http://weatherspark.com/\#!dashboard;q=Gdansk,+Pomorskie,+Poland}{http://weatherspark.com/\#!dashboard; q=Gdansk,+Pomorskie,+Poland}.
\end{description}}}

\paragraph{$\bigstar$ Podsumowanie}
\begin{checklist}
\item Przyjrzeć się historycznym danym pogodowym dla wybranego rejonu i~terminu obozu.
\item Warunki pogodowe sprawdzać w~serwisach, które zajmują się tym tematem profesjonalnie.
\end{checklist}

\subsubsection{Stan techniczny sprzętu [4.11]}
Jak wiadomo stan techniczny sprzętu posiadanego przez drużyny bywa różny i~zależy od zaangażowania i~zdolności osoby, która się nim opiekuje. O~ile o~własny sprzęt drużyna może (a~wręcz powinna) zatroszczyć się sama, o~tyle wpływ na stan sprzętu ,,obcego'' jest poza jej zasięgiem. Dlatego decydując się na wykorzystanie wypożyczonego sprzętu, należy zwrócić szczególną uwagę na zawarcie w~umowie dokładnego opisu sprzętu, o~który chodzi. Wielokrotnie zdarzało się np. przy wypożyczaniu kajaków na spływ, że w~umowie był tylko zapis dotyczący ilości kajaków i~nic ponadto. W~momencie dostarczenia kajaków na miejsce okazywało się, że kajaki są nie takie jak kadra obozu oczekiwała, a~ich stan był opłakany. W~ten sposób nieuczciwe wypożyczalnie żerują na osobach, które postawione przed faktem dokonanym przeważnie nawet nie próbują protestować, tylko potulnie zgadzają się na takie traktowanie. Dlatego przed podpisaniem umowy należy zażądać umożliwienia obejrzenia sprzętu lub chociaż jego zdjęć. Zdjęcia te i~opis sprzętu zawierający to na czym kadrze zależy powinny znaleźć się w~umowie. Czyli nie powinno się podpisywać umowy na ,,wypożyczenie kajaków'', tylko na ,,wypożyczenie kajaków typu X, wyposażonych w~Y i~Z itd.'' Podobnie sprawa wygląda w~przypadku innego wypożyczanego sprzętu. W~celu zabezpieczenia się przed nieuzasadnionymi roszczeniami dotyczącymi uszkodzeń lub zniszczenia wypożyczonych rzeczy w~momencie odbioru warto sporządzić protokół zdawczo-odbiorczy uzupełniony dokumentacją zdjęciową.
\\
\\
\small{
\emph{Podczas przygotowań do spływu kajakowego po Łupakowie kadra miała trudności ze znalezieniem kajaków, ponieważ ich właściciele obawiali się, że zostaną zniszczone --- rzeka ma charakter górski, jest płytka i~usiana kamieniami. W~końcu udało się wypożyczyć kajaki starego typu, w~średnim stanie, z~połatanym poszyciem. Po kilku dniach okazało się, że łaty zostały nałożone w~niewłaściwy sposób --- zaczęły odpadać, a~kajaki zaczęły nabierać wody. Łatano je na własną rękę czym się dało. Po zakończeniu spływu właściciel stwierdził, że ,,kajaki są podziurawione'' i~zażądał dodatkowej opłaty na poczet naprawy.}}
\\
\\
Zupełnie inaczej sytuacja wygląda, gdy drużyna decyduje się na wykorzystanie sprzętu należącego do uczestników obozu.
\\
\\
\small{
\emph{Świetnym przykładem są rowery, które drużyna planuje wykorzystać na obozie rowerowym. Każdy z~uczestników będzie zapewniał, że jego rower jest sprawny i~na chodzie, bo używa go ,,codziennie''. Na rodzicach uczestników też w~większości nie można polegać. A~to dlatego, że ani jedni ani drudzy niejednokrotnie nie mają pojęcia jaka jest różnica między jazdą na rowerze dla zabawy przez krótki czas co kilka dni, a~całodziennym wykorzystaniem roweru w~terenie, obciążonego dodatkowo sakwami, namiotem itp. Rowery uczestników w~większości nie zostaną odpowiednio przygotowane do obozu rowerowego, nawet mimo wcześniejszych próśb --- przyczyna jest trywialna --- ludzie nie wiedzą jak to zrobić albo nie chce im się, albo jedno i~drugie. W~takiej sytuacji przyczyną problemów kolejny raz okazują się być błędne założenia i~naiwna wiara kadry, że można polegać na zapewnieniach osób, które właściwie\ldots nie znają się na rzeczy. A~to oznacza pasmo usterek na trasie obozu, niemal codziennie --- należy sobie to uświadomić i~być na to przygotowanym, żeby nie tracić czasu na długotrwałe wyprawy do sklepów czy serwisów rowerowych w~większych miastach.}}
\\
\\
Zaniedbania ze strony kadry oraz błędne założenia w~kwestii sprzętu są najczęstszym źródłem poważnych problemów na obozach, których powodzenie w~głównej mierze zależy od wykorzystania tego sprzętu właśnie. Problemy te skutecznie psują atmosferę, harmonogram, plany, rozkład dnia itd. Warto przyłożyć się poważnie do sprawy i~nie zlekceważyć tego aspektu, w~przeciwnym razie odbije się to bardzo niekorzystnie na realizacji obozu.

Na pocieszenie warto zauważyć, że nieuczciwe firmy wypożyczające zdezelowany sprzęt konsekwentnie są eliminowane z~rynku i~coraz częściej okazuje się, że sprzęt jest bardzo dobrej jakości.

Każdy sprzęt  (własny, wypożyczony) należy zabezpieczyć przed kradzieżą, zwłaszcza w~nocy. Rowery można spiąć zapinkami, kajaki powiązać łańcuchem itp. Taka ochrona nie daje żadnej gwarancji, zawsze jednak może opóźnić działanie złodziei lub ich zniechęcić. Wystawienie nocnej warty dodatkowo zmniejszy ryzyko. Należy także koniecznie ubezpieczyć ten sprzęt nie tylko od kradzieży, ale także zniszczenia.
\\
\\
\small{
\emph{Na pewnym obozie skradziono kilka rowerów podczas powrotu z~obozu. W~trakcie załadunku sprzętu była ulewa --- nawet wtedy znaleźli się ,,dobrzy ludzie'', którzy ,,przyszli z~pomocą''. ,,Okazja czyni złodzieja'', szczególnie gdy dotyczy ufnych i~nierzadko łatwowiernych harcerzy\ldots}}

\paragraph{$\bigstar$ Podsumowanie}
\begin{checklist}
\item Decydując się na wykorzystanie wypożyczonego sprzętu, zawrzeć w~umowie jego dokładny opis.
\item Przed podpisaniem umowy zażądać umożliwienia obejrzenia sprzętu lub chociaż jego zdjęć.
\item Sporządzić protokół zdawczo-odbiorczy uzupełniony dokumentacją zdjęciową.
\item Problemy ze sprzętem skutecznie psują atmosferę, harmonogram, plany, rozkład dnia itd.
\item Wystawienie nocnej warty zmniejszy ryzyko kradzieży sprzętu.
\end{checklist}

\subsection{Dzień obozowy [5]}
\subsubsection{Wędrowanie [5.6]}
Jak wiadomo głównym punktem rozkładu dnia na obozach wędrownych jest\ldots węd\-rowanie, na spływach --- pływanie kajakiem, na obozach rowerowych --- jazda na rowerze itd. I~bardzo dobrze, bo tak właśnie ma być. Jednak to nie znaczy wcale, że nie powinny zostać wprowadzone inne aktywności w~ramach urozmaicenia np. przerwy na jakieś interesujące zajęcia, krótką wizytę w~ciekawym miejscu albo zabawę itp. Takie elementy dodatkowe nie tylko służą wytchnieniu od wysiłku, ale są fantastycznym narzędziem zapoznającym uczestników z~tym co ciekawego znajduje się ,,na trasie'' --- a~znajduje się na pewno, tylko kadra musi chcieć, żeby to odkryć i~zaplanować, najlepiej jeszcze przed obozem. Na podobnej zasadzie przerw można przeprowadzać gry, czy krótkie zawody.
\begin{figure}[htp]
\centering
\includegraphics[scale=0.19197341513292433537832310838446]{DSC03902.JPG}
\caption{Krótka przerwa na huśtawce przy przydrożnym sklepie.}\label{fig:hustawka}
\end{figure}

\paragraph{$\bigstar$ Podsumowanie}
\begin{checklist}
\item Dzień obozowy to nie tylko same wędrówki.
\item Wprowadzić urozmaicenia w~postaci przerw na interesujące zajęcia, krótkie wizyty w~ciekawych miejscach albo zabawę itp.
\end{checklist}

\subsection{Wędrówki [6]}
\subsubsection{Podział na grupy [6.1]}
Z~podziałem na grupy podczas wędrówek bywa różnie. Bardzo rzadko zdarza się, że cały obóz wędruje razem, w~jednej grupie, tzn. w~szyku. W~przeważającej liczbie przypadków obóz jest podzielony na grupy. Doświadczona kadra jest tego świadoma i~dokonuje takiego podziału zaraz na początku obozu. W~przeciwnym wypadku podział nastąpi samorzutnie. Kryteria podziału bywają różne. Naturalny i~najczęstszy podział to ,,czołówka'', ,,peleton'' i~,,maruderzy'' (na spływach zwani ,,poławiaczami pereł''). Inne to: według zastępów, drużyn, zadań itp.

Bez względu na kryterium podziału, w~każdej grupie powinna znajdować się właściwa ilość opiekunów czy instruktorów określona przepisami. Należy pamiętać o~tym przy kompletowaniu kadry na obóz. Jeśli obóz będzie podzielony na grupy, to każda grupa musi posiadać wymaganą przepisami liczbę opiekunów. W~związku z~tym przeważnie trzeba znaleźć ich więcej niżby wynikało to z~prostego podziału całkowitej liczby uczestników obozu.

Grupy muszą być odpowiednio wyposażone, np. w~apteczkę, powinny też mieć możliwość komunikowania się ze sobą. Dodatkowo powinny się spotykać na szlaku co jakiś czas. W~każdej grupie musi być doświadczona osoba prowadząca, umiejąca czytać mapę i~potrafiąca poruszać się po szlaku. Wszyscy powinni wiedzieć którymi szlakami w~danym dniu wędrować i~dokąd należy dotrzeć.
\\
\\
\small{
\emph{\label{akcja-kroscienko-szczawnica}Któregoś dnia na obozie, w~czasach kiedy nie było jeszcze telefonów komórkowych, zaplanowano przejście górami ze Szczawnicy do Krościenka nad Dunajcem. Wszyscy zostali poinformowani, że najpierw należy wejść na Pasmo Radziejowej szlakiem niebieskim do zielonego, następnie zielonym do czerwonego, i~czerwonym aż do celu. Nastąpił naturalny podział na 3~grupy. Czołówka nieznacznie wyprzedzała peleton tzn. była cały czas w~zasięgu wzroku. Ostatnia grupa została daleko w~tyle. Spotkanie grup zostało wyznaczone na skrzyżowaniu szlaku zielonego z~czerwonym. Peleton dogonił czołówkę właśnie w~tym miejscu i~wspólnie czekali tam na ostatnią grupę. Po dwóch godzinach oczekiwania zdecydowano się na dalszą wędrówkę, bo ,,trzecia grupa na pewno zaraz przyjdzie''. Po dojściu do Krościenka obie grupy zameldowały się na kwaterze. O~godzinie 18 stwierdzono, że trzecia grupa powinna już dawno być na miejscu. Kadra postawiła wartę na moście nad Dunajcem, ponieważ przebiegał po nim czerwony szlak --- na wypadek gdyby zaginiona grupa tam dotarła. Część osób czekała na kwaterze. Kadra poprosiła o~pomoc GOPR, ale odesłano ich na policję. Tam kadra próbowała złożyć zawiadomienie o~zaginięciu, co okazało się niemożliwe, bo ,,minęło zbyt mało czasu od momentu zaginięcia''. Policyjnym radiowozem komendant został zawieziony do Szczawnicy na miejsce poprzedniego kwaterowania, aby sprawdzić czy przypadkiem grupa nie zawróciła z~drogi z~jakiejś przyczyny. Po bezskutecznym godzinnym krążeniu po Szczawnicy radiowóz wrócił do Krościenka. Około godziny 21 warta na moście zauważyła zaginioną grupę podążającą czerwonym szlakiem śladem poprzednich grup. Na szczęście wszyscy byli zdrowi i~cali, tylko bardzo zmęczeni i~wystraszeni, że mogli nie zdążyć przed nocą. Okazało się, że grupa dotarła niebieskim szlakiem do zielonego, tak jak powinna. Z~tego miejsca ruszyła szlakiem niebieskim, ale, niestety, w~stronę przeciwną niż należało. Grupa zorientowała się, że popełniła błąd dopiero gdy dotarła niebieskim szlakiem z~powrotem do Szczawnicy zamiast na Prehybę. Członkowie grupy zdecydowali, że zawrócą i~wejdą niebieskim szlakiem na pasmo, a~po dotarciu tam kontynuowali wędrówkę czerwonym szlakiem w~stronę Krościenka. Analiza tej historii pokazała, że:
\begin{itemize}
\item Feralne skrzyżowanie zielonego i~niebieskiego szlaku leży na wypłaszczonych fragmentach ścieżek.
\item Na skrzyżowaniu nie było oznakowania informującego dokąd dany szlak prowadzi w~każdą stronę. To wprowadziło grupę w~błąd.
\item Czołówka i~peleton nie powinny były kontynuować wędrówki z~umówionego miejsca spotkania na skrzyżowaniu niebieskiego i~czerwonego szlaku przed przybyciem ostatniej grupy. Powinny były cofnąć się do skrzyżowania szlaku niebieskiego z~zielonym i~stamtąd wysłać dwa patrole na poszukiwania. Jeden patrol powinien był zejść do Szczawnicy szlakiem niebieskim, a~następnie wrócić na wspomniane skrzyżowanie szlakiem zielonym. Drugi patrol odwrotnie --- powinien był ruszyć do Szczawnicy szlakiem zielonym i~wrócić stamtąd szlakiem niebieskim. W~ten sposób jeden z~patroli natknąłby się na zagubioną grupę i~sprowadził na właściwe miejsce. Drugi patrol wróciłby z~niczym albo natknął się na patrol pierwszy.
\item Czołówka i~peleton błędnie założyły, że ,,trzecia grupa na pewno zaraz przyjdzie''.
\item Kadra powinna była wyznaczyć miejsca spotkań nie rzadziej niż na każdym skrzyżowaniu szlaków.
\item Po drodze było jeszcze jedno skrzyżowanie szlaku, czerwonego z~żółtym, na którym nie poczekano na ostatnią grupę.
\item Kadra postępowała prawidłowo po uświadomieniu sobie, że trzeba zacząć poszukiwania.
\item Pozostawienie kilku osób na kwaterze było sensowne, ponieważ członkowie zaginionej grupy wiedzieli dokąd powinni dotrzeć --- znali cel wędrówki.
\item Ustawienie warty na moście w~Krościenku było bardzo dobrym posunięciem, ponieważ zaginieni harcerze wiedzieli, że powinni wędrować czerwonym szlakiem i~zakładając, że będą się tego trzymać, musieli przejść przez ten most, ponieważ nie ma tam innego przejścia przez Dunajec.
\item Zwrócenie się o~pomoc do GOPR było prawidłowym odruchem.
\item Poproszenie o~pomoc policji i~sprawdzenie poprzedniej kwatery było bardzo dobrą reakcją.
\end{itemize}
\begin{figure}[htp]
\centering
\includegraphics[scale=0.48141025641025641025641025641026]{szlaki-szczawnica.jpg}
\caption[Cantin for LOP]{Szlaki w~rejonie Szczawnicy i~Krościenka.\footnotemark}\label{fig:szlaki}
\end{figure}
\footnotetext{Fragment mapy pochodzi z~serwisu \protect\href{http://www.szczawnica.na-pulpit.pl}{http://www.szczawnica.na-pulpit.pl}}
Kadra tego obozu popełniła karygodny błąd planując niewłaściwie miejsca spotkań i~dopuszczając do konturowania wędrówki bez zaczekania na wszystkich uczestników, łamiąc w~ten sposób reguły bezpieczeństwa, które sama ustaliła. Po uświadomieniu sobie problemu podjęte czynności związane z~poszukiwaniami były jednak prawidłowe. Członkowie zaginionej grupy trzymali się zadanej trasy i~poruszali się po szlakach wyznaczonych przez kadrę, popełnili błąd źle orientując się na skrzyżowaniu szlaku niebieskiego z~zielonym. Po spostrzeżeniu swojego błędu grupa zareagowała prawidłowo, tj. cofnęła się ,,po własnych śladach'' do feralnego skrzyżowania i~kontynuowała wędrówkę już bez żadnych wpadek.}}
\\
\\
%\newpage
\indent Podział naturalny wynika oczywiście z~tempa i~wydolności poszczególnych osób, ale należy bardzo z~tym uważać. Jest to często źródłem konfliktów, drwin i~nieprzyjemnych sytuacji wśród uczestników. Czołówka znacznie wyprzedza peleton, wręcz ucieka mu i~jednocześnie narzeka na całą resztę, że są zbyt wolni --- poziom zarozumiałości i~pogardy rośnie tutaj w~czasie. Peleton to największa grupa, w~której ludzie wędrują swoim tempem, niektórym zależy na dogonieniu czołówki, niektórym niekoniecznie, ale są dostatecznie zmotykować aby po prostu wspólnie wędrować. Dominujące uczucia w~peletonie to niedosyt i~coraz większa złość na czołówkę. Maruderzy to grupa, która wlecze się na końcu. Składa się ona z~osób, które pojechały na obóz przygotowane nieodpowiednio, czy to kondycyjnie czy sprzętowo, a~także z~osób, które nadwyrężyły w~jakiś sposób swoje siły. Nastroje w~tej grupie są złe, nie ma radości z~wędrowania, wręcz staje się ono udręką. Na skutek docinków osób z~czołówki i~peletonu samoocena osób z~tej grupy jest bardzo niska i~codziennie się zmniejsza. Dlatego bardzo ważne jest trafne wybranie instruktora, który będzie wędrował z~najsłabszymi aby potrafił ich odpowiednio motywować, podtrzymywać na duchu i~rozmawiać w~sposób, który pozwoli im zapomnieć o~tym w~jakiej grupie są.

Z podziałem naturalnym nie warto walczyć, tylko należy wykorzystać go we właściwy sposób. Np. czołówce można wymyślić cały zestaw czynności opóźniających, które będą musiały być wykonane po drodze, albo można nawet wyznaczyć im dłuższą trasę przebiegającą wzdłuż trasy peletonu i~przecinającą się z~nią kilka razy w~danym dniu. W~miejscach tych przecięć tras czy szlaków należy zaplanować spotkania wszystkich grup --- zdarzy się wtedy na pewno kilka razy sytuacja, że to peleton i~maruderzy będą czekali na czołówkę --- morale wśród maruderów i~peletonu wzrośnie w~takiej sytuacji. Członkowie czołówki za to będą zadowoleni z~przebycia dłuższej trasy i~zobaczenia czy zwiedzenia ciekawego miejsca. Peletonowi można robić dłuższe przerwy na jakieś gry, które również mają na celu opóźnienie --- wszystko to po to, aby maruderzy raz za razem dołączali do peletonu i~po krótkim odpoczynku wyruszali na jego czele w~dalszą drogę. Skład tych grup nie jest stały i~może się codziennie zmieniać, podobnie jak powinno się zmieniać instruktorów czy opiekunów każdej z~tych grup, zwłaszcza w~celu umożliwienia odpoczynku psychicznie wypalonemu opiekunowi grupy maruderów i~fizycznie wycieńczonemu opiekunowi czołówki.

Istnieją różne odmiany podziału naturalnego, np. niekiedy występuje jeszcze czwarta grupa ,,pościgowa'' --- są to ambitne osoby z~peletonu, które starają się dogonić czołówkę, a~jednocześnie tempo peletonu jest dla nich zbyt wolne. Nie potrzebują one żadnej dodatkowej motywacji, gdyż nakręca ich ambicja. To właśnie na nich trzeba bardzo uważać, ponieważ nieprzerwana pogoń za czołówką bardzo szybko ich wycieńcza, zbyt wysokie tempo powoduje, że stają się nieostrożni i~podatni na kontuzje, w~konsekwencji właśnie oni regularnie zasilają grupę maruderów.

\paragraph{$\bigstar$ Podsumowanie}
\begin{checklist}
\item Uczestnicy samorzutnie podzielą się na kilka grup wędrujących w~różnym tempie.
\item Podział często bywa źródłem konfliktów.
\item Grupom o~zbyt wysokim tempie zlecać dodatkowe zadania i~dłuższe warianty tras.
\item W~każdej grupie powinna być wymagana przepisami liczba opiekunów.
\item Grupy muszą być odpowiednio wyposażone.
\item Grupy powinny spotykać się na szlaku co jakiś czas.
\item W~każdej grupie musi być doświadczona osoba prowadząca.
\item Wszyscy powinni znać cel i~trasę zaplanowane na dany dzień.
\end{checklist}

\subsubsection{Długość wędrówek [6.5] \label{dlugosc-wedrowek}}
Wędrowanie to fantastyczna sprawa, dopóki nie trwa zbyt długo. W~ciągu dnia obozowego musi się znaleźć wystarczająca ilość czasu na inne rzeczy. Czas poświęcony na wędrowanie zależy wprost od odległości i~odwrotnie od tempa wędrowania (jazdy rowerem czy płynięcia kajakiem). Należy mieć to na uwadze planując dzienne trasy obozowych wędrówek. Wpływ na czas mają także czynniki, które czasem bardzo ciężko przewidzieć, jednak doświadczona kadra powinna być ich świadoma.
\begin{figure}[htp]
\centering
\includegraphics[scale=0.10803075396825396825396825396825]{P7210065.JPG}
\caption{Szlak po deszczu.}\label{fig:szlakpodeszczu}
\end{figure}
Np. jeśli wędrówka przypada w~czasie kiedy pada deszcz albo tuż po deszczu to szlak może być zabłocony i~śliski, a~małe strumyki, które normalnie można przeskoczyć będą prowadziły więcej wody i~trzeba się będzie przez nie przeprawiać --- te i~tym podobne rzeczy będą wydłużały czas wędrówek, mimo, że odległość się nie zmieni.
\begin{figure}[htp]
\centering
\includegraphics[scale=0.10813803376365441906653426017875]{P7210087.JPG}\\
~~~~~~~~\\
\includegraphics[scale=0.10813803376365441906653426017875]{P7210090.JPG}
\caption{Strumyki ,,nie do przeskoczenia''.}\label{fig:przeprawa}
\end{figure}

PTTK stworzyło system odznak\footnote{Więcej o~tym w~sekcji \ref{odznaki-pttk} na stronie \pageref{odznaki-pttk}.} za wędrówki piesze, górskie, rowerowe, kajakowe. Każda przebyta trasa podczas wędrówek może zostać zaliczona na poczet jakiejś odznaki. Ilość tych punktów wynika z~regulaminów poszczególnych odznak i~jest obwarowana pewnymi limitami wiekowymi. Starsze wersje tych regulaminów podawały też maksymalne limity dzienne ilości punktów możliwych do zdobycia dla osób w~danym wieku. Warto znaleźć te limity, bo można z~nich bardzo łatwo wyprowadzić odległość jakiej nie powinny przekraczać dziennie osoby w~danym wieku. Zbyt duże odległości pokonywane w~zbyt długim czasie prowadzą wprost do zmęczenia, a~później do wycieńczenia i~zniechęcenia uczestników obozu.

Ustalając trasę i~dzienne etapy trzeba mieć na względzie możliwości drużyny, która tę trasę ma pokonać. Możliwości te wynikają bezpośrednio z~dyspozycji poszczególnych osób. Jeśli są one słabe, za młode i~niedoświadczone w~wybranej formie wędrówek, to prawdopodobnie okaże się, że to właśnie potencjał tych osób jest wyznacznikiem możliwości drużyny. Planując trasę należy zaplanować jej podstawowy wariant dla najsłabszych i~kilka rozszerzonych wariantów dla osób o~większych zdolnościach.

\paragraph{$\bigstar$ Podsumowanie}
\begin{checklist}
\item Czas trwania wędrówek zależy od długości trasy i~tempa wędrowania.
\item Wpływ na czas wędrówek mają czynniki, które trudno przewidzieć, zwłaszcza nagła zmiana pogody.
\item W~ciągu dnia musi się znaleźć wystarczająca ilość czasu na inne rzeczy.
\item Regulaminy odznak turystycznych PTTK definiują dzienne limity punktów za wędrówki dla osób w~różnym wieku.
\item Możliwości drużyny i~poszczególnych osób zależą od dyspozycji i~doświadczenia w~danej formie wędrówek.
\item Podstawowy wariant trasy planować pod kątem najsłabszych uczestników.
\end{checklist}

\subsubsection{Lista ekwipunku [6.8]}
Aby uniknąć zabierania przez uczestników na obóz niepotrzebnych rzeczy, które przecież swoje ważą i~będą zajmowały miejsce w~plecakach czy sakwach, a~także w~celu zapewnienia, że każdy uczestnik będzie wyposażony w~rzeczy niezbędne, potrzebne na obozie, należy jak najwcześniej przygotować pełną listę szczegółowo wymieniającą co wchodzi w~skład ekwipunku uczestnika obozu. Listę trzeba rozdystrybuować do rodziców uczestników obozu, aby byli świadomi jakie wydatki ich czekają i~mogli je zaplanować w~swoim rodzinnym budżecie.

\begin{figure}[htp]
\centering
\includegraphics[scale=0.68200633579725448785638859556495]{jednolite-sandaly.jpg}
\caption{Efekt skoordynowanych zakupów --- jednolite sandały.}\label{fig:jednolite-sandaly}
\end{figure}
Nie wszyscy rodzice znają się na sprzęcie i~nie potrafią wybrać odpowiedniego. Część rodziców, których na to stać, wybierze i~kupi najdroższe rzeczy zgodnie z~zasadą ,,najdroższe = najlepsze''. Spora część kupi najtańsze wychodząc z~założenia, że dziecko i~tak prędzej wyrośnie z~danej rzeczy niż rzecz ta się zniszczy. Bardzo niewielka część kupi rzeczy ze średniej półki, sensownej jakości i~w przystępnej cenie. Wobec powyższego warto skoordynować te zakupy. Polega to na wspólnej wizycie kadry i~uczestników z~rodzicami w~jednym czy kilku sklepach ze sprzętem czy odzieżą i~dokonaniem tam zakupów pod okiem kadry. Kadra wcześniej negocjuje zniżki i~asortyment z~personelem sklepu. W~ten sposób rodzice kupią sensowny sprzęt po dobrej cenie, a~kadra będzie miała podczas obozu mniej kłopotów związanych z~usterkami, czy w~ogóle ze sprzętem nie nadającym się do użytku. Pozostawiając kwestię wyposażenia uczestników rodzicom kadra nie kontroluje ryzyka związanego z~niezrozumieniem swoich intencji przez rodziców.
\\
\\
\small{
\emph{Na liście sprzętu na pewien obóz wędrowny w~Bieszczadach wpisane były „mocne skórzane buty z~cholewką nad kostkę”. Jakież było zdziwienie kadry połączone z~niedowierzaniem i~wręcz szokiem kiedy jedna z~uczestniczek wyjęła z~plecaka\ldots zimowe kozaki. Rozpadły się one po 2 dniach wędrowania i~konieczna była wycieczka z~Cisnej do Leska po nowe buty, które wytrzymały do końca wspomnianego obozu, cały następny rok i~kolejny obóz.}}

\paragraph{$\bigstar$ Podsumowanie}
\begin{checklist}
\item Przygotować listę ekwipunku aby każdy uczestnik miał ze sobą odpowiedni sprzęt.
\item Listę rozdać rodzicom uczestników kilka miesięcy przed obozem.
\item Kontrolować zakupy --- nie wszyscy rodzice znają się na sprzęcie.
\end{checklist}

\subsubsection{Ekwipunek uczestnika [6.9]}
Sporządzając listę ekwipunku uczestnika należy wydzielić najważniejsze rzeczy, które są wymagane i~konieczne na planowanym obozie. Będą to np. plecak (sakwa, worek kajakowy), buty, kurtka, śpiwór. Rzeczy te powinny być porządne (tzn. odpowiedniej jakości), odpowiednio dobrane i~dopasowane do uczestnika. To kadra jest zobowiązana aby tego dopilnować. Uczestnicy powinni być wyekwipowani w~te rzeczy wystarczająco wcześnie przed obozem, aby podczas przygotowań mogły one zostać przetestowane i~sprawdzone w~warunkach terenowych. Sprawa ta jest wyjątkowo często zaniedbywana przez kadrę skutkiem czego są problemy takie jak: za duże plecaki (nie chodzi o~pojemność plecaka tylko o~to na jaki wzrost jest plecak), niewygodne buty, przemakające kurtki, nie działające i~ciężkie latarki, zbyt grube lub zbyt cienkie śpiwory itd.

Oprócz wymienionego powyżej ekwipunku uczestnicy zabierają na obóz mnóstwo niepotrzebnych rzeczy, które rodzice niezmiennie wciskają im do plecaków, bo ,,mogą się przydać''. Ale bywa odwrotnie, mimo dostarczonej listy uczestnicy zapominają o~spakowaniu niektórych rzeczy. Dlatego warto jest zorganizować akcję sprawdzania i~ważenia plecaków. Polega ona na tym, że uczestnicy przynoszą plecaki spakowane na obóz do harcówki dzień przed wyjazdem. Tam kadra sprawdza czy plecak zawiera to co trzeba, braki wypisywane są na kartce i~następnego dnia muszą zostać doniesione, natomiast rzeczy nadmiarowe zostają w~harcówce (lub wracają od razu do domu). Taką zbiórkę należy przeprowadzić rano, aby uczestnicy mieli czas na skompletowanie (zakupienie) brakujących rzeczy. Jeśli spotkanie będzie wieczorem istnieje zagrożenie, że komuś nie uda się tego zrobić. Dodatkowo plecaki są ważone --- waga plecaka nie powinna przekraczać 1/4 wagi osoby, która będzie go nosić. W~dniu wyjazdu plecaki są zabierane z~harcówki i~jadą prosto na obóz. W~trakcie obozu może okazać się, że plecaki są za ciężkie, albo jakieś rzeczy nie są jednak potrzebne. W~takiej sytuacji, zamiast nosić je ze sobą, należy po prostu włożyć je np. do worków, podpisać, zapakować do kartonu (można kupić na poczcie) i~wysłać w~jednej paczce na adres np. drużynowego. Taki manewr jest zwykle stosowany, kiedy rodzice uczestników nie są w~stanie zaakceptować skutków akcji sprawdzania zawartości plecaków --- oczywiście wiążą się z~tym pewne koszta, które obóz musi ponieść --- na szczęście są one niewielkie.

\paragraph{$\bigstar$ Podsumowanie}
\begin{checklist}
\item Na liście ekwipunku zaznaczyć najważniejsze rzeczy.
\item Dopilnować jakości i~dopasowania niezbędnego ekwipunku.
\item Podstawowy sprzęt kilkukrotnie przetestować w~warunkach terenowych.
\item Sprawdzić zawartość plecaków dzień przed wyjazdem.
\item Niepotrzebne rzeczy zostawić w~harcówce lub odesłać jedną paczką pierwszego dnia obozu.
\end{checklist}

\subsubsection{Plecaki [6.10]}
\begin{figure}[htp]
\centering
\includegraphics[scale=0.6]{plecak-pakowanie.jpg}
\caption[Caxton for LOK]{Pakowanie: Najcięższe rzeczy powinny znajdować się jak najbliżej pleców i~powinny być spakowane maksymalnie do poziomu ramion. Śpiwór, ubrania i~inne lekkie rzeczy pakujemy do dolnej i~środkowej komory --- w~komorze środkowej układamy je w~przedniej części korpusu plecaka. Rzeczy drobne, wymagające szybkiego dostępu, umieszczamy na górze i~w~klapie plecaka. W~ten sposób środek ciężkości plecaka znajduje się blisko ciała.\footnotemark}
\label{fig:pakowanie}
\end{figure}
Sztuka prawidłowego pakowania plecaka czy sakwy stała się niestety wiedzą tajemną i~ze świecą szukać można drużyny, której ktoś pokazał jak to prawidłowo robić. Wybierając się na obóz wędrowny należy posiąść tajniki tej sztuki i~zapoznać z~nią uczestników obozu, aby sami byli w~tym bardzo dobrze wyszkoleni, wręcz robili to automatycznie, z~zamkniętymi oczami. Inwestując czas w~nauczenie tego przed obozem zyskamy oszczędność czasu i~zmniejszymy ryzyko kontuzji na obozie.

Idealnie byłoby, gdyby każdy uczestnik dysponował podczas nauki tym samym plecakiem, sakwą czy workiem kajakowym, z~którym pojedzie na obóz. Dzięki temu będzie miał wystarczającą ilość czasu na zapoznanie się z~nim i~odkrycie do czego służą różne troki, zapinki, klamry itp. I~tutaj dochodzimy do meritum sprawy. O~ile o~sakwie czy worku kajakowym należy pamiętać, żeby po prostu nie były za duże, to plecak na obóz wędrowny musi być dostosowany do tego, kto będzie go dźwigać. I~nie chodzi tu tylko o~regulację troków czy szelek, ale przede wszystkim o~dobranie plecaka do wzrostu osoby, która będzie z~nim wędrować. Nie każdego stać na nowy, porządny, dopasowany plecak. Zdarza się, że plecaki są pożyczane od starszego rodzeństwa lub innych osób. Okazać się wtedy może, że plecaka nijak nie da się wyregulować, bo po prostu zakres regulacji jest za mały --- szczególnie widoczne jest to na szelkach, pasie biodrowym i~czasami na pasie piersiowym.
\\
\\
\footnotetext{Obrazek i~opis pochodzą z~magazynu \protect\href{4outdoor.pl}{4outdoor.pl} nr~4.}
% obrazki i opisy pochodzą z magazynu 4outdoor.pl nr 4
\begin{figure}[htp]
\centering
\includegraphics[scale=0.6]{plecak-ukladanie01.jpg}~\includegraphics[scale=0.6]{plecak-ukladanie02.jpg}

\includegraphics[scale=0.6]{plecak-ukladanie03.jpg}~~~~~~~~\includegraphics[scale=0.6]{plecak-ukladanie04.jpg}
\caption[Capin for OF]{Najpierw należy poluzować wszystkie taśmy. Potem założyć plecak, ułożyć pas biodrowy i~zaciągnąć go. Środek pasa biodrowego powinien znajdować się na kościach biodrowych, ale nie powyżej, gdyż będzie wtedy uciskał żołądek. Następnie należy zaciągnąć szelki, ale nie za mocno, ponieważ ciężar plecaka powinien być w~jak największej części przenoszony przez pas biodrowy. W~następnej kolejności należy sprawdzić ułożenie szelek. Ich punkt zamocowania powinien znajdować się pomiędzy łopatkami, a~same szelki powinny łagodnie rozchodzić się na ramiona. Jeżeli ułożenie szelek jest nieprawidłowe, być może konieczna będzie dodatkowa regulacja plecaka (niektóre modele umożliwiają dopasowanie systemu nośnego do wzrostu) lub nawet zmiana modelu plecaka na właściwy do wzrostu. Kolejna czynność to dociągnięcie stabilizujących pasków przy pasie biodrowym. Następna to wyregulowanie stabilizujących pasków ramiennych --- przy ich pomocy ładunek jest przyciągany bliżej środka ciężkości.\footnotemark}
\label{fig:dopasowywanie1}
\end{figure}
\footnotetext{Obrazki i~opisy pochodzą z~magazynu \protect\href{4outdoor.pl}{4outdoor.pl} nr~4.}
\begin{figure}[htp]
\centering
\includegraphics[scale=0.6]{plecak-ukladanie05.jpg}
\caption[Caption for LOF]{Na końcu należy zaciągnąć pas piersiowy, by poprawić ułożenie szelek.\footnotemark}
\label{fig:dopasowywanie5}
\end{figure}
\small{
\emph{Kadra musi wiedzieć jak skonstruowany jest plecak i~jak powinien być prawidłowo wyregulowany aby mogła dopilnować i~nauczyć tego samego uczestników. Najważniejsze to zapamiętać, że:
\begin{itemize}
\item plecak musi być dostosowany do wzrostu osoby (szelki muszą mieć odpowiedni zakres regulacji, aby plecak nie wisiał poniżej bioder),
\item plecak nosimy na plecach, ale jego ciężar jest przenoszony przez pas biodrowy, a~nie przez szelki. Jednym słowem plecak musi być podniesiony odpowiednio wysoko, aby pas biodrowy był na właściwej wysokości i zapięty na tyle ciasno, by sam był w~stanie utrzymać plecak, aby nie wisiał on na szelkach – tzn. szelki nie mogą uciskać barków od góry,
\item po ustawieniu pasa biodrowego regulujemy szelki, aby przytrzymywały plecak przy plecach, tzn, żeby nie odchylał się on do tyłu. Górne troki szelek powinny odciągać szelki w~górę, tak aby nie dotykały barków od góry --- powinna być tam przestrzeń na 1 -- 2 palce,
\item pas piersiowy zapobiega rozsuwaniu się szelek na boki i~powinien być również napięty, ale niezbyt ciasno, żeby można było swobodnie oddychać. Większość plecaków kupowanych przez uczestników to plecaki męskie i~mają pas piersiowy na wysokości nieodpowiedniej dla harcerek --- mało kto zwraca na to uwagę, skutkiem czego harcerki nie korzystają z~tego pasa, bo jest im po prostu niewygodnie --- w~takiej sytuacji należy ten pas zdjąć i~zamontować około 10cm wyżej --- wtedy da się z~niego korzystać. W~damskich wersjach plecaków ten pas jest oryginalnie mocowany we właściwym miejscu.
\end{itemize}}}

Kadra ma obowiązek sprawdzić plecaki uczestników przed wyjazdem i~w~razie niemożności wyregulowania doprowadzić do zmiany plecaka na odpowiedni. W~ten sposób eliminuje się główne źródło problemów --- marudzenie na ,,niewygodny plecak'' i~otarcia od plecaka.

\paragraph{$\bigstar$ Podsumowanie}
\begin{checklist}
\item Przed obozem nauczyć uczestników prawidłowego pakowania plecaka.
\item Do nauki używać tych samych plecaków, które uczestnicy będą mieli na obozie.
\item Sprawdzić czy plecaki są dopasowane do wzrostu osób, które będą je nosić.
\item Kadra musi zrozumieć jak skonstruowany jest plecak i~jak powinien być prawidłowo regulowany.
\end{checklist}

\subsubsection{Kondycja uczestników [6.12] \label{kondycja-uczestnikow}}
Kiedy podczas wizytacji pytam czy uczestnicy zostali przygotowani kondycyjnie do obozu, prawie zawsze napotykam na niezrozumienie i~lekceważenie tego tematu. A~szkoda, bo w~obecnych czasach, kiedy dzieci i~młodzież spędzają swój wolny czas siedząc przed komputerem, ich kondycja pozostawia bardzo wiele do życzenia. Popracowanie nad kondycją nie dość, że wyjdzie im na zdrowie, to przygotuje ich do trudów obozu wędrownego, który ich czeka. I~nie wystarczą dwa czy trzy wypady w~teren w~maju czy w~czerwcu. To musi być kilkumiesięczny program, wpleciony w~roczny plan pracy drużyny.
\\
\\
\small{
\emph{Oto kilka przykładów takich przygotowań, które realizowaliśmy w~drużynie przed różnymi obozami:
\begin{description}
\item[bieganie na dystans] Dwa razy w~tygodniu, wcześnie rano od września do czerwca biegaliśmy na dystansie 5 -- 10 km. Spotykaliśmy się w~harcówce, zostawialiśmy rzeczy na przebranie i~biegliśmy, po powrocie przebieraliśmy się w~suche rzeczy i~każdy szedł do swoich codziennych spraw. [obóz wędrowny, obóz rowerowy],
\item[rajdy terenowe] Raz na 6 tygodni wybieraliśmy się w~ciężki teren na całodzienną wędrówkę, ubrani tak jak na obóz wędrowny, w~te same buty, kurtki, z~zapakowanymi plecakami. W~okolicach Trójmiasta były to np. wędrówki na azymut po Lasach Oliwskich (czyli co chwila podchodzenie i~schodzenie), wędrówki w~Jarze Raduni wzdłuż jednego brzegu rzeki --- powrót wzdłuż drugiego brzegu [obóz wędrowny, zwłaszcza w~górach. Dla obozu ,,nie w~górach'' można wybrać łatwiejszy teren]
\item[wycieczki rowerowe] Podobnie jak wyżej opisane rajdy --- raz na 6 tygodni całodzienna wycieczka rowerowa po okolicy, rowery obciążone sakwami [obóz rowerowy],
\item[pływanie na basenie] Raz w~tygodniu zajęcia całej drużyny na basenie. Udało się nam wynegocjować niższą stawkę za wynajęcie całego szkolnego basenu tylko dla nas. Mieliśmy swoich ratowników, którzy prowadzili zajęcia --- w~tym ten, który popłynął potem z~nami na spływ. Drużyna była podzielona na grupy ze względu na umiejętności pływackie. Nauczyliśmy wszystkich doskonale pływać i~po kilku miesiącach zajęcia przerodziły się w~treningi wytrzymałościowe. Na końcu ratownicy przeprowadzili zajęcia związane z~wypadkami na wodzie, naukę skoków ratowniczych, holowania topielca itd. Jadąc na spływ wiedzieliśmy, że w~razie czego uczestnicy sobie poradzą. [spływ kajakowy]
\item[zajęcia na ergometrze kajakowym] Każdy zastęp raz w~tygodniu przychodził do harcówki na trening na ergometrze kajakowym. \footnotetext{Obrazek i~opis pochodzą z~magazynu \protect\href{4outdoor.pl}{4outdoor.pl} nr~4.} Ergometr wprawia w~ruch większość partii mięśni, głównie brzucha, nóg, rąk i~barków\footnote{Więcej informacji: \href{http://runners-world.pl/trening/Forma-w-domu-trening-na-ergometrze-3620-4.html}{http://runners-world.pl/trening/Forma-w-domu-trening-na-ergomet rze-3620-4.html}}. Doskonale odwzorowuje czynności wykonywane na wodzie podczas wiosłowania w~kajaku. Dzięki temu byliśmy w~stanie przygotować do wiosłowania osoby, które nigdy nie były na spływie, a~nawet nie siedziały w~kajaku [spływ kajakowy],
\begin{figure}[htp]
\centering
\includegraphics[scale=0.47368119266055045871559633027523]{ergometr-kajakowy.jpg}~\includegraphics[scale=0.08612385321100917431192660550459]{Concept_cw2_h1.jpg}
\caption[Caption for LOF]{Z~lewej: ergometr kajakowy. Z~prawej: ergometr wioślarski.\footnotemark}
% http://www.makina.user.icpnet.pl/olimpijczyk/pl/dansprint.html
% http://www.fitness4you.pl/wioalarz-klubowy-concept-2-model-e-najlepszy,det,778.html
% http://www.azs.po.opole.pl/?page_id=799
% Do bibliografii: http://runners-world.pl/trening/Forma-w-domu-trening-na-ergometrze-3620-4.html
\label{fig:ergometry}
\end{figure}
\footnotetext{Źródło: \href{http://www.makina.user.icpnet.pl/olimpijczyk/pl/dansprint.html}{http://www.makina.user.icpnet.pl/olimpijczyk/pl/dansprint.html},  \href{http://www.fitness4you.pl/wioalarz-klubowy-concept-2-model-e-najlepszy,det,778.html}{http://ww w.fit\-ness4you.pl/wioalarz-klubowy-concept-2-model-e-najlepszy,det,778.html}}
%\newpage
\begin{figure}[htp]
\centering
\includegraphics[scale=1.0]{kajakiem.jpg}
\caption[Caption for LOF]{Kajakiem po Gdańsku.\footnotemark}
\label{fig:kajakiem-po-gdansku}
\end{figure}
\footnotetext{Źródło: \href{http://www.gdansk.pl/kultura?id=9581}{http://www.gdansk.pl/kultura?id=9581}}
\item[kilkugodzinne wycieczki kajakami] W~kwietniu, maju i~czerwcu wybieraliśmy się wypożyczonymi kajakami na kilkugodzinne wycieczki po rzekach i~kanałach Gdańska. Dało nam to możliwość sprawdzenia jak radzą sobie pary w~kajakach, mieliśmy możliwość dokonania zmian i~wyłapania słabszych osób [spływ kajakowy].
\end{description}}}
%\newpage
\paragraph{$\bigstar$ Podsumowanie}
\begin{checklist}
\item Przygotować uczestników kondycyjnie do obozu.
\item Przygotowania realizować jako kilkumiesięczny program wpleciony w~śródroczną pracę drużyny.
\item Nie lekceważyć tego tematu.
\item Przed spływem sprawdzić czy uczestnicy są dobrymi pływakami, żeby nie zabrać kogoś, kto np.~na środku jeziora lub po wpadnięciu do wody dostanie ataku paniki.
\end{checklist}

\subsection{Sprawy organizacyjne [7]}
\subsubsection{Zwiad kwatermistrzowski [7.2]\label{zwiad-kwatermistrzowski}}
Zwiad kwatermistrzowski przed obozem wędrownym brzmi niewiarygodnie, ale zdarzają się i~takie rzeczy. Raczej nie trzeba nikomu tłumaczyć, że sprawdzenie trasy, miejsc kwaterowania, noclegów, atrakcji itd. na miejscu daje możliwość stwierdzenia czy to jest dokładnie to co chcemy mieć na obozie.

Jeśli mamy do dyspozycji czas albo osoby z~kadry, które tym czasem dysponują, warto zainwestować w~bilety i~wysłać ekipę na trasę planowanego obozu w~celu sprawdzenia wszystkiego i~zweryfikowania czy poczynione założenia są prawdziwe. Przy okazji można też sprawdzić jak faktycznie wyglądają kwatery o~wynajęcie których zostały podpisane umowy. Ekipa zwiadu powinna robić dokumentację zdjęciową i~po powrocie zdać raport, czy zrelacjonować, co zastała na miejscu ilustrując to fotografiami.

Na wykonanie zwiadu można wykorzystać jakiś długi weekend w~maju czy czerwcu.\newpage
\paragraph{$\bigstar$ Podsumowanie}
\begin{checklist}
\item Jeśli tylko to możliwe wysłać zwiad kwatermistrzowski w~celu sprawdzenia trasy i~kwater.
\item W~trakcie zwiadu robić dokumentację zdjęciową dla pozostałej kadry.
\end{checklist}

\subsubsection{Przygotowanie kadry [7.4]}
Nie mniej ważne niż przygotowaniu uczestników jest przygotowanie kadry. Należy zrobić rozeznanie czy członkowie kadry są odpowiednio przygotowani do prowadzenia obozu danego typu, np. czy kadra ma wiedzę o~górach przez które wiedzie trasa, o~rzece po której mają płynąć, regionie przez który będą jechać, czy byli już na obozie tego typu, w~jakiej są kondycji fizycznej? Takie sprawdzenie da komendantowi i~kadrze obraz stanu przygotowania instruktorów i~czas na podjęcie działań, które polepszą sytuację. Skoro kadra ma przygotować do obozu uczestników to sama musi mieć o~tym odpowiednie pojęcie, w~przeciwnym wypadku ślepy będzie prowadził kulawego.

\paragraph{$\bigstar$ Podsumowanie}
\begin{checklist}
\item Sprawdzić czy kadra jest przygotowana do prowadzenia obozu wybranego typu.
\item Chociaż jedna osoba z~kadry powinna mieć doświadczenie z~obozu tego samego typu.
\end{checklist}

\subsubsection{Atrakcyjność [7.7] \label{atrakcyjnosc}}
Obozy odbywają się w~wakacje, a~wakacje to okres odpoczynku. Zadaniem kadry jest umożliwienie realizacji tego odpoczynku w~atrakcyjnej formie. Na obozie musi być atrakcyjnie. Trasa nie może być monotonna, plan dnia powinien być urozmaicony, wskazane jest by codziennie działo się coś oprócz wędrowania, pedałowania czy wiosłowania. To kadra musi pomyśleć i~przygotować się do tego.
\\
\\
\small{
\emph{Jeśli do przeprowadzenia jakichś atrakcyjnych zajęć potrzebne są dodatkowe materiały, ich transportowania można łatwo uniknąć wysyłając je paczką na Poste Restante (Poczta Polska) albo do Paczkomatu (Impost), jeśli jest w~okolicy. Można również umówić się z~właścicielem schroniska, szkoły, pola namiotowego, że wyślemy paczkę na potrzeby obozu. Wysłane materiały będą czekały w~określonym miejscu, a~po ich wykorzystaniu można je odesłać.}}
\\
\\
Zamiast wymyślać na siłę ,,atrakcje'', należy wykorzystać walory okolic, które obóz przemierza (turystyczne, krajoznawcze, etnograficzne) i~na ich bazie oprzeć te atrakcje. Może to być zorganizowana wycieczka po ruinach zamku czy czynnym kamieniołomie, albo wizyta w~placówce Straży Granicznej (z którą ZHR ma podpisaną umowę --- wystarczy tylko odpowiednio wcześnie przed obozem wysłać pismo z~zapotrzebowaniem, terminem, liczbą osób --- komendant takiej placówki przyjmie wtedy obóz i~pokaże placówkę, sprzęt itd.). Spotkanie z~miejscowym rzeźbiarzem, zaproszenie ludowego grajka czy gawędziarza na ognisko to kolejne możliwości. Wystarczy tylko rozejrzeć się po okolicy --- świetnie nadaje się do tego zwiad kwatermistrzowski. Zwykle na odwrocie map znajdują się wypisane różne ciekawe miejsca. Warto też zajrzeć do starych przewodników opisujących dany region czy pasmo górskie, rzekę itd.

Ogromną atrakcją na spływie mogą się okazać początkowe lub końcowe odcinki rzek jak również odbicie na jakiś ładny dopływ (na dłużej lub krócej), co oczywiście powinno się uwzględnić podczas przygotowań, przy planowaniu ile kilometrów ma być przepłynięte danego dnia. Bardzo istotnym jest aby sprawdzić wcześniej czy dany odcinek rzeki był uczęszczany w~ostatnich latach, a~jeśli tak, to czy przez pojedyncze osoby czy większe grupy. Wszystko po to, aby wiedzieć, czy szlak nie jest kompletnie zarośnięty i~możliwy do przebycia. Początkowe odcinki i~małe dopływy przeważnie przepływają lub graniczą z~rezerwatami przyrody lub ostojami zwierzyny. Powinno się dokładnie sprawdzić czy pod tym względem się coś nie zmieniło w~ostatnich latach, ponieważ wpływanie w~takie miejsca może być zabronione (i~dlatego właśnie mogą być nie uczęszczane). Po co się pakować w~kłopoty?
\\
\\
\small{
\emph{Pewien spływ na Drawie nie zaczynał się tak jak większość, w~Czaplinku, tylko nad jeziorem Siecino, z~którego wypływa rzeka Rakoń, która wpada potem do Kokny, a~Kokna do Drawy. Jak już Rakoń wpadnie do Kokny, można od razu płynąć na Drawę lub Kokną w~stronę jeziora Dołgie. Cała grupa postanowiła popłynąć właśnie w~tym kierunku, z~ciekawości. Okazało się, że ostatni raz ktoś tamtędy płynął chyba ,,za Niemca'' bo rzeka była mocno zarośnięta, prawie nie dało się płynąć. Nagle osoba siedząca z~przodu pierwszego kajaka zaczęła strasznie krzyczeć, bo\ldots pogryzły ją osy, które gdzieś tam miały gniazdo. Ostatecznie cała grupa się wycofała.}}
\\
\\
\small{
\emph{Na szlaku Krutyni jest taki piękny dopływ Babant, którym kiedyś można było normalnie płynąć do jeziora Tejsowo, a~teraz jest tam rezerwat czy coś podobnego. W~każdym razie pływanie tamtędy jest już nielegalnie --- uczestnicy pewnego spływu zostali zatrzymani i~ukarani mandatem przez Straż Leśną.}}
\\
\\
\small{
\emph{Na szlaku Czarnej Hańczy jest pewien odcinek będący granicą rezerwatu. Kajak płynący jako pierwszy został zaatakowany przez łabędzia, kiedy podpłynął zbyt blisko do gniazda. Tylko na prędce zaimprowizowana technika obrony wiosłami uchroniła kajakarzy przed dotkliwym pokąsaniem.}}
\\
\\
\small{
\emph{Odcinek trasy pewnego obozu rowerowego na Mazurach został wyznaczony wzdłuż zwykłego szlaku turystycznego opisanego w~przewodniku z~początków lat 90-tych, który znajdował się w~biblioteczce drużyny. Droga wiodła przez coraz bardziej zarośnięty las, a~znaki szlaku były coraz mniej widoczne. Drużyna zdecydowała się wycofać i~pojechać okrężną drogą kiedy gałęzie zaczęły wyginać szprychy i~wkręcać się w~zębatki przerzutek. Okazało się później, że szlak ten od dawna nie był już zaznaczany na mapach i~nie był utrzymywany.}}
\begin{figure}[htp]
\centering
\includegraphics[scale=0.19180746561886051080550098231827]{krutynia-strumien.jpg}
\caption{Niespodzianka: za mało wody.}\label{fig:krutynia-strumien}
\end{figure}
\\
\\
\small{
\emph{Kiedyś spływy szlakiem Krutyni kończyły się tak daleko, jak tylko się dało, czyli w~Jabłoni. Stamtąd jest kilka km. do Pisza, z~którego kiedyś był łatwy powrót np. do Warszawy. Teraz ludzie kończą wiosłowanie np. w Rucianem-Nidzie i,~niestety, omija ich to, co na tym szlaku jest najlepsze --- jezioro Nidzkie. Jeśli ktoś ambitnie płynie dalej, do samego końca jeziora Nidzkiego, wpływa na rzeczkę Wiartelnicę, dopływa do miejscowości Wiartel, przepływa przez jezioro Wiartel, to znajdzie się w miejscu, w~którym można zrobić dwie rzeczy: albo przenosić kajaki 600~m., albo przebijać się wąskim i~płytkim strumieniem 1~km. do jeziora Brzozolasek, na końcu którego jest wspomniana miejscowość Jabłoń. W~2005~r. pewna ekipa dotarła do tego miejsca i~ciągnęła kajaki ten kilometr, nie wiedząc, że najprawdopodobniej dawno nikogo tam nie było. Napotykała różne przeszkody typu kładki z~belek itp., nie mówiąc o~butelkach na dnie. (Lepiej mieć w~takiej sytuacji na nogach tenisówki niż sandały, zwłaszcza, że woda była zupełnie nieprzejrzysta, ludzie po kostki zapadali się w mulistym dnie.) To była bardzo fajna przygoda. Decydując się właśnie na takie atrakcje wypada sprawdzić, czy ktokolwiek ,,robi'' jeszcze te odcinki, czy są na nich jakieś przeszkody nie do przebycia czy wręcz przeciwnie --- stanowią najciekawszy odcinek na całej rzece.}}

\paragraph{$\bigstar$ Podsumowanie}
\begin{checklist}
\item Na obozie musi być atrakcyjnie.
\item Powinno dziać się coś oprócz wędrowania.
\item Nie wymyślać ,,atrakcji'' na siłę.
\item Wykorzystać walory okolic, przez które wiedzie trasa obozu.
\item Uważać na mało uczęszczane odcinki szlaków.
\end{checklist}

\subsubsection{Ciekawostki [7.8]}
\begin{figure}[htp]
\centering
\includegraphics[scale=0.09438255726272668445737449796512]{stukan.jpg}
\caption{Nieplanowany turniej ,,stukana''.}\label{fig:stukan}
\end{figure}
Ponieważ nie da się nigdy przewidzieć na kogo lub na co można natrafić na trasie obozu, kadra musi być czujna i~powinna wykorzystywać niespodziewanie nadarzające\begin{figure}[htp]
\centering
\includegraphics[scale=0.09438255726272668445737449796512]{wieczor-z-duchami.jpg}\\
~~~~~~~~\\
\includegraphics[scale=0.07538606798315245771277581723061]{lato-dla-ducha.jpg}~~~~\includegraphics[scale=0.09413612760929919964155889875373]{festiwal-piosenki-jura.jpg}
\caption{Napotkane ogłoszenia. Okazja do spotkania z~duchami, seria nabożeństw i~koncertów w~ramach III Piwniczańskiego ,,lata dla ducha'' i~koncerty podczas festiwalu ,,Jura 2012''.}\label{fig:ogloszenia}
\end{figure} się okazje, mimo, że nie ma na nie czasu według harmonogramu. Chodzi tutaj o~akcje w~trakcie dnia, które nie spowodują dużego opóźnienia w~dotarciu do kwatery albo które można przeprowadzić rezygnując z~czegoś co było akurat zaplanowane, a~można to przełożyć na inny termin.
\\
\\
%\newpage
\small{
\emph{Przykłady:
\begin{itemize}
\item obóz napotyka ludowego rzeźbiarza, który zgadza się pokazać jak pracuje, może się zgodzi wyrzeźbić coś dla obozu w~zamian za pomoc w~czymś; można zorganizować jakiś konkurs, w~którym nagrodą będzie to co on wyrzeźbi w~trakcie tego półgodzinnego czy godzinnego spotkania,
\item okazuje się, że w~szkole, w~której obóz ma nocować jest kilka stołów i~sprzęt do ping-ponga --- można przeprowadzić wieczorne zawody o~puchar oboźnego,
\item obóz przechodzi ścieżką nad rzeką / jeziorem i~trafia na plażę ze strzeżonym kąpieliskiem --- kadra umawia się z~miejscowym ratownikiem i~obóz spędza 45 minut w~wodzie (niekoniecznie wszyscy na raz),
\item obóz przechodzi przez wieś i~trafia na porządki w~remizie OSP --- po chwili rozmowy ze strażakami obóz spędza pół godziny bawiąc się w~strażaków i~podpalaczy,
\item 23~lipca, w~Dzień Włóczykija, wszyscy uczestnicy obozu przebierają się za włóczykijów,
\item obóz kwateruje na odludziu i~trafia na bezchmurną noc podczas księżyca w~nowiu --- kadra organizuje godzinne zajęcia dotyczące gwiazdozbiorów, nawigacji itp.
\end{itemize}}}
\small{\label{kolowrotek}
\emph{Na pewnym obozie, umówiona na 20 minut wizyta u~góralki posiadającej kołowrotek przedłużyła się o~ponad godzinę. Skutkiem tego obóz nie zdążył zejść ze szlaku i~dotrzeć do schroniska przez zapadnięciem ciemności. Na innym obozie ciasto u~gospodyni nie zdążyło upiec się w~wygospodarowane pół godziny, przez co plan pozostałej części dnia musiał być zmieniony, bo nie wypadało po prostu wstać i~wyjść.}}

\paragraph{$\bigstar$ Podsumowanie}
\begin{checklist}
\item Wykorzystywać niespodziewanie nadarzające się okazje.
\item Nie ma nic złego w~zmianie planów jeśli nie robi się w~nich rewolucji.
\end{checklist}

\subsubsection{Rodzaj obozu [7.10]}
Zazwyczaj zdarza się, że kadra planuje bardzo ambitną trasę obozu, nie zważając na limity, o~których mowa w~punkcie \ref{dlugosc-wedrowek} na stronie \pageref{dlugosc-wedrowek}, ani na swoje niedostateczne doświadczenie organizacyjne. Skutek bywa taki, że zamiary przerastają kondycyjne i~organizacyjne możliwości drużyny. Zorganizowanie obozu to nie jest wcale prosta sprawa i~w trakcie obozu może się okazać, że organizacja się sypie, a~wyzwanie przerosło możliwości kadry. Planując trasę warto odłożyć plany na dzień czy dwa na półkę i~przemyśleć raz jeszcze wypisując wady, zastanowić się czy kadra i~uczestnicy podołają organizacyjnie i~kondycyjnie. Warto pokazać trasę i~omówić plany z~opiekunem obozu --- być może wnioski wyciągnięte z~przemyśleń czy dyskusji doprowadzą do zmiany tych planów, np. rezygnacji z~noclegów w~namiotach czy~w ogóle zabierania namiotów, albo rezygnacji z~kwaterowania codziennie w~innym miejscu w~celu ograniczenia wędrówek z~plecakami, albo w~ogóle wybrania 2 -- 3 kwater jako bazy i~organizowania z~nich wycieczek w~rożne miejsca --- można wtedy zdecydować się na noclegi w~namiotach itd.

\paragraph{$\bigstar$ Podsumowanie}
\begin{checklist}
\item Wybrać rodzaj obozu, w~którym drużyna dobrze się ,,czuje''.
\item Nie planować ,,ambitnej'' trasy obozu.
\item Przedyskutować plany z~opiekunem obozu i~innymi doświadczonymi instruktorami.
\end{checklist}

\subsubsection{Uczestnicy wymagający szczególnej uwagi [7.12]}
Na każdym obozie znajdzie się co najmniej jedna osoba wymagająca szczególnej troski. Kadra musi mieć tego świadomość i~w razie potrzeby powinna szybko i~fachowo reagować aby problem się nie rozprzestrzeniał. Nie chodzi tu tylko o~osoby kontuzjowane czy maruderów i~symulantów, a~raczej o~osoby, które są czymś zdemotywowane, ,,nie chcą lub nie mogą już dalej iść / wiosłować / pedałować''. Ich postawa jest zaraźliwa i~udziela się innym uczestnikom obozu. Źródło demotywacji niekoniecznie musi być w~danej osobie, może być zewnętrzne, np. tydzień deszczu, mokre ubrania, zimno, beznadziejne jedzenie, brak napojów, zła organizacja, nietrafione oczekiwania co do obozu.

Aby skutecznie reagować na tego typu przypadki należy wiedzieć w~jaki sposób można daną osobę zmotywować czy zachęcić do dalszej wędrówki. Nie jest to łatwe zadanie, sposobów jest tyle ile przypadków --- leży to w~kwestii drużynowego.
\\
\\
\small{
\emph{Na jednym z~obozów, jako ostateczność, zastosowano kilka razy technikę prowadzenia takich osób za rękę przez komendanta na początku peletonu.}}

\paragraph{$\bigstar$ Podsumowanie}
\begin{checklist}
\item Kadra musi szybko i~fachowo reagować na problemy zgłaszane przez uczestników.
\item Demotywacja jest ,,zaraźliwa'' wśród uczestników.
\item Najlepsze efekty przynoszą interwencje drużynowego.
\end{checklist}

\subsubsection{Harcerskie znajomości [7.13]}
Warto nawiązywać harcerskie znajomości na szlaku. Aby nie pozostawiać tego przypadkowi można sprawdzić w~czerwcu w~Naczelnictwie czy na trasie naszego obozu możemy spotkać innych harcerzy, a~jeśli tak, to zdobyć namiary na komendanta i~umówić się na odwiedziny.

\paragraph{$\bigstar$ Podsumowanie}
\begin{checklist}
\item Nawiązywać harcerskie przypadkowe znajomości na szlaku.
\item Zaplanować odwiedziny w~obozach położonych w~pobliżu trasy.
\end{checklist}

\subsubsection{Punkty GOT, TOK, KOT itp. [7.14]\label{odznaki-pttk}}
Wspomniałem już o~tym w~punkcie \ref{dlugosc-wedrowek} na stronie \pageref{dlugosc-wedrowek}, ale warto powtórzyć. PTTK stworzyło system odznak\footnote{Odznaki PTTK: \href{http://www.pttk.pl/pttk/przepisy/index.php?co=odznaki}{http://www.pttk.pl/pttk/przepisy/index.php?co=odznaki}} za wędrówki piesze, górskie, rowerowe, kajakowe i~inne. Każda przebyta trasa podczas wędrówek może zostać zaliczona na poczet jakiejś odznaki w~postaci pewnej liczby punktów GOT\footnote{Górska Odznaka Turystyczna: \href{http://www.pttk.pl/pttk/przepisy/index.php?co=ro_got}{http://www.pttk.pl/pttk/przepisy/index.php?co=ro\_Goty}.}, TOK\footnote{Turystyczna Odznaka Kajakowa: \href{http://www.pttk.pl/pttk/przepisy/index.php?co=ro_tok}{http://www.pttk.pl/pttk/przepisy/index.php?co=ro\_ tok}.}, KOT\footnote{Kolarska Odznaka Turystyczna: \href{http://www.pttk.pl/pttk/przepisy/index.php?co=ro_kot}{http://www.pttk.pl/pttk/przepisy/index.php?co=ro\_ kot}} czy OTP\footnote{Odznaka Turystyki Pieszej: \href{http://www.pttk.pl/pttk/odznaki/ro\_itp.pif}{http://www.pttk.pl/pttk/odznaki/ro\_otp.pdf}.} --- wystarczy tylko do specjalnych książeczek zbierać pieczątki w~miejscowościach, przez które się wędruje i~w miejscach, w~których się nocuje. Ilość tych punktów wynika z~regulaminów tych odznak. Opracowanie punktacji trasy obozu na podstawie regulaminu odznaki wymaga trochę czasu, ale wystarczy to zrobić raz, np. przed obozem, a~potem tylko nanosić ewentualne poprawki --- wtedy uczestnicy będą mogli uzupełniać swoje książeczki na bieżąco. Po obozie wszystkie książeczki zanosi się razem do siedziby oddziału PTTK w~celu weryfikacji. Osoba weryfikująca sprawdzi punktację, wyliczenia i~potwierdzi osiągnięcie limitu (limitów) punktów wymaganych na daną odznakę. Po zakupieniu odznak można je wręczyć na pierwszej zbiórce drużyny po wakacjach, co zdecydowanie zmotywuje harcerzy do działania podczas przygotowań do kolejnego obozu\ldots wędrownego.
\begin{figure}[htp]
\centering
\includegraphics[scale=0.3]{got.jpg}\includegraphics[scale=0.3]{tok.jpg}\includegraphics[scale=1.0]{kot.jpg}\includegraphics[scale=0.3]{otp.jpg}
\caption[Caption for LOF]{Odznaki PTTK.\footnotemark}
\label{fig:odznaki}
\end{figure}
\footnotetext{Źródło: PTTK.}

\paragraph{$\bigstar$ Podsumowanie}
\begin{checklist}
\item PTTK stworzyło system odznak za wędrówki piesze, górskie, rowerowe, kajakowe itp.
\item Wystarczy tylko zbierać pieczątki do specjalnych książeczek.
\item Opracowanie punktacji trasy obozu zrobić przed obozem.
\item Osiągnięcie limitu punktów uprawnia do noszenia odznaki.
\item Odznaki wręczyć od razu po wakacjach.
\end{checklist}

\begin{figure}[htp]
\centering
\includegraphics[scale=0.65]{pieczatki-got.png}
\caption{Zbieranie pieczątek to frajda dla uczestników obozu, ponieważ schroniska zazwyczaj mają ozdobne duże kolorowe pieczątki pamiątkowe.}
\label{fig:pieczątki}
\end{figure}

\cleardoublepage
\section{W~trakcie obozu.}
Pokaż mi swój obóz, a~powiem Ci\ldots\\
\\
Ta sekcja zawiera aspekty, o~których instruktorzy powinni pamiętać w~trakcie obozu. Jest ich bardzo dużo, ale nie wolno się przerażać --- z~każdym kolejnym obozem będą wchodziły one w~nawyk, aż w~końcu staną się czymś oczywistym.
\subsection{Dokumentacja finansowa i~nie tylko [1]}
Pokaż mi dokumentację obozu, a~powiem Ci\ldots\\
\\
Tematy poruszane w~tym punkcie są szczegółowo omawiane przed każdą akcją letnią na spotkaniach i~szkoleniach dla kwatermistrzów. Przepisy dotyczące dokumentacji i~kontroli finansów zmieniają się dość często, dlatego informacje podane poniżej mogą wkrótce okazać się nieaktualne. Przykładem są choćby wychodzące z~użycia kwity KP opisane w~punkcie \ref{kwity-kp}. Przestrzeganie tych przepisów wymaga sumienności, staranności i~czasu, który należy przewidzieć w~obowiązkach kwatermistrza. Skoro obozy mają być prowadzone profesjonalnie po prostu nie można zaniedbywać tych kwestii.
\subsubsection{Rachunki i~faktury [1.1]}\label{rachunki-i-faktury}
Rachunki i~faktury zbierane na obozie muszą być przechowywane w~należytym porządku. Nie oznacza to, że trzeba nosić ze sobą ciężki i~duży segregator --- wystarczy teczka lub koszulka foliowa. Najrzadziej co dwa dni kwatermistrz powinien sprawdzać porządek w~fakturach, układać je w~kolejności chronologicznej oraz dokonywać brakujących opisów. W~ten prosty sposób, od samego początku obozu, będzie zapewniony porządek w~dokumentacji finansowej. Aby przetrwała ona w~dobrym stanie aż do rozliczenia obozu, trzeba prawidłowo zabezpieczyć ją przed zniszczeniem, a~szczególnie zamoknięciem. Włożenie do reklamówki i~umieszczenie na ,,plecach'' plecaka to najczęściej stosowany nieskuteczny sposób. Teczka z~fakturami powinna zostać umieszczona w~specjalnej, odpowiednio zamykanej, plastikowej koszulce wodoodpornej. Innym sposobem jest włożenie teczki do worków na mrożonki, które, w~przeciwieństwo do foliowych reklamówek, nie przepuszczają wody. Dopiero tak zabezpieczone dokumenty można umieścić na ,,plecach'' plecaka.
\paragraph{$\bigstar$ Podsumowanie}
\begin{checklist}
\item Układać faktury i~rachunki chronologicznie w~specjalnej teczce lub koszulce.
\item Najrzadziej co drugi dzień dokonywać brakujących opisów na fakturach.
\item Chronić faktury przez zgubieniem, zniszczeniem, zabrudzeniem, zamoknięciem.
\item Przechowywać faktury w~nieprzemakalnym worku.
\end{checklist}
\subsubsection{Kontrola nad stanem finansów obozu [1.2]}
Dzięki wprowadzeniu kont obozowych oraz możliwości dokonywania wypłat w~bankomatach, kadra nie musi posiadać przy sobie wszystkich pieniędzy obozowych. Są jednak sytuacje, zwłaszcza na obozach wędrownych, że po drodze wcale nie ma bankomatów. Wtedy obóz wędruje z~całą gotówką. W~takim przypadku należy pamiętać o~rozdzieleniu pieniędzy pomiędzy osoby z~kadry aby zminimalizować ryzyko strat w~wypadku zgubienia czy kradzieży. Kwatermistrz po rozdysponowaniu kolejnej kwoty ,,odbiera'' następnej osobie z~kadry gotówkę, którą wcześniej u~niej zdeponował.

Kwatermistrz dysponuje kwotami przeznaczonymi na zakupy, powierza je różnym osobom. Przy kontrolowaniu pieniędzy, które wyjmowane są z~kasy obozu pomocny jest tzw. zeszyt zaliczek --- wpisuje się tam każdą sumę powierzoną komuś przez kwatermistrza. Po zakupach, przy rozliczaniu tej sumy, oddaje się fakturę i~resztę pieniędzy. W~ten sposób kwatermistrz może łatwo ustalić komu i~kiedy jaką ilość pieniędzy przekazał i~czy została ona rozliczona.
\paragraph{$\bigstar$ Podsumowanie}
\begin{checklist}
\item Jeśli tylko to możliwe trzymać pieniądze na koncie w~banku i~wypłacać je w~bankomatach.
\item Gotówkę podzielić pomiędzy wszystkich członków kadry.
\item Korzystać z~zeszytu zaliczek w~momencie przekazywania komukolwiek gotówki na zakupy.
\end{checklist}
\subsubsection{Książka finansowa [1.3]}
Wśród instruktorów pokutuje od lat opinia, że książki finansowej na obozie wędrownym prowadzić się ,,nie da''. I~faktycznie, praktyka pokazuje, że mało który obóz wędrowny ją prowadzi. Tłumaczenia są prawie zawsze takie same: ,,nie ma czasu'' oraz ,,i~tak wszystko potem trzeba będzie przepisać''.

Na kursach kwatermistrzowskich przekazywane są informacje dlaczego uzupełnianie tej książki jest takie ważne. Stanowi ona dokumentację finansową oraz źródło informacji na temat stanu i~struktury wydatków obozowych. Jeżeli kwatermistrz nie ma czasu aby chociaż co drugi dzień wpisać do niej kilka pozycji to znaczy, że ma za dużo innych obowiązków i~trzeba go odciążyć, aby mógł to robić.
\paragraph{$\bigstar$ Podsumowanie}
\begin{checklist}
\item Wpisywać faktury do książki finansowej najrzadziej co dwa dni.
\item Jeżeli kwatermistrz nie ma czasu trzeba go odciążyć.
\end{checklist}
\subsubsection{Wydatki [1.4]}
Kontrola bieżących wydatków na obozie polega na wiedzy ile pieniędzy wydano, ile jeszcze trzeba wydać oraz ile zostało w~kasie. Dzięki temu wiadomo czy wystarczy funduszy do końca obozu. Inną ciekawą informacją, którą można znaleźć w~książce finansowej, jest struktura wydatków. Polega to na zsumowaniu kwot z~poszczególnych kategorii, jak wyżywienie, zakwaterowanie, transport itd. i~porównaniu z~kwotami wpisanymi do preliminarza. Uwzględniając numer dnia obozowego można ustalić czy obóz wydaje pieniądze zgodnie z~preliminarzem czy np. wydaje dużo mniej na wyżywienie i~jednocześnie dużo więcej na transport. Przyglądając się się tym wyliczeniom oraz temu co się dzieje na obozie można stwierdzić, że różnica wydatków poniesionych na wyżywienie wynika np. z~nieprzewidzianego spadku cen żywności albo z~faktu oszczędzania na jedzeniu przez kadrę, a~różnica w~wydatkach poniesionych na transport to skutek nadprogramowej wycieczki do pobliskiego miasta np. na koncert.
\paragraph{$\bigstar$ Podsumowanie}
\begin{checklist}
\item Podczas wpisywania faktur do książki finansowej uzupełniać od razu odpowiednie kategorie wydatków.
\item Uwzględniając numer dnia obozowego sprawdzać czy obóz wydaje pieniądze zgodnie z~preliminarzem.
\end{checklist}
\subsubsection{KP [1.6]\label{kwity-kp}}
Po przyjęciu wpłaty gotówką należy od ręki wypisać kwit potwierdzający ten fakt. Wszelkie opóźnienia wpłyną źle na porządek w~finansach obozowych oraz dołożą kolejną czynność do zrobienia ,,potem''.
\paragraph{$\bigstar$ Podsumowanie}
\begin{checklist}
\item Po przyjęciu wpłaty gotówką wypisać od ręki kwit potwierdzający.
\end{checklist}
\subsubsection{Karty obozowe [1.7]}
Karty obozowe są dokumentami zawierającymi ważne informacje o~uczestnikach obozu, również tych pełnoletnich (patrz rozdział \ref{wyciag-z-kart} na stronie \pageref{wyciag-z-kart}). Aby zatwierdzić obóz należy posiadać komplet prawidłowo wypełnionych kart obozowych. Niestety, po zatwierdzeniu, zainteresowanie kartami znika aby powrócić dopiero podczas rozliczania obozu. Tymczasem są tam sekcje do wypełnienia przez kadrę zaraz na początku oraz w~trakcie obozu. Np. sekcja o~zakwalifikowaniu uczestnika na obóz powinna być wypełniona najpóźniej w~momencie otrzymania opłaty za obóz od danego uczestnika. Braki takich wpisów są podstawą dla wizytatorów do stwierdzenia, że dokumentacja obozowa jest niekompletna albo niewłaściwie prowadzona.
\paragraph{$\bigstar$ Podsumowanie}
\begin{checklist}
\item W~kartach obozowych są sekcje do wypełnienia przez kadrę obozu zaraz na początku oraz w~trakcie obozu.
\end{checklist}
\subsubsection{Książka pracy obozu [1.10]}
Kolejnym niedocenianym dokumentem, z~którym instruktorzy powinni pracować na co dzień, a~bardzo interesującym dla wizytatorów, jest wypełniona i~prowadzona na bieżąco książka pracy obozu. Obóz, jako placówka wypoczynku dzieci i~młodzieży, musi posiadać dokument opisujący to, co zostało faktycznie zrealizowane każdego dnia. Nie chodzi tu o~bezsensowne, codzienne, przepisywanie programu obozu, ale o~kilka zdań komentarza zawierających informacje co i~jak się udało wykonać. Takie źródło informacji będzie dla wizytatora czy hufcowego streszczeniem tego co i~jak się na obozie działo, a~dla kadry obozu będzie podstawą do pracy z~harcerzami w~kolejnym roku. Umieszczane tam komentarze mogą być po prostu hasłami opisującymi jak udało się coś wykonać, mogą to być zapiski przypominające pamiętnik czy kronikę, ale jeszcze lepiej, kiedy członkowie kadry zapisują swoje przemyślenia dotyczące tego co się zdarzyło, co w~związku z~tym można zrobić, plany i~pomysły na przyszłość, komentarze dotyczące prób i~rozwoju uczestników, spraw organizacyjnych. Jest tam też miejsce na rozkazy obozowe. Ze względu na ważkość tych informacji o~książkę należy dbać i~chronić ją w~sposób taki jak dokumenty finansowe (opis w~punkcie \ref{rachunki-i-faktury} na stronie \pageref{rachunki-i-faktury}).
\paragraph{$\bigstar$ Podsumowanie}
\begin{checklist}
\item Książka pracy obozu powinna być uzupełniana codziennie.
\item Wpisów do książki mają prawo i~powinni dokonywać wszyscy członkowie kadry obozu.
\item Nie przepisywać bezmyślnie programu obozu.
\item Wystarczy wpisywać kilka zdań komentarza zawierających informacje co i~jak się udało wykonać.
\item Dbać o~książkę jak o~dokumenty finansowe.
\end{checklist}
\subsection{Kwatery i~porządek [2]}
Pokaż mi jak i~gdzie odpoczywasz po wędrówce, a~powiem Ci\ldots
\subsubsection{Przestrzeganie zasad higieny [2.5]}
Nawyki dotyczące przestrzegania higieny każdy wynosi z~domu. Bez względu na wiek zdarzają się osoby, które są z~higieną na bakier i~uważają ją za zbędną ceremonię czy stratę czasu. Kadra musi być czujna aby zaraz na początku obozu ,,zidentyfikować'' brudasów, porozmawiać z~nimi oraz zobowiązać oboźnego do ich dyskretnego ,,pilnowania''. W~cięższych przypadkach to drużynowy powinien rozwiązać problem, pracując indywidualnie z~harcerzami, np. poprzez zadawanie harców. Wizytacje z~ostatnich kilku lat pokazują, że z~higieną osobistą nie ma wielkiego problemu, jednakże kwestia czystej i~suchej odzieży przedstawia się znacznie gorzej. Bardzo często spotkać można harcerzy w~brudnych i~,,pachnących'' stęchlizną ubraniach. A~najgorsze, że to samo można powiedzieć o~kadrze\ldots W~tym miejscu pole do wykazania się mają oboźni. To w~ich mocy leży doprowadzenie wszystkich do tego aby byli czyści, aby ich odzież była wyprana i~sucha. Żeby to osiągnąć oboźny musi dopilnować, by uczestnicy rzeczywiście dokonywali codziennej toalety w~czasie na to przeznaczonym. Tego czasu musi być wystarczająca ilość, aby każdy z~uczestników zdążył się umyć, a~także wyprać swoje rzeczy. Ponadto wyzwaniem jest doprowadzenie do wysuszenia upranej odzieży. Podobna kwestia dotyczy czyszczenia i~suszenia butów po wędrówkach. Jak zawsze kadra musi dawać przykład i~pociągnąć harcerzy za sobą. Czasem wystarczy zwykłe ,,idziemy robić pranie'', bez zbiórek, ustawiania się itp. aby uczestnicy chętnie poszli to zrobić.
\paragraph{$\bigstar$ Podsumowanie}
\begin{checklist}
\item Zaraz na początku obozu ,,zidentyfikować'' brudasów i~dyskretnie ich ,,pilnować''.
\item Nie dopuszczać aby harcerze chodzili w~brudnych i~,,pachnących'' stęchlizną ubraniach.
\item Oboźny musi pilnować, by uczestnicy rzeczywiście dokonywali codziennej toalety.
\item Pamiętać o~czyszczeniu i~suszeniu butów po wędrówkach.
\item Czasem wystarczy zwykłe ,,idziemy robić pranie''.
\end{checklist}
\subsubsection{Porządek i~czystość na obozie [2.6]}
Z~utrzymaniem porządku i~czystości na kwaterze sytuacja wygląda podobnie jak z~przestrzeganiem higieny osobistej. Niektórym rodzicom udało się wpoić swoim dzieciom umiejętność utrzymania porządku, jednak na obozie zawsze znajdą się takie osoby, które w~ogóle o~to nie dbają. Jest to kolejne wyzwanie dla oboźnego, aby wykorzenić złe nawyki i~nauczyć bałaganiarzy odpowiedniego postępowania.
\begin{figure}[htp]
\centering
\includegraphics[scale=0.19011236]{DSC03913.JPG}
\caption{Bałagan w~schronisku.}\label{fig:balagan-w-schronisku}
\end{figure}
\\
\\
W~nawiązaniu do poruszonego wyżej tematu prania ubrań, do obowiązków oboźnego należy wyznaczenie i~przygotowanie miejsca na suszenie prania, a~ponadto doprowadzenie do jego wysuszenia. Nie może tak być, że pranie suszy się wszędzie na terenie kwatery. Bielizna, spodnie, koszulki i~inne części garderoby wiszą bez ładu na naciągach namiotów, poręczach schodów w~schroniskach czy szkołach, na oparciach krzeseł, na gałęziach okolicznych drzew albo leżą na plecakach, śpiworach lub wprost na trawie. Ile razy pranie wisiało na dworze całą noc, mimo padającego deszczu lub wichury? Ile razy nie można było znaleźć właściciela ubrań, które spadły na ziemię? Wyznaczenie miejsca na suszarnię oraz rozwieszenie kilku linek naprawdę nie jest czymś co przekracza możliwości każdego oboźnego, da się to wykonać w~ciągu kilku minut.
\paragraph{$\bigstar$ Podsumowanie}
\begin{checklist}
\item Oboźny pilnuje przestrzegania porządku na obozie.
\item Bałaganiarzy uczyć odpowiedniego postępowania.
\item Oboźni muszą wyznaczyć i~przygotować miejsce na suszenie prania.
\item Pranie doprowadzać szybko do wysuszenia.
\end{checklist}
\subsubsection{Kadrówka [2.7]}
Podobną kwestią jest sprawa porządku w~kadrówce. Jeżeli nie ma w niej porządku, to hipokryzją jest wymaganie porządku od harcerzy. Kadra musi mieć idealny porządek. Wydaje się być dobrym zwyczajem, stosowanym przez niektóre drużyny, włączenie sprawdzania porządku w~kadrówce do porannego rytuału przygotowania do apelu. Harcerze muszą widzieć i~wiedzieć, że w~kadrówce zawsze jest porządek. Niech harcem dla zastępów będzie osiągnięcie lepszego porządku niż jest w~kadrówce.

Kadra musi przestrzegać porządku, musi dawać przykład jak to robić. Nie ma innej opcji.
\paragraph{$\bigstar$ Podsumowanie}
\begin{checklist}
\item Kadra musi mieć idealny porządek.
\item Sprawdzać porządek w~kadrówce podczas przygotowania do apelu.
\end{checklist}
\subsubsection{,,Pozostałości'' [2.8]}
Doskonałą miarą pozwalającą stwierdzić czy na kwaterze był porządek i~czy harcerze umieją o~niego dbać, jest obserwacja tego, co pozostaje w~miejscu kwaterowania obozu. Jeżeli po spakowaniu plecaków, obozowego sprzętu i~materiałów, pozostają pojedyncze śmieci, które znikają w~ciągu kilku minut to bardzo dobrze. Natomiast jeśli potrzebne jest długotrwałe sprzątanie, zbieranie śmieci, przeprowadzanie śledztwa do kogo należą znalezione rzeczy itd. to od razu wiadomo, że harcerze nie dbali o~porządek, że kadra nie umiała uczestników przypilnować. Dobrze chociaż, jeśli były chęci do posprzątania po sobie. Jest bowiem nie do zaakceptowania sytuacja, w~której drużyna zostawia po sobie nie posprzątaną kwaterę. Takie zachowanie prowadzi wprost do jej tzw. ,,spalenia'', tzn. jej właściciel lub zarządca po prostu nie będzie chciał więcej gościć u~siebie harcerzy, ponieważ będzie się obawiał o~stan kwatery po jej opuszczeniu przez obóz.
\paragraph{$\bigstar$ Podsumowanie}
\begin{checklist}
\item Obóz musi zostawić po sobie posprzątaną kwaterę.
\item Zostawienie nieposprzątanej kwatery prowadzi wprost do jej tzw. ,,spalenia''.
\end{checklist}
\subsection{Żywienie [3]}
Pokaż mi co i~kiedy jesz, a~powiem Ci\ldots
\subsubsection{Zakupy żywności [3.1]}
\begin{figure}[htp]
\centering
\includegraphics[scale=0.38262125]{zakupy.jpg}
\caption{Nie kupować za dużo jedzenia, żeby nie trzeba było go nosić.}\label{zakupy}
\end{figure}
Zakupy, zwłaszcza żywności, muszą być mądrze zaplanowane. W~niektórych rejonach ciężko jest z~zaopatrzeniem: nie ma sklepów w~ogóle, są czynne tylko wcześnie rano albo po południu, nieczynne w~soboty, niedziele i~święta, zaopatrzenie jest obliczone na potrzeby lokalnej społeczności. Jeśli temat nie zostanie odpowiednio ,,rozpracowany'' konieczne będzie organizowanie specjalnych, niespodziewanych wypraw po zakupy. Jeśli to tylko możliwe zakupy należy robić ,,po drodze'', tzn. w~trakcie wędrówek, bowiem każdy inny sposób powoduje zakłócenie harmonogramu obozu. Kwatermistrz powinien wiedzieć, w~których sklepach i~kiedy trzeba zrobić zakupy. Co więcej, powinien telefonicznie, kilka dni wcześniej, złożyć zamówienie, aby sklep był zaopatrzony w~odpowiednio większą liczbę potrzebnych produktów.
\paragraph{$\bigstar$ Podsumowanie}
\begin{checklist}
\item Planować gdzie i~kiedy trzeba zrobić zakupy.
\item Unikać organizowania specjalnych, niespodziewanych wypraw po zakupy.
\item Zaopatrzenie w~wiejskich sklepach jest obliczone na potrzeby lokalnej społeczności.
\item Składać telefoniczne zamówienia kilka dni wcześniej.
\end{checklist}
\subsubsection{Wyżywienie podczas wędrówki [3.2]}
Każdy wysiłek fizyczny, a~więc również wędrówki, wiosłowanie, pedałowanie itd. powodują zużywanie energii. Powstałe braki należy uzupełniać często i~regularnie, zwłaszcza u~dzieci. W~związku z~tym kwatermistrz jest odpowiedzialny za przygotowanie prowiantu dla uczestników, który zostanie spożyty na szlaku. Każdy uczestnik powinien dysponować w~trakcie wędrówki swoją porcją. Głowa przewodnika lub oboźnego w~tym, aby podczas przerw prowiant ten był zjadany. Przygotowanie żywności na drogę zajmuje sporo czasu, dlatego należy sprytnie podejść do sprawy. Stary sposób, od którego już większość drużyn odeszła, to zaangażowanie zastępu kuchennego, od świtu, aby oprócz śniadania, przygotowywał kanapki na drogę. Obecnie najczęściej spotykany sposób to taki, że podczas śniadania, zorganizowanego jako rodzaj ,,szwedzkiego stołu'' każdy jest zobowiązany do zrobienia dla siebie kanapek. Oboźny lub inna wyznaczona osoba z~kadry pilnuje tylko, żeby ilość kanapek była wystarczająca dla każdego i~aby były one różnorodne. Do ,,bazy'' kanapkowej należy dołożyć jeszcze jakiś soczysty i słodki owoc, np. jabłko czy nektarynkę, dodatkowo przyda się dawka magnezu w~postaci czekoladowego batona. Unikamy wszelkich miękkich owoców, które z~reguły nie są w~stanie przetrwać ,,plecakowego'' transportu. Do kanapek na drogę nie dajemy ryb z~puszki ani niczego co się łatwo psuje, roztapia i~wypływa. Unikać należy wszelkiego rodzaju jogurtów i~im podobnych przetworów mlecznych, które w~gorące letnie dni bardzo często powodują problemy żołądkowe.
\begin{figure}[htp]
\centering
\includegraphics[scale=0.289230769]{DSC03917.JPG}
\caption{Posiłek na szlaku}\label{prowiant}
\end{figure}
\\
\\
Dzieci i~młodzież są przyzwyczajeni do ciągłego podjadania różnych przekąsek. Jeżeli kadra obozu nie zapewni im wyżywienia podczas wędrówki skutkiem będzie zwielokrotnienie tego efektu --- przy każdym napotkanym sklepie odegra się ten sam scenariusz: najpierw błagania o~zgodę na wejście do sklepu, a~potem kupowanie słodyczy, czipsów, paluszków itd. w~zastraszających ilościach. Podobne historie mają miejsce, kiedy ilość prowiantu na drogę jest zbyt mała, albo jest on ubogi w~pewne składniki, do których dzieci są przyzwyczajone, a~także w~te, które są tracone na skutek wysiłku fizycznego. Obserwuje się wtedy kupione egzotyczne zestawy, np: czekoladę i~ogórki, albo lody i~kiełbasę. Sposobem na okiełznanie tego zjawiska są zakupy kontrolowane przez kadrę. Polegają one na zezwoleniu na kupowanie tylko i~wyłącznie konkretnych produktów, gdzie przy kolejnym sklepie ich lista powinna być inna.

Kadra obozu ma obowiązek zapewnić wyżywienie na obozie --- to oznacza również prowiant na drogę --- musi być on na odpowiednim poziomie. Aby tak się stało należy uwzględnić w~dziennej stawce żywieniowej koszt prowiantu.
\paragraph{$\bigstar$ Podsumowanie}
\begin{checklist}
\item Każdy powinien przygotować kanapki dla siebie podczas śniadania.
\item Do kanapek dołożyć owoce i~czekoladę.
\item Każdy uczestnik powinien dysponować prowiantem w~trakcie wędrówki.
\item Pilnować aby podczas przerw prowiant był zjadany.
\item Unikać produktów, które się roztapiają, łatwo zgniatają i~szybko psują, a~także jogurtów itp.
\item Zapewnić właściwą ilość wyżywienia na szlaku aby uniknąć kupowania słodyczy i~przekąsek przez uczestników.
\item Uwzględnić w~dziennej stawce żywieniowej koszt prowiantu.
\end{checklist}
\subsubsection{Napoje [3.3]}
Jeszcze ważniejszą kwestią niż wyżywienie podczas wędrówek, omówione powyżej, są napoje. Każdy uczestnik musi mieć ze sobą picie, przenaczone tylko na jego użytek, w~ilości odpowiedniej dla danego etapu wędrówki, tj. np. na 4 czy 6 godzin. Wiadomo, że wraz ze wzrostem temperatury zapotrzebowanie na napoje będzie znacznie rosło. Kwatermistrz musi być wyczulony na to i~zapewnić ich odpowiednią ilość. Oboźny jest odpowiedzialny, aby każdy uczestnik wziął je ze sobą na wędrówkę, a~potem wypijał podczas każdej przerwy. Nie wolno dopuścić do odwodnienia. Należy unikać sytuacji kiedy napoje są wypijane od razu na początku wędrówki albo dopiero pod jej koniec. Regularność jest tutaj jednym z~kluczowych czynników. Inną kwestią jest rodzaj napojów. Czy jest to czarna herbata pozostała po śniadaniu, czy zwykła woda z~kranu lub jeszcze gorzej --- najtańsza butelkowana? Wraz z~potem organizm traci minerały, które muszą być ciągle uzupełniane. Najłatwiej osiągnąć to stawiając na różnorodność. Wystarczy zaopatrzyć uczestników w~małe butelki 0,33 litra i~do każdej nalać czego innego. Jeśli herbata ze śniadania, to nie za mocna, posłodzona i~z~cytryną lub --- w~chłodny dzień --- z~miodem i~imbirem. Jeśli woda, to średniozmineralizowana lub z~kranu, jeśli nadaje się do picia bez przegotowania. Unikać wody niskozmineralizowanej i~przegotowanej, a~kiedy nie ma innej opcji, to rozpuszczać w~niej raz na kilka dni różne tabletki musujące albo wycisnąć do niej cytrynę i~posłodzić, jak herbatę. Można też kupić herbatę granulowaną i~dosypywać ją do butelek. Na końcu każdej wędrówki dać do picia napój izotoniczny, ale nie więcej niż 0,25 litra. Unikać mleka do picia rano i~w~ciagu dnia. Napoje gazowane, a~także kwaśne soki nie są dobrym pomysłem. Wieczorem można wykorzystać mleko i~zrobić gorące kakao lub płatki na mleku.
\begin{figure}[htp]
\centering
\includegraphics[scale=0.10803075396825396825396825396825]{P7310251.JPG}
\caption{Czego by się tu napić?}\label{plecak-z-napojami}
\end{figure}
\\
\\
Jeżeli ilość napojów będzie za mała, albo będą nieurozmaicone, wystąpi efekt dokupowania ich przez uczestników, zwłaszcza przez tych, którzy mają nawyk częstego picia słodkich i~gazowanych. Większość dzieci jest obecnie do tego przyzwyczajona, więc wyjściem jest dołączanie jednej butelki z~takim napojem do codziennego ,,przydziału'' --- oczywiście codziennie innym.
\paragraph{$\bigstar$ Podsumowanie}
\begin{checklist}
\item Każdy uczestnik powinien dysponować dużą ilością napojów w~trakcie wędrówki.
\item Napoje powinny być różnorodne.
\item Unikać ciągłego picia zwykłej wody niskozmineralizowanej.
\item Zapewnić napoje uzupełniające elektrolity.
\item Napojów na bazie mleka nie podawać przed wędrówkami ani w~upalne dni.
\end{checklist}
\subsubsection{Stawka żywieniowa [3.4]}
Przed obozem, podczas układania preliminarza i~planowania wydatków, kadra obozu ustala dzienną stawkę żywieniową. Jeśli wszystkie wydatki na żywność były zapisywane od początku obozu, to bardzo łatwo jest ustalić rzeczywistą wartość tej stawki. Porównując ją z~tą planowaną, oraz uwzględniając to, co i~ile uczestnicy jedzą i~wypijają można łatwo stwierdzić czy obóz ,,oszczędza'' na jedzeniu i~napojach. Niedoświadczona kadra ma tendencję do kupowania najtańszych produktów spożywczych, podejrzanej jakości, w~celu ,,zaoszczędzenia''. Zdarzają się także zaniechania, np. dotyczące przygotowania prowiantu i~napojów na wędrówki. Wszystkie śniadania i~kolacje są podobne, bo zjada się na nich tylko biały chleb z~serkiem topionym i pasztetem z~puszki. Obiady są dziełem przypadku i~mają przeważnie wtedy postać ryżu lub makaronu z~sosem. Tego typu ,,oszczędnościom'' trzeba powiedzieć stanowcze ,,nie'', ponieważ każdy obóz wędrowny ma zatwierdzony jadłospis, a~kwatermistrz podjął się świadomie wypełniania swoich obowiązków, a~w~konsekwencji również realizacji jadłospisu.
\paragraph{$\bigstar$ Podsumowanie}
\begin{checklist}
\item Nie ,,oszczędzać'' na jedzeniu i~napojach.
\item Nie kupować najtańszych produktów spożywczych ani podejrzanej jakości w~celu ,,zaoszczędzenia''.
\item Nie rezygnować z~prowiantu i~napojów na wędrówki.
\item Zapewnić urozmaicone posiłki.
\end{checklist}
\subsubsection{Warunki przechowywania żywności [3.5]}
Specyfiką obozów wędrownych jest, niegromadzenie dużych zapasów żywności, aby nie stały się balastem. W~związku z~tym problem przechowywania żywności na kwaterach praktycznie nie istnieje, gdyż jest ona zjadana najpóźniej następnego dnia. Oczywiście zakupione produkty przeznaczone do spożycia na następny dzień powinny być przechowywane w~odpowiednich warunkach. Dlatego warto zapewnić na kwaterze dostęp do lodówki. Produkty, których nie trzeba przechowywać w~lodówce powinny leżeć w~czystym, suchym, zacienionym miejscu.

Wyzwaniem natomiast jest transport produktów spożywczych pomiędzy kwaterami, zwłaszcza, gdy trwa to od rana do wieczora w~upalne dni. Przy wysokiej temperaturze masło i~margaryna roztapiają się, soki i~mleko fermentują, sery zlepiają się i~wysychają, twarogi gliwieją\footnote{Zgliwiały czyli oślizły, pożółkły i~niezbyt ładnie pachnący twaróg można uratować w~następujący sposób: posypać lekko sodą oczyszczoną, która ułatwia topienie i~usmażyć z~kminkiem (można dodać co się chce, np. inne zioła, koperek, oliwki, szynkę) na maśle z~jajkami (lub bez).}, wędliny zielenieją itp. --- ogólnie produkty szybko się psują. O~ile margaryna czy ser są w~stanie przetrwać kilka godzin, to pozostałe wymienione produkty nie powinny w~ogóle być transportowane. Należy kupować ich taką ilość, aby nic nie zostawało. Spożycie produktów w~podejrzanym stanie grozi zbiorowym zatruciem, do czego absolutnie nie można dopuścić.

Prowiant na wędrówkę, który każdy uczestnik powinien mieć ze sobą, warto pakować do lekkich, zamykanych, plastikowych pojemniczków. W~ten sposób produkty żywnościowe przetrwają bez deformacji, nie będą narażone na zabrudzenie ani same nie pobrudzą pozostałej zawartości plecaka czy sakwy. Wiele drużyn z~powodzeniem stosuje ten sposób.
\paragraph{$\bigstar$ Podsumowanie}
\begin{checklist}
\item Nie gromadzić dużych zapasów żywności, bo trzeba będzie je nosić.
\item Kupować tyle, aby nic nie zostawało.
\item Produkty transportowane w~upale mają małe szanse przetrwania.
\item Niektórych produktów w~ogóle nie ma sensu przenosić, bo wiadomo, że nie przetrwają.
\item Uważać aby nie podać zepsutego, podejrzanego po transporcie produktu.
\end{checklist}
\begin{figure}[htp]
\centering
\includegraphics[scale=0.09431787188559634380067816600324]{IMAG0399.jpg}\\
~~~~~~~~\\
\includegraphics[scale=0.15034988790463692038495188101487]{DSC01099.JPG}
\caption{Brudno vs. czysto}\label{fig:brudny-stol}
\end{figure}
\subsubsection{Warunki i~sposób przygotowywania posiłków [3.6]}
Na kwaterach są różne warunki, w~których trzeba przygotować posiłki. Bardzo często obóz nie ma dostępu do zaplecza kuchennego. Sam stół musi wystarczyć, ale nie zwalnia to w~żaden sposób z~konieczności zachowania pewnych zasad czystości. Osoba odpowiedzialna musi dopilnować, aby miejsce było czyste, a~jeśli się nie da, należy użyć ceraty, serwetek czy czegokolwiek do przykrycia stołu, ławy, kajaka, deski itd. Nie wolno kłaść żywności bezpośrednio na brudne miejsca. Jeżeli posiłki są przygotowywane przez zastęp kuchenny, to instruktor odpowiedzialny za posiłki ma obowiązek kontrolować te czynności, choćby nawet to był najlepszy zastęp. Osoba wyznaczona do nadzorowania posiłków musi mieć o~tym pojęcie oraz naprawdę sumiennie tego pilnować. Jeżeli obóz korzysta z~systemu ,,szwedzkiego stołu'', to ów wspomniany instruktor powinien mieć oko na każdy stół, a~najlepiej aby przy każdym stole siedział ktoś z~kadry. Z~wielu powodów, które powinny być oczywiste dla każdego drużynowego, nie do zaakceptowania są sytuacje, kiedy kadra siedzi osobno przy ,,swoim'' stole, a~uczestnicy przy ,,swoich''.

Inna możliwość to korzystanie z~lokali gastronomicznych, których znajduje się sporo w~miejscowościach turystycznych. Restauracje i~bary są nastawione na sezonowy zysk, dlatego kwatermistrz powinien zorientować się wcześniej czy w~danej miejscowości nie ma jakiegoś domu wczasowego, sanatorium czy zakładu pracy --- posiadają one bardzo często własne stołówki, gdzie można dobrze i~niedrogo zjeść --- oczywiście po uprzednim zamówieniu telefonicznym. Tak czy inaczej, przed zjedzeniem czegokolwiek warto sprawdzić w~jakich warunkach są przygotowywane posiłki.

\paragraph{$\bigstar$ Podsumowanie}
\begin{checklist}
\item Dopilnować, aby miejsce przygotowywania i~spożywania posiłków było czyste.
\item Instruktor musi kontrolować proces przygotowywania posiłków przez harcerzy.
\item Kadra powinna spożywać posiłki wraz z~harcerzami, a~nie osobno.
\item Przy korzystaniu z~lokali gastronomicznych zwrócić szczególną uwagę na warunki w~jakich przygotowywane są posiłki.
\end{checklist}
\subsubsection{Terminowość wydawania posiłków [3.7]}
Regularne odżywianie się to podstawa na obozach. Da się to zrobić także na obozach wędrownych. Zgodnie z~planem dnia, codziennie: śniadanie o~tej samej porze, potem regularne przerwy na szlaku, podczas których oboźny sprawia, że uczestnicy jedzą prowiant i~piją napoje, po wędrówce obiad, a~potem kolacja albo obiado--kolacja, codziennie o~tej samej porze. Dopilnowanie terminowości wydawania posiłków spoczywa na barkach oboźnego.
\paragraph{$\bigstar$ Podsumowanie}
\begin{checklist}
\item Starać się, aby posiłki były wydawane codziennie o~tej samej porze.
\item Unikać opóźnień w~wydawaniu posiłków.
\item Harcerze nie mogą być głodni.
\end{checklist}
\subsubsection{Zgodność posiłków z~jadłospisem [3.8]}
\begin{figure}[htp]
\centering
\includegraphics[scale=0.09431787188559634380067816600324]{IMAG0406.jpg}
\caption{Niewiele warta mieszanka.}\label{fig:ogorki-z-pomidorami}
\end{figure}
Raporty z~wizytacji pokazują smutną prawdę: jadłospisy nie są prawie w~ogóle przestrzegane, w~większości przypadków powstają tylko na potrzeby zatwierdzenia obozu. To pokazuje niski stan świadomości o~normach żywienia wśród kadr obozów. Jadłospis jest po to, żeby żywienie było racjonalne. Skład posiłków powinien uwzględniać potrzebę dostarczenia różnych wartości w~zależności od wysiłku i~temperatury. Jadłospis umożliwia rozłożenie potraw tak, aby posiłki nie były monotonne. Rzeczywistość pokazuje, że harerze na obozach jedzą to, co się akurat da najtaniej kupić, czyli ciągle kanapki z~pasztetem i serkiem topionym doprawione kaczupem, dożywiają się zupkami ,,chińskimi'' i~kanapkami z~dżemem, a~na obiady pochłaniają góry makaronu lub ryżu z~sosami. Rzadko gdzie pojawia się ser żótły, wędliny, pomidory, ogórki, melony czy płatki na mleku. Natomiast musli, twaróg, serki twarogowe (np. wiejski lub almette) czy ryby, choćby z~puszki, kakao, kawa zbożowa, sałata pojawiają się w~śladowych ilościach. Posiłki powinny spełniać normy żywienia właściwe dla specyfiki obozów wędrownych i~być bogate w~potas i~magnez. Powszechną praktyką jest kupowanie tylko i~wyłącznie białego pieczywa. To naprawdę nie jest trudne, aby na stole znalazło się codziennie pieczywo razowe lub graham. Biały chleb powinien być stosowany, kiedy nie ma żadnego innego. Dodatkowo należy uważać, aby nie przesadzić z~ilością i~mieszaniem dodatków, jak np. słynnym już i~powszechnym mieszaniem pomidorów z~ogórkami na jednej kanapce lub w~jednej sałatce w~celu ,,dostarczenia zajdujących się w~nich witamin''\footnote{Ogórki oprócz walorów smakowych i~wody posiadają znikome ilości witamin A, B, C i~E. Askorbinaza, zawarta w~zaledwie łyżeczce soku z~tego warzywa, zniszczy całą witaminę C w~aż trzech litrach soku z~pomidorów. Dlatego nie należy ich łączyć. Na szczęście z~każdej sytuacji jest jakieś wyjście. Ogórki i~pomidory lepiej jadać osobno, a~do sałatek czy kanapek dodawać konserwowe lub kiszone. Ogórków lepiej też nie podawać na kolację, bo uważane są ogólnie za ciężkostrawne.}.~~\newpage
\paragraph{$\bigstar$ Podsumowanie}
\begin{checklist}
\item Przygotować przed obozem sensowny jadłospis i~przestrzegać go.
\item Unikać monotonii w~żywieniu.
\item Nie jeść ciągle białego chleba z~dżemem, pasztetem, serkiem topionym i~keczupem.
\item Stosować dodatki: pomidory, ogórki, sałatę, ale ich nie mieszać.
\item Kupować nabiał, ryby, wędliny, żółty ser i~inne.
\end{checklist}
\subsection{Bezpieczeństwo [4]}
Pokaż mi co potrafisz, a~powiem Ci\ldots
\subsubsection{Telefony kontaktowe [4.2]}
W~kartach obozowych podane są kontaktowe numery telefonów do kadry obozu. Robi się to po to, aby w~każdej chwili można było nawiązać kontakt z~danym obozem. Może to być potrzebne w~różnych sytuacjach, jak rozmowy z~rodzicami uczestników, informacje o~zagrożeniach itp. Ze względu na fakt wykorzystywania tych numerów także w sytuacjach zagrożenia przez instytucje i~służby, gdzie obóz został zgłoszony, ważne jest, aby wyznaczony instruktor miał przy sobie cały czas włączony telefon o~tym numerze. Co więcej, ze względów bezpieczeństwa nie należy tego telefonu wyłączać ani wyciszać na noc. Bardzo dużo drużyn wykorzystuje taki numer jako jedyny do kontaktów uczestników z~rodzicami, zakazując uczestnikom korzystać, a~nawet posiadać, na obozie swoich telefonów.
\paragraph{$\bigstar$ Podsumowanie}
\begin{checklist}
\item Telefony kontaktowe do obozu muszą być dostępne cały czas.
\item Nie wyłączać tych telefonów na noc.
\item Wyznaczyć instruktora odpowiedzialnego za te telefony.
\end{checklist}
\subsubsection{Znajomość regulaminów przez uczestników [4.3]\label{regulaminy}}
Na początku każdego obozu uczestnicy i~kadra zapoznają się z~różnymi regulaminami. Na obozie wędrownym najlepszym momentem na to jest czas podróży na początek trasy obozu, przeważnie w~pociągu czy autokarze. Jest to dogodna chwila, kiedy wszyscy uczestnicy zebrani są siłą rzeczy w~jednym miejscu. Najgorszą możliwą formą przedstawienia regulaminów jest ich przeczytanie przez kogoś z~kadry. Można to uatrakcyjnić w~jakiś sposób, aby harcerze się w~to zaangażowali --- znowu inwencja instruktora jest tutaj kluczowa. Samo przeczytanie regulaminów to dopiero początek, lecz niestety prawie wszystkie drużyny na tym kończą. Ważne jest jeszcze wyrywkowe sprawdzenie znajomości tych regulaminów za pomocą prostej gry. Samo przeczytanie skutkuje nieznajomością regulaminów, co bardzo łatwo zauważy każdy wizytator. Wystarczy choćby rzut oka na to w~jaki sposób drużyna porusza się po drodze i~jak jest oznakowana. Okazuje się, że kadra czyta uczestnikom regulaminy, ale sama ich nie przestrzega. Często uczestnicy dobrze znają regulaminy, nawet potrafią wyrecytować niektóre punkty z~pamięci, jednak nikt nigdy nie zastosował się do nich na obozie. Składając swój podpis pod regulaminami, każdy kto jest na obozie, a~zwłaszcza kadra, zobowiązuje się do ich przestrzegania.
\paragraph{$\bigstar$ Podsumowanie}
\begin{checklist}
\item Zapoznać uczestników z~regulaminami podczas podróży na obóz, jeszcze w~pociągu czy autokarze.
\item Regulamin przedstawić w~jak najbardziej atrakcyjny sposób.
\item Co jakiś czas sprawdzać wyrywkowo znajomość regulaminów organizując proste gry.
\end{checklist}
\subsubsection{Apteczki [4.4]}
Apteczki są nieodłącznym wyposażeniem każdego obozu, albowiem wypadki, kontuzje i~zachorowania się zdarzają. Kadra musi zadbać o~przygotowanie apteczek jeszcze przed obozem, zaczynając od sporządzenia listy wyposażenia. Apteczki mogą skompletować sami harcerze, ale wtedy, przed wyruszeniem na obóz, trzeba je koniecznie skontrolować. Ze względu na prawdopodobny lub planowany podział uczestników na grupy na obozie musi być kilka apteczek zawierających podstawowe materiały opatrunkowe i~środki dezynfekcyjne, tak aby każda grupa była wyposażona w~jedną. Wyposażenie apteczek musi być regularnie sprawdzane i~uzupełniane w~trakcie obozu --- warto powierzyć taki obowiązek jednej osobie z~kadry. W~przeciwnym wypadku może okazać się, że czegoś będzie brakować kiedy akurat będzie potrzebne. Ostatnie wizytacje pokazują, że drużyny bazują na apteczkach kupowanych z~wyposażeniem i~nikt do nich nie zagląda dopóki nie są potrzebne --- niestety są one skromnie wyposażone, nawet słowo ,,jednorazowe'' tutaj nie pasuje. Trzeba je koniecznie uzupełnić. Ze względu na specyfikę obozów wędrownych, zwłaszcza górskich, w~każdej takiej aptecze powinny się znaleźć dwa bandaże elastyczne, których używamy gdy ktoś nadwyręży kostkę czy kolano, a~mieszczuchom niewiele ruszającym się w~ciągu roku zdarza się to bardzo często.
\begin{figure}[htp]
\centering
\includegraphics[scale=0.2]{apteczka_plecakowa_tatonka_first_aid_pack_1_4458.jpg}~\includegraphics[scale=0.4]{apteczka_janysport1.jpg}
\caption{Główną apteczkę najlepiej przygotować w~formie małego plecaka, który podczas wędrówek można przytroczyć, a~w~czasie wycieczek łatwo i~bez konieczności przepakowywania, można zabrać ze sobą.}\label{fig:apteczka-plecak}
\end{figure}


Oprócz tego każdy obóz powinien posiadać jedną apteczkę ,,główną'' stanowiącą źródło wyposażenia dla apteczek grup, ale także posiadającą dodatkowe, jak np. termometr, maści rozgrzewające itd.

We wszystkich apteczkach muszą być utrzymane idealny porządek i~czystość, a~ponadto powinny one być zabezpieczone przez zamoknięciem w~czasie deszczu czy przeprawy przez rzekę.
\paragraph{$\bigstar$ Podsumowanie}
\begin{checklist}
\item Na obóz zabrać kilka mniejszych apteczek dla grup i~jedną większą --- ,,główną''.
\item Sporządzić listę wyposażenia apteczek.
\item Skontrolować zawartość wszystkich apteczek przed wyjazdem.
\item Sprawdzać stan apteczek podczas obozu i~uzupełniać braki.
\item Dbać aby apteczki nie zamokły i~były czyste.
\end{checklist}
\subsubsection{Wyciąg z~kart obozowych [4.5]\label{wyciag-z-kart}}
Posiadając komplet wypełnionych kart obozowych kadra powinna sporządzić na swój użytek wyciąg informacji m.~in. o~uczuleniach, przyjmowanych stale lekach itd. Każda osoba z~kadry powinna posiadać przy sobie jego kopię, aby w~razie dziwnych objawów czy wypadku można było zawsze szybko tam zajrzeć i~zasięgnąć informacji. Informacje te szczególnie przydają się podczas planowania jadłospisu lub robienia zakupów (np. można uniknąć kupienia czekolady na drogę podczas gdy połowa uczestników jest na nią uczulona) albo po użądleniu przez osę czy szerszenia nie czekać na obrzęk i~zapaść tylko wieźć od razu do szpitala osobę na to uczuloną. Podczas posiłków można łatwo kontrolować czy uczestnicy przyjmujący stale leki zażyli je.
\paragraph{$\bigstar$ Podsumowanie}
\begin{checklist}
\item Przygotować wyciąg informacji z~kart obozowych m.~in. o~uczuleniach, przyjmowanych stale lekach itd.
\item Zrobić kopię tego wyciągu dla każdej osoby z~kadry.
\item Zaglądać do wyciągu podczas planowania jadłospisu, robienia zakupów, posiłków, po użądleniu.
\end{checklist}
\subsubsection{Karta zabiegów i~urazów [4.6]}
Przepisy regulujące prowadzenie obozów nakładają na kadrę obowiązek prowadzenia ,,Karty zabiegów i~urazów''. Wystarczy do tego zwykły zeszyt 16-kartkowy, który najlepiej przechowywać razem z~długopisem w~apteczce. Po udzieleniu jakiejkolwiek pomocy, kiedy użyte zostało cokolwiek z~apteczki, w~tym właśnie zeszycie należy po prostu zapisać datę, czas, miejsce, dane osoby, której udzielona została pomoc oraz w~prostych słowach co rozpoznano (np. otarcie kolana na skutek przewrócenia się) i~jak wyglądała pomoc (np. ranę umyto czystą wodą z~butelki, zdezynfekowano wodą utlenioną, zaopatrzono gazą jałową i~bandażem). Na koniec wyraźny podpis osoby, która udzielała pomocy. To nie jest nic trudnego, a~dzieki temu jest kontrola i~wiadomo potem co się na obozie stało. W~razie wizytacji będzie można uargumentować na co zostały zużyte środki z~apteczki. Ponadto taki zeszyt stanowi źródło informacji dla rodziców uczestników po obozie, jak równeż dostarcza danych statystycznych przed kolejnym obozem, dzięki czemu wiadomo np. że potrzeba więcej bandaży elastycznych, bo członkowie drużyny są bardziej niż przeciętnie podatni na kontuzje kolan.
\begin{figure}[htp]
\centering
\includegraphics[scale=0.10803075396825396825396825396825]{P7210091.JPG}\\
~~~~~~~~\\
\includegraphics[scale=0.10803075396825396825396825396825]{P7220121.JPG}
\caption{Nareszcie można coś wpisać do karty zabiegów i~urazów: ,,3~sierpnia~2008 15:43, strumyk na szlaku Komańcza--Chryszczata, Marta~J. Uraz: obcięcie fragmentu pięty prawej stopy podczas przechodzenia boso na drugą stronę strumienia. Pomoc: ranę umyto wodą z~butelki, zwisający na płacie skóry fragment pięty odcięto nożyczkami, ranę zdezynfekowano wodą utlenioną i~zaopatrzono kilkoma warstwami gazy jałowej i~bandażami. Zalecenia: zmieniać opatrunek co kika godzin, a~do końca obozu rana się zagoi. Podpis: Maciej~L.''}\label{fig:pieta}
\end{figure}


W~zeszycie tym warto zapisywać również informacje o~wizytach u~lekarza lub pobytach w~szpitalu. Oczywiście w~tych dwóch ostatnich przypadkach niezwłocznie należy powiadomić rodziców pacjenta.
\paragraph{$\bigstar$ Podsumowanie}
\begin{checklist}
\item Założyć ,,Kartę zabiegów i~urazów'' w~zwykłym zeszycie 16-kartkowym.
\item Trzymać ten zeszyt razem z~długopisem w~apteczce.
\item Zapisywać w~nim informacje związane z~wydaniem lub wykorzystaniem jakiegokolwiek elementu wyposażenia apteczki.
\end{checklist}
\subsubsection{Przestrzeganie wyznaczonej trasy [4.8]}
Zgłaszając obóz do zatwierdzenia, kadra przedstawia plan trasy na każdy dzień. Plany są tylko planami i~w~razie konieczności komendant obozu może podjąć decyzję o~zmianie trasy, a~nawet o~zmianie planowanej kwatery na inną. Zdarzyło się już nie raz, że harcerze mieli przemoczone wszystkie ubrania, było zimno i~nie zdążyły one wyschnąć do rana, a~deszcz padał nadal i~harcerze pozostawali na kolejną dobę na tej samej kwaterze, chociaż według planu powinni kontynuować wędrówkę, albo na spływie zostawali na cały dzień na tym samym polu namiotowym, żeby nie płynąć w~deszczu. Takie pojedyncze odstępstwa są często konieczne. Podobnie rzecz się ma z~tzw. zamienianiem programu, co często pociąga za sobą właśnie zmianę trasy i~celu wędrówki. Komendant jest na obozie i~może podjąć tego typu decyzję, zwłaszcza, że suma sumarum plan zostanie wypełniony, tylko w~innej kolejności lub innym miejscu. Nie ma jednak zgody na samowolne skracanie lub drastyczne zmiany trasy z~mało racjonalnych powodów, tak jak na pewnym obozie, który miał wędrować po Beskidzie Niskim 10~dni, a~niespodziewanie został napotkany przez wizytatora w~Krakowie, gdzie od siódmego dnia harcerze nie robili nic konkretnego.
\paragraph{$\bigstar$ Podsumowanie}
\begin{checklist}
\item Starać się przestrzegać trasy podanej w~zatwierdzonym planie obozu.
\item W~uzasadnionych przypadkach komendant może zmienić trasę lub miejsce noclegu.
\item O~wszelkich większych zmianach komendant powinien poinformować Koordynatora Akcji Letniej w~Okręgu.
\item Wszelkie patologie są niedopuszczalne.
\end{checklist}
\subsubsection{Pogoda [4.9] \label{pogoda2}}
Prognozę pogody trzeba sprawdzać codziennie. W~górach pogoda jest bardzo zmienna, więc warto to robić kilka razy dziennie, bo można nie tylko zmoknąć czy ,,usmażyć się na słońcu'', ale wręcz narazić uczestników obozu na niebezpieczeństwo:
\begin{description}
\item[wiatr] \hfill \\ Silny wiatr utrudnia poruszanie się, zwiększa utratę ciepła (wychłodzenia i~odmrożenia), obniża sprawność fizyczną i~psychiczną, zimą zwiększa zagrożenie lawinowe, powoduje zawieje, zamiecie i~tworzy zaspy na szlakach.
\item[mgła] \hfill \\ Powoduje problemy z~orientacją w~terenie, zwiększa utratę ciepła, narasta presja psychiczna.
\item[temperatura powietrza] \hfill \\ Niska: utrata ciepła, wychłodzenia i~odmrożenia. Wysoka: przegrzanie organizmu.
\item[opady deszczu lub śniegu] \hfill \\ Zwiększają utratę ciepła, utrudniają orientację, pogarszają warunki w~terenie.
\item[burze] \hfill \\ Towarzyszy im pogorszenie pogody i~groźne dla życia wyładowania atmosferyczne.
\item[słońce] \hfill \\ Możliwość przegrzania organizmu i~poparzeń.
\end{description}
Gdy zbliża się burza, należy unikać przebywania na otwartych przestrzeniach, pod pojedynczymi lub wysokimi drzewami, na graniach, w~okolicach metalowych przedmiotów takich jak łańcuchy, klamry, drabinki. Unikać także strumieni, stawów. Odległość od burzy można obliczyć mnożąc czas pomiędzy błyskiem, a~hukiem przez 330 --- wynik w~metrach lub przez 0,33 --- wynik w~kilometrach.
\\
\\
Jak już było wspomnine w~rozdziale \ref{pogoda} na stronie \pageref{pogoda}, prognozę pogody należy sprawdzać w~profesjonalnych serwisach internetowych, np. ICM.
\begin{figure}[htp]
\centering
\includegraphics[scale=0.39882765531062124248496993987976]{meteorogram.png}
\caption{Meteorogram z~ICM.}\label{fig:meteorogram}
\end{figure}
Numeryczna prognoza pogody w~serwisie \href{http://meteo.icm.edu.pl}{http://meteo.icm.edu.pl} pokazuje nowoczesne podejście do prezentowania danych dotyczących tego co się w~pogodzie powinno stać według prognoz. Po wejściu na główną stronę można wybrać prognozę pogody o~długości 48 lub 84 godz. na siatce 4 km lub 13 km odpowiednio. Po dokonaniu wyboru ukazuje się mapa, na której trzeba zaznaczyć interesujące miejsce. Można też wybrać z~listy jedno z~miast wojewódzkich albo wpisać współrzędne geograficzne. Pokazany zostanie meteorogram opatrzony legendą, która pomaga zrozumieć zastosowaną notację.
\\
\\
\small{
\emph{Na pewnym obozie w~Bieszczadach kadra zaplanowała bardzo długie przejście na ostatni dzień obozu. Trasa wiodła z~Wołosatego czerwonym szlakiem do Przełęczy Bukowskiej. Stamtąd dalej czerwonym szlakiem przez Rozsypaniec, Halicz, Kopę Bukowską do Przełęczy pod Tarnicą\footnote{\href{http://www.twojebieszczady.pl/mapa-on/biesz2.php}{http://www.twojebieszczady.pl/mapa-on/biesz2.php}}. Stamtąd wejście i~zejście na Tarnicę żółtym szlakiem. Następnie niebieskim szlakiem przez Bukowe Berdo do wsi Widełki, gdzie kadra chciała zorganizować nocleg w~zaznaczonej na mapie bazie namiotowej. Kadra nie wzięła pod uwagę, że trasa jest za długa, w~związku z~czym uczestnicy nie mieli wystarczającej ilości prowiantu ani napojów, a~od rana było upalnie. W~okolicy Kopy Bukowskiej drużyna natrafiła na kilka sączących się strumyczków, z~których nabrała wody do butelek. Podczas dalszej wędrówki niektóre osoby zaczęły się skarżyć na bóle brzucha. Na Tarnicy drużyna podzieliła się na dwie grupy: zdrowych i~cierpiących na ból brzucha. Zdrowi byli niecierpliwi i postanowili kontynuować wędrówkę. Druga grupa zdecydowała się odpocząć na Tarnicy i~poczekać aż bóle brzucha same przejdą i~dopiero wtedy kontynuować wędrówkę. Odpoczynek przekształcił się w~drzemkę. Cała grupa obudziła się nagle, gdy poczuła gwałtowne podmuchy chłodnego wiatru. W~czasie ich drzemki pogoda się zmieniła, niebo zaciągnęło się chmurami, zrobiło się zimno. Harcerze biegiem rzucili się w~kierunku Bukowego Berda, gdzie dogoniła ich burza z~piorunami gęsto trafiającymi w~okoliczne drzewa. Zrobiło się ciemno, zwłaszcza, że minęło kilka godzin i~zbliżał się zmierzch. Strach spowodował, że w~niezłym tempie znaleźli się cali mokrzy i~zziębnięci w~Widełkach, gdzie przy zejściu ze szlaku czekała na nich pierwsza grupa. Tutaj czekała kolejna przykra niespodzianka. Studencka baza namiotowa zaznaczona na mapie okazała się jednym namiotem typu NS, bez okien, rozbitym na błotnistym klepisku, bez żadnego wyposażenia. Wewnątrz namiotu leżał wielki wilczur, przywiązany łańcuchem do stelaża namiotu. Personelu bazy nie udało się znaleźć ani wieczorem, ani następnego dnia rano. Wilczur okazał się na szczęście łagodny, więc harcerze tłocząc się wokół niego, z~dala od okien ułożyli się do snu. Po noclegu na zimnym błocie drużyna wyglądała jak siedem nieszczęść. Ponieważ burza skończyła się w~nocy i~ranek był już słoneczny, udało im się doprowadzić do trochę lepszego stanu. Drużyna opuściła to miejsce z~ulgą, tak samo szybko jak się w~nim zjawiła. Ignorancja kadry sprowadziła niebezpieczeństwo na uczestników:
\begin{itemize}
\item Trasa była za długa.
\item Uczestnicy mieli za mało prowiantu i~napojów więc pili wodę z~napotkanych ,,źródełek''.
\item Zdrowi zostawili cierpiących na ból brzucha samych.
\item Kadra nie sprawdziła prognozy pogody i~nie kontrolowała jej stanu w~trakcie wędrówki.
\item Odpoczywający na Tarnicy zasnęli na kilka godzin i~nie zauważyli załamania pogody.
\item Zaskoczeni deszczem i~burzą biegli wyznaczonym szlakiem zamiast schować się w~goprówce pod Tarnicą lub zejść niebieskim szlakiem do Wołosatego.
\item Kwatera nie była zarezerwowana ani nawet telefonicznie sprawdzona czy jest czynna.
\item Nocleg w~namiocie był ryzykowny ze względu na wilczura.
\end{itemize}}}

\paragraph{$\bigstar$ Podsumowanie}
\begin{checklist}
\item Sprawdzać prognozę pogody kilka razy dziennie w~profesjonalnym serwisie meteo.
\item Nie wierzyć bezkrytycznie w~prognozy i~obserwować zmiany pogody.
\item Pogoda bywa zmienna, zwłaszcza w~górach.
\item Odpowiednio reagować przy ekstremalnych warunkach pogodowych.
\end{checklist}
\subsubsection{Sprzęt do oznaczania kolumny w~marszu [4.10]}
O~konieczności wyposażenia kolumny w~marszu w~chorągiewki białą i~żółtą, latarkę białą i~czerwoną, kamizelki odblaskowe wszyscy wiedzą z~regulaminów obozowych, o~których wspomniane jest w~punkcie \ref{regulaminy} na stronie \pageref{regulaminy}, jednak warto tutaj podkreślić to jeszcze raz. Każdy kierowca potwierdzi, że niewłaściwie idący pieszy stwarza zagrożenie w~ruchu drogowym. Jeśli jest to wieczorową porą albo nocą, zagrożenie niepomiernie wzrasta. Harcerze w~większości preferują ciemne kolory odzieży, co jeszcze pogarsza sytuację. Nie wolno tego lekceważyć. Kierowcy w~ogólności jeżdżą coraz gorzej, poziom umiejętności jest zły, jest wielu pijanych. Oświetlenie kolumny w~marszu może uratować komuś życie. To nie jest żadna kosztowna inwestycja --- wystarczą zwykłe lampki ze sklepu rowerowego za kilkanaście złotych sztuka. Większy problem będzie ze zdobyciem chorągiewek potrzebnych do oznaczenia kolumny w~marszu, dlatego najlepiej wykonać je własnym sumptem na wiosnę, na długo przed obozem. Przydadzą się także na obozach stałych, rajdach, wycieczkach itp. Chociaż czasem nawet poprawne oznakowanie nie pomoże, jak w~wypadku opisanym na stronie \pageref{pijany-wjechal-w-kolumne}. Jeśli tylko to możliwe lepiej unikać poruszania się po drogach i~nie dać się zabić.
\paragraph{$\bigstar$ Podsumowanie}
\begin{checklist}
\item Wykonać lub kupić i~zabrać na obóz chorągiewki do oznaczania kolumny w~marszu.
\item Kupić światełka do oznaczania kolumny w~marszu.
\item Pamiętać i~korzystać z~tego sprzętu podczas poruszania się po drogach.
\item Jeśli to tylko możliwe unikać poruszania się po drogach o~zbyt dużym natężeniu ruchu.
\item Nie ryzykować i~nie dać się zabić przez piratów drogowych.
\end{checklist}
\subsection{Dzień obozowy [5]}
Pokaż mi co się dzieje, a~powiem Ci\ldots
\subsubsection{Rozkład dnia [5.1]}
Nieprzestrzeganie zaplanowanego rozkładu dnia na obozach wędrownych wiąże się głównie z~tendencją do pomijania pewnych czynności oraz do przedłużania innych. Problem wynika wprost z~lenistwa i~niekonsekwencji kadry. Główny ciężar pilnowania rozkładu dnia spoczywa na oboźnym. Kompetentny instruktor pełniący tę funkcję nie pozwoli nikomu na pominięcie ani przedłużenie zaplanowanej czynności i~doprowadzi do jej wykonania lub przerwania. Z~drugiej strony problemu jest kwestia czy zatwierdzony rozkład jest wykonalny i~odpowiada realiom obozów wędrownych. Najczęstsze błędy i~problemy to:
\begin{description}
\item[pomijanie porannej rozgrzewki] usprawiedliwiane często zmęczeniem, brakiem celowości ćwiczeń fizycznych, skoro uczestnicy męczą się wędrując,
\item[pomijanie poranego apelu] usprawiedliwiane zmęczeniem, brakiem ,,istotnych spraw, o~których warto wspominać w~rozkazie'', niechęcią do musztry,
\item[opóźnianie posiłków] spowodowane zwłaszcza kiedy to zastęp kuchenny przygotowuje posiłki, kiedy nie zdecydowano się na ,,szwedzki stół'',
\item[przeciąganie czasu wędrówki --- często aż do zmroku] spowodowane nieumiejętną dystrybucją wysiłku i~odpoczynków lub po prostu przecenieniem sił albo wyznaczeniem zbyt ambitnej trasy,
\item[rezygnacja z~ogniska na rzecz ,,kominka''] spowodowane lenistwem, brakiem chęci do zbierania chrustu, brakiem pozwolenia na palenie ogniska na terenie kwatery,
\item[przedłużanie czasu ogniska] usprawiedliwione tym, że ,,jest fajnie'' i~wszyscy to lubią,
\item[opóźnianie rozpoczęcia ciszy nocnej] spowodowane brakiem dyscypliny,
\item[przedłużanie czasu na toaletę] spowodowane często zbyt małą liczbą łazienek (umywalni), albo usprawiedliwiane faktem, że punkt programu następujący po toalecie i~tak nie zacznie się o~czasie.
\end{description}
\paragraph{$\bigstar$ Podsumowanie}
\begin{checklist}
\item Przestrzegać zaplanowanego rozkładu dnia.
\item Oboźny jest panem i~władcą czasu na obozie.
\item Pozwalając sobie na tzw. ,,odpuszczenie sobie'' czegoś kadra uczy harcerzy niekonsekwencji.
\item Rozkład dnia na obóz wędrowny różni się od odpowiednika na obóz stały.
\end{checklist}
\subsubsection{Powitanie dnia [5.2]}
Każdy dzień zwyczajowo rozpoczynany jest pobudką. Konia z~rzędem temu oboźnemu, który umie to zrobić jak należy. Przeważnie spotyka się skrajności: przeraźliwy wojskowy gwizdek i~wrzaski, po których można dostać zawału albo delikatne lub wręcz błagalne prośby aby harcerze raczyli wstać. Harcerstwo to nie wojsko ani towarzystwo wzajemnej adoracji, ma swoje metody. Każde środowisko ma swoją tradycję przekazywaną z~pokolenia na pokolenie. Warto się zastanowić, czy nie czas na wzniesienie się na wyższy poziom i~wprowadzenie jakichś zmian. Grunt to nie przesadzać i~robić to z~wyczuciem. Na którym obozie można się spotkać z~sygnałem pobudki wygrywanym na trąbce? Byłoby wspaniale aby drużyny podjęły rękawicę i~spróbowały czegoś naprawdę sensownego. Można zacząć od gitary, skoro harcerscy trębacze już dawno wyginęli\ldots
\begin{figure}[htp]
\centering
%\includegraphics[scale=1.8228476821192052980132450331126]{p_treb1.jpg}
\includegraphics[scale=1.8]{p_treb1.jpg}
\caption[Cantin for LOP]{Pocztówka harcerska wydana w~Warszawie w~czerwcu 1915~r.\footnotemark}\label{fig:trebacz}
\end{figure}
\footnotetext{Pocztówka pochodzi z~serwisu \protect\href{http://kkraj.pttk.pl/bk-nks/nks02-07.htm}{http://kkraj.pttk.pl/bk-nks/nks02-07.htm}}

Po pobudce jest czas na powitanie dnia. Niemal każda drużyna ma swoje specyficzne obrzędy z~tym związane. Jak one zostaną przeprowadzone zależy od chęci oboźnego. Jeśli będzie robił to niechętnie, harcerze też nie będą widzieli w~tym sensu i~całość będzie wyglądała żałośnie. Spotyka się też często drużyny, które zastąpiły harcerskie obrzędy samymi modlitwami. Wygląda to sztucznie, wręcz faryzejsko, gdyż jak się potem okazuje nikt z~uczestniczących nie postępuje w~ten sposób poza obozem. Przesadzanie nie prowadzi do niczego dobrego.

Zaprawa poranna na obozie wędrownym nie powinna być długa ani wyczerpująca. Powinna być nastawiona tylko i~wyłącznie na delikatne rozciągnięcie zmęczonych mięśni. Wszelkie obciążające i~siłowe ćwiczenia doprowadzą szybko do kontuzji. Oboźny, czy inny instruktor prowadzący powinien sobie starannie przygotować zestawy ćwiczeń i~przeprowadzać je potem z~wyczuciem. Celem nie jest wykończenie harcerzy ani udowodnienie im, że są cieniasami, tylko rozbudzenie i~rozciągnięcie zakwaszonych, zmęczonych mięśni.

\paragraph{$\bigstar$ Podsumowanie}
\begin{checklist}
\item Nie robić wojskowej ani delikatnej pobudki.
\item Wykorzystać trąbkę lub gitarę i~wprowadzić piosenkę zamiast gwizdków, krzyków i~liczenia.
\item Odprawić poranne harcerskie obrzędy, nie pomijać ich ani nie zastępować samymi modlitwami.
\item Przeprowadzać obrzędy na wysokim poziomie, aby harcerze znali ich wartość.
\end{checklist}
\subsubsection{Apel [5.3]}
Po porannej toalecie zwyczajowo przychodzi czas na apel. Niektóre drużyny przeprowadzają go wieczorem, co jest też akceptowalne. Z~apelu nie powinno się rezygnować. Choćby tylko ze względu na tradycję, budowanie tożsamości drużyny oraz fakt, że jest on wpisany w~plan dnia. Przygotowania do apelu powinny trwać krótko i~obejmować sprawdzenie umundurowania i~porządku na kwaterze. Miejsce na apel powinno być godne, czyste i~wystarczająco duże. Niemal w~każdych warunkach lepiej jest wyjść na zewnątrz. Na apelu harcerze powinni być w~mundurach, ostatecznie innych jednolitych strojach, jeśli obowiązują one na obozie. Oprócz meldowania zastępami powinien być odczytany rozkaz zawierający choćby tylko sensowne informacje dotyczące planów na dany dzień oraz służb. Odśpiewanie Hymnu Harcerskiego to harcerska powinność i~przywilej. Obecność bandery na obozach wędrownych to rzadkość, ale jest to jak najbardziej wykonalne. Wystarczy się postarać i~przygotować. Apel musi być przeprowadzony sprawnie i~trwać naprawdę krótko. Nie jest to miejsce na ćwiczenie musztry --- można ją ćwiczyć kiedy indziej, jeśli tylko nauka musztry znalazła się w~zatwierdzonych celach obozu. Swoją drogą regulamin musztry, jest kolejnym, o~którym wszyscy wiedzą, że jest, ale mało kto go czytał, a~jeszcze mniej osób go przestrzega. Nawet podstawowe komendy są podawane i~wykonywane źle. Kto z~instruktorów odważy się zmienić nawyki i~nauczy swoją drużynę przed obozem idealnie wykonywać choćby te elementy musztry wykorzystywane podczas apeli i~przemarszów?
\begin{figure}[htp]
\centering
\includegraphics[scale=0.09438255726272668445737449796512]{musztra.jpg}
\caption{Przygotowania do apelu.}\label{fig:musztra}
\end{figure}
\paragraph{$\bigstar$ Podsumowanie}
\begin{checklist}
\item Robić codziennie apel w~mundurach.
\item Przygotowania i~sam apel przeprowadzać szybko i~sprawnie.
\item Odczytywać na każdym apelu rozkaz, choćby zawierał tylko informacje dotyczące planów na dany dzień oraz służb.
\item Śpiewać Hymn Harcerski.
\item Nie ćwiczyć musztry na apelu --- nauczyć jej przed obozem.
\end{checklist}
\subsubsection{Sprawność przygotowań [5.4]}
Po śniadaniu obóz zwykle rozpoczyna przygotowania do wymarszu. Nadzór nad tym procesem to kolejna domena oboźnego, co nie znaczy, że pozostałe osoby z~kadry nie powinny mu pomagać --- bynajmniej: cała kadra musi się do tego włączyć aby wszystko było przeprowadzane sprawnie. Każdy powinien wiedzieć co ma robić. Celem jest wykonanie tych przygotowań w~jak najkrótszym czasie, ale po harcersku, bez krzyków. Kwaterę trzeba opuścić i~wyjść jak najwcześniej na szlak, zgodnie z~rozkładem dnia albo i~wcześniej, jeśli zrobione zostało wszystko co powinno było zostać zrobione (nie tak jak w~przykładzie ze strony \pageref{kolowrotek}). W~górach dodatkowo motywuje do tego fakt, że latem burze najczęściej zdarzają się popołudniu i~lepiej ich unikać. Dodatkowo na trasie mogą się zdarzyć niespodzianki, które opóźnią wędrówkę i~uniemożliwią dotarcie do celu i~zakwaterowanie przed zmierzchem.
\begin{figure}[htp]
\centering
\includegraphics[scale=0.68387395736793327154772937905468]{most-niespodzianka.jpg}\\
~~~~~~~~\\
\includegraphics[scale=0.10794979079497907949790794979079]{przechodzenie-po-drzewie.jpg}
\caption{Niespodzianka: zerwany most na szlaku. Znalezienie innego przejścia może zająć sporo czasu.}\label{fig:blondyni}
\end{figure}
\paragraph{$\bigstar$ Podsumowanie}
\begin{checklist}
\item Przygotowania do wymarszu przeprowadzać sprawnie.
\item Oboźny dowodzi, pozostali instruktorzy mu pomagają.
\item Każdy musi wiedzieć co ma robić.
\item W~górach latem burze najczęściej zdarzają się popołudniu i~lepiej ich unikać.
\item Na trasie zdarzają się niespodzianki, które opóźniają wędrówkę i~uniemożliwiają dotarcie do celu i~zakwaterowanie przed zmierzchem.
\end{checklist}
\subsubsection{Sprawność na szlaku [5.5]}
Na szlaku harcerze powinni poruszać się sprawnie jako zgrana grupa dowodzona przez instruktora. To na barkach opiekuna grupy leży odpowiedzialność aby tak się stało. Niewskazane jest poruszanie się w~luźnych grupach, rozsypanych po szlaku poza zasięgiem wzroku instruktorów. Lepiej trzymać grupy w~zwartym szyku. Na czele każdej grupy stawiać należy najsłabsze osoby i~maruderów. Jeśli takich nie ma, to co jakiś czas i~tak lepiej zmienić osobę prowadzącą, aby dać jej odpocząć i~dać szansę innym. Każdy harcerz powinien dostać mapę do ręki i~przez jakiś czas prowadzić, inaczej nie dane mu będzie poczuć jak to jest ani nie będzie miał okazji pomylić drogi i~zabłądzić. Nie dopuszczać aby harcerze wlekli się noga za nogą, nie powinni też forsować się za bardzo, choćby nawet chcieli.
\paragraph{$\bigstar$ Podsumowanie}
\begin{checklist}
\item Tempo poruszania się powinno być odpowiednie do wieku uczestników i~zgodne z planem wędrówki.
\item Obóz ,,w~marszu na szlaku'' powinien wyglądać porządnie.
\end{checklist}
\subsubsection{Sprawność kwaterunkowa [5.7]}
Po przybyciu do nowej kwatery oboźny ma kolejną szansę udowodnienia swojej wartości. To on, po ewentualnym uzgodnieniu koniecznych szczegółów z~drużynowymi, wydaje polecenia dotyczące kto gdzie i~co ma robić. Harcerze po kilku dniach powinni się do tego przyzwyczaić i~wyrobić w~sobie nawyk wykonywania tych poleceń bez oglądania się na innych. Oboźny musi mieć w~głowie sprytny plan zakwaterowania, a~harcerze i~pozostała kadra powinni mu w~tym sprawnie pomagać. Jeżeli obóz kwateruje w~namiotach, ich rozstawianie powinno przebiegać sprawnie, harcerze powinni umieć rozstawiać (i~składać) namioty, których używają. Obóz to nie czas ani miejsce na uczenie tak podstawowych rzeczy. Namioty powinny być transportowane, rozstawiane i~składane przez harcerzy, którzy w~nich śpią, nie inaczej, gdyż nigdy się nie nauczą o~nie dbać, sprzątać, suszyć itd. W~razie kłopotów ze sprawnością kwaterunkową można wprowadzić nagrody lub przywileje dla zastępów za najlepszy czas lub najmniejszy chaos.
\paragraph{$\bigstar$ Podsumowanie}
\begin{checklist}
\item Zakwaterowanie obozu po przybyciu na miejsce noclegu musi przebiegać krótko i~sprawnie.
\item Oboźny dowodzi.
\item Wszyscy realizują polecenia oboźnego.
\item Harcerze rozstawiają zawsze ten sam namiot, w~którym śpią --- powinni być tego nauczeni przed obozem.
\end{checklist}
\subsubsection{Czas na pranie odzieży i~konserwację wyposażenia [5.8]}
Niewiele obozów w~swoim rozkładzie dnia przeznacza czas, kiedy uczestnicy mogą wyprać ubrania, umyć i~zaimpregnować buty, wyczyścić inny sprzęt jak np. rowery, kajaki. Nie może to być symboliczne 5~minut, ale ilość wystarczająca, aby można było dokładnie wykonać te czynności. Celem jest wyrobienie w~harcerzach nawyku dbania o~własne rzeczy i~sprzęt drużyny. Kadra powinna w~tym czasie robić\ldots to samo co harcerze. Własny przykład działa najlepiej i~oboźny nie będzie musiał używać swojego daru przekonywania.
\paragraph{$\bigstar$ Podsumowanie}
\begin{checklist}
\item Wyznaczyć w~rozkładzie dnia osobny czas na pranie ubrań, mycie i~impregnowanie butów, czyszczenie sprzętu, np. rowerów, kajaków.
\item Czasu musi być wystarczająco dużo aby każdy uczestnik miał szansę zdążyć.
\item Kadra obozu w~tym czasie powinna robić to samo co harcerze, razem z~nimi.
\item Oboźny powinien pilnować czy ten czas jest wykorzystywany zgodnie z~intencją.
\end{checklist}
\subsubsection{Czas wolny [5.9]}
\begin{figure}[htp]
\centering
\includegraphics[scale=0.11974522292993630573248407643312]{zajecia-w-czasie-wolnym.jpg}
\caption{W~czasie wolnym harcerze powinni robić coś konkretnego, kadra na tym spływie również\ldots}\label{fig:czas-wolny}
\end{figure}
Na wielu obozach pokutuje pojęcie czasu wolnego zapożyczone z~letnich kolonii. Z~definicji uczestnicy robią wtedy co chcą, a~kontrola nad nimi nie istnieje. Jeżeli komukolwiek udało się zatwierdzić rozkład dnia obozowego, zawierający taką pozycję, powinien głęboko się zastanowić czy to jest ,,coś co harcerze lubią najbardziej''. Obóz harcerski to nie kolonia i~nie powinno się dawać uczestnikom zupełnie wolnej ręki, co nie znaczy, że powinni być krótko trzymani przez cały czas. Jeżeli zaplanowano taki czas z~myślą o~chwili wytchnienia, należy zastosować zasadę ograniczonego wyboru, czyli dać harcerzom kilka opcji czynności, które mogą robić, np. gra w~koszykówkę, pobyt w~saunie, pisanie listów czy cokolwiek innego, byle konkretnego. Nie po to uczestnicy wyjechali na obóz, aby robić nie wiadomo co.
\paragraph{$\bigstar$ Podsumowanie}
\begin{checklist}
\item Jeśli istnieje tzw. czas wolny dla uczestników obozu nie powinien trwać zbyt długo.
\item Stosować zasadę ograniczonego wyboru konkretnych czynności, które harcerze mogą robić.
\item Na obozie harcerskim czas jest zorganizowany w~sposób harcerski, nie kolonijny.
\end{checklist}
\subsubsection{Ognisko [5.10]}
Wydawało by się, że o~ważności obrzędowego ogniska nie trzeba harerzom nic tłumaczyć. Niestety, smutna rzeczywistość pokazuje, że ogniska są tak samo często pomijane jak gimnastyka poranna czy apel. Kadra powinna uzyskać zgodę zarządcy kwatery na palenie ogniska oraz zorganizować wszystko co potrzebne, aby ognisko mogło się odbyć. Jeśli nie jest to możliwe należy zorganizować ,,kominek'' przy świecach. Jak wiadomo, po całym dniu wiosłowania, pedałowania czy wędrowania uczestnicy ,,lecą z~nóg'', dlatego ogniska powinny być krótkie. Harcerze nie powinni na nich zasypiać. Kadra powinna zawczasu mieć przygotowany program każdego ogniska, wybrane piosenki i~wymyśloną gawędę. Prowadzenie chaotyczne i~,,na żywioł'' pokazuje harcerzom, że instruktorzy są po prostu nieprzygotowani. Jest to kolejny moment, kiedy zły przykład niszczy efekty harcerskiego wychowania. Na obozie powinna być gitara, właśnie ze względu na ogniska. Bez gitary nie ma klimatu. Innym elementem tworzącym klimat ogniska jest tematyka, dlatego ważne jest przygotowanie piosenek związanych z~obozem. Jeśli obóz jest w~górach wybierać należy piosenki o~górach, jeśli na Mazurach --- o~Mazurach, jeśli to obóz żeglarski --- szanty. Na koniec ogniska zostawić należy czas na dwie--trzy dowolne piosenki zaproponowane przez harcerzy. Nie bać się uczyć piosenek, przygotować wydruki z~tekstem i~akordami i~codziennie śpiewać. Gawędy muszą być krótkie, najlepiej w~postaci krótkich historyjek również związanych z~rejonem obozu lub z~tematem przewodnim danego dnia. Nie mogą to być opowieści oderwane od kontekstu obozu. Góry, jak i~każdy inny rejon, mają swoje lokalne legendy --- należy je znać i~wykorzystać.
\begin{figure}[htp]
\centering
\includegraphics[scale=1.02]{e-gitarra.jpg}
\caption{Gitarę można spróbować pożyczyć od tubylców. Bez gitary nie ma nastroju.}\label{fig:e-gitarra}
\end{figure}
\paragraph{$\bigstar$ Podsumowanie}
\begin{checklist}
\item Codziennie urządzać krótkie obrzędowe ogniska (w~ostateczności kominki).
\item Przygotować program każdego ogniska i~prowadzić je sprawnie, nie chaotycznie ani~na żywioł.
\item Zapewnić akompaniament gitary lub innego instrumentu.
\item Przygotować i~wygłaszać sensowne gawędy w~kontekście rejonu lub tematyki obozu.
\item Uczyć śpiewać piosenek związanych z~rejonem lub tematyką obozu --- przygotować wydruki z~tekstem i~akordami.
\end{checklist}
\subsubsection{Zakończenie dnia [5.11]}
Zakończenie dnia jest lustrzanym odbiciem powitania. Podobnie jak wtedy, drużyny odprawiają swoje specyficzne rytuały --- i~bardzo dobrze. Ważne jest aby nie trwały one zbyt długo ani nie przejawiały jakichś skrajności --- harcerze muszą rozumieć ich znaczenie i~czuć ich wartość. Oboźny powinien przeprowadzać je sprawnie, jeśli jest coś do powiedzenia, powinno to zostać wyartykuowane krótko i~szybko. Sporo drużyn porzuciło wieczorne harcerskie obrzędy na rzecz modlitw oraz rytuałów rodem z~oazy, jak zbiorowa spowiedź itd. I~tak jak przy powitaniu dnia --- chęci szlachetne, ale wykonanie niewłaściwe. Jeśli ludzie są zmęczeni, słaniają się na nogach nie ma sensu robić pewnych rzeczy na siłę. Często naprawdę wystarczy Modlitwa Harcerska.
\paragraph{$\bigstar$ Podsumowanie}
\begin{checklist}
\item Odprawić wieczorne harcerskie obrzędy, nie pomijać ich ani nie zastępować samymi modlitwami lub długimi rytuałami.
\item Przeprowadzać obrzędy na wysokim poziomie, sprawnie, aby harcerze docenili ich wartość.
\end{checklist}
\subsubsection{Rada obozu [5.12]}
Rada obozu to zjawisko prawie niespotykane. Z~rozmów podczas wizytacji wynika, że instruktorzy nie widzą sensu w~kolejnym ,,formalnym'' spotkaniu, wieczorem, kiedy są zmęczeni. To kolejny stereotyp. Kto powiedział, że to spotkanie musi być formalne i~późno wieczorem? Chodzi o~naprawdę krótką rozmowę kadry, podczas której odbywa się głównie wymiana informacji na temat problemów i~konfliktów wśród kadry, uczestników. To miejsce i~czas, gdzie każdy instruktor może otwarcie przyznać, że sobie z czymś nie radzi, że ma żal czy pretensję do kogoś o~coś, albo może stwierdzić, że coś wyszło znakomicie i~pochwalić się tym (lub kogoś). Wtedy buduje się wartość i~siła grupy instruktorów danego obozu. Rezygnując z~tego, kadra pozbawia się potężnego narzędzia działającego na jej spójność. Jeśli spotkania Rady są już organizowane, to przeważnie po rozpoczęciu ciszy nocnej, kiedy wszystkim chce już się spać. Dobrym pomysłem jest przeniesienie tego spotkania na czas wieczornej toalety albo nawet przygotowania do kolacji. Chodzi o~znalezienie takiej pory, kiedy główny program dnia obozowego praktycznie został wykonany i~zostały same ,,rutynowe'' czynności, a~jednocześnie na tyle wczesnej, aby instruktorzy nie zasypiali na siedząco. W~znakomitej większości przypadków czas takiego spotkania powinien się zamknąć w~15~minut. Można sobie nawet wyznaczyć pewien reżim: korzystać z~minutnika, zobowiązać każdego do odpowiedzi na 3 pytania: co według niego dzisiaj poszło dobrze, co poszło źle i~jakie zauważył problemy. W~ten sposób każdy będzie zorientowany w~odczuciach innych instruktorów, a~jeśli zidentyfikowany zostanie jakiś problem, komendant zarządzi osobne spotkanie poświęcone jego rozwiązaniu, na którym już nie będą musieli być wszyscy obecni.
\paragraph{$\bigstar$ Podsumowanie}
\begin{checklist}
\item Organizować codziennie krótkie spotkania Rady Obozu, prowadzone przez komendanta.
\item Unikać spotkań po ogłoszeniu ciszy nocnej.
\item Jeśli jest coś do przekazania zastępowym nie robić tego po ogłoszeniu ciszy nocnej, tylko następnego dnia rano.
\item Wszelkie problemy zgłaszać śmiało i~otwarcie na forum Rady.
\item Sensowne spotkania instruktorów budują tożsamość i~wzmacniają wartość kadry jako grupy.
\end{checklist}
\subsection{Wędrówki [6]}
Pokaż mi jak wędrujesz, a~powiem Ci\ldots
\subsubsection{Świadomość trasy i~celu [6.2]}
Po opuszczeniu kwatery przed wędrówką zwyczajowo następuje zbiórka, na której sprawdza się czy wszyscy są obecni, czy wszystko zostało zabrane oraz w~jakim stanie są plecaki, sakwy i~rowery, worki i~kajaki itd. Jest to moment, kiedy wszyscy powinni zostać szczegółowo poinformowani o~trasie i~celu danego dnia. Należy rozdać mapy wszystkim grupom, oraz krok po kroku wytłumaczyć wszystkim, którędy trzeba wędrować, jechać czy płynąć. Robi się to po to, aby każdy miał świadomość dokąd powinien się udać w~przypadku zabłądzenia czy jakiegokolwiek innego zdarzenia skutkującego rozbiciem grupy, co się czasem zdarza. Podczas tłumaczenia każdy musi mieć przed oczami mapę. Trzeba pokazać punkt początkowy dziennego etapu, a~potem kolejno opisać drogę wymieniając koniecznie kolory szlaków oraz opisując każde ich skrzyżowanie, zwracając szczególną uwagę na takie, gdzie zmienia się kolor szlaku. Końcowa informacja zawierać powinna opis punktu docelowego. Przy okazji ustalić należy czas i~miejsca spotkań wszystkich grup. Postępuje się tak ze względów bezpieczeństwa, gdyż trasa wiedzie przez nieznany teren, gdzie oznakowanie się zmienia albo mogło zostać zniszczone lub zatarte przez czas.
\\
\\
\small{
\emph{Na pewnym obozie w~Beskidzie Niskim okazało się, że została wycięta bardzo duża połać lasu, co do jednego drzewa --- wraz z~tymi, na których były namalowane znaki szlaku. Harcerzom zajęło około godziny ustalenie, którędy dalej powinni iść.}}
\paragraph{$\bigstar$ Podsumowanie}
\begin{checklist}
\item Rozdać mapy wszystkim grupom.
\item Przed wyruszeniem na szlak pokazać wszystkim na mapie cel i~przebieg trasy dziennego etapu.
\item Na spływie mapy muszą być wodoodporne, przynajmniej jedna na dwa kajaki.
\item Podać kolejno kolory wszystkich szlaków i~omówić wszystkie skrzyżowania ze zmianą koloru szlaków.
\item Ustalić czas i~miejsca spotkań wszystkich grup.
\end{checklist}
\subsubsection{Rozkład grup w~czasie [6.3] i~przerwy [6.4]}
Jeżeli obóz podzielony jest na grupy, odstępy między nimi nie powinny być zbyt duże. Ze względów bezpieczeństwa każda grupa powinna być w~stanie dotrzeć szybko do poprzedzającej lub następnej. Podczas wędrówek, jazdy rowerem, spływu kajakiem zdarzają się kontuzje i~inne nieprzewidziane wypadki. Często grupą opiekuje się jeden instruktor i~w~razie czego nie może on być pozbawiony wsparcia. Praktyka pokazuje, że unikać należy odstępów dłuższych niż pół godziny, a~grupy powinny się spotykać na szlaku co mniej więcej półtorej godziny. W~terenie zasięg telefonii komórkowej bywa różny, nie tak jak w~mieście, dlatego zawsze przed wyruszeniem trzeba ustalić z~instruktorem grupy prowadzącej miejsce i~za ile czasu będą czekać na pozostałe grupy. Podczas oczekiwania na wolniej poruszające się grupy, oprócz odpoczynku, należy zagospodarować sensownie czas harcerzom. Jeżeli pozwoli się im cały czas odpoczywać łatwo doprowadzić można do sytuacji kiedy grupa prowadząca nie dość, że będzie najszybsza, to jeszcze najbardziej wypoczęta. Cały szkopuł tkwi w~tym, aby tę grupę czymś dodatkowo zmęczyć, aby uniknąć frustracji harcerzy z~pozostałych grup. Z~braku pomysłów można nawet w~tym czasie zarządzić zbieranie jagód czy malin, a~grupa szybko się połapie, że nie warto tak gnać do przodu. Chyba, że taki cel przyświeca drużynowemu --- wtedy to zupełnie inna historia. Jeżeli grupa czeka ponad 45 mint na kolejną i~nie może się z~nią skontaktować telefonicznie, powinna wyruszyć jej naprzeciw, bo taka zwłoka przeważnie oznacza jakieś kłopoty.

Oprócz momentów kiedy spotykają się wszystkie grupy, konieczne są krótsze przerwy zarządzane przez przewodnika, oboźnego lub opiekuna grupy. Nie należy się znęcać nad uczestnikami i~za wszelką cenę dążyć do pokonania jakiegoś etapu bez przerw. Praktyka pokazuje, że przerwy zarządzane są na 10 minut co 45--50 minut. W~czasie przerwy opiekun musi dopilnować aby uczestnicy coś zjedli oraz koniecznie wypili odpowiednią ilość napojów. Zalecić można pozycję leżącą z~uniesionymi lekko nogami, aby dać odpłynąć krwi. Jeśli jest zimno nie pozwalać zdejmować butów. W~górach ani na obozach rowerowych nie powinno się robić przerw na podejściach, gdyż świadomość konieczności kontynuowania drogi pod górę działa deprymująco i~naprawdę ciężko jest po takiej przerwie ruszyć grupę z~miejsca. Wybierać należy miejsca osłonięte od wiatru, podczas upału --- w~cieniu, w~chłodny dzień --- na słońcu.

Na spływach kajakowych należy uformować specyficzny szyk, który zwiększa bezpieczeństwo uczestników. W~pierwszym i~ostatnim kajaku powinien płynąć ktoś z~kadry --- np.~przewodnik jako pierwszy, a~komendant ostatni. Oboźny powinien zmieniać swoje miejsce, podobnie jak drużynowi, ze względu na aspekty opisane w~następnym rozdziale. Kajaki muszą być w~zasięgu wzroku ratownika, który zazwyczaj płynie w~środku szyku --- jednak nie za blisko, żeby na siebie nie wpadać na przeszkodach. Ratownik nie może być zbyt daleko, gdyby ktoś się topił. Harcerze będący na początku szyku powinni koniecznie informować tych z~tyłu o~przeszkodach zauważonych w~wodzie.
\paragraph{$\bigstar$ Podsumowanie}
\begin{checklist}
\item Unikać odstępów między grupami dłuższych niż pół godziny.
\item Grupy powinny spotykać się na szlaku co mniej więcej półtorej godziny.
\item Jeśli oczekiwanie trwa dłużej niż 15~minut od ustalonego odstępu w~czasie, wyczekująca grupa powinna wyjść naprzeciw.
\item Podczas czekania na inne grupy, oprócz odpoczynku należy zagospodarować sensownie czas harcerzom.
\item Konieczne są krótkie przerwy co 45--50 minut.
\item Nie robić przerw na podejściach.
\item Wybierać miejsca na odpoczynek w~zależności od pogody.
\end{checklist}
\subsubsection{Formacja na szlaku [6.6]}
Ustawienie harcerzy w~grupie najczęściej wynika z~tradycji drużyny i~potrzeby chwili. Są drużyny, którę zawsze poruszają się rzędem. Inne preferują szyk wolny, jeszcze inne tworzą mniejsze grupki idące dwójkami. Nie ma w~tym nic złego, dopuki się to sprawdza. Jeśli w~grupie są słabsze osoby, a~reszta grupy im dokucza, najlepszym sposobem jest właśnie utworzenie rzędu czy nawet kolumny dwójkowej, jeśli jest miejsce. Sposobem na to jest postawienie najsłabszych na czele, co ucina wszelkie dyskusje. Nawet jeśli grupa porusza się w~luźnej formacji, opiekun powinien doprowadzić do tego, aby najsłabsi szli na początku --- dzieki temu unika się ,,ucieczki'' do przodu silniejszych osób i~uczucia bezsilności, przeradzającej się z~czasem we frustrację tych słabszych. Praktyka pokazuje, że harcerze wędrujący w~kolumnie dwójkowej czy rzędzie bardzo rzadko ze sobą rozmawiają, a~jeśli już wydobywają z~siebie jakiś głos są to przyśpiewki marszowe. Nie o~to jednak chodzi na obozie wędrownym. Przemarsze w~formalnym szyku można uskuteczniać podczas krótkich przejść, podczas poruszania się po ruchliwej drodze, czy nocnych wypraw, kiedy potrzebny jest spokój i~kontrola nad wszystkimi. Natomiast na szlaku, w~lesie, na łące potrzebne jest zupełnie co innego. To jest niepowtarzalna okazja do formowania harcerzy. Drużynowy czy opiekun grupy powinien znaleźć czas aby porozmawiać codziennie z~każdym swoim harcerzem, właśnie w~trakcie wędrówki. To nie muszą być rozmowy na śmiertelnie poważne tematy, ale większość z~nich powinna dotyczyć rozwoju danego harcerza, a~zwłaszcza jego formacji duchowej. To jest sposobność wpojenia harcerzom zamiłowania do wędrowania, poznania jego wartości, docenienia sensu wysiłku, zmęczenia i~wytrwałości. Wtedy właśnie nawiązuje się najbardziej wartościowa więź między harcerzem, a~jego drużynowym, która utrzyma go w~drużynie, umocni jego harcerstwo i~chęć służby, może nawet być zalążkiem zmiany czy języczkiem u~wagi jego przyszłości, nie tylko jako harcerza, ale przede wszystkim człowieka. Nie da się porozmawiać o~osobistych sprawach maszerując w~szyku --- nikt nie będzie chciał się otworzyć mając świadomość, że pozostali będą słuchać tego mimo woli.
\paragraph{$\bigstar$ Podsumowanie}
\begin{checklist}
\item Formalny szyk stosować podczas poruszania się po ruchliwych drogach, nocnych wypraw, kiedy potrzebny jest spokój i~kontrola nad wszystkimi.
\item Na szlaku zezwolić na szyk luźny, ale taki aby mieć wszystkich w~zasięgu wzroku.
\item W~trakcie wędrówki drużynowy powinien znaleźć czas na codzienną, osobistą, rozmowę z~każdym swoim harcerzem --- nawiązuje się wtedy najbardziej wartościowa więź między harcerzem, a~jego drużynowym.
\item Traktować wędrówkę jako szczególną okazję do formacji duchowej harcerzy, a~nie zwykły marsz.
\end{checklist}
\subsubsection{Wygląd na szlaku [6.7]}
\begin{figure}[htp]
\centering
\includegraphics[scale=0.60384615384615384615384615384615]{blondyni1.jpg}~\includegraphics[scale=0.60384615384615384615384615384615]{blondyni2.jpg}
\caption{Zapomniało się o~koszulkach? Nie szkodzi: przecież można kupić szampon koloryzujący dla każdego.}\label{fig:blondyni}
\end{figure}
Na obozy wędrowne koniecznie trzeba zabrać pełne umundurowanie, jednak nie należy używać go podczas codziennego pedałowania, wiosłowania czy wędrowania. Mundurów używać trzeba podczas apeli, mszy św. czy innych oficjalnych wyjść, można~także podczas zwiadów terenowych i~podobnych akcji. Oczywiście nie znaczy to, że harcerze na szlaku mogą wyglądać jak luźna grupa turystów. Przed obozem warto pomyśleć o~jednolitych strojach, np. mundurach polowych albo koszulkach drużyny. Bardzo sprytnym pomysłem jest podanie kolorów koszulek, które uczestnicy i~kadra powinni wziąć na obóz i~każdego dnia polecać wszystkim nałożenie koszulek w~wybranym kolorze. Sam strój to oczywiście nie wszystko. Bardzo ważne jest zachowanie, zwłaszcza podczas marszu w~luźnym szyku. Trzeba zwracać uwagę na otoczenie, nie zakłócać spokoju innym, np. podczas przejścia obok cmentarza, kościoła czy przydrożnego krzyża, przy~którym akurat odprawiane jest nabożeństwo.
\\
\\
\small{
\emph{Na pewnym obozie harcerze wracali główną ulicą miejscowości na kwaterę. Szli ustawieni w~kolumnę dwójkową śpiewając bardzo głośno przyśpiewki marszowe. Ich repertuar rozciągał się od ,,Siała baba mak'' aż do ,,ZHR to potęga'', jak to zwykle bywa. Im bliżej było kwatery, tym poziom przyśpiewek się obniżał. Kiedy już cała miejscowość wiedziała ,,jak się bawi ZHR'', do instruktora prowadzącego podbiegł ubrany na czarno mężczyzna i~ze skrajnym wzburzeniem zażądał ,,zamknięcia się'', ponieważ na mijanym właśnie cmentarzu żałobnicy nie byli w~stanie kontynuować ceremonii pogrzebowej ze względu na wrzaski harcerzy. Cisza zapadła momentalnie --- jak ,,makiem zasiał''\ldots}}
\\
\\
Należy pamiętać, że obóz jest gościem lokalnych społeczności. Nie można wchodzić w~ich środowisko z~butami i~burzyć ludziom spokoju --- nie o~taki wizerunek chodzi.
%\begin{figure}[htp]
%\centering
%\includegraphics[scale=0.063529412]{wygl-na-szlaku-2.jpg}
%\caption{Harcerze powinni wyglądać porządnie.}\label{fig:wygl}
%\end{figure}
\begin{figure}[htp]
\centering
\includegraphics[scale=0.10794979079497907949790794979079]{na-drodze.jpg}
\caption{Chorągiewka sygnałowa na końcu, odblask, gwizdek i~jednolite koszulki --- o~to chodzi.}\label{fig:na-drodze}
\end{figure}
\paragraph{$\bigstar$ Podsumowanie}
\begin{checklist}
\item Nie używać mundurów podczas codziennych wędrówek, bo są niepraktyczne i~szybko się zniszczą.
\item Postarać się o~umundurowanie polowe lub jednolite koszulki.
\item Jednolite czapki lub chustki to atrybut grup kolonijnych i~wycieczek szkolnych.
\item Zwracać uwagę na otoczenie, nie zakłócać spokoju innym.
\item Harcerze na szlaku powinni wyglądać jak zorganizowana grupa.
\end{checklist}
\subsubsection{Ubiór [6.11]}
Odpowiedni ubiór harcerzy i~instruktorów podczas wędrówek to kolejny aspekt, który zdobywa przeważnie negatywne oceny podczas wizytacji w~ostatnich latach. Okazuje się, że harcerze wychodzą na szlak ubrani często jak chcą, według własnego uznania, oboźni nie zajmują się sprawdzeniem czy ubiór jest odpowiedni do aktualnych warunków pogodowych. Kadra powinna przed wymarszem sprawdzić w~co ubrali się uczestnicy, a~co najważniejsze, sama powinna świecić przykładem w~tej kwestii. Jeżeli kadra będzie lekceważyć tę sprawę, harcerze również nie będą do tego przykładać wagi. Nie może być tak, że jest upał, a~harcerze ubrani są w~ciepłe polary, albo podczas deszczu mokną, bo nikt nie założył kurtki. Gdy słońce praży oboźny musi doprowadzić do tego aby każdy miał jakieś nakrycie głowy i~posmarował się kremem z~dużym filtrem UV. Gdy jest zimno lub wieje zimny wiatr harcerze nie powinni chodzić w~krótkich spodenkach, na spływie kajakowym powinni mieć cały czas nałożone kamizelki ratunkowe i~rękawiczki ochronne, a~na obozie rowerowym --- kaski, rękawiczki lub nawet ochraniacze.

W~chwili zmiany warunków atmosferycznych instruktor bez ociągania powinien zatrzymać grupę i~polecić zmianę odzieży, np. nałożenie kurtek gdy zaczęło padać, zdjęcie polarów gdy zrobiło się gorąco, posmarowanie się kremem gdy zaczęło prażyć słońce. Sam powinien zrobić to samo, jako pierwszy --- kadra przecież daje właściwy przykład, nieprawdaż?
\\
\\
\small{
\emph{Na pewnym obozie dawno temu, harcerze wędrowali kolejny dzień w~ulewnym deszczu przez bukowy las. Nastroje były beznadziejne, morale dawno upadło. Wszystkie ubrania mieli przemoczone, również te, które były w~plecakach. Coraz więcej osób się załamywało, zaczynało płakać. W~pewnym momencie oboźny wykonał coś, co przeszło do legendy w~drużynie. Kiedy deszcz się nasilił i~wszystkich opuściła już nadzieja na poprawę pogody, oboźny zarządził postój, kazał harcerzom (a~właściwie zmusił ich) zdjąć wszystkie mokre ubrania i~zostać w~samej bieliźnie, rozwiesić między drzewami, w~wybranych miejscach tropiki od namiotów i~zbierać olbrzymie ilości chrustu. Zostało rozpalone olbrzymie ognisko, przy którym przez kilka godzin mogli się ogrzać. Udało im się także wysuszyć część ubrań i~mogli je nałożyć. Napili się też gorącej herbaty i~coś zjedli. Atmosfera się poprawiła, morale wzrosło. Następnego dnia rano w~zupełnie innych nastrojach kontynuowali wędrówkę mimo deszczu, bo wiedzieli, że są w~stanie przetrwać.}}
\paragraph{$\bigstar$ Podsumowanie}
\begin{checklist}
\item Ubiór harcerzy i~instruktorów podczas wędrówek musi być dostosowany do aktualnych warunków atmosferycznych.
\item Kadra musi dawać odpowiedni przykład i~kontrolować ubiór uczestników.
\item W~chwili zmiany warunków atmosferycznych zatrzymać grupę i~polecić zmianę odzieży.
\item W~czasie deszczu mieć na sobie kurtki.
\item Gdy jest upał nałożyć nakrycie głowy i~posmarować się kremem z~dużym filtrem UV.
\item Gdy jest zimno lub wieje zimny wiatr nie chodzić w~krótkich spodenkach.
\item Na spływie kajakowym mieć cały czas nałożone kamizelki ratunkowe i~rękawiczki ochronne.
\item Na obozie rowerowym używać kasków, rękawiczek, a~nawet ochraniaczy.
\end{checklist}
\subsection{Sprawy organizacyjne [7]}
Pokaż mi jak jest, a~powiem Ci\ldots
\subsubsection{Obowiązkowość kadry [7.3]}
Każda osoba, która znalazła się w~kadrze obozu, musi mieć jakiś zakres obowiązków, inaczej nie ma żadnego uzasadnienia, aby w~kadrze była. Kiedy tylko powstanie pomysł zorganizowania obozu drużynowi zaczynają kompletowanie kadry od komendanta, kwatermistrza, następnie ktoś zgadza się zostać oboźnym, instruktorem programowym. W~końcu zostają jeszcze zatrudnieni ratownik, przewodnik lub pielegniarka. Konstelacje są różne, od 3--osobowego minimum do newet 10--osobowej ekipy. Kiedy skład jest znany należy jednoznacznie zdefiniować zakres obowiązków dla każdej osoby, aby każdy wiedział co i~kiedy musi robić. Nikt jednak nie powinien ograniczać się tylko i~wyłącznie do swojej ,,działki'', ale ofiarować swoją pomoc, kiedy ktoś inny jej potrzebuje. Od razu umówić się trzeba, że obowiązki będą wykonywane z~pełnym zaangażowaniem i~poświęceniem, w~przeciwnym wypadku coś na obozie będzie kulało. Obóz jest szkołą życia dla harcerzy, dla kadry również --- uczestnicy obozu będą się uczyć nowych rzeczy, utrwalać już zdobyte umiejętności, podobnie kadra, musi starać się z~całych sił, aby być skuteczna, musi uczyć się od siebie i~doskonalić to, co już umie. Aby to wszystko faktycznie zadziałało, każda osoba z~kadry musi czuć, że to jest ,,jej obóz'', musi mieć wizję i~pomysł na swoją rolę na nim, musi orientować się w~planie obozu i~w~dokładnym rozkładzie każdego dnia obozowego. Bez tego będzie tylko figurantem, któremu komendant będzie musiał ciągle przypominać i~mówić co w~danym momencie powinien robić. Każda osoba z~kadry musi być samodzielna, musi znać ważność swojej roli i~wypełniać ją od początku do końca, musi śmiało i~zdecydowanie podejmować decyzje i~konsekwentnie się ich trzymać.

Poniżej znajduje się przykładowa, niekompletna lista obowiązków, które są standardowo ,,przypisane'' do funkcji na obozie:
\begin{description}
\item[komendant:] nadzór nad bezpieczeństwem uczestników i~kadry, podejmowanie strategicznych decyzji, zwoływanie Rady Obozu i~kontrola wykonywania obowiązków przez innych członków kadry, nadzór nad finansami, pisanie rozkazów, podpisywanie ich, obecność na codziennym apelu,
\item[oboźni:] nadzór nad wykonywaniem rozkładu dnia przez wszystkich bez wyjątku, robienie pobudki, przeprowadzanie rozgrzewki porannej, nadzór podczas toalety porannej i~wieczornej, przygotowanie do apelu, wykwaterowanie i~zakwaterowanie obozu, robienie wszystkich zbiórek, zlecanie i~nadzór nad przygotowaniami do ogniska, nadzór i~zlecanie pracy służbie porządkowej, nadzór nad wartami,
\item[kwatermistrz:] nadzór nad gotówką i~wyżywieniem, zlecanie zakupów, nadzór nad zastępem kuchennym, uzupełnianie książki finansowej, sprawdzanie stanu gotówki, porównywanie struktury wydatków do założeń preliminarza,
\item[zaopatrzenie:] dokonywanie i~transport zakupów do obozu
\item[drużynowi:] praca z~harcerzami, realizacja programu obozu, prowadzenie rozwoju duchowego harcerzy, rozmowy formujące na szlaku, prowadzenie prób na sprawności i~stopnie, wspieranie na duchu w~chwilach zwątpienia, rozwiązywanie wszelkich problemów zgłaszanych przez harcerzy, wypełnianie książki pracy obozu, przygotowanie i~wygłaszanie gawęd na ogniskach,
\item[instruktorzy programowi, przyboczni:] pomoc drużynowym przy realizacji programu obozu, prowadzeniu gier,
\item[pielęgniarka:] wszelkiego rodzaju pierwsza pomoc medyczna, wypełnianie karty zabiegów i~urazów, nadzór nad apteczkami,
\item[ratownik:] nadzór nad wszystkimi i~podejmowanie decyzji podczas zajęć w~wodzie i~na~wodzie,
\item[przewodnik:] prowadzenie obozu trudnymi szlakami, w~trudnych warunkach, powyżej 1000 m. n.p.m., kontrola ubioru, poprawności ułożenia plecaków, stanu kondycji harcerzy, zarządzanie przerw i~posiłków na szlaku, wprowadzanie doraźnych korekt zaplanowanej trasy w~zależności od aktualnych warunków.
\end{description}
Oczywiście może być tak, że jedna osoba będzie pełnić więcej niż jedną funkcję, jednak będzie to zawsze ze szkodą dla obozu, ponieważ nadmierne obciążenie obowiązkami spowoduje zaniedbywanie tych, które nie są absolutnie niezbędne. Tylko funkcji komendanta i~kwatermistrza nie można łączyć.
\paragraph{$\bigstar$ Podsumowanie}
\begin{checklist}
\item Każda osoba, która znalazła się w~kadrze obozu, musi mieć jakieś obowiązki.
\item Obowiązki wykonywać z~pełnym zaangażowaniem i~poświęceniem.
\item Nikt nie powinien ograniczać się tylko i~wyłącznie do swojej ,,działki''.
\item Starać się być skutecznym, uczyć się od siebie i~doskonalić swoje umiejętności.
\item Każdy musi orientować się w~planie obozu i~w~dokładnym rozkładzie każdego dnia obozowego.
\item Być samodzielnym, znać ważność swojej funkcji, śmiało i~zdecydowanie podejmować decyzje i~konsekwentnie się ich trzymać.
\end{checklist}
\subsubsection{Służby obozowe [7.1]}
Służby obozowe to podstawowe trybiki sprawiające, że obóz działa mimo nielicznej kadry. Harcerze kierowani przez wyznaczonych instruktorów wykonują codziennie pracę na rzecz obozu. Służba kuchenna, porządkowa i~wartownicza, to trzy, z~którymi można się zetknąć niemal na każdym obozie. Ich konstelacja oraz zakres obowiązków różni się z~obozu na obóz i~nie ma w~tym nic złego. Ważne jest, aby powołać służby w~rozkazie na początku obozu oraz jasno zdefiniować co mają robić.

Służba kuchenna może robić na obozie wszystko co jest związne z~przygotowaniem i~wydawaniem posiłków, a~także sprzątaniem po nich. Jeśli nie ma wyodrębnionej służby zaopatrzenia --- także robienie i~przynoszenie zakupów na kwaterę. Odkąd drużyny zaczęły wprowadzać system ,,szwedzkiego stołu'' żywot zastępów kuchennych uległ znacznej poprawie --- nie muszą się już zrywać skoro świt i~przygotowywać gór kanapek na śniadanie.

Służba porządkowa jest wykorzystywana do utrzymania porządku podczas pobytu na kwaterze. Jeżeli oboźny jest w~stanie zapanować nad harcerzami, a~oni sami są nauczeni dbania o czystość, to taka służba jest zbędna.

Służbę wartowniczą powołać trzeba jedynie wtedy, kiedy na obozie jest sprzęt, którego nie można zabezpieczyć w~zamkniętym pomieszczeniu lub kiedy uczestnicy śpią w~namiotach. Taki sprzęt to np. kajaki na spływie czy rowery na obozie rowerowym. Tego po prostu należy pilnować ze względu na wysokie ryzyko kradzieży. Podczas noclegu pod namiotami warta jest konieczna ze względu na bezpieczeństwo uczestników. Warta ma na celu ostrzeżenie wszystkich przed jakimkolwiek niebezpieczeństwem czy wyrządzeniem szkód przez osoby trzecie. Najlepszy system wart na obozie wędrownym, to równy podział godzin od ogłoszenia ciszy nocnej do pobudki wśród wszystkich uczestników i~kadry, bez wyjątku. Przypada wtedy kilkanaście minut warty na osobę. Jest to czas tak krótki, że nie zaburzy wypoczynku.

Czegokolwiek taka służba by nie obejmowała musi to być jasno zdefiniowane, aby harcerze wiedzieli czego się od nich wymaga. Oprócz tego musi być jasno powiedziane kiedy dany zastęp zaczyna i~kończy służbę. Jeżeli grafik służb jest ułożony na początku obozu i~podany do wiadomości to bardzo dobrze. Zawsze też należy wymienić w~rozkazie, który zastęp pełni jaką funkcję danego dnia i~który zastęp będzie następny. Unikamy wtedy zaskakiwania uczestników i~sprawiania wrażenia, że ,,znowu im'' przypadła jakaś służba. Grafik służby ucina wszelkie tego typu wątpliwości, ponieważ każdy może sprawdzić kiedy jaką ma służbę i~jakby nie liczył zawsze mu wyjdzie, że podział jest sprawiedliwy.
\paragraph{$\bigstar$ Podsumowanie}
\begin{checklist}
\item Zdefiniować służby obozowe i~ich zakres obowiązków.
\item Na początku obozu ustalić i~ogłosić grafik kto kiedy pełni jaką służbę.
\item Codziennie wymieniać w~rozkazie, który zastęp pełni jaką funkcję danego dnia i~który zastęp będzie następny.
\end{checklist}
\subsubsection{Służba [7.5]}
Jak już wspomniane było powyżej, obóz wędrowny jest gościem lokalnej społeczności. Harcerze powinni zostawić coś po sobie oprócz opadającego kurzu i~gotówki w~mijanych sklepach. Obozy są po to, aby wypocząć, ale przede wszystkim są doskonałym miejscem, gdzie realizować można służbę na rzecz mieszkańców rejonu, przez który wiedzie trasa. Niestety, ostatnimi laty obozy przemykaja przez napotkane miejscowości niczym Struś Pędziwiatr. Naprawdę wystarczy chwila orientacji aby znaleźć kogoś, komu można pomóc, choćby przez chwilę. Trzeba i~tak być musi, inaczej co warte będzie to wszystko?
\\
\\
\small{
\emph{Na pewnym obozie rowerowym harcerze jechali drogą środkiem malutkiej wioski zapomnianej chyba przez wszystkich. Nagle jeden z~harcerzy zauważył staruszkę, która trzęsącymi się rękami malowała farbą płot. Po 30 minutach płot był pomalowany przez harcerzy, a~starsza pani nie posiadała się ze szczęścia.}}
\\
\\
Pomoc w~podobnych sytuacjach naprawdę niewiele kosztuje. W~planie dnia musi być zostawiony zapas na nieprzewidziane sytuacje, w~tym również na służbę.
\paragraph{$\bigstar$ Podsumowanie}
\begin{checklist}
\item Przewidzieć w~planie dnia czas na nieprzewidziane okazje do wykonania służby na rzecz lokalnej społeczności.
\item Wykorzystywać każdą nadarzającą się okazję do wypełnienia służby.
\item Od drużynowego zależy czy drużyna się czegoś podejmie czy nie, ponieważ może to całkowicie zrujnować plany na dany dzień.
\end{checklist}
\subsubsection{Pierwsze wrażenie [7.6]}
Każdy ,,obcy harcerz'', który napotkałby harcerski obóz wędrowny, powinien być w~stanie bezbłędnie poznać na pierwszy rzut oka, że to nie kolonia, wycieczka klasowa ani przypadkowa grupa.
\paragraph{$\bigstar$ Poniższe cechy powinny charakteryzować grupę harcerzy:} %$\bigstar$
\begin{checklist}
\item Dowodzi jedna osoba, pozostali są jej posłuszni.
\item Poruszają się w~zwartej grupie, często w~szyku.
\item Są podobnie ubrani, często mają jednolite koszulki.
\item Niosą lub wiozą ze sobą plecaki, sakwy czy inne bagaże.
\item Wiedzą co mają robić i~dokąd iść.
\item Nie hałasują, nie słychać wylgaryzmów.
\item Często śpiewają idąc.
\item Poproszeni o~pomoc chetnie jej udzielą, jeśli tylko nie przekracza ich możliwości.
\item W~sytuacji kryzysowej ruszają na pomoc innym.
\end{checklist}
\subsubsection{Atmosfera i~samopoczucie [7.9]}
Zdarza się, że pierwsze wrażenie jest pozytywne, ale szybko okazuje się, że na obozie jest drętwo. Kadra i~uczestnicy powinni się postarać aby stworzyć dobrą atmosferę, aby odczucia i~nastroje wśród uczestników i~kadry były jak najbardziej korzystne. Wypracowanie takiego ,,stanu ducha'' wymaga kilku dni, ale gwarantuje potem pozytywne podejście w~chwilach niepowodzeń, problemów i~zwątpienia. Nie jest możliwe osiągnięcie czegoś takiego bez odpowiednich relacji między kadrą, a~uczestnikami. Muszą być one oparte na wzajemnym szacunku i~zaufaniu. Jeśli na obóz jedzie jedna drużyna, to kłopotów z~tym nie powinno być, ale jeśli kilka, do tego jeszcze z~różnych środowisk, konflikty i~wzajemne animozje są gwarantowane i~całego obozu może nie starczyć na ,,dotarcie się'' grup. W~takich momentach otwartość i~szczerość są najlepszym co można zrobić. Jeśli dotyczy to kadry, spotkania Rady Obozu są właściwym miejscem. Czasami sytuacja jest tak napięta, że nie wolno czekać aż do spotkania Rady, tylko rozwiązać problem na miejscu bez udziału uczestników. Jeśli pokłócą się harcerze, ich drużynowi powinni dać im jak najszybciej szansę na samodzielne dogadanie się bez świadków, zanim uraz okrzepnie.

Jeżeli ,,coś się nie klei'' widać to od razu, poprzez wzajemne animozje między poszczególnymi grupami. Wtedy harcerze nie są radośni, nie śpiewają w~drodze, nie akceptują zgodnie pomysłów innych osób, idą w~milczeniu, często smutni i zniechęceni. Pobyt na takim obozie nie należy do przyjemnych i~kadra musi interweniować, bo skutki mogą być opłakane, włącznie z~odejściem niektórych osób z~drużyny po obozie. Ratowanie integralności ekipy może się odbyć nawet kosztem planu i~programu obozu. Czasem warto zostać na kwaterze jeden dzień dłużej, zorganizować gry integracyjne, porozmawiać z~harcerzami, zebrać od nich informacje --- może nawet anonimowo --- co im się nie podoba i~co proponują aby zostało na obozie zmienione. Te uwagi i~pomysły są bezcenne, ponieważ przez sam fakt ich zgłoszenia harcerze pokazują, że im zależy na polepszeniu czy uratowaniu sytuacji. W~takim ciężkim momencie drużynowi powinni sobie po prostu poprzestawiać priorytety, skonsultować się ze swoimi mentorami, a~jeśli takich nie mają, to z~hufcowymi i~porozmawiać rzeczowo o~problemie. Nie chodzi tu o~narzekanie, tylko o~sam fakt rozmowy, który sam w~sobie wymusza wyartykułowanie sytuacji w~zrozumiały sposób, co już jest początkiem jej wyjaśniania i~rozwiązywania. Instruktor na drugim końcu kraju prawie nigdy nie będzie miał gotowej recepty, ale może podpowiedzieć co on sam zrobił w~podobnych sytuacjach ze swoją drużyną w~innym czasie i~miejscu. Każdej drużynie zdarzyć się może jakiś kryzys i~różne mogą być jego przyczyny, nie tylko zła wola kogokolwiek, ale nawet tak prozaiczne jak deszcz padający przez cały obóz.
\paragraph{$\bigstar$ Podsumowanie}
\begin{checklist}
\item Stworzyć dobrą atmosferę na obozie aby odczucia i~nastroje były pozytywne.
\item Relacje między kadrą, a~uczestnikami muszą być oparte na wzajemnym szacunku i~zaufaniu.
\item Proces ,,docierania się'' różnych drużyn może trwać dłużej niż ich obóz.
\item Otwartość i~szczerość są najlepszym co można zrobić w~przypadku konfliktów.
\item Konflikty wśród kadry rozwiązywać na forum Rady Obozu, a~jeśli sprawa jest poważna, od ręki, bez udziału świadków.
\item Harcerzom dać szansę na samodzielne rozwiązanie sporów, na osobności, im szybciej tym lepiej.
\item Jeżeli na obozie ,,źle się dzieje'' koniecznie doprowadzić do polepszenia sytuacji, nawet kosztem planów, programu, wędrówek itd.
\item Przed podjęciem jakichkolwiek decyzji zebrać opinie od harcerzy i~porozmawiać z~zaufanym instruktorem spoza obozu.
\item Nawet najbardziej zgrane drużyny przeżywają czasem kryzys --- każdemu może się zdarzyć.
\item Braterstwo jest kluczem do rozwiązania.
\end{checklist}
\subsubsection{Podsumowanie i~ewaluacja obozu [7.11]}
Ostatnią ważną sprawą, której nie można pominąć jest podsumowanie obozu, jeszcze zanim kadra rozejdzie się do swoich domów. Bardzo dobrym momentem jest podróż powrotna z~obozu. Tak jak w~przypadku podpisywania regulaminów w~drodze na obóz, wszyscy są razem w~jednym pociągu czy autobusie. Nawet jeśli podróż nie jest długa, wystarczy. Trzeba to zrobić póki jeszcze wszyscy wszystko pamiętają, i~jeszcze żywe są emocje, a~wydarzenia nie zatarły się w~pamięci żadnego z~instruktorów. Ewaluację przeprowadzać najłatwiej na zasadzie zwykłej retrospekcji. Najpierw każda osoba musi podać listę spraw, które według niej zostały wykonane nieprawidłowo, albo jej się nie podobały. Kontunuować ,,narzekanie'' do momentu aż nikt już nie będzie miał nic do dodania. Następnie wykonać podobny manewr wymieniając pozytywne aspekty obozu i~te momenty, które najbardziej się podobały. Należy przy tym zastanowić się czy i~w~jakim zakresie udało się zrealizować założone cele obozu. Bogatym źródłem tematów jest książka pracy obozu, jeśli była rzetelnie wypełniana. Wszystko to należy zapisać np. na dwóch osobnych kartkach, jako minusy i~plusy, bez powtórzeń. Gdy obie listy są gotowe przychodzi kolej na przeprowadzenie krótkiej dyskusji na temat każdego punktu, aby każdy wiedział co zgłaszająca osoba miała dokładnie na myśli. Na samym końcu odbywają się dwa osobne głosowania najpierw nad sprawami ocenionymi negatywnie, potem tymi pozytywnie, z~których każde należy przeprowadzić w~następujący sposób: każda osoba dostaje do dyspozycji 10 głosów, które może dowolnie przyznać każdej z~wymienionych spraw. Może przyznać jakiejś sprawie nawet wszystkie swoje głosy albo rozdzielić po jednym na 10 spraw. Każdy powienien głosować na sprawy, które uważa za najbardziej istotne. Po obu głosowaniach należy wybrać z~obu list po 10 aspektów, które ,,zdobyły'' najwięcej głosów. ,,Zwycięskie minusy'' to tematy, na które trzeba zwrócić uwagę, lepiej dopracować, albo zrobić zupełnie inaczej podczas przygotowań do kolejnego obozu, jak i~na samym obozie. Wybrane ,,plusy'' są tymi, które koniecznie trzeba utrzymać, rozwinąć, jeszcze lepiej przygotować i~powtórzyć na kolejnym obozie. Na tym kończy się ewaluacja obozu. Wypada jeszcze aby komendant podziękował kadrze za służbę na obozie. Obie listy z~wybranymi tematami --- każdy koniecznie opatrzony kilkoma zdaniami komentarza i~liczbą głosów --- po obozie, należy przepisać na czysto, dołączyć do dokumentacji obozowej, rozesłać każdemu członkowi kadry, a~przede wszystkim pokazać i~przedyskutować z~hufcowym. Przykładowa lista minusów zamieszczona jest w~załączniku \ref{ewaluacja-minusy} na stronie \pageref{ewaluacja-minusy}.

Oczywiście spotkanie w~pociągu czy autobusie nie jest jedynym słusznym sposobem. Można to zrobić zupełnie inaczej. Istotne, aby to zrobić jak najszybciej, pod koniec obozu albo zaraz po nim. Większość drużyn, jesli w~ogóle, robi to w~formie spotkania poobozowego, jednak to nie jest to samo i~mało co z~takich spotkań wynika.
\\
\\
\small{
\emph{Na pewnym obozie kadra obozu urządziła ewaluację w~saunie, w~ostatni dzień obozu. Można więc sprawić, aby to spotkanie było oryginalne i~miało ,,mniej formalną'' formę.}}
\paragraph{$\bigstar$ Podsumowanie}
\begin{checklist}
\item Pod koniec obozu albo zaraz po nim zrobić ewaluację.
\item Ewaluację przeprowadzić na zasadzie retrospekcji z~głosowaniem.
\item Tematy wybrane w~głosowaniu przedyskutować z~hufcowym i~wziąć szczególnie pod uwagę przed kolejnym obozem.
\item Bez ewaluacji kadra obozu nigdy się nie nauczy niczego na własnych błędach.
\end{checklist}
% \mainmatter
\cleardoublepage
\section{Obowiązujące przepisy}
Organizatorzy harcerskich obozów wędrownych są zobowiązani do przestrzegania obowiązujących przepisów prawa, w~tym:
\subsection{Ogólne --- organizacja wypoczynku}
\begin{itemize}
\item Ustawa z~dnia 7~września 1991~r. o~systemie oświaty. Dz.~U. 2004.256.2572 j.~t.
\item Rozporządzenie Ministra Edukacji Narodowej z~dnia 21~stycznia 1997~r. w~sprawie warunków, jakie muszą spełniać organizatorzy wypoczynku dla dzieci i~młodzieży szkolnej, a~także zasad jego organizowania i~nadzorowania. Dz.~U. 1997.12.67.
\item Rozporządzenie Rady Ministrów z~dnia 6~maja 1997~r. w~sprawie określenia warunków bezpieczeństwa osób przebywających w~górach, pływających, kąpiących się i~uprawiających sporty wodne. Dz.~U. 1997.57.358.
\item Rozporządzenie Ministra Edukacji Narodowej i~Sportu z~dnia 31~grudnia 2002~r. w~sprawie bezpieczeństwa i~higieny w~publicznych i~niepublicznych szkołach i~placówkach. Dz.~U. 2003.6.69.
\item Rozporządzenie Ministra Edukacji Narodowej i~Sportu z~dnia 8~listopada 2001~r. w~sprawie warunków i~sposobu organizowania przez publiczne przedszkola, szkoły i~placówki krajoznawstwa i~turystyki. Dz.~U. 2001.135.1516.
\end{itemize}

\subsection{Sanepid, badania, żywienie}
\begin{itemize}
\item Ustawa z~dnia 14~marca 1985~r. o~Państwowej Inspekcji Sanitarnej. Dz.~U. 2011.212.1263 j.~t.
\item Ustawa z~dnia 5~grudnia 2008~r. o~zapobieganiu oraz zwalczaniu zakażeń i~chorób zakaźnych u~ludzi. Dz.~U. 2008.234.1570 --- brak rozporządzeń wykonawczych.
\item Ustawa z~dnia 25~sierpnia 2006~r. o~bezpieczeństwie żywności i~żywienia. Dz.~U. 2010.136.914 j.~t.
\item Rozporządzenie Ministra Zdrowia z~dnia 17~kwietnia 2007~r. w~sprawie pobierania i~przechowywania próbek żywności przez zakłady żywienia zbiorowego typu zamkniętego. Dz.~U. 2007.80.545.
\end{itemize}

\subsection{Woda do~picia, kąpieliska}
\begin{itemize}
\item Ustawa z~dnia 7~czerwca 2001~r. o~zbiorowym zaopatrzeniu w~wodę i~zbiorowym odprowadzaniu ścieków. Dz.~U. 2006.123.858 j.~t.
\item Rozporządzenie Ministra Zdrowia z~dnia 29~marca 2007~r. w~sprawie jakości wody przeznaczonej do spożycia przez ludzi. Dz.~U. 2007.61.417.
\item Ustawa z~dnia 18~lipca 2001~r. Prawo wodne. Dz.~U. 2005.239.2019 j.~t.
\item Rozporządzenie Ministra Zdrowia z~dnia 8~kwietnia 2011~r. w~sprawie prowadzenia nadzoru nad jakością wody w~kąpielisku i~miejscu wykorzystywanym do kąpieli. Dz.~U. 2011.86.478.
\end{itemize}

\subsection{Pozostałe}
\begin{itemize}
\item Ustawa z~dnia 25~czerwca 2010~r. o~sporcie. Dz.~U. 2010.127.857.
\item Ustawa z~dnia 29~sierpnia 1997~r. o~usługach turystycznych. Dz.~U. 2004.223.2268 j.~t.
\item Ustawa z~dnia 13~września 1996~r. o~utrzymaniu czystości i~porządku w~gminach. Dz.~U. 2005.236.2008 j.~t.
\item Ustawa z~dnia 20~czerwca 1997~r. Prawo o~ruchu drogowym. Dz.~U. 2005.108.908 j.~t.
\item Ustawa z~dnia 29~lipca 2005~r. o~przeciwdziałaniu narkomanii. Dz.~U. 2005.179.1485.
\item Ustawa z~dnia 28~września 1991~r. o~lasach. Dz.~U. 2011.12.59 j.~t.
\item Ustawa z~dnia 27~kwietnia 2001~r. Prawo ochrony środowiska. Dz.~U. 2008.25.150 j.~t.
\item Ustawa z~dnia 3~lutego 1995~r. o~ochronie gruntów rolnych i~leśnych. Dz.~U. 2004.121.1266 j.~t.
\item Ustawa z~dnia 24~sierpnia 1991~r. o~ochronie przeciwpożarowej. Dz.~U. 2009.178.1380 j.~t.
\item Ustawa z~dnia 12~grudnia 2003~r. o~ogólnym bezpieczeństwie produktów. Dz.~U. 2003.229.2275.
\item Zarządzenie Ministra Edukacji Narodowej z~dnia 3~lipca 1992~r. w~sprawie warunków zapewniania prawa wykonywania praktyk religijnych dzieciom i~młodzieży przebywającym w~zakładach wychowawczych i~opiekuńczych oraz na obozach i~koloniach. M.P.1992.25.181.
\end{itemize}
\cleardoublepage
\section{Gdy zdarzy się wypadek}
Gdy na obozie zdarzy się wypadek trzeba postępować według zaleceń Instrukcji zamieszczonej w~załączniku \ref{instrukcja-postepowania-wypadek}.
Niestety, instrukcja ta załatwia kwestię ,,co należy robić w~tym momencie, kiedy wypadek się zdarzył'' w~pierwszym, lakonicznym punkcie. Pozostałe punkty dotyczą już czasu ,,po'' wypadku.

Oprócz wypadków na obozie mogą zdarzyć się inne sytuacje, jak ta opisana w~przykładzie na stronie \pageref{akcja-kroscienko-szczawnica}, które skutkują nadzwyczajnymi działaniami, np. poszukiwawczymi. Wędrujący obóz może być też świadkiem wypadku lub innej sytuacji kryzysowej. Jeśli na miejscu zdarzenia (wypadku, pożaru itp.) nie ma służb ratowniczych, zgodnie z~prawem, mamy obowiązek udzielenia natychmiastowej pomocy poszkodowanym.

Najważniejsza rzecz to oczywiście ,,nie tracić zimnej krwi''. Zgodnie z pierwszym punktem wspomnianej Instrukcji, należy ,,chronić osoby, które nie zostały pokrzywdzone'' oraz ,,udzielić pierwszej pomocy osobom poszkodowanym''. W~tym celu trzeba wyznaczyć osobę lub osoby, które zajmą się udzielaniem pomocy poszkodowanym. Pozostałą grupę pod opieką odpowiedniej liczby instruktorów należy jak najszybciej odizolować, najlepiej poza zasięg wzroku. Po pierwsze, aby nic się im nie stało, a~po drugie, żeby nie przeszkadzali. Może okazać się wskazane zawieszenie wykonywania programu obozu do momentu aż pierwsze emocje opadną i~cała kadra będzie świadoma sytuacji. Jeżeli uczestnicy obozu są w~szoku to należy jak najszybciej zorganizować pomoc psychologa i~umożliwić im rozmowę telefoniczną z~rodzicami. Jeżeli zdarzenie nie ma tragicznych następstw, instruktorzy powinni zająć grupę bazując na programie rezerwowym. Jednym słowem grupa czeka na decyzję kadry ,,co dalej'', ale się nie nudzi tylko ma czas zorganizowany. Po ustabilizowaniu się sytuacji grupa może realizować dalej program obozu, np. wędrówkę do nowej kwatery.

Grupa udzielająca pomocy, prowadząca akcję poszukiwawczą czy jakąkolwiek inną powinna skupić się tylko i~wyłącznie na tym co robi. Najlepiej wyznaczyć jedną osobę do kontaktu z~czekającą grupą i~drugą osobę do kontaktu z~rodzinami poszkodowanych/zaginionych osób. Obie te osoby muszą mieć stały kontakt z~komendantem, bez względu w~której grupie się on znajduje. Może się bowiem okazać, że osoby udzielające pomocy będą potrzebowały wsparcia, zmienników lub czegokolwiek innego. Osoby wyznaczone do kontaktu idealnie nadają się do realizowania punktów 2--6 wspomnianej Instrukcji. Skład grupy ratowniczej opisany jest nawet w~Rozporządzeniu Rady Ministrów (Dz.~U. 05.212.1765): \S~11 pkt.~2:

Grupa ratownicza, w~zależności od zakresu przewidywanych do realizacji zadań, może składać się z następujących zespołów:
\begin{enumerate}
\item dowodzenia i~koordynacji (dowódca, oficer łącznikowy, tłumacz oraz koordynator medyczny);
\item ratowniczych;
\item zabezpieczenia logistycznego.
\end{enumerate}

Jeżeli akcja jest skomplikowana i~grupa udzielająca pomocy nie jest pewna swoich decyzji, można wykonać telefon do opiekuna obozu z~ramienia Okręgu lub do Szefa Akcji Letniej. To są doświadczone osoby, które mogą bez zbędnych ceregieli wprost zasugerować właściwe rozwiązanie jakiejś sytuacji. Najważniejsze to nie podejmować pochopnie decyzji, nie działać w~pojedynkę, konsultować się z~członkami grupy, ,,głośno myśleć'', żeby wszyscy wiedzieli co się dzieje i~co będzie się działo zaraz. Co dwie głowy to nie jedna --- zawsze ktoś może zaproponować lepsze rozwiązanie. Warto wyznaczyć na samym początku osobę kierującą akcją. Jeśli nie ma na miejscu osoby zdecydowanej i~kompetentnej, która z~racji np. zajmowanego stanowiska ,,dowodzi'' akcją, z~chwilą przystąpienia do działania jeden z~instruktorów może pełnić tę rolę. Staje się ,,kierującym działaniami ratowniczymi''. Wszystkie decyzje i~polecenia kierującego są niepodważalne i~ostateczne, a~wszyscy bez wyjątku mają obowiązek im się podporządkować.

Kierujący działaniami ratowniczymi ma prawo:
\begin{itemize}
\item zarządzić ewakuację ludzi i~mienia,
\item wstrzymać ruch drogowy,
\item wprowadzić zakaz przebywania osób trzecich,
\item przejąć w~użytkowanie na czas określony sprzęt (np. środki transportu) --- podstawa prawna: Kodeks cywilny art. 142 \S~1~i~2,
\item żądać niezbędnej pomocy od instytucji, organizacji, podmiotów prawnych oraz osób fizycznych.
\end{itemize}
Kierujący ma obowiązek powiadomić służby profesjonalne oraz zabezpieczyć miejsce wypadku (osobiście lub wydając odpowiednie polecenia). Staje się odpowiedzialny za przebieg działań ratowniczych oraz osoby będące na miejscu i~w~pobliżu zdarzenia, niezależnie od tego, czy mu pomagają, wykonują jego polecenia, czy są tylko przypadkowymi ,,gapiami''.

Każdy, kogo poprosicie ma obowiązek udzielić Wam pomocy. Nieudzielenie lub odmowa pomocy jest karalna. Mimo to, jeśli prosicie kogoś o~wezwanie pogotowia, policji czy straży, bo sami nie możecie tego zrobić, liczcie się z~tym, że może tego nie zrobić. Dlatego, jeśli tylko jest taka możliwość, poproście np. o~powiadomienie kilka osób równocześnie. W~żadnym razie nie wolno wyolbrzymiać skutków wypadku. Może się zdarzyć, że w~tym czasie ktoś będzie potrzebował pomocy bardziej. Z~miejsca wypadku oddalajcie się tylko w~celu wezwania pomocy lub dopiero po przyjeździe służb ratowniczych na ich wyraźne zezwolenie.

Obóz harcerski nie jest jednostką ratowniczą i~nie posiada wystarczających umiejętności, uprawnień ani sprzętu do prowadzenia długotrwałych działań. W~związku z~tym trzeba się skupić na ochronie osób, które nie zostały poszkodowane i~na pierwszej pomocy osobom poszkodowanym. W~zależności od sytuacji pole manewru może być bardzo ograniczone.
\begin{description}
\item[Choroby zwierzęce] Od czasu do czasu następują wybuchy epidemii poważnych chorób zwierzęcych, które są w~stanie się szybko rozprzestrzenić przekraczając granice krajów, lub mają zdrowotne konsekwencje dla obozu.

Jeżeli obóz kwateruje w~gospodarstwie agroturystycznym, gdzie jest żywy inwentarz lub ptactwo i~podejrzewasz wybuch epidemii poważnej choroby
zwierzęcej (zwierzęta wyglądają na apatyczne i~chore lub mają krwawiące rany, znajdujesz padłe osobniki itp.):

\begin{itemize}
\item Wyprowadź uczestników obozu poza teren gospodarstwa.
\item Skontaktuj się z~weterynarzem lub Sanepidem.
\item Jeśli obóz napotkał większą ilość martwych dzikich ptaków lub innych zwierząt, kadra powinna zgłosić ten fakt do Sanepidu.
\end{itemize}
\item[Krwawienie] \hfill \\ Co robić:
\begin{itemize}
\item Kontroluj obfite krwawienia poprzez mocny nacisk na ranę przy użyciu czystego, suchego opatrunku.
\item Ułóż ciało rannej osoby tak, aby unieść ranę powyżej poziomu serca celem zredukowania utraty krwi.
\item Jeśli podejrzewasz, że kończyny są uszkodzone, poruszaj nimi bardzo delikatnie.
\item Pomóż położyć się rannemu, uspokój go, ogrzej i~poluzuj obcisłe elementy odzieży.
\end{itemize}
\item[Niebezpieczne wycieki chemiczne] \hfill \\ Kontakt z~niebezpiecznymi chemikaliami, które zostały rozlane bądź inaczej uwolnione może spowodować poważne lub nawet śmiertelne obrażenie. W~przypadku takiego wydarzenia służby ratownicze zidentyfikują rodzaj i~poziom niebezpieczeństwa i~poinformują kadrę, co robić. Obóz może zostać poproszony o~pozostanie na kwaterze oraz o~uszczelnienie okien i~drzwi, lub o~ewakuację z~danego terenu.

Co robić:
\begin{itemize}
\item Poinformuj służby ratownicze.
\item Trzymaj się z~daleka od miejsca zdarzenia.
\item Jeśli jest to bezpieczne spróbuj stanąć pod wiatr tyłem do zanieczyszczonego terenu.
\item Postępuj zgodnie ze wskazówkami służb ratowniczych.
\end{itemize}
\item[Nieprzytomność] \hfill \\ Jeśli osoba jest nieprzytomna:
\begin{itemize}
\item Zadzwoń po pogotowie ratunkowe.
\item Jeśli jest to możliwe, upewnij się, że osoba ma drożne drogi oddechowe.
\item Jeśli osoba przestała oddychać i/lub wydaje się być martwa, oraz jeśli posiadasz potrzebne umiejętności, powinieneś podjąć odpowiednie kroki celem jej/jego reanimacji.
\item Jeśli osoba oddycha ułóż ją w~pozycji bezpiecznej.
\end{itemize}
\item[Odwodnienie] \hfill \\ Odwodnienie jest bardzo niebezpieczne. Przy dużym wysiłku fizycznym organizm może wydalać 1-2 litrów wody na godzinę (w~normalnych warunkach 1-2 litrów na dobę).

Objawy:
\begin{itemize}
\item Suchy język.
\item Zmiana koloru moczu na ciemnozłoty.
\end{itemize}
\newpage
Co robić:
\begin{itemize}
\item Musisz pamiętać o~zabieraniu ze sobą na wędrówki sporej ilości płynów. Szczególnie przy wysokich temperaturach należy dużo pić. Jeżeli po drodze mijasz źródła --- uzupełnij zapas wody. Chłodna woda jest lepiej absorbowana przez organizm.
\item Kawa, herbata, alkohol zazwyczaj potęgują utratę płynów przez organizm. Jeżeli możesz --- lepiej pij soki i~napoje energetyczne.
\end{itemize}
\item[Oparzenia] \hfill \\ Co robić:
\begin{itemize}
\item Wszystkie oparzenia schładzaj zimną wodą przez co najmniej 10 minut, jednak nie opóźniaj przewiezienia osoby do szpitala.
\item Aby zapobiec zakażeniu, owiń oparzoną powierzchnię sterylnym opatrunkiem. Jeśli nie jest on dostępny, owiń oparzoną powierzchnię przezroczystą folią do pakowania żywności --- nie stosuj suchych opatrunków.
\item Trzymaj pacjenta w cieple.
\end{itemize}
\item[Pogryzienie przez psa (lub inne dzikie zwierzę)] \hfill \\ Najbardziej narażeni na ukąszenie przez psa są rowerzyści podczas jazdy przez wieś lub przejażdżki po terenach parkowych, jak również osoby biegające w~podobnych miejscach. Można też paść ofiarą ataku kota, zaś w okolicach leśnych --- dzikiego zwierzęcia (lisa, jenota, szczura, itp.).

Co robić:
\begin{itemize}
\item Niemal zawsze należy podejrzewać zagrożenie wścieklizną --- dlatego ranie należy najpierw pozwolić przez pewien czas na krwawienie.
\item Następnie miejsce ukąszenia należy przemyć ciepłą wodą z mydłem, można też użyć wody utlenionej, riwanolu lub roztworu nadmanganianu potasu.
\item Nie można używać jodyny ani spirytusu.
\item Po oczyszczeniu rany zakładamy opatrunek jałowy i~udajemy się do lekarza. Musi on zdecydować, czy konieczne jest zastosowanie surowicy przeciw wściekliźnie.
\item Jeżeli zaatakował nas pies domowy, należy zgłosić ten fakt na policję, która powinna jego właściciela pociągnąć do odpowiedzialności, zaś psa skieruje na obserwację przez służby weterynaryjne. Jeśli zaś było to zwierzę bezpańskie lub dzikie, policja powinna doprowadzić do jego schwytania i~obserwacji.
\end{itemize}
\item[Połamane kości] \hfill \\ O~uszkodzeniu kręgosłupa świadczy fakt bezwładności lub brak czucia w kończynach. Często też zdarza się, że osoba której pozornie nic nie dolega, siada na ziemi z~podkulonymi kolanami i~podtrzymuje rękami głowę. Takie zachowanie sugeruje uszkodzenie rdzenia kręgowego na odcinku szyjnym. Złamania żeber czasem dają się ,,wymacać'', mogą świadczyć też o~nich obrzęki, dziwne nierówności pod skórą, zasinienia czy trudności w~oddychaniu lub świszczący oddech.
\newpage
Co robić:
\begin{itemize}
\item Unikaj poruszania danej osoby i~zadzwoń po pogotowie ratunkowe.
\end{itemize}
\item[Porażenie prądem] \hfill \\ Objawy porażenia prądem:
\begin{itemize}
\item Ból.
\item Poparzenia skóry.
\item Zaburzenia w oddychaniu.
\item Utrata przytomności.
\end{itemize}
Co robić:
\begin{itemize}
\item Nie wolno dotykać osoby porażonej prądem, zanim nie odłączy się jej od źródła prądu. Odłącz bezpieczniki (korki), wyjmij z~gniazdka wtyczkę urządzenia elektrycznego, które spowodowało porażenie. Użyj do tego przedmiotu który nie przewodzi prądu (np. drewnianego kija od szczotki), odsuń kabel elektryczny od poszkodowanego.
\item Sprawdź stan poszkodowanego:
\begin{itemize}
\item Czy jest przytomny.
\item Czy oddycha.
\end{itemize}
\item Wezwij Pogotowie Ratunkowe.
\item Jeśli ratowany nie oddycha przystąp do reanimacji.
\item Jeśli ratowany jest nieprzytomny, ale oddycha, ułóż go w pozycji bocznej.
\item Załóż opatrunek na oparzone miejsce.
\item Zostań z~poszkodowanym do czasu przybycia Pogotowia Ratunkowego i~przejęcia opieki nad poszkodowanym.
\end{itemize}
\item[Powódź] \hfill \\ Powodzie są zwykle spowodowane przez kombinację zdarzeń włącznie z~wylewem brzegów rzek, sztormami przybrzeżnymi albo zablokowaniem lub przeciążeniem kanalizacji. Jeżeli powódź zagraża miejscu, w~którym przebywa obóz, aby zminimalizować szkody mienia, powinno się podjąć pewne kroki, pamiętając, że najważniejsze powinno być zawsze bezpieczeństwo.

Co robić:
\begin{itemize}
\item Poinformuj służby ratownicze.
\item Nie próbuj przejść lub przejechać przez spływ wód powodziowych.
\item Jeśli jest to możliwe unikaj kontaktu ze spływem wód powodziowych, ponieważ może on być zanieczyszczony lub skażony.
\item Nigdy nie próbuj płynąć przez szybko płynącą wodę --- możesz zostać zniesiony lub uderzony przez obiekt w~wodzie.
\item Zawsze noś odpowiednią odzież podczas pracy w~lub w~pobliżu spływu wód powodziowych.
\end{itemize}
\item[Pożar] \hfill \\ Pożar jest bardzo częstą przyczyną sytuacji kryzysowych, zarówno mały w~gospodarstwach domowych jak i~duży w~miejscach pracy oraz budynkach użyteczności publicznej, jak schroniska czy szkoły. We wszystkich przypadkach podstawowe wskazówki pozostają te same.

Co robić:
\begin{itemize}
\item Podnieś alarm.
\item Ewakuuj budynek tak szybko i~bezpiecznie jak to tylko możliwe i~udaj się do ustalonego punktu zbiórki.
\item Nie używaj wind.
\item Jeśli jesteś otoczony dymem, pozostawaj blisko podłogi i~przeczołgaj się do wyjścia (ponieważ dym, trujące gazy i~gorąco wzniosą się wyżej).
\item Jeśli to możliwe, zakryj nos i~usta mokrą szmatką i~przykryj odkrytą skórę.
\item Zanim otworzysz drzwi, sprawdź je tylną częścią dłoni. Jeśli są gorące, nie otwieraj ich, ponieważ po drugiej stronie może znajdować się ogień.
\item Jeśli nie możesz uciec, pozostań w~nienaruszonym pomieszczeniu. Zamknij drzwi, podejdź do okna i~przyciągnij uwagę, aby zaalarmować ratowników o~Twojej obecności.
\end{itemize}
\item[Przegrzanie] \hfill \\ Przegrzanie --- inaczej udar cieplny. Może zdarzyć się także w~dni pochmurne, gdyż występuje w~wyniku działania promieniowania podczerwonego i~ultrafioletowego. Charakteryzuje się:
\begin{itemize}
\item gwałtownym bólem głowy,
\item uczuciem gorąca,
\item zaczerwienieniem twarzy,
\item nudnościami,
\item wymiotami,
\item skurczami żołądka i~mięśni nóg,
\item ciężkim oddechem,
\item zaburzeniami wzroku i~słuchu,
\item może doprowadzić do utraty przytomności.
\end{itemize}
Co robić:
\begin{itemize}
\item Chorego należy odizolować od słońca,
\item posadzić w~pozycji półsiedzącej z~głową umieszczoną wyżej niż reszta ciała,
\item podawać dużo zimnych płynów zawierających roztwór soli (1~łyżeczka na litr wody), jednak gdy chory wymiotuje nie wolno podawać słonych napojów.
\item schładzać ciało kompresami.
\end{itemize}
\item[Rozstrój żołądka] \hfill \\ Tą przykrą dolegliwość może spowodować bakteryjne zatrucie pokarmowe (np. woda ze źródełka) lub błąd w diecie.

Co robić:
\begin{itemize}
\item Podać 6-10 rozdrobnionych w~wodzie tabletek węgla leczniczego albo można rozpuścić kilka torebek specyfiku dla niemowląt ,,Smecta''.
\item By zapobiec odwodnieniu, należy podawać gorzką herbatę lub napój izotoniczny.
\item Gdy zatrucie nie ustępuje i~obok biegunki pojawiają się wymioty i~gorączka, konieczna jest wizyta u~lekarza.
\end{itemize} %http://www.e-horyzont.pl/pierwsza-pomoc-turysty-i-cyklisty_30d#art138
\item[Szok] \hfill \\ Co robić:
\begin{itemize}
\item Nie doprowadzaj osoby do wychłodzenia lub przegrzania.
\item Jeśli nie podejrzewasz złamania kości, unieś nogi na wysokość około 30~cm.
\item Nie podawaj jedzenia ani picia.
\end{itemize}
\item[Ukąszenie przez żmiję] \hfill \\ Gdy zaatakuje żmija skutki ukąszenia mogą być tragiczne. Żmija ma trójkątną głowę i~pionowe źrenice. Może mieć różne ubarwienie, ale charakterystycznym jej znakiem jest czarny zygzak na grzbiecie. Często bywa mylona z nieszkodliwym zaskrońcem. Nadepnięcie na gada zwykle kończy się ukąszeniem. Jeżeli powstała w~wyniku tego rana ma postać dwóch punktowych skaleczeń (położonych 1,5-2~cm od siebie), krwawi, towarzyszy temu silny ból, obrzęk i~zasinienie, najprawdopodobniej doszło do zatrucia jadem.

Co robić:
\begin{itemize}
\item Uspokoić poszkodowanego aby jak najmniej się poruszał. Ruch przyśpiesza rozprzestrzenianie się substancji toksycznej w~organizmie. Jeśli ukąszona została ręka lub noga, należy ją unieruchomić. Z~ręki trzeba natychmiast zdjąć zegarek i~biżuterię.
\item Ranę zabezpieczyć jałowym opatrunkiem.
\item Rannego ułożyć w~takiej pozycji, by ukąszone miejsce znajdowało się niżej niż serce.
\item Nie wysysać rany ani nie rozcinać jej nożem. Może to spowodować powstanie nowych uszkodzeń i sprawić dodatkowy ból poszkodowanemu. Nie wolno też stosować lodu ani opasek uciskowych.
\item Jeśli osoba ukąszona traci świadomość, trzeba uważnie ją obserwować i~sprawdzać, czy ma zachowane krążenie i~czy oddycha. Nieprzytomną należy ułożyć w~pozycji bezpiecznej na boku. Jeżeli zajdzie potrzeba, trzeba rozpocząć sztuczne oddychanie i~masaż serca.
\item Po dotarciu do placówki medycznej należy poinformować personel, że doszło do ukąszenia przez żmiję --- surowica przeciw jadowi musi być podana jak najszybciej.
\end{itemize}
\item[Utonięcie] \hfill \\ Najczęstszą przyczyną wypadków, a~w~skrajnym przypadku śmierci, dzieci i~młodzieży podczas okresu letniego są utonięcia lub podtopienia.

Co robić:
\begin{itemize}
\item Oceń sytuację.
\item Wezwij pomoc --- poproś gapiów o~pomoc, o~wezwanie pogotowia oraz WOPR-u (nr do WOPR-u to 601 100 100) lub telefon pod 112.
\item Wybierz najbardziej bezpieczny sposób ratowania, wyciągania tonącego:
\begin{itemize}
\item za pomocą koła ratunkowego;
\item za pomocą liny;
\item z łodzi;
\item z pomostu podając rękę lub długi przedmiot tonącemu;
\item można podpłynąć i przyholować tonącego do brzegu.
\end{itemize}
\item Jeżeli to jest możliwe, nie wolno w~żadnym wypadku dopuścić do faktu ,,oddychania'' wodą przez tonącego. Jak najszybciej należy wydostać go spod wody, a~przy utracie przytomności, oddechu już w~momencie wydobycia na powierzchnię zalecane jest podanie mu pierwszego sztucznego oddechu. Przy konieczności holowania go do brzegu, powtórzyć należy ten zabieg kilkakrotnie na płytkiej wodzie.
\item Po wyniesieniu na brzeg lub w~ogóle przed rozpoczęciem ożywiania nie należy stosować żadnych zabiegów w~celu usunięcia wody z~przewodu pokarmowego i~dróg oddechowych. Powodują one stratę cennego czasu. W~przewodzie pokarmowym nie dochodzi do oddychania.
\item Wykonać kolejno czynności reanimacyjne wedle schematu.
\item Być psychicznie przygotowanym na różne komplikacje, jakie mogą wystąpić przy stosowaniu ożywiania.
\item Kiedy uda się przywrócić oddech i~krążenie, należy osobę ratowaną ułożyć na boku w~pozycji bocznej ustalonej, ponieważ w~każdej chwili należy spodziewać się wystąpienia wymiotów. Nie wolno dopuścić do zachłyśnięcia się wymiocinami.
\item Osobom, które odzyskały przytomność można podać do picia nie za gorącą, dobrze osłodzoną herbatę. Absolutnie nie wolno podawać alkoholu. Należy zapewnić suchą odzież i~ciepłe okrycie np. kocem termicznym.
\end{itemize}
\item[Użądlenie przez owada] \hfill \\ Jest to bardzo często spotykana przygoda rowerzystów. Podczas jazdy owad (pszczoła, osa, szerszeń, itp.) może wpaść do gardła, we włosy, pod odzież lub kask. Najgroźniejsze jest użądlenie okolicy krtani, powiększający się obrzęk może doprowadzić do uduszenia poszkodowanego. Równie groźne skutki może mieć użądlenie dla osoby z~uczuleniem lub dziecka, może dojść wówczas do niewydolności krążenia i zapaści.
\newpage
Co robić:
\begin{itemize}
\item Jeśli użądlenie nastąpiło w~okolicy krtani i~obrzęk się powiększa, należy poszkodowanego natychmiast dostarczyć do lekarza, może być nawet konieczna tracheotomia.
\item Jeśli użądlona została osoba uczulona i~doszło do niewydolności krążenia i zapaści, konieczna jest reanimacja i~sztuczne oddychanie oraz natychmiastowa pomoc lekarska.
\item W~przypadku zwykłego ukąszenia największą ulgę przynosi okład z~roztworu sody oczyszczonej lub amoniaku, można też wypić wapno.
\item Gdy trafimy na kleszcza należy go usunąć --- samodzielnie (przez wykręcanie pęsetą), lub przez lekarza. Ze względu na coraz powszechniejsze zagrożenie chorobami roznoszonymi przez te owady (borelioza, kleszczowe zapalenie opon mózgowych) --- po ukąszeniu przez kleszcza wskazana jest wizyta u~lekarza.
\end{itemize}
\item[Wybuchy i~podejrzane paczki] \hfill \\ Jeśli znajdziesz się w~pobliżu miejsca wybuchu:
\begin{itemize}
\item Oddal się z~miejsca wybuchu/budynku tak szybko i~spokojnie jak to możliwe.
\item W~przypadku zawalenia elementów wewnętrznych budynku, schroń się pod solidnym stołem lub biurkiem dopóki sytuacja się wystarczająco nie ustabilizuje, abyś mógł bezpiecznie wyjść. Kiedy będzie już bezpiecznie opuść szybko miejsce zdarzenia uważając na osłabione podłogi i~schody.
\item Upewnij się, że jesteś bezpieczny zanim spróbujesz pomóc innym.
\end{itemize}
Zwracaj uwagę na pozostawione nietypowe, podejrzane paczki. Zwróć uwagę na:
\begin{itemize}
\item Odbarwienie, kryształki, dziwne zapachy lub tłuste plamy.
\item Ślady pudru na kopercie lub znajdujący się na niej podobny do pudru osad.
\item Nietypowy rozmiar, wagę lub kształt.
\item Widoczne kable lub baterie.
\end{itemize}
Co robić:
\begin{itemize}
\item Zostaw paczkę w miejscu gdzie się znajduje.
\item Ewakuuj siebie i~innych z~bezpośredniego sąsiedztwa paczki/budynku.
\item Zadzwoń po policję.
\item W~stosownych przypadkach zaalarmuj personel zajmujący się ochroną budynku.
\item Jeśli otworzyłeś paczkę zawierającą podejrzany materiał, umyj ręce lub weź prysznic używając mydła i~wody oraz nie dotykaj rękoma ust ani oczu.
\end{itemize}
\item[Wychłodzenie] \hfill \\ Nawet w~lecie podczas załamania pogody (burzy, silnego zimnego wiatru, długotrwałych opadów deszczu) może dojść do wychłodzenia organizmu, podatnego na to zwłaszcza po dużym wysiłku. Większość z~nas doświadczyła tego choćby raz w życiu, a~już na pewno zetknęli się z~tym ci, którym przydarzyło trząść się z~zimna. To jeden z~pierwszych objawów zjawiska, które w~swych bardziej zaawansowanych stadiach prowadzi nawet do zagrożenia życia. Największe straty ciepła dotyczą najbardziej ukrwionych, powierzchownych okolic ciała, czyli głowy, szyi i~kończyn --- tą drogą pozbywamy się ponad 50\% wyprodukowanej energii cieplnej. Oprócz temperatury otoczenia największe znaczenie dla komfortu termicznego mają jeszcze dwa inne czynniki pogodowe: wiatr i~wilgoć. Ta ostatnia zwiększa przewodnictwo cieplne i~ułatwia jego ucieczkę. Dotyczy to zarówno wilgoci atmosferycznej, jak i~mokrej odzieży. Wiatr z~kolei niszczy cienką warstwę nieruchomego powietrza tuż przy skórze. Powietrzna ,,powłoczka'' ma własności termoizolacyjne i~w~warunkach bezwietrznych broni przed chłodem.

Jak się chronić:
\begin{itemize}
\item Dobrać odzież odpowiednią do warunków, jakich się spodziewamy.
\item Zabierać ze sobą ciepłą, nieprzemakalną odzież.
\item Zastosować warstwowy sposób ubierania się ,,na cebulkę'' --- pozwala on wykorzystać termoizolacyjne właściwości powietrza zatrzymanego pomiędzy kolejnymi elementami garderoby.
\item Koniecznie trzeba odpowiednio ochronić głowę.
\item Na deszczowe wędrówki zabierać zapasowe rękawiczki, czapkę i~skarpetki na wypadek przemoczenia.
\item Warto zabrać ze sobą termos z~ciepłym, osłodzonym napojem.
\item Można wspomagać się za pomocą specjalnych ogrzewaczy dostępnych w~sklepach ze sprzętem turystycznym.
\end{itemize}
%http://www.e-gory.pl/index.php/Bezpieczenstwo/Pierwsza-pomoc/Wychlodzenie.html
Są 4~stadia wychłodzenia:
\begin{enumerate}
\item Stan wzrastającego pobudzenia:
\begin{itemize}
\item temperatura 36,5-34$^\circ$C;
\item pobudzenie psycho--ruchowe;
\item skóra blada lub sina;
\item przyśpieszone tętno i~oddech;
\item dreszcze i~drżenie mięśniowe niezależne od woli.
\end{itemize}
\item Zanikanie reakcji pobudzenia:
\begin{itemize}
\item temperatura 34-30$^\circ$C;
\item apatia;
\item oddech nieregularny, powierzchowny;
\item senność, z~możliwością obudzenia;
\item brak odczucia bólu;
\item częściowa utrata świadomości.
\end{itemize}
\item Utrata przytomności:
\begin{itemize}
\item temperatura 30-27$^\circ$C;
\item poszkodowany nie daje się obudzić;
\item źrenice nie reagują na światło;
\item tętno słabe, przerywane, długie przerwy w~oddechu;
\item brak odruchów, brak przytomności.
\end{itemize}
\item Śmierć pozorna:
\begin{itemize}
\item temperatura poniżej 27$^\circ$C;
\item szerokie, sztywne źrenice;
\item zanik oddechu;
\item niewyczuwalne tętno.
\end{itemize}
\end{enumerate}

Pierwsza pomoc:
\begin{itemize}
\item zmienić mokrą odzież na suchą;
\item chronić przed zimnem, wilgocią i~wiatrem (zawinąć w~folię NRC, koc wełniany, śpiwór);
\item izolować od podłoża;
\item zakazać jakichkolwiek ruchów;
\item nie stosować masażu biernego;
\item nie podawać alkoholu i~leków rozszerzających naczynia krwionośne!
\item poszkodowanego przenieść do chłodnego pomieszczenia, a~dopiero po wstępnym ogrzaniu ciała przenieść do ciepłego;
\item o~ile pozwala stan poszkodowanego należy przede wszystkim stosować ogrzewanie wewnętrzne, poprzez podanie ciepłych i~słodzonych napojów;
\item przy temperaturze ciała powyżej 30$^\circ$C, ogrzewać tylko brzuch i~klatkę piersiową gorącym kompresem, ale nie bezpośrednio na skórę (przez folię i~sweter);
\item przy temperaturze ciała poniżej 30$^\circ$C, nie stosować gorących kompresów, ani innych sposobów szybkiego ogrzewania!
\item zabrania się stosowania ciepłych kąpieli;
\item w~przypadku cięższych wychłodzeń, jak najszybciej zapewnić poszkodowanemu fachową pomoc medyczną.
\item jeśli ze względu na swój stan poszkodowany wymaga reanimacji, trzeba ją prowadzić do czasu ogrzania ciała lub przybycia służb ratowniczych, bądź przywrócenia funkcji życiowych. ,,Szczęściem w~nieszczęściu'' jest to, że niska temperatura wywiera ochronny wpływ na komórki mózgu i~innych narządów. Przy temperaturze ciała 18$^\circ$C mózg toleruje zatrzymanie krążenia trwające nawet dziesięć razy dłużej niż przy 37$^\circ$C. To istotna wiadomość dla ratujących  --- czas reanimacji takich osób powinien być znacznie wydłużony. Obrazowa zasada mówi, że ,,nikt nie jest martwy, dopóki nie jest ciepły i~martwy''. Poza tym, podstawowe procedury resuscytacyjne przy wychłodzeniu nie odbiegają od ogólnych reguł. %http://www.egzotyka.tv/bezpieczenstwo51__hipotermia_%EF%BF%BD%E2%80%9C_nie_tylko_dla_polarnikow.html
\end{itemize}
\item[Wypadek komunikacyjny] \hfill \\ Wypadki komunikacyjne w~chwili obecnej należą do najczęstszych przyczyn obrażeń. Typowe obrażenia podczas zderzenia samochodu przedstawiają się następująco:
\begin{itemize}
\item złamania: kręgosłupa szyjnego, klatki piersiowej, miednicy,
\item krwawienia,
\item urazy narządów wewnętrznych: śledziony, serca, wątroby,
\item oparzenia.
\end{itemize}
Obrażenia powstające na skutek wypadków komunikacyjnych są często bardzo groźnymi obrażeniami wielonarządowymi. Obrażenia te są często niewidoczne nie tylko dla laika, ale na skutek szoku powypadkowego, nieodczuwalne dla poszkodowanych. Można o~nich często tylko domniemywać w~oparciu o~zaobserwowane zachowanie ofiar. Często, ich ,,zwiastunem'', mimo zapewnień o~dobrym samopoczuciu, mogą być niewielkie mdłości, zawroty lub bóle głowy, bladość skóry. We wszystkich tych przypadkach najważniejszą sprawą jest właściwy sposób wyjęcia poszkodowanego z~fotela pojazdu. W~związku z~tym, jeśli samochód się nie pali, nie wycieka z~niego paliwo ani nie ma absolutnej konieczności nie wyciągamy samodzielnie poszkodowanego. Jeśli jest to jednak  konieczne to zawsze zakładamy, że do opisanych wyżej urazów doszło. Właściwym postępowaniem jest założenie kołnierza usztywniającego i~wyciągnięcie poszkodowanego chwytem pod brodę i~ułożenie pleców rannego ,,na siebie'' z~zastosowanym wyciągiem głowy w~osi długiej ciała. W przypadku (najczęściej) braku kołnierza usztywniającego, można takowy zaimprowizować owijając szyję np. kilkakrotnie złożoną, w~około 15-to centymetrowy pas, gazetą. Obrażenia komunikacyjne są bardzo złudne i~często trudne lub nawet niemożliwe do zdiagnozowania w~warunkach ,,polowych''. Przytoczone przypadki są tylko przykładową i~bardzo niepełną sygnalizacją objawów. Nie ma dobrej recepty na właściwą pomoc przedmedyczną. Możemy ograniczyć się do stwierdzonych ponad wszelką wątpliwość urazów.

Co robić:
\begin{itemize}
\item Jeśli jesteś świadkiem kolizji lub wypadku na drodze, a~nie ma Policji, Pogotowia Ratunkowego lub Straży Pożarnej zatrzymaj się i~zobacz co się stało.
\item Jeżeli są służby profesjonalne, nie zatrzymuj się (chyba, że Cię zatrzymają), nie blokuj ruchu, nie rób zbiegowiska.
\item Jeśli nie ma służb ratowniczych wyłącz zapłon w~uszkodzonych samochodach oraz zabezpiecz miejsce wypadku, żeby Ciebie nikt nie przejechał.
\item Odłącz akumulator, ustaw trójkąt ostrzegawczy, włącz światła awaryjne, jeśli uznasz to za właściwe, poproś innych o~zatrzymanie ruchu na drodze. W~nocy oświetl miejsce wypadku.
\item Wezwij pomoc profesjonalną.
\item Jeżeli nie wiesz jak należy wyjmować rannych z samochodu, nie wyjmuj ich, chyba że musisz (brak oddechu, pali się samochód).
\item W~miarę swoich umiejętności i~możliwości udziel pomocy poszkodowanym.
\end{itemize}
\item[Wypadek na wodzie] \hfill \\ Na wodzie spotykają się żeglarze, kajakarze, użytkownicy łódek, skuterów, motorówek, windsurferzy, kitesurferzy itd. Tu również zdarzają się wypadki, które w~uproszczeniu można uznać za kombinację wypadków komunikacyjnych z~utonięciami. Kajaki zderzają się z~żaglówkami, kitesurferzy wpadają na windsurferów, motorówki wpadają na łódki itp. Obrażenia mogą być najróżniejsze, a~dodatkowo poszkodowani mogą utonąć.

Co robić:
\begin{itemize}
\item Nie dopuścić do utonięcia poszkodowanych.
\item Jeżeli w~incydencie brał udział sprzęt pływający posiadający silnik, podobnie jak przy wypadku komunikacyjnym, wyłączyć zapłon, uważać na rozlane paliwo pływające na powierzchni wody.
\item Skupić się na ratowaniu ludzi, uszkodzony sprzęt zostawić w~wodzie i~nie zajmować się nim, chyba, że stwarza dodatkowi zagrożenie, wtedy zabezpieczyć go zależnie od możliwości.
\end{itemize}
\item[Zachorowania większej ilości osób] \hfill \\ Jeżeli więcej niż 2 osoby mają podobne podejrzane objawy może to być zatrucie lub poważna infekcja.

Co robić:
\begin{itemize}
\item Odizoluj osoby chore od zdrowych.
\item Nie lekceważ tych objawów.
\item Nie próbuj leczyć chorych na własną rękę.
\item Jeśli dysponujesz własnym środkiem transportu zawieź chorych do lekarza, w~przeciwnym wypadku ściągnij lekarza do obozu.
\end{itemize}
\item[Zaginięcie] \hfill \\ Podczas gier terenowych, rajdów itp. możliwe są zaginięcia pojedynczych osób, patroli czy zastępów. Zaginięcie jednak nie musi być zwykłym zagubieniem na szlaku czy w~lesie, może być skutkiem przestępstwa popełnionego przez osoby trzecie.

Jeżeli samodzielne poszukiwania siłami obozu nie przynoszą rezultatu nie wahaj się poprosić o~pomoc GOPR/TOPR (w~górach) lub Policję. Procedury tych instytucji mogą wydać się bezduszne, postępowanie zależy od osoby przyjmującej zgłoszenie. Często policjant odmawia przyjęcia zgłoszenia przed upłynięciem 48~godzin, co nie jest zgodne z~procedurami policji, zapisanymi w~Zarządzeniu nr 352 Komendanta Głównego Policji z~dnia 16~lipca 2003~r.

Jak zgłosić zaginięcie policji?
\begin{itemize}
\item Idź osobiście na najbliższy komisariat w~miejscu zaginięcia (ewentualnie najbliższy miejsca kwaterowania obozu).
\item Poinformuj, że chcesz zgłosić zaginięcie. Przyjęcie trwa około godziny, policjant musi poznać jak najwięcej faktów, aby móc Ci pomóc w~poszukiwaniach.
\item Zostaniesz poproszony o~zdjęcie zaginionego. Powinieneś je mieć przy sobie.
\item Zapytaj policjanta, czy dane zaginionej osoby jeszcze tego samego dnia zostaną wprowadzone do policyjnego systemu komputerowego.
\item Poproś o~zaświadczenie potwierdzające, że zaginięcie zostało przyjęte. Przyda Ci się ono w trakcie prowadzonych poszukiwań oraz do zgłoszenia zaginięcia do ITAKI\footnote{Centrum Poszukiwań Ludzi Zaginionych \href{http://zaginieni.pl}{http://zaginieni.pl}}.
\end{itemize}

Pamiętaj:
\begin{itemize}
\item Policjant ma obowiązek przyjąć zgłoszenie od razu. Nie ma żadnego przepisu mówiącego, że od chwili zaginięcia do przyjęcia zgłoszenia przez policję musi upłynąć jakiś czas (np. 48~godzin).
\item Policjant powinien wydać zaświadczenia o~przyjęciu zgłoszenia na żądanie osoby zgłaszającej.
\item Policjant nie ma prawa komentować Twojej sprawy w~sposób dla Ciebie nieprzyjemny lub obraźliwy.
\item Jeżeli nie jesteś zadowolony z~pracy policji --- możesz poskarżyć się do zwierzchników policjanta zajmującego się Twoją sprawą. O~skardze poinformuj ITAKĘ (przyślij kopię pisma).
\item Instytucją, do której zadań i~obowiązków należy poszukiwanie osób zaginionych, jest policja. Dlatego tak ważne jest, aby --- w~przypadku zaginięcia osoby --- jak najszybciej zgłosić to w~najbliższej komendzie policji.
\end{itemize}
\end{description}

Każda krytyczna sytuacja jest inna i~nie ma jednolitej recepty na postępowanie w~takiej chwili. Czytając opisy wypadków każdy może myśleć jak zachowałby się na miejscu tamtych osób. Jest to jednak tylko czcze gdybanie, gdyż działanie w~stresie doprowadza ludzi do stanu, którego nie są w~stanie przewidzieć. Każdego może sparaliżować strach, widok krwi czy połamanych kości, groza sytuacji może wywołać odrętwienie, w~końcu stres może doprowadzić do nieprzewidzianych i~nerwowych zachowań. Nie każdy przecież może być ratownikiem\ldots
%\\
%\\
\newpage
\small{
\emph{
\begin{flushright}Warszawa, 5-08-2012\end{flushright}
\begin{center}Komunikat prasowy nr 1/08/2012\end{center}
\textbf{Około godziny 16:00 w~dniu 5~sierpnia br. podczas nawałnicy na obozie harcerskim w~trakcie ewakuacji doszło do wypadku. Drzewo przygniotło 11-letnią uczestniczkę, mimo akcji ratunkowej dziecko zmarło.}\\\\
Na obozie harcerskim Szczepu Harcerskiego ,,Czternastka'' nad jeziorem Wędromierz koło miejscowości Borowy Młyn w~powiecie międzyrzeckim (woj. lubuskie) wypoczywało pod okiem instruktorów harcerskich (wychowawców) ok.~40 dzieci.\\\\
Około godziny 16:00 podczas nawałnicy, w~trakcie ewakuacji doszło do wypadku. Drzewo przygniotło 11-letnią uczestniczkę, mimo akcji ratunkowej dziecko zmarło.\\\\
W~trakcie zarządzonej ewakuacji zostały wezwane służby ratunkowe, a~pozostałe dzieci przetransportowane w~bezpieczne miejsce. Kadra obozu w~tak krytycznej sytuacji postępowała bardzo sprawnie. ,,Jesteśmy w~trakcie organizowania pomocy psychologicznej dla harcerek i~harcerzy, rodziców i~kadry obozu. W~miarę potrzeb będą odbywać się spotkania indywidualne i~grupowe. Poprosiliśmy również o~pomoc Miejski Punkt Interwencji Kryzysowej'' --- mówi Leszek Masklak, Przewodniczący Okręgu Wielkopolskiego ZHR. Po konsultacjach z~rodzicami uczestników obozu, komenda wraz z~przedstawicielami władz Okręgu Wielkopolskiego zdecydowali o~wcześniejszym zakończeniu obozu. Dzieci są już pod opieką rodziców.\\\\
Rokrocznie podczas Harcerskiej Akcji Letniej pod opieką instruktorów harcerskich (wykwalifikowanych wychowawców i~kierowników wypoczynku) przebywa ponad 10000 dzieci i~młodzieży na ponad 200 obozach i~zgrupowaniach harcerskich. Proces wewnętrzny i~zewnętrzny zatwierdzania obozów jest bardzo rygorystyczny. W~Związku Harcerstwa Rzeczypospolitej wewnętrzna procedura weryfikacyjna dopuszczająca obóz do realizacji jest trójstopniowa. Obóz w~Borowym Młynie spełniał wszystkie wymagane przepisami prawa oraz regulaminami wewnętrznymi ZHR wymagania.\\\\
Na wszystkie pytania w sprawie informacji udzielają:\\
Druhna Anna Kowalska --- rzecznik prasowy ZHR\\
Druh Leszek Masklak --- Przewodniczący Okręgu Wielkopolskiego ZHR}}
%\\
%\\
\newpage
\small{
\emph{ %http://tito0110.republika.pl/harcerka.htm
\begin{center}Pijany kierowca zabił harcerkę\end{center}\label{pijany-wjechal-w-kolumne}
\textbf{Zaczął od dwóch piw. Później była wódka --- wychylił dwa razy po pół szklanki, poprawił pięćdziesiątką. Gdy kolega poprosił go o~podwiezienie, wsiadł do samochodu, włączył magnetofon i~mocno nadepnął na pedał gazu. Przez jego lekkomyślność zginęła harcerka, a~kilka innych osób zostało rannych.}\\\\
W~środę Sławomir~L. stanął przed stołecznym Sądem Rejonowym.\\\\
Prokuratura zarzuciła mu spowodowanie katastrofy przez ,,umyślne naruszenie zasad bezpieczeństwa''. W~sierpniu 1998~r. mając we krwi 2,17 promila alkoholu i~pędząc polonezem ponad 100~km/h, wjechał w~kolumnę harcerzy wracającą z~rajdu. Idąca z~tyłu grupy Agnieszka~G. ucierpiała najbardziej. Złamana kość ramieniowa, cztery żebra i~wylew krwi do mózgu nie dały lekarzom żadnych szans. Dziewczyna zmarła w~szpitalu, nie odzyskawszy przytomności.\\\\
Według opinii biegłego harcerze poruszali się zgodnie ze wszystkimi przepisami. Idącą prawą stroną szosy kolumnę otwierała harcerka z~latarką świecącą białym światłem, a~zamykała Agnieszka z~czerwoną sygnalizacją. Piesi w~żaden sposób nie przyczynili się do wypadku, nie mieli też żadnych możliwości jego uniknięcia --- mówił prokurator, powołując się na ekspertów. Poza Agnieszką obrażenia odniosło jeszcze dwoje harcerzy i~pasażer poloneza. Kilkanaście innych osób miało siniaki i~zadrapania. Według oskarżyciela Sławomir~L. zagroził życiu i~zdrowiu każdego z~maszerujących.\\\\
Oskarżony przyznał się do winy. Zaprzeczył natomiast, by z~tyłu kolumny ktoś niósł czerwone światło. Po dociekaniach prokuratora przyznał jednak, że to on go nie widział, a~jak było faktycznie, nie wie. Czy po takiej ilości alkoholu nie zauważył pan u~siebie spowolnionych reakcji? --- pytał oskarżyciel. Cały czas dobrze mi się jechało. Gdyby nie było tam tak ciemno, nic by się nie stało --- stwierdził podsądny. Dodał, że bezpośrednio po wypadku badano jego refleks i~test wypadł pomyślnie. Być może jechałem trochę za szybko --- przyznał w~końcu. Zaznaczył jednak, że na pewno nie z~prędkością 100 km/h. Wiem to, bo po przekroczeniu dziewięćdziesiątki silnik zaczynał wyć --- oświadczył.\\\\
Następnie sąd zaczął przesłuchiwanie świadków wypadku. 23---letnia Beata~R., jedna z~opiekunek grupy, kategorycznie stwierdziła, że kolumna była oznakowana zarówno z~przodu, jak i~z~tyłu. Kierowca krzyczał, używając wulgaryzmów. Już z~daleka czułam alkohol --- mówiła. Dlaczego nie wspomniała pani o~tym w~czasie pierwszego przesłuchania? --- pytał obrońca, dodając, że wówczas świadek twierdziła, iż po tragedii jego klient płakał przykucnąwszy w~rowie. Dziś nic takiego sobie nie przypominam --- odpowiedziała kobieta.\\\\
Agnieszka zginęła mając 23 lata. Jej matka (w~procesie występuje w~roli oskarżyciela posiłkowego) nigdy nie pogodzi się z~jej śmiercią. Była ideałem córki. Zaliczano ją do najlepszych studentek. Skończyła kurs języka migowego, dawała korepetycje z~języka angielskiego. Jej marzeniem było uczyć małe dzieci --- mówi. Po tragicznej śmierci prezydent RP odznaczył Agnieszkę Medalem za Ofiarność i~Odwagę, PCK --- Odznaką Honorową, a~ZHP --- Złotym Krzyżem za Zasługi.\\\\
Sławomir~L. ma 29 lat. Nie po raz pierwszy staje przed sądem. Skazywano go już za zagarnięcie mienia społecznego znacznej wartości (rok i~sześć miesięcy) oraz zabór pojazdu w~celu krótkotrwałego użycia (pięć miesięcy).}}
\\
\\
\small{
\emph{ %http://wiadomosci.gazeta.pl/wiadomosci/1,114873,6202770,Policja_wyjasnia_przyczyny_smierci_14_latka_w_Tatrach.html
\begin{center}Policja wyjaśnia przyczyny śmierci 14-latka w~Tatrach\end{center}
\textbf{Zakopiańska policja wyjaśnia okoliczności śmiertelnego wypadku narciarza.}\\\\
W~sobotę na trasie narciarskiej w~Czarnej Górze 14-letni chłopiec z~Siedlec zjechał z~trasy i~uderzył w~skarpę polnej drogi.\\\\
Mimo natychmiastowej pomocy i~transportu śmigłowcem do szpitala nastolatek zmarł. Chłopiec przebywał na Podhalu na zimowisku w~Małem Cichem. Kolonia liczyła 22~chłopców w~wieku od 14 do 17 lat i~miała trzech opiekunów.\\\\
Monika Kraśnicka Broś --- rzecznik zakopiańskiej policji --- potwierdza, że opiekunowie byli z~grupą i~byli trzeźwi. Prawdopodobnie chłopiec odłączył się od grupy, zjechał poza wyznaczony tor jazdy i~uderzył w~skarpę polnej drogi --- dodała. Nastolatek, mimo że jeździł na nartach w~kasku, zmarł na skutek bardzo poważnych obrażeń głowy.\\\\
Ze wstępnych ustaleń wiadomo, że chłopiec od kilku lat jeździł na nartach. Zimowisko, na którym przebywał, zostało zorganizowane przez klub sportowy.}}
\\
\\
\small{
\emph{ %http://www.kurierlubelski.pl/artykul/496861,gnojno-15-letnia-harcerka-zabladzila-w-nocy-w-lesie,id,t.html
\begin{center}15-letnia harcerka zabłądziła w~nocy w~lesie\end{center}
\textbf{15-letnia uczestniczka obozu harcerskiego w~Gnojnie (gm. Konstantynów, powiat bialski) straciła orientację i~zaginęła w~lesie. Dziewczyny szukali pogranicznicy i~strażacy.}\\\\
W~niedzielę po godz. 2:00 w~nocy policjanci dostali zgłoszenie o~zaginięciu 15-letniej uczestniczki obozu wypoczynkowego w~miejscowości Gnojno, gm. Konstantynów. Dziewczyna pochodząca z~Siedlec zagubiła się w~czasie nocnej zabawy w~podchody.\\\\
Dziewczyna miała przy sobie mapę, jednak najprawdopodobniej ciemności sprawiły, że nie potrafiła jej prawidłowo zorientować --- mówi podkom. Jarosław Janicki, z~Komendy Miejskiej Policji w~Białej Podlaskiej. Sytuację utrudniał fakt, że nie posiadała przy sobie telefonu komórkowego.\\\\
W~poszukiwaniu nastolatki uczestniczyło siedmiu policjantów z~psem, 17 strażaków z~OSP oraz dwuosobowy patrol Straży Granicznej. Dziewczynka została znaleziona przez patrol z~Komendy Powiatowej Policji z~Łosic, który na prośbę bialskiego oficera dyżurnego, sprawdzał drogę do Gnojna od strony granicy z~woj. mazowieckim.\\\\
15-latka została znaleziona kilka minut po godzinie 4:00. Była w~dobrym stanie fizycznym i~nie potrzebowała pomocy lekarskiej. Okazało się, że sama wyszła z~lasu, jednak potem pomyliła kierunki i~zamiast do schroniska skierowała się w~przeciwną stronę.}}

% Brakuje mi działu - co robić gdy zdarzy się wypadek - jest to co prawda opisane, gdy jakaś wycieczka się zgubiła i Komendant zgłosił się do GOPR i policji, ale warto by temu poświęcić cały rozdział ty by jasno było wiadomo co robić - GOPR,112, policja, Okręg, rodzice itd.
% Warto też omówić przykładowe wypadki jakie się zdarzały, w związku z tym na co uważać np. były wypadki gdy kierowca wpadł w kolumnę harcerzy, gdy ktoś spadł w Tatrach, dostał udaru w górach, złamał nogę itd. Tak by ludzie automatycznie wiedzieli co robić. Nie chodzi o poradnik medyczny, ale jak zachować się w pierwszym momencie i co robić z pozostałą grupą.
\cleardoublepage
\appendix

\flushbottom
\section{Umowa rezerwacji noclegów\label{wzor-umowy-rezerwacji}}
\subsection{Wzór}
\begin{center}UMOWA REZERWACJI MIEJSC NOCLEGOWYCH\end{center}

\noindent Zawarta w~dniu \uline{data} w~\uline{miejscowość} pomiędzy
\begin{enumerate}
\item \uline{Nazwa schroniska, szkoły, operatora pola biwakowego itp.} reprezentowanym przez \uline{imię i~nazwisko osoby, która ma prawo podpisywać umowy}, legitymującym się dowodem osobistym \uline{seria i~numer dowodu osobistego}, zwanym w~treści umowy ,,Zleceniobiorcą'', a
\item Okręgiem Pomorskim Związku Harcerstwa Rzeczypospolitej z~siedzibą w~Gdańsku, ul.~Zator Przytockiego~4, 80-245 Gdańsk, NIP: 957-090-39-62, reprezentowanym przez \uline{imię i~nazwisko osoby, która ma prawo podpisywać umowy}, legitymującym się dowodem osobistym \uline{seria i~numer dowodu osobistego}, zwanym w~treści umowy ,,Harcerzami'',
\end{enumerate}
w~sprawie wynajmu \uline{miejsc noclegowych i~wyżywienia} dla uczestników obozu harcerskiego \uline{nazwa obozu i~numery drużyn}.

\begin{center}\S 1\end{center}
Zleceniobiorca oświadcza, że jest właścicielem/operatorem \uline{nazwa schroniska, szkoły, pola biwakowego itp.} znajdującego się w~\uline{adres i~ewentualnie dokładny opis lokalizacji obiektu}.

\begin{center}\S 2\end{center}
Zleceniobiorca dokona rezerwacji miejsc noclegowych na rzecz Harcerzy i~odda im do~korzystania wymieniony w~\S~1 obiekt w~następującym zakresie, np.:
\begin{enumerate}[a)]
\item ilość osób:
\begin{itemize}
\item[-] 20~harcerek,
\item[-] 20~harcerzy,
\end{itemize}
\item ilość kadry:
\begin{itemize}
\item[-] 3~instruktorki,
\item[-] 3~instruktorów,
\end{itemize}
\item termin pobytu (dzień i~godzina):\\
\uline{np. od 7~lipca 2012 godz: 17:00 do 9~lipca 2012 godz: 10:00,}
\item udostępniony obszar/pomieszczenia:\\
\uline{Dokładnie wymienić co zostanie udostępnione dla harcerzy podczas pobytu, np.:}
\begin{itemize}
\item[-] Fragment pola biwakowego o~powierzchni 100~m$^{2}$ i~wymiarach 10~m.~x~10~m., znajdujący się w~odległości co najmniej 25~m. od toalet, umywalni, ustępów, ulicy, \uline{wymienić inne}. 2~toalety, 2~umywalnie z~bieżącą zimną wodą dostępne 24~godz./dobę, miejsce do przygotowywania posiłków. Miejsce do palenia ogniska.
\item[-] 3~sale lekcyjne o~powierzchni co najmniej 50~m$^{2}$ każda, 2~toalety zlokalizowane na tej samej kondygnacji co wspomniane sale lekcyjne, sala gimnastyczna \uline{ze sprzętem: piłki, materace itd.}, szatnia na \uline{ilość} osób, stołówka, kuchnia z~dostępem do lodówki, kuchenki gazowej/elektrycznej/pieca itd. Ławki szkolne oraz krzesła zostaną wyniesione z~sal lekcyjnych, a~szafki oraz pomoce naukowe zostaną zabezpieczone/zaplombowane przez Zleceniobiorcę przed przybyciem Harcerzy.
\item[-] 6~pokoi 3-osobowych na tej samej kondygnacji, wyposażonych w~jednoosobowe łóżka/łóżka piętrowe z~materacami, bez pościeli. 3~łazienki wspólne znajdujące się w~piwnicy schroniska z~dostępem do bieżącej zimnej wody 24~godz./dobę oraz bieżącej ciepłej wody codziennie w~godz. 7:00~-~9:00 i~18:00~-~20:00. W~łazienkach będą natryski dostępne dla Harcerzy codziennie w~godz.: 18:00~-~20:00 z~bieżącą zimną i~ciepłą wodą. Miejsce do palenia ogniska.
\item[-] 4~sale do przeprowadzania zajęć teoretycznych,
\item[-] salę gimnastyczną do przeprowadzania zajęć sportowych,
\item[-] itp.
\end{itemize}
\end{enumerate}

\begin{center}\S 3\end{center}
W~trakcie pobytu Harcerzy w~wymienionym w~\S~1 obiekcie nie będą prowadzone żadne prace budowlane ani remontowe, a~pomieszczenia udostępnione Harcerzom będą wykończone i~w stanie nadającym się do bezpiecznego korzystania.

\begin{center}\S 4\end{center}
Obiekt wymieniony w~\S~1 będzie udostępniony Harcerzom na wyłączność.\\
\uline{Ewentualnie:}\\
Sposób korzystania z~obiektu wymienionego w~\S~1 przez inne osoby w~czasie pobytu harcerzy: \uline{opisać}.\\
Harcerze mogą przebywać w~wymienionym w~\S~1 obiekcie 24~godz./dobę. \uline{(Może być inaczej, np.: Harcerze mogą przebywać w~wymienionym w~\S~1 obiekcie w~godz: 17:00~-~10:00 dnia następnego.)}

\begin{center}\S 5\end{center}
Koszt brutto wynajmu wymienionego w~\S~1 obiektu wynosi np.:
\begin{itemize}
\item[-] namiot/pokój 2-osobowy: \uline{kwota}/dobę/pobyt
\item[-] namiot/pokój/sala lekcyjna \uline{ilość}-osobowy: \uline{kwota}/dobę/pobyt
\item[-] dostęp do sali gimnastycznej: \uline{kwota}/dobę/pobyt
\item[-] dostęp do kuchni: \uline{kwota}/dobę/pobyt
\item[-] pobyt za 1 osobę: \uline{kwota}
\item[-] itd.
\end{itemize}
\newpage
\begin{center}\S 6\end{center}
Koszt brutto wyżywienia dla 1 osoby wynosi odpowiednio:
\begin{itemize}
\item[-] śniadanie: \uline{kwota}
\item[-] obiad: \uline{kwota}
\item[-] kolacja: \uline{kwota}
\end{itemize}

\begin{center}\S 7\end{center}
Zaliczka na poczet dokonania rezerwacji wynosi \uline{kwota}, słownie: \uline{kwota słownie}, płatna na konto Zleceniobiorcy do dnia \uline{data}. W~przypadku nienadesłania zaliczki w~podanym terminie rezerwacja nie będzie dokonana.

Dane Zleceniobiorcy do przelewu zaliczki: \\ \uline{dokładna nazwa i~adres, numer rachunku bankowego.}

\begin{center}\S 8\end{center}
Harcerze mają możliwość dokonania zmiany terminu pobytu oraz ilości osób.

\begin{center}\S 9\end{center}
Rozliczenie nastąpi na podstawie faktycznego ilościowego wykorzystania noclegów w~momencie opuszczenia przez Harcerzy obiektu wymienionego w~\S~1.

Zleceniobiorca po zakończeniu pobytu wystawi fakturę VAT na następujące dane:\\
Okręg Pomorski ZHR, ul.~Zator Przytockiego 4, 80-245 Gdańsk, NIP: 957-090-39-62.

\begin{center}\S 10\end{center}
Po dokonaniu rezerwacji bez zgody Harcerzy nie może nastąpić odwołanie rezerwacji miejsc noclegowych. W~przypadku naruszenia tego postanowienia Zleceniobiorca udostępni taką samą ilość miejsc/pokojów o~tym samym standardzie i~w~tej samej kwocie w~innym obiekcie znajdującym się najdalej w~sąsiedniej miejscowości w~stosunku do obiektu wymienionego w~\S~1.

\begin{center}\S 11\end{center}
W~przypadku odwołania rezerwacji przez Harcerzy w~terminie \uline{ilość} dni przed terminem pobytu wpłacona przez Harcerzy zaliczka przepada na rzecz Zleceniobiorcy.
 
\begin{center}\S 12\end{center}
W~sprawach nie unormowanych niniejszą umową zastosowanie mają:
\begin{itemize}
\item[-] regulamin wymienionego w~\S~1 obiektu (w~załączniku do umowy),
\item[-] odpowiednie przepisy kodeksu cywilnego.
\end{itemize}
\newpage
\begin{center}\S 13\end{center}
Niniejszą umowę sporządzono w dwóch jednobrzmiących egzemplarzach po jednym dla każdej ze stron.
\\
\\
\\
\indent \hspace{1cm} Zleceniobiorca: \hspace{5cm} Harcerze:\\
\\
\\
\indent \hspace{1cm} ........................ \hspace{4.5cm} .........................\\
\\
\\
\\
Załączniki:
\begin{enumerate}
\item Regulamin obiektu \uline{(schroniska, szkoły, pola biwakowego itd. --- przykład na stronie \pageref{regulamin-schroniska})}.
\end{enumerate}
\newpage
\subsection[Regulamin Szkolnego Schroniska Młodzieżowego w~Łodzi]{Regulamin Szkolnego Schroniska Młodzieżowego\\w~Łodzi\footnote{Źródło: \href{http://bip.uml.lodz.pl/edu/organizacja\_przedmiot.php?id=352}{http://bip.uml.lodz.pl/edu/organizacja\_przedmiot.php?id=352}}}
\begin{enumerate}\label{regulamin-schroniska}
\item Prawo do korzystania ze schroniska młodzieżowego w~Łodzi przysługuje młodzieży szkolnej i~studenckiej, nauczycielom i~wychowawcom oraz członkom Polskiego Towarzystwa Schronisk Młodzieżowych, jak również obcokrajowcom należącym do Międzynarodowej Federacji Schronisk Młodzieżowych.

Z~noclegu w~schronisku, w~razie wolnych miejsc, mogą korzystać również inne osoby pod warunkiem przestrzegania obowiązującego regulaminu.
\item Zasady rezerwacji miejsc.

Po uzyskaniu informacji drogą telefoniczną, internetową lub listowną o~dostępności miejsc w~danym terminie, osoby prywatne lub przedstawiciele grup dokonują wstępnej rezerwacji w~recepcji schroniska. Następnie pracownik recepcji przygotowuje umowę dla grupy, wysyła ją faxem, pocztą lub elektronicznie. Instytucja zamawiająca odsyła podpisaną umowę w~ciągu 10~dni od jej otrzymania, przyjmując warunki w~niej zawarte i~deklarując wpłatę zaliczki wyliczoną przez schronisko, zaliczka nie powinna przekraczać kwoty 25~\% rezerwowanej usługi.

Osoby prywatne, rezerwując miejsca na miesiąc wcześniej, powinny wpłacić zaliczkę na adres lub konto Schroniska. Grupy mogą dokonywać rezerwacji z~wyprzedzeniem nawet rocznym. Osoby indywidualne jeden miesiąc przed terminem.

Zwrot zaliczki może nastąpić w~przypadku nieskorzystania ze schroniska z~wyjątkowo ważnych i~udokumentowanych przyczyn uznanych przez dyrektora schroniska, po ewentualnym potrąceniu wcześniej poniesionych kosztów przez schronisko.

W~przypadku sporów sprawę należy kierować do Wydziału Edukacji Urzędu Miasta Łodzi ul.~Sienkiewicza~5. W~przypadku nienadesłania zaliczki 7~dni przed przybyciem do schroniska grupy, dyrektor może anulować zamówienie.
\item Turyści indywidualni w~grupach poniżej 5~osób są przyjmowani na noclegi bezpośrednio w~schronisku, jeżeli są wolne miejsca, z~tym, że pierwszeństwo ma młodzież szkolna i~studencka, a~następnie inne osoby.
\item Przybywający do schroniska (powinno to nastąpić w~godz. 16:00 --- 20:00), wypełniają ,,Kartę gościa'', pokazują dyżurującemu recepcjoniście legitymacje PTSM, legitymacje uczniowskie, studenckie lub dowód osobisty.

W~przypadku grupy --- dowód kierownika grupy i~listę uczestników, regulują należne opłaty, według cennika wywieszonego w~recepcji schroniska w~widocznym miejscu.
\item Ze schroniska korzystać można nie dłużej niż przez 3~kolejne noce, chyba że są wolne miejsca i~dyrektor schroniska wyrazi zgodę na dłuższy pobyt.
\item Osoby płci męskiej i~żeńskiej kwaterują się w~oddzielnych pokojach.
\item Osoby, które przebyły chorobę zakaźną, nie mogą w~okresie kwarantanny korzystać ze schroniska.
\item Kierownik zespołu wycieczkowego i~opiekunowie obowiązani są nocować w~schronisku razem z~uczestnikami. W~przeciwnym przypadku zespół nie może być przyjęty do schroniska.
\item Każdy nocujący otrzymuje świeżą bieliznę pościelową w~recepcji, sam dokonuje powleczenia, a~opuszczając schronisko zdaje bieliznę i~wszystkie wypożyczone przedmioty w recepcji.
\item Od godz. 22.00 do 6.00 obowiązuje w~schronisku cisza nocna z~wygaszeniem światła. Turyści, którzy przychodzą późno lub bardzo rano wychodzą ze schroniska, nie powinni zakłócać wypoczynku pozostałym osobom. Odchylenia od ustalonych godzin ciszy są możliwe tylko w~wyjątkowych przypadkach za wiedzą dyrektora schroniska.
\item Korzystający ze schroniska powinni najpóźniej do godz. 10.00 zasłać łóżka i~sprzątnąć pomieszczenia schroniska.
\item Przygotowywanie posiłków może się odbywać jedynie w~kuchniach samoobsługowych schroniska. Po spożyciu posiłku należy pozmywać naczynia kuchenne, a~kuchnię dokładnie sprzątnąć.
\item Wszelkie zniszczenia i~uszkodzenia przedmiotów stanowiących własność schroniska należy zgłaszać w~recepcji schroniska, następnie dyrektor określa wysokość odszkodowania.
\item W~schronisku obowiązuje schludny ubiór, spokojne i~uprzejme zachowanie się; picie alkoholu i~uprawianie gier hazardowych jest zabronione.
\item Palenie tytoniu w~sypialniach jest zakazane; palić wolno tylko w~miejscu wyznaczonym.
\item Zwierząt do budynku schroniska wprowadzać nie wolno.
\item W~razie przekroczenia regulaminu schroniska lub nieodpowiedniego zachowania się, dyrektor schroniska jest uprawniony do zatrzymania winnemu legitymacji (uczniowskiej, studenckiej, członkowskiej) i~usunięcia go ze schroniska, a~ponadto zawiadamia o~tym właściwą szkołę (uczelnię, organizację).
\item Korzystający ze schroniska mogą wszelkie pozytywne i~negatywne uwagi wpisywać do książki życzeń schroniska lub w~ważnych wypadkach kierować je do Wydziału Edukacji lub Kuratorium Oświaty w~Łodzi.
\item We wszystkich sprawach nie ujętych w~regulaminie, a~dotyczących toku życia w schronisku, jak: zapewnienie porządku, ochrona mienia, przestrzeganie zasad kultury itp., korzystający ze schroniska są obowiązani stosować się do wskazań dyrektora lub osób przez niego upoważnionych do pełnienia dyżuru.
\end{enumerate}
\noindent Regulamin opracowany na podstawie Statutu Schronisk Młodzieżowych w~Polsce i~zgodny ze statutem Szkolnego Schroniska Młodzieżowego w~Łodzi z~roku 2007.
\section{Przykładowa informacja o~spływie\newline (Drawa'2006)\label{info-o-splywie}}
W~niedzielę, 16 lipca, wieczorem dojeżdżamy do Czaplinka. Tam, nad jeziorem rozbijamy biwak na noc.

Wcześniej będziemy musieli podzielić się na pary --- każda para spędzi ze sobą cały spływ w~jednym kajaku. Postaramy się to zrobić tak, aby wszystkie pary były mniej więcej jednakowo silne. Jeśli zrobimy to dobrze, to taki podział pozostanie do końca spływu. Może się jednak okazać, że przeceniliśmy czyjeś siły --- w takim przypadku dokonamy pojedynczych zmian. Oprócz tego utworzymy grupy, które będą spały razem w~jednym namiocie --- te grupy nie będą się również zmieniały przez cały spływ. Postaramy się zrobić tak, aby osoby płynące w~tym samym kajaku spały w~tym samym namiocie. Większość namiotów będzie 4-osobowa, co oznacza, że w~jednym namiocie znajdą się obsady dwóch kajaków. Za transport namiotu odpowiadają obie ekipy, dzieląc się równo ładunkiem. Jednym słowem każda ekipa pilnuje i~transportuje sobie ,,połówkę'' namiotu, w~którym śpi.

Każdy dzień spływu będzie wyglądał podobnie: pobudka całego obozu o~godz. 7:00, potem część osób przygotowuje śniadanie, część składa namioty itp. Po śniadaniu pakujemy sprzęt i~bagaże do kajaków i~wypływamy na szlak. Po drodze będą krótkie przerwy i jedna dłuższa. W~miarę możliwości podczas tej dłuższej przerwy przygotujemy i~zjemy obiad. Po obiedzie płyniemy dalej. Zakupy robimy w~miejscowościach, przez które będziemy przepływać. Pływanie kończymy około godz. 18:00. Na biwaku wyciągamy kajaki z~wody, wypakowujemy namioty i~natychmiast je rozstawiamy. Potem wypakowujemy resztę sprzętu i~bagaży i~przebieramy się w~suche rzeczy. Część osób zabiera się za przygotowanie kolacji, pozostała część porządkuje, myje i~układa kajaki tak, żeby były gotowe na następny dzień. Jeśli będzie to możliwe, to wieczorem spędzimy trochę czasu przy ognisku, ale zależy to od czasu i~regulaminu pola, na którym będziemy biwakować. W~nocy warty będą pilnowały kajaków. Wartę będą pełnić wszyscy bez wyjątku w~każdą noc --- dzięki temu warty będą krótkie --- około 20 minut na osobę.

W~zamykanej komorze na rufie będziemy przewozić większość sprzętu i~bagaży. Dla zachowania wyważenia kajaków część rzeczy będzie zapakowana w~dziobie kajaka. Plecaki w~kajakach muszą być ułożone ,,plecami'' (szelkami) do góry, co oznacza, że podczas pakowania na ,,plecach'' plecaka mają się znaleźć wasze dokumenty, a zaraz pod nimi rzeczy, w~które macie się przebrać na biwaku i~które muszą być suche za wszelką cenę. Rzeczy, które będą na górze plecaka, znajdą się na spodzie, gdy plecak będzie leżał w~kajaku, i~będą najbardziej narażone na zamoczenie.

Mimo, że tylna komora jest zamykana, to nie można zagwarantować, że będzie całkowicie wodoszczelna. Lepiej więc dmuchać na zimne niż spać w~mokrym śpiworze, w~mokrym namiocie i~w~mokrych ubraniach. Dlatego wszystkie rzeczy osobiste zapakujecie w~worki foliowe w~następujący sposób:\\
Każda pojedyncza rzecz ma być zapakowana w~osobny worek foliowy: wkłada się rzecz do woreczka, ściskając woreczek wysysa powietrze, zawiązuje woreczek na ,,kokardkę'', potem wkłada się to do drugiego woreczka, wysysa powietrze i~znowu zawiązuje (nie używajcie żadnych plastikowych klipsów ani metalowych drucików do zawiązywania woreczków, bo dziurawią one sąsiednie worki). Czyli: każda koszulka, majtki, para skarpetek jest włożona osobno w~dwa woreczki. Potem można to sobie posortować w~reklamówki, żeby nie było bałaganu. Tak zabezpieczone wszystkie rzeczy wkładacie do dwóch grubych foliowych worków i~dopiero taką paczkę wkładacie do plecaka. Wtedy zawiązujecie worki i~zapinacie plecak. Alumata (karimata) też ma być zapakowana w~dwa worki. Jeśli wydaje się Wam to przesadą, to pewnie zdziwi was fakt, że taki sposób gwarantuje tylko 70\% bezpieczeństwo. Jeśli nie będzie padał deszcz ani kajak nie nabierze wody to zabezpieczenia się nie przydadzą. Jednak codziennie zdarza się, że: pada deszcz, lub jest duża fala na jeziorze, która przelewa się przez burty, albo ktoś za mocno chlapie wiosłami --- wtedy kajak działa jak wanna i~zbiera wodę w~środku --- w~ skrajnych przypadkach wasze rzeczy będą pływać w~100 litrach brudnej wody i~jeśli zaniechacie pakowania w~worki, albo worki będą dziurawe, albo perforowane, albo gdy zamiast worków użyjecie reklamówek, to wieczorem może się okazać, że:
\begin{itemize}
\item macie mokry śpiwór,
\item macie mokry namiot,
\item macie mokre rzeczy, w~które zamierzaliście przebrać się wieczorem,
\item macie mokre rzeczy, w~których mieliście spać,
\item macie mokre dokumenty, legitymację, portfel, apteczkę, latarkę itd.,
\item jeśli włożyliście wszystkie koszulki albo bieliznę do jednego worka, zamiast każdą rzecz osobno, to okaże się, że nie macie się w~co przebrać, albo dobitniej: nie macie nic suchego, więc możecie zostać w~mokrych kąpielówkach, albo przebrać się w~mokrą bieliznę (przeziębienie gotowe).
\end{itemize}
W kajaku jesteście ubrani~w:
\begin{itemize}
\item koszulkę, na którą będziecie mieli nałożoną i~zapiętą kamizelkę ratunkową,
\item strój kąpielowy / kąpielówki,
\item sandały (na gołe stopy),
\item rękawiczki bez palców,
\item kapelusz,
\item okulary przeciwsłoneczne (kto chce, bo mogą nie dotrwać do końca spływu, ale warto je wziąć).
\end{itemize}
Oprócz tego jesteście posmarowani kremem z filtrem przeciw opalaniu, a~pod ręką (czyli nie w~plecaku, tylko w~osobnym podwójnym worku) macie: krem przeciwsłoneczny, krótkie spodenki, kubek albo menażkę, koszulę flanelową, kurtkę od deszczu --- te rzeczy przydadzą się, gdy zrobi się chłodniej, gdy zacznie padać deszcz lub zacznie wiać wiatr albo na postoju, gdy trzeba będzie iść do sklepu na zakupy.

Rzeczy, w~które będziecie ubrani w~kajaku nie powinny być w~ciemnych kolorach (czyli nie czarne, nie granatowe itp.), bo gdy świeci słońce jest w~nich bardzo gorąco, co może skończyć się przegrzaniem. Wybierzcie ubrania w~jasnych i~żywych kolorach.

Wasza legitymacja szkolna, karta pływacka i~prywatne pieniądze mają być zapakowane w~osobnym, małym, podwójnym worku w~kieszeni plecaka albo na wierzchu plecaka, tak, żebyście mogli je wyjąć w~ciągu minuty bez potrzeby rozpakowywania plecaka. Książeczka RUM musi być zapakowana osobno w~2~worki i~porządnie zabezpieczona, też schowana w~kieszeni albo na wierzchu plecaka.

Oprócz tego w~kajaku macie butelki z~piciem dla was na drogę (,,big łyki''). Dostaniecie także kilka butelek 1,5~litrowych z~wodą pitną, która zostanie wykorzystana do przygotowania obiadu, zrobienia napoju po obiedzie albo kolacji itd. --- tej wody nie będzie wolno wam zużyć bez zgody kadry.

Wszystkie rzeczy, które znajdą się luzem w~kajaku muszą być powiązane ze sobą cienką linką. Ta linka przyda się też gdy trzeba będzie holować kajak przez jakiś uciążliwy odcinek rzeki, a także do suszenia mokrych rzeczy na biwakach.

Nie bierzcie dużych zapasów słodyczy, kosmetyków itd. --- lepiej wziąć trochę więcej pieniędzy i~w~razie czego dokupić sobie rzeczy, które się Wam skończą, niż wozić je ze sobą. Nie będziemy płynąć przez odludzie.
 
\subsection{Co każdy ma zabrać}
\subsubsection{Rzeczy używane podczas pływania\label{podczas_plywania}}
\begin{checklist}
\item koszulki (max. 2 szt.) (najlepiej T-shirt’y, ale jeśli ktoś bardzo chce może wziąć zamiast 1~T-shirt’a koszulkę bez rękawków / na ramiączkach),
\item kąpielówki / strój kąpielowy --- musi być wygodny, będziecie spędzać w~nim całe dnie, nie może być za ciasny ani nigdzie uwierać. Jeśli ktoś ma, może zabrać krótkie spodenki z~pianki,
\item sandały turystyczne z~rzemykami z~pasków parcianych (nie skórzane), ostatecznie klapki. Postarajcie się za wszelką cenę mieć zapinane sandały, które trzymają piętę --- w~tych sandałach będziecie wchodzić często do wody, więc muszą się trzymać mocno na nodze a~nie latać jak klapki,
\item kapelusz materiałowy z~rondem na głowę (ostatecznie czapka z~daszkiem, ale kapelusz jest dużo lepszy),
\item wodoodporny krem z~filtrem (minimum 12) do opalania --- 1~duże opakowanie (nie olejek do opalania, tylko krem ochronny przeciw opalaniu. Dla wrażliwych filtr 16 minimum),
\item cienka, oddychająca kurtka przeciwdeszczowa z~kapturem, ostatecznie płaszcz foliowy (kilka sztuk --- są bardzo nietrwałe),
\item okulary przeciwsłoneczne, najlepiej na sznurku,
\item koszula flanelowa, ciepła bluza lub sweter --- przyda się podczas płynięcia, gdy zrobi się chłodniej. Najlepsza jest koszula flanelowa, przydaje się w~różnych sytuacjach, bo można ją związać u~dołu,
\item krótkie spodenki (najlepiej do połowy uda) --- bądźcie przygotowani na to, że czasem trzeba będzie w~nich wskoczyć do wody, więc nie bierzcie żadnych ,,workowatych'',
\item rękawiczki bez palców (wiosła będą prawdopodobnie aluminiowe --- mogą bardzo brudzić ręce),
\item 10 metrów mocnej linki,
\item kubek / menażka (do wylewania wody z kajaka),
\item 3~butelki ,,big łyki'' (koniecznie z~dużym otworem) co najmniej po pół litra każda (Kto chce może wziąć mały metalowy termos zamiast 1 butelki. Szklane termosy się nie nadają).
\end{checklist}

\subsubsection{Rzeczy używane na postojach\label{na_postojach}}
\begin{checklist}
\item przybory toaletowe i kosmetyki,
\item ręcznik średniej wielkości, zapakowany w~dwa worki,
\item bielizna (bez przesady z~ilością, zawsze można zrobić pranie),
\item T-shirt (max. 1~szt.) --- w~ten T-shirt przebieracie się po przypłynięciu na biwak, żeby nie chodzić w~tym, w~którym płynęliście (bo będzie on po prostu mokry i~przepocony),
\item skarpetki (max. 3 pary),
\item menażka / kubek, niezbędnik,
\item ostry nóż, zabezpieczony w~futerale,
\item śpiwór. Śpiwór składacie odpowiednio, zwijacie w rulon, wkładacie w~dwa mocne duże worki i~to wszystko razem wkładacie do worka kompresyjnego. Potem, ściskając śpiwór, zawiązujecie pierwszy worek foliowy, potem drugi a~na końcu worek kompresyjny. Jeśli macie mały worek kajakowy, to weźcie go zamiast worka kompresyjnego od śpiwora. Tak zapakowany śpiwór na czas podróży ma być przytroczony na zewnątrz plecaka tak, żeby można go było łatwo odczepić,
\item komplet do spania, czyli krótkie spodenki bawełniane (mogą być bokserki) i~T-shirt albo jakaś lekka, cienka piżama. Jeśli ktoś marznie w~śpiworze, to bierze w~zamian coś cieplejszego
\item alumata, ostatecznie karimata --- zapakowana w~dwa worki,
\item buty (lekkie i~małe), najlepiej jakieś lekkie buty trekkingowe pod kostkę, adidasy lub coś podobnego. Buty podczas płynięcia podróżują porządnie zapakowane w~dwa worki, tak jak śpiwór --- czyli nie włożone do plecaka
\item porządne, mocne długie spodnie w~ciemnym kolorze (M-65 lub podobne, jeśli macie), odradzam dżinsy --- schną bardzo długo (rano i~wieczorem nad wodą często jest rosa) --- długie spodnie potrzebne są wieczorami i~rano, gdy jest chłodno,
\item polar rozpinany --- będzie przydatny, podobnie jak długie spodnie, tylko wieczorami i~rano. Koszula flanelowa (bluza lub sweter) mogą być mokre po całym dniu płynięcia --- musicie mieć coś suchego i~ciepłego do włożenia na biwakach,
\item cienkie rękawiczki wełniane z~palcami (zimowe) --- wieczorami i~w~nocy nad wodą bywa bardzo zimno,
\item latarka, najlepiej czołówka (pamiętajcie o~nowych bateriach),
\item środek przeciw komarom (,,Off'' lub coś podobnego).
\end{checklist}

\subsubsection{Przechowywanie i~zabezpieczenie ,,bagażu''}
\begin{checklist}
\item plecak typu kostka lub coś podobnego, ale nie większy, bez stelaża (można wyjąć) albo worek kajakowy, jeśli ktoś ma, o~pojemności takiej jak plecak kostka (czyli 20 litrów),
\item mocne i~grube worki (np. na śmieci) --- cienkie i~perforowane worki się nie nadają,
\item woreczki do produktów spożywczych (mocne, najlepsze są takie jak do mrożonek. Woreczki do kanapek są przeważnie cienkie i~perforowane i~w~większości się nie nadają),
\item zapasowe worki, duże i~małe --- przydadzą się, gdy poprzednie się przedziurawią.
\end{checklist}

\subsubsection{Pozostałe}
\begin{checklist}
\item zegarek (kto chce),
\item chusteczki higieniczne,
\item ważna legitymacja szkolna,
\item karta pływacka (kto ma),
\item pieniądze (bez przesady),
\item apteczka osobista --- leki, które musicie przyjmować (dla alergików itp. itd.),
\item troki parciane (nie skórzane) --- kilka sztuk do przytroczenia śpiwora, alumaty, butów, menażki, ,,big łyków'' itd. Im więcej troków weźmiecie, tym lepiej. Najlepsze są o~długości co najmniej 1~metra. Troki podpiszcie, żeby było wiadomo czyje są.
\end{checklist}

\subsubsection{Zestaw na powrót}
\begin{checklist}
\item bielizna --- 1 para,
\item skarpetki --- 1 para,
\item T-shirt --- 1 sztuka,
\item lekkie, cienkie długie spodnie --- 1 para.
\end{checklist}
Zestaw na powrót ma być zapakowany porządnie, tak jak wszystko, czyli każda rzecz w~dwa worki, do tego to wszystko razem włożone do dwóch porządnych worków i~spakowane na dno plecaka. Wyjmiecie to dopiero po zakończeniu spływu, przed wyruszeniem w~drogę powrotną do domu.

\subsection{Zestaw na wyjazd}
Nie bierzecie żadnych innych rzeczy oprócz już wymienionych, co oznacza, że na podróż z~domu na obóz musicie się ubrać w~rzeczy wymienione powyżej, czyli:
\begin{itemize}
\item bielizna (pkt. \ref{na_postojach}),
\item T-shirt (pkt. \ref{podczas_plywania}),
\item skarpetki (pkt. \ref{na_postojach}),
\item buty (adidasy) (pkt. \ref{na_postojach}),
\item długie spodnie (pkt. \ref{na_postojach}), chyba, że będzie upał, to krótkie spodenki (pkt. \ref{podczas_plywania}),
\item jeśli będzie zimno, to koszula flanelowa (pkt. \ref{podczas_plywania}) albo polar (pkt. \ref{na_postojach}),
\item jeśli będzie padał deszcz, to kurtka od deszczu (pkt. \ref{podczas_plywania}),
\item jeśli nie będzie padał deszcz, to kapelusz na głowę (pkt. \ref{podczas_plywania}),
\item jeśli będzie słońce, to okulary przeciwsłoneczne (pkt. \ref{podczas_plywania}),
\item na drogę weźcie sobie coś do picia w butelki ,,big łyki'' (pkt. \ref{podczas_plywania}) i~jakieś kanapki albo coś do jedzenia, tak, żeby starczyło wam do wieczora (pkt. \ref{podczas_plywania}),
\item sandały, śpiwór, alumatę, menażkę i~wszystko, co się nie zmieściło do plecaka, przytroczcie porządnie trokami do plecaka tak, żebyście mieli wolne ręce.
\end{itemize}
\section{Przykładowa lista ekwipunku uczestnika\\obozu wędrownego w~górach\label{lista-ekwipunku-uczestnika}}
\subsection{Podstawa}
\begin{itemize}
\item Jedziemy w~góry, wszystko trzeba będzie nosić, więc bierzemy najmniej jak się da. Bieliznę i~inne ubrania będziemy prać, więc bierzemy tylko po kilka sztuk.
\item Jeśli czegoś zabraknie, to na miejscu kupimy, nie jedziemy na odludzie, na miejscu są sklepy.
\item Plecak ma być spakowany porządnie, wszystko ma być posortowane i~włożone do worków na mrożonki.
\item Każdy sam pakuje swój plecak w~taki sposób, aby wiedział, gdzie co jest.
\item Wszystko powinno być wewnątrz plecaka, nic na zewnątrz, a~jeśli już, to porządnie przytroczone trokami, żadnych sznurków.
\item W~plecaku musi być jeszcze miejsce na jedzenie obozowe, które będziemy nosić ze sobą.
\item W~przypadku wątpliwości dotyczących kupowania nowego sprzętu polecam testy sprzętu: \href{http://www.ngt.pl}{http://www.ngt.pl} i~opinie użytkowników: \href{http://ngt.pl/forum/}{http://ngt.pl/forum/}.
\end{itemize}

\subsection{Obuwie}
\begin{checklist}
\item Buty trekingowe na porządnej podeszwie, sznurowane. Nie muszą być z~membraną. Mają być wygodne i~rozchodzone. Za duże o~1/2 numeru.
\item Sandały (nie skórzane) --- powinny mieć paski pod całą podeszwą, a nie tylko mocowane po bokach.
\item Adidasy lub inne lekkie buty do chodzenia/biegania ,,nie na szlaku''.
\end{checklist}

\subsection{Bielizna}
\begin{checklist}
\item Majtki --- maksymalnie 6~par --- jeśli ktoś ma z~tkaniny termoaktywnej to wziąć koniecznie.
\item Cienkie skarpetki z coolmaxem --- 3 pary.
\item Grube skarpetki z coolmaxem --- 3 pary.
\item Koszulki z krótkim rękawem --- 3 pary --- jeśli ktoś ma z~tkaniny termoaktywnej to wziąć koniecznie.
\end{checklist}

\subsection{Odzież}
Nie brać ubrań, szczególnie spodni i~krótkich spodenek, w~jasnych kolorach, bo szybko się brudzą.
\begin{checklist}
\item Spodnie długie, najlepiej M-65 lub podobne --- 1 para.
\item Krótkie spodenki z~nogawkami nad kolana --- 2 pary.
\item Pasek do spodni.
\item Koszula flanelowa z~długim rękawem.
\item Polar 200 albo 300.
\item Kurtka od deszczu, z~kapturem, cienka, z~membraną, nie brać kurtek zimowych.
\item Spodnie od deszczu z~cienkiej membrany --- kto ma, niekoniecznie.
\item Kapelusz z rondem / czapka z daszkiem.
\item Chustka na głowę / apaszka.
\item Koszulka drużyny i~umundurowanie letnie, jeśli jest.
\item Kąpielówki.
\item Zestaw na powrót (lekkie cienkie długie spodnie, majtki, koszulka, cienkie skarpetki --- porządnie zapakowane w~wodoszczelny worek).
\end{checklist}

\subsection{Akcesoria kuchenne}
\begin{checklist}
\item Menażka okrągła, żadnych jajowatych, ,,lornetek'' ani innego wojskowego złomu.
\item Łyżka.
\item Widelec.
\item Nóż w~pokrowcu / futerale: finka lub scyzoryk (naostrzony).
\item Kubek, najlepiej termiczny.
\end{checklist}

\subsection{Kosmetyki i~przybory toaletowe}
\begin{checklist}
\item Ręcznik mały.
\item Ręcznik kąpielowy.
\item Mydło.
\item Pasta do zębów.
\item Szampon do włosów.
\item Szczoteczka do zębów.
\item Środek przeciw komarom.
\item Obcinaczka do paznokci / skórek.
\item Grzebień.
\item Krem ochronny z~filtrem.
\item Chusteczki higieniczne.
\item Papier toaletowy.
\end{checklist}
Wszystkie płynne kosmetyki warto przelać do mniejszych buteleczek, około 100~ml., aby nie nosić potem zbędnego ciężaru.

\subsection{Spanie}
\begin{checklist}
\item Piżama / coś do spania.
\item Śpiwór w pokrowcu / worku kompresyjnym.
\end{checklist}

\subsection{Transport}
\begin{checklist}
\item Plecak, dopasowany do wzrostu --- po spakowaniu wszystkiego plecak nie może ważyć więcej niż 1/4 wagi ciała.
\item Pokrowiec przeciwdeszczowy na plecak.
\item Troki zapasowe do plecaka.
\item Woreczki foliowe do mrożonek, bez metalowych klipsów, 3 litrowe --- 1 paczka.
\end{checklist}

\subsection{Pozostałe}
\begin{checklist}
\item Lekarstwa osobiste (na alergię, chorobę lokomocyjną i~inne).
\item Sznurek / linka ok. 2 metry.
\item Długopis.
\item Cienki zeszyt w kratkę --- 16 kartek / notatnik / notes.
\item Pieniądze na drobne wydatki (w~małym portfelu).
\item Ważna legitymacja szkolna ze zdjęciem właściciela i~wszystkimi wymaganymi pieczątkami.
\item Okulary słoneczne.
\item Okulary korekcyjne (jeśli ktoś nosi).
\item Latarka --- czołówka z~zapasową żarówką.
\item Igła i~nici.
\end{checklist}
\section{Instrukcja postępowania w~przypadku zdarzenia ubezpieczeniowego mogącego skutkować odpowiedzialnością cywilną instruktora lub Związku\label{instrukcja-postepowania-wypadek}}
\begin{center}(Uchwała Naczelnictwa ZHR nr 190/4 z~dnia 22 kwietnia 2007~r.)\end{center}

Wystąpienie zdarzenia ubezpieczeniowego o~charakterze utraty zdrowia, życia lub mienia osób trzecich. (Np. wypadek śmiertelny; wypadek który zakończył się trwałym uszczerbkiem na zdrowiu; inne zdarzenia, w~trakcie których ktoś poniósł stratę materialną --- pożar lasu, domu, kamienicy, stodoły, zalanie mieszkania, biura, zniszczenie wynajętego lub użyczonego samochodu, poczęcie dziecka przez osobę nieletnią, molestowanie seksualne przez innych uczestników lub instruktorów, mobbing, wyciek danych osobowych do osób nieuprawnionych --- np. informacje o~stanie zdrowia, informacje teleadresowe itp.)
\begin{enumerate}
\item W~miarę możliwości ogranicz negatywne konsekwencje zdarzenia (chroń osoby, które nie zostały pokrzywdzone, udziel pierwszej pomocy osobom poszkodowanym, wzywaj fachową pomoc (lekarz, straż, policja, psycholog itp.).
\item Zadzwoń do przełożonych na poziomie Zarządu Okręgu lub Naczelnictwa i~przedstaw sprawę.
\item Jeżeli sprawa dotyczy niepełnoletniego uczestnika, ustal jedną osobę do kontaktów z~rodzicami tego uczestnika. Osoba ta niezwłocznie powinna powiadomić rodziców o~zdarzeniu i~być stale dostępna pod telefonem komórkowym dla rodziców.
\item Spisz jak najdokładniej wszystkie okoliczności zdarzenia dopóki je jeszcze pamiętasz na specjalnym formularzu. Pamiętaj o~datach i~godzinach. Wyślij taki opis faxem do biura Ubezpieczyciela tel +48 22 452 39 89 jak najszybciej, lecz nie później niż 3~dni od daty zdarzenia.
\item W~razie jakichkolwiek roszczeń wysuwanych przez osoby poszkodowane (rodzice, sąsiedzi, właściciele utraconego mienia, itp.) nie podejmuj żadnych rozmów i~ustaleń. Przekaż nr polisy OC Związku i~powiedz, że nie jesteś upoważniony do odpowiadania na jakiekolwiek pytania. Możesz przekazać nr telefonu i/lub~adresy do Okręgu i~Naczelnictwa z~sugestią, aby osoby poszkodowane tam zgłaszały swoje roszczenia --- najlepiej na piśmie.
\item Bądź miły i~ludzki. Staraj się pomóc osobom poszkodowanym, ale bądź powściągliwy, jeżeli chodzi o~udzielanie informacji. Nigdy do końca nie wiesz, jak odbierze je Twój rozmówca i~jak potem zrelacjonuje przebieg sprawy.
\item Osoba z~biura Ubezpieczyciela możliwie najszybciej będzie starała się udzielić Ci odpowiedniej pomocy, a~być może odwiedzi Cię osobiście i~przejmie sprawę, jeżeli sytuacja będzie tego wymagała.
\item W~przypadku zainteresowania ze strony mediów w~porozumieniu z~osobą z~Okręgu lub Naczelnictwa oraz Ubezpieczycielem przygotuj krótką informację nt. zdarzenia podkreślając słowo wypadek. Nie wpuszczaj mediów na teren obozu (żadnych kontaktów z~uczestnikami i~instruktorami!). W razie problemów proś o~pomoc policję.
\end{enumerate}
\section{Ewaluacja obozu --- minusy\label{ewaluacja-minusy}}
Poniżej znajduje się autentyczna lista minusów z~komentarzem opracowanym po pewnym obozie, na który pojechały wspólnie: drużyna harcerek, drużyna wędrowniczek i~drużyna wędrowników. Ewaluacja została przeprowadzona przez kadrę podobozu harcerek. Od myślników wymieniono pojedyncze głosy instruktorek.
\subsection{Dezorientacja wśród kadry}
Nie wszyscy znają plan dnia po dojściu do schroniska. W~przyszłości należy zapoznać całą kadrę z~najistotniejszymi ustaleniami i, oczywiście, dać jej prawo głosu.
\begin{itemize}
\item[-] Plan obozu, czyli wszystkie zajęcia, powinny być omówione z~wszystkimi osobami z~kadry przed obozem, żeby mogła zostać uzgodniona ich forma oraz aby utrwaliła się kolejność.
\item[-] Przed obozem powinno się przedyskutować kompetencje oboźnej i~komendantki, bo później jedna nie chce wchodzić w~kompetencje drugiej. W~efekcie nikt nie wydaje stosownych decyzji i~robi się zamieszanie.
\item[-] Każda osoba z~kadry powinna mieć wydrukowany, na własny użytek, program obozu.
\item[-] Przypominaniem czegokolwiek po dojściu do schroniska powinna się zająć oboźna lub komendantka podobozu. Głosować nie ma nad czym, bo wszystko jest już dawno ustalone. Błahe decyzje, które należy podjąć w~określonym momencie, powinny być wydawane jednoosobowo bez zbędnych głosowań (informując przy tym inne osoby z~kadry, aby nie zostały wydawane sprzeczne polecenia, ale także dlatego, żeby cała kadra wiedziała co się dzieje w~danej chwili, a~nie dowiadywała się równo z~uczestnikami, a~nawet po nich).
\item[-] Wątpliwości dotyczące logistyki powinny być przedyskutowane w~gronie kadry w~sposób konkretny i~szybki. Jałowe, przeciągające się dyskusje powinny być tępione.
\end{itemize}
\subsection{Członkowie kadry myślą głównie o~zajęciach prowadzonych przez siebie, niedostatecznie interesując się zajęciami prowadzonymi przez innych}
Przed obozem nie zapoznaliśmy się dokładnie z~zajęciami, które miały być prowadzone przez inne osoby, przez co pewne pomysły się powielały. Za rzadko z~własnej inicjatywy pomagaliśmy sobie w~prowadzeniu zajęć, lub np. w~rozłożeniu punktów na grę. Gdy ktoś miał czas, powinien był zaproponować choćby niewielką pomoc.
\begin{itemize}
\item[-] Zajęcia powinny być prowadzone wspólnie i~dobrze przygotowane przed obozem. W~momencie, w~którym trzeba było rozłożyć grę, a~wszyscy byli zmęczeni, każdy liczył, że ktoś inny to zrobi. Dlatego do każdych zajęć powinny być wydelegowane 2~osoby prowadzące, odpowiedzialne za przygotowanie i~przebieg zajęć (1~główna osoba prowadząca i~1~do pomocy w~prowadzeniu zajęć, np. przy drugim końcu stołu albo sali, pilnująca, aby uczestnicy się nie rozpraszali). Pomocnik przydaje się gdy trzeba coś wytłumaczyć jednej lub dwóm osobom nie męcząc grupy powtarzaniem czegoś, co dla reszty jest jasne.
\item[-] Osoba z~niedopracowanymi zajęciami lub potrzebująca chociażby rozstawienia punktów gry nie może bać się prosić o~pomoc. Liczenie, że coś się zrobi samo lub później do niczego nie prowadzi, a~rozpoczęcie zajęć może się opóźnić.
\item[-] Wszystkie osoby z~kadry, które nie były zajęte aktualnymi sprawami organizacyjnymi, powinny brać udział w~zajęciach innych osób, choćby dlatego, aby wiedzieć, jaka wiedza zostaje przekazana ich podopiecznym i~czego mogą wymagać w~ciągu roku lub jaką wiedzę należy zweryfikować ponownie za jakiś czas. Dzisiaj żadna osoba z~kadry nie jest w~stanie stwierdzić, czego ludzie nauczyli się na zajęciach innych osób. Poza tym uczestnicy widzą, że kadra też jest na zajęciach, a~nie ,,siedzi i~się obija albo wariuje z~chłopakami''.
\end{itemize}
\subsection{Spotkania przedobozowe i~zajęcia}
Przed wyjazdem powinniśmy poświęcić więcej czasu sprawie zajęć programowych. Powinny być dopracowane i~ciekawiej przygotowane. Dobrze spotkać się więcej razy wyłącznie w~tym celu i~wymienić się pomysłami, przygotować pewne rzeczy razem, a~jeśli podzielimy się pracą, dać wszystkim dostęp do tego, co stworzyliśmy m.in. aby w~razie czego coś poprawić. Problemem było jeszcze to, że niektóre zajęcia kończyły się przed upływem czasu, który był na nie przeznaczony i~powstawały luki, podczas których ludzie nie mieli nic do roboty więc zakłócali spokój w~schronisku, rozbiegali się itd. Znów rozwiązaniem jest przygotowanie dodatkowych gier i~zajęć do wykorzystania w~takiej sytuacji, np. kiedy osoba odpowiedzialna za jakąś grę jest w~szpitalu.
\begin{itemize}
\item[-] Przed przygotowaniem zajęć sam ich pomysł powinien zostać wspólnie przedyskutowany. Powinien być atrakcyjny i~maksymalnie ,,wypasiony'' --- w~razie problemów jest pełno osób, które znają temat --- można je poprosić o~podpowiedzenie pewnych rozwiązań. Zawsze też można poszukać w~internecie. Jednym słowem --- trzeba się przygotować do tego i~to już od początku kwietnia.
\item[-] Wszystko, co się da, powinno być profesjonalnie przygotowane na długo przed obozem, pokazane reszcie kadry. Powielanie pewnych elementów, jak np. bieganie wg azymutów, na różnych zajęciach nie jest niczym złym, a~utrwala wiedzę i~pokazuje jej użyteczność.
\item[-] Opracowanie zajęć rezerwowych jest konieczne, podobnie jak opracowanie zajęć na czas deszczu. Umarło gdzieś pojęcie pomocy na ochotnika --- każdy był zmęczony, a~przecież kadra powinna się wspierać. Zajęcia powinny być przygotowywane i~prowadzone przez co najmniej 2 osoby. W~tym czasie pozostałe osoby powinny biegać po krzakach i~rozkładać gry.
\item[-] Zajęcia trzeba dostosować do potrzeb uczestników, a~nie kierować się jedynie sztampowym układem zajęć z~każdego obozu. Ważniejszy jest cel, który chcemy osiągnąć.
\item[-] Przed obozem powinnyśmy się zapoznać z~planem zajęć innej osoby by nie powielać formy zajęć lub po prostu doradzić komuś swoimi pomysłami. Przygotowanie nadmiarowych zajęć było by dobrym rozwiązaniem by zapobiec lukom miedzy zajęciami.
\end{itemize}
\subsection{Współpraca drużynowej i~przybocznej}
Nie może być tak, że przyboczna nie uczestniczy w~istotnych dla drużyny rozmowach, nie wie o~nich. Drużynowa powinna często konsultować się z~przyboczną i~podejmować pewne decyzje razem z~nią, a~co najmniej zapoznać ze swoimi planami.
\begin{itemize}
\item[-] Przyboczna musi trzymać rękę na pulsie, być blisko drużynowej i~interesować się wszystkim, a~nie czekać na podanie pod nos informacji i~zaproszenia na spotkanie w~jakiejś sprawie.
\item[-] Przyboczna miała za mało na głowie na tym obozie i~była tak jakby w~połowie uczestnikiem a~w~połowie kadrą --- to nie skończyło się dobrze. Sama powinna wyjść z~inicjatywą, a~nie uciekać od problemów związanych z~drużyną, mówić, że np. ,,i~tak ma za małe doświadczenie i~nic nie wie i~ktoś inny ma decydować''.
\end{itemize}
\subsection{Zajęcia dla wędrowniczek i~harcerek powinny być prowadzone na różnych poziomach lub osobno}
Kiedy jedne słyszą o~pewnych rzeczach pierwszy raz, inne nudzą się, a~czasem nie mają ochoty czynnie uczestniczyć w~zajęciach --- nawet, jeśli nie wszystko wiedzą czy rozumieją. Wiadomo, że niemożliwe jest to, aby drużynowa wędrowniczek sama przygotowała wszystkie zajęcia dla swoich dziewczyn, a~pozostałe osoby --- dla grupy harcerek. Trzeba się rozsądnie podzielić pracą. A~może wędrowniczki same przygotują jakąś grę korzystając z~własnej wiedzy i~umiejętności?
\begin{itemize}
\item[-] Zajęcia mają być atrakcyjne. Aby zainteresować wędrowniczki powinny były być o~wiele bardziej atrakcyjne, żeby utrzymać ich uwagę.
\item[-] Najlepszym pomysłem byłby całkowicie osobny obóz. Połączenie dwóch metod pracy (harcerskiej i~wędrowniczej) nie idzie w~parze. Niektóre wędrowniczki brakiem karności psują zajęcia harcerkom i~dają zły przykład, a~przy tym same się nudzą. Należy ograniczyć ich kontakty do odwiedzin i~sporadycznej pomocy w~czasie wyjazdów. Szczególnie, gdy ,,wędrowniczki'' są w~wieku zastępowych harcerek a~ponadto ich ego nie jest dobrym przykładem. W~ostateczności powinien to być osobny podobóz.
\item[-] Wędrowniczki powinny raczej przygotowywać zajęcia dla harcerek, może to by je usatysfakcjonowało i~poczułyby, że to wcale nie takie proste. Powinno to jednak być bardzo kontrolowane, żeby później nie czuły się jak nie wiadomo kto.
\end{itemize}
\subsection{Drużynowa i~czas spędzany z harcerkami}
Na szlaku nie było widać drużynowej ze swoimi harcerkami, podobnie przy innych okazjach. Z~tego powodu może ona nie wiedzieć o~wszystkim, co się dzieje między harcerkami, jakie mają problemy. A~przecież podczas wędrówki można kogoś lepiej poznać, zrozumieć i~dzięki rozmowie np. domyślić się, jakie zadania na stopień dla danej osoby będą najodpowiedniejsze, dlatego następnym razem trzeba naprawić ten błąd.
\begin{itemize}
\item[-] Drużynowe muszą być świadome, że są kadrą i~całym swym postępowaniem dają przykład innym. Nie można dać sobie chwil na pofolgowanie, bo cały czas jest się obserwowanym chociażby przez przyboczną, która podświadomie chce dorównać. Nawet w~tych złych rzeczach.
\item[-] Na obozie nie powinno się zdarzyć, że ktoś z~kadry nie ma nic do roboty --- zawsze można komuś pomóc w~zajęciach, w~kuchni, czy porozmawiać z~kimś z~uczestników i~przy okazji wpływać na postawę tej osoby.
\item[-] To jest chyba problem dużych obozów (jak np. 25 na obozie wędrownym) gdy nie jesteśmy w~stanie zapanować i~ogarnąć całej grupy ludzi. Rozwiązaniem są wyjazdy mniejszym gronem lub duże zaplecze kadry jednej drużyny, mające świadomość, jak kształtować postawę ludzi w~czasie drogi.
\end{itemize}
\subsection{Brak czasu na pranie}
Często prać można było tylko podczas godziny wyznaczonej na mycie się. Jedni zdążyli, inni nie (np. z~powodu pechowego miejsca w~kolejce), niektórzy w~ogóle nie wpadli na taki pomysł, a~potem chodzili w~brudnych ubraniach. Dobrze jest przeznaczyć np. raz na kilka dni godzinę czy dwie tylko na tę czynność.
\begin{itemize}
\item[-] Najlepiej wyznaczyć taki czas w~dzień wędrówki, po dotarciu do schroniska, albo zaraz następnego dnia rano po śniadaniu.
\item[-] Czas na pranie powinien być wyznaczony osobno poza czasem na mycie się czy jakimś innym czasem wolnym.
\end{itemize}
\subsection{Torby}
Obóz był o~tematyce ekologicznej i~jednym z~jego elementów było zachęcenie uczestniczek do korzystania z~materiałowych toreb zamiast reklamówek. Można było zabierać je na zakupy, aby przyzwyczaić dziewczyny do tego pomysłu. Niestety, wyszywanie na nich haftu łemkowskiego zajęło sporą część obozu. Poza tym niektóre dziewczyny nie chciały już nawet na nie patrzeć, a~więc cały ekologiczny podtekst padł. Następnym razem trzeba zastanowić się, czy zawsze połączenie dwóch pomysłów ma sens.
\begin{itemize}
\item[-] Powinno się zabierać te torby na zakupy obozowe. A~tak wyszło kompletnie bez sensu. A~hafty łemkowskie można było wyszyć na osobistych proporczykach czy jeszcze czymś innym, a~potem ewentualnie naszyć to na taką torbę.
\item[-] Robienie różnych rzeczy powinno mieć coś na celu. Wyszywanie haftu było ciekawe i~ćwiczyło umiejętności hafciarskie (chociaż rozwlekło się w~czasie i~nie miałyśmy na jego temat konkretnych informacji), ale nie miało dalszego przełożenia, stworzone torby były bezużyteczne.
\item[-] Tematyka ekologiczna nie była odczuwalna. Miało na to wpływ pozyskanie środków finansowych z~różnych źródeł --- musieliśmy spełnić określone wymagania, które nie miały zbyt wiele wspólnego z~ekologią.
\end{itemize}
\subsection{Karność wędrowniczek}
Wędrowniczki czuły się w~pewien sposób lepsze od harcerek, chociaż między niektórymi przedstawicielkami obu pionów jest mała lub żadna różnica wieku, a~nawet umiejętności, zaradności itd. Nie dawały dobrego przykładu młodszym dziewczynom, które potem czasem dziwiły się, że wędrowniczkom coś wolno, a~im nie.
\begin{itemize}
\item[-] Postawa i~przemyślenia wynikały jedynie z~nudy, a~przejście do drużyny wędrowniczek rozbudowało ego do niesamowitych rozmiarów (miejsce do pracy śródrocznej). Problemem był także brak autorytetu wśród kadry dziewcząt.
\item[-] Wędrowniczki i~wędrownicy mieli na siebie zły wpływ.
\item[-] Zabrakło autorytetu wśród kadry podobozu harcerek.
\end{itemize}
\subsection{Słaba współpraca z~drużyną wędrowników}
Wędrownicy za mało pomagali przed obozem w~przygotowaniach, a~w~trakcie obozu w~prowadzeniu zajęć, w~sprawach organizacyjnych itd. Czasem nawet nieodpowiednio się zachowywali. Wcześniej mieli obóz drużyny, który był dla nich priorytetem, dlatego teraz było im ,,wszystko jedno''. W~przyszłości, gdyby zaistniała podobna sytuacja, wcześniej trzeba dokładnie określić, jakie są obowiązki członków drużyny męskiej jadącej na obóz z~licznym podobozem żeńskim.
\begin{itemize}
\item[-] Chłopaki nie pomagali przed ani na obozie. A~jeszcze na dodatek często się obijali i~jeśli czegoś nie zrobili to nikt tego nie zrobił. Swoją postawą (rzekomo wędrowniczą) dawali zły przykład.
\item[-] Zachowywali się niedojrzale. Trzeba było zwracać im uwagę, czasem w~podstawowych kwestiach, np. związanych z~kulturą osobistą. Często bardziej utrudniali swoją obecnością, zamiast pomagać. Gdyby taka sytuacja miała miejsce po raz kolejny, należy zadbać o~normalną ilość uczestników z~drużyny męskiej, aby ich kadra się nie nudziła. Poza tym trzeba obiektywnie weryfikować drużyny, z~którymi chce się jechać --- jeśli się ma z~kimś dobre relacje prywatnie, to nie gwarantuje wcale dobrej organizacji wyjazdu.
\item[-] Harcerze zawiedli na całej linii. Myślę, że należy im o~tym bardzo dobitnie powiedzieć, żeby też zrozumieli swój błąd i~nie powtarzali tego przy współpracy z~innymi drużynami.
\end{itemize}
\subsection{Mało czasu dla siebie (odpoczynek)}
Czas wolny wynikał głównie z~tego, że zajęcia ,,się skróciły''. Natomiast czas wolny, który przewidziany był w~planie, był stanowczo za krótki. Ludzie spędzali go głównie stojąc w~kolejce do mycia.
\begin{itemize}
\item[-] Dawna praktyka instruktorska, zawsze mówiła, że uczestnicy na obozie nie mają mieć chwili tzw. czasu wolnego. Być może teraz zmieniły się standardy i~należy równać w~dół do poziomu kolonii, gdzie przed obiadem są słabo przygotowane ,,zajęcia'', a po obiedzie ,,czas wolny'' i~ludzie robią co chcą. Jedyny czas wolny, wg tej praktyki, kiedy ludzie mają chwilę oddechu to:
\begin{itemize}
\item[-] toaleta poranna i~wieczorna (każdy czeka na swoją kolejkę --- przed i~po tym, zanim wszyscy skończą jest czas wolny),
\item[-] czas na pranie (każdy czeka na swoją kolejkę --- przed i~po tym, zanim wszyscy skończą jest czas wolny),
\item[-] cisza poobiednia,
\item[-] chwila czasu przed ciszą nocną i~czas po ciszy nocnej, zanim oboźna się wkurzy, że jest za głośno.
\end{itemize}
Jeśli ma być inaczej, to jest kwestia ustalenia tego na forum Hufca czy Okręgu, bo to raczej powinno być jednolite wszędzie.
\item[-] Wyjść trzeba od solidnego przygotowania zajęć, żeby były ciekawe, intensywne i~odpowiednio długie. Żeby ludzie rzeczywiście odpoczywali w~przewidzianym na to czasie, a~nie szwendali się bez sensu bo zajęcia ,,się skróciły''.
\item[-] Czasu wolnego powinno być jak najmniej, ale w~warunkach prawie codziennego ekstremalnego wysiłku podczas wędrówek, ludzie, w~czasie dnia bez wędrówek, nie mieli już siły aktywnie uczestniczyć w~zajęciach, co też wpływało na ich poziom i~atmosferę --- wiadomo, że gdy ktoś jest zmęczony i~każe mu się iść na bieg, nie będzie zachwycony. Nie było złe to, że dawano ludziom czasami trochę luzu, trzeba tylko pilnować by nie przeszkadzali innym mieszkańcom schroniska.
\item[-] Kwestia sporna i~do ustalenia przed obozem.
\end{itemize}
\subsection{Uczestnicy nie byli kondycyjnie przygotowani do wędrówek i~do długotrwałego wysiłku}
\begin{itemize}
\item[-] Mimo tego, że forma obozu była znana od września ludzie nie zostali przygotowani kondycyjnie do wędrówek. Tylko wędrowniczkom udało się zorganizować pojedyncze krótkie wędrówki przed obozem. Długofalowe przygotowania nie zostały zrealizowane, a~dokładniej: zostały potraktowane jako nieważne. Kadra drużyn nie znalazła w~sobie minimum chęci aby zorganizować w~ciągu roku harcerskiego zajęcia poprawiające kondycję, takie jak: bieganie raz lub dwa razy w~tygodniu, zajęcia na basenie, wędrówki w~trudnym terenie, rajdy. Gorzej --- nikomu nie przyszło do głowy wysłać ludzi na imprezy organizowane przez kogoś innego, np. biegi i~marsze na orientację. Jednym słowem, rady płynące z~doświadczenia komendanta zostały kompletnie zignorowane --- przygotowanie ludzi do obozu i~świadomość konieczności przygotowania kondycyjnego tak naprawdę nie istniała wśród ludzi, a~zwłaszcza wśród kadry. Śródroczna praca drużyn nie miała nic wspólnego z~przygotowaniami do obozu, chociaż miejsce i~cel były znane na początku roku harcerskiego.
\item[-] Drużynowe skupiły się głównie na tym, żeby z~ich drużyn pojechało jak najwięcej osób, wiedząc o~tym, że nie wszystkie dadzą radę. Trzeba obiektywnie i~ostro postawić warunki przed obozem, gdyż później ludzie, nie dający sobie rady, stają się zmorą całej kadry i~innych uczestników, spowalniając tempo marszu i~utrudniając jego przebieg. Dodatkowo ma to bardzo demotywujący wpływ na słabszych uczestników i~obniża morale całego obozu, psując przy okazji atmosferę.
\end{itemize}
\flushbottom
\cleardoublepage
\begin{center}\textbf{\Large Zakończenie}\end{center}
\normalsize
\vspace{0.7cm}
\noindent Powiedz, dokąd znów wędrujesz?\\
\noindent Czy daleko jest twój sad?\\
\noindent Hen, w krainy buczynowe\\
\noindent Ze mną tam układa pieśni wiatr\\
~\\
\noindent Hen, w krainy buczynowe\\
\noindent Ze mną tam nikogo tylko wiatr\\
~\\
\noindent Zmierzchy grają, a przestrzenie\\
\noindent Własny mi podają dźwięk\\
\noindent Takie śpiewy z nimi lub milczenie\\
\noindent W którym znika każdy dawny lęk\\
~\\
\noindent W takich śpiewach i milczeniu\\
\noindent W szumie świętych buków zginął lęk\\
~\\
\noindent Zaszumiały cię powietrza\\
\noindent I ruszyłeś sam na szlak\\
\noindent Ten ostatni, ten najlepszy\\
\noindent Przyszedł czas, Pan dał ci znak\\
~\\
\noindent Ten ostatni, ten najlepszy\\
\noindent Przyszedł czas, Pan dał ci znak\footnote{Piosenka dla Wojtka Bellona, SDM. Wykonanie: Ryczące Czterdziestki} \textbf{\Large \includemovie[text=\eightnote, controls=true, attach=true]{}{}{piosenka-dla-wojtka-bellona.mp3}}
\newpage
\pagestyle{empty}
~
\newpage
\end{document}
\bye
%
%%%
% Document ends.
%%%

